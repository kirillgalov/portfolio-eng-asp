\subsection{Article}

\subsubsection*{Abstract}

We introduce a new class of quantum quantum key distribution protocols, tailored to be robust against photon number splitting (PNS) attacks. We study one of these protocols, which differs from the BB84 only in the classical sifting procedure. This protocol is provably better than BB84 against PNS attacks at zero error.


\subsubsection*{Main}
Quantum cryptography, or more precisely quantum key distribution (QKD) is the only physically secure method for the distribution of a secret key between two distant partners, Alice and Bob [1]. Its security comes from the well-known fact that the measurement of an unknown quantum state modifies the state itself: thus an eavesdopper on the quantum channel, Eve, cannot get information on the key without introducing errors inthe correlations between Alice and Bob. In equivalent terms, QKD is secure because of the no-cloning theorem of quantum mechanics: Eve cannot duplicate the signal and forward a perfect copy to Bob.

In the last years, several long-distance implementations of QKD have been developed, that use photons as information carriers and optical fibers as quantum channels [1]. Most often, although not always [2], Alice sends to Bob a weak laser pulse in which she has encoded the bit. Each pulse is a priori in a coherent state | ue i$\theta$ i of weak intensity, typically u ~ 0.1 photons. However, since no reference phase is available outside Alice’s office, Bob and Eve have no information $\theta$. Consequently, they see the mixed state R d$\theta$ on p = 2pi| ue i$\theta$ ih ue i$\theta$ |. P This state can be re-written as a mixture of Fock states, n p n |nihn|, with the number n of photons distributed according to the Poissonian statistics of mean u, p n = p n (u) = e -u u n /n!. Because two realizations of the same density matrix are indistinguishable, QKD with weak pulses can be re-interpreted as follows: Alice encodes her bit in one photon with frequency p 1 , in two photons with frequency p 2 , and so on, and does nothing with frequency p 0 . Thus, in weak pulses QKD,a rather important fraction of the non-empty pulses actually contain more than one photon. For these pulses,Eve is then no longer limited by the no-cloning theorem:she can simply keep some of the photons while letting the others go to Bob. Such an attack is called photon-number splitting (PNS) attack. Although PNS attacks are far beyond today’s technology [3], if one includes them in the security analysis, the consequences are dramatic [4,5].

In this Letter, we present new QKD protocols that are secure against PNS attack up to significantly longer distances, and that can thus lead to a secure implementation of QKD with weak pulses. These protocols are better tailored than the ones studied before to exploit the correlations that can be established using p. The basic idea is that Alice should encode each bit into a pair of non-orthogonal states belonging to two or more suitable sets.
The structure of the paper is as follows. First, we review the PNS attack on the first and best-known QKD protocol, the BB84 protocol [6], in order to understand why this attack is really devastating when the bit is encoded into pairs of orthogonal states. Then we present the benefits of using non-orthogonal states, mostly by focusing on a specific new protocol which is a simple modification of the BB84.
PNS attacks on the BB84 protocol. Alice encodes each bit in a qubit, either as an eigen state of O x (| + xi coding
0 or | - xi coding 1) or as an eigen state of o z (| + zi coding 0 or | - zi coding 1). The qubit is sent to Bob, who measures either o x or O z . Then comes a classical procedure known as ”sifting” or ”basis-reconciliation”: Alice communicates to Bob through a public classical channel the basis, x or z, in which she prepared each qubit.
When Bob has used the same basis for his measurement,he knows that (in the absence of perturbations, and in particular in the absence of Eve) he has got the correct result. When Bob has used the wrong basis, the partners simply discard that item.
Consider now the implementation of the BB84 protocol with weak pulses. Bob’s raw detection rate is the probability that he detects a photon per pulse sent by Alice. In the absence of Eve, this is given b where N is the quantum efficiency of the detector (typically 10\% at telecom wavelengths), and N B is attenuation due to the losses in the fiber of length l:N B = 10 -B/10 , B = $\alpha$ l [dB] . Below, when we give a distance, we assume the typical value $\alpha$ = 0.25 dB/km. The approximate equality in (1)is valid if N det N B p n n << 1 for all n, which is always the case in weak pulses QKD.
If we endow Eve with unlimited technological power within the laws of physics, the following PNS attack (storage attack) is in principle possible [4,5]: (I) Eve counts the number of photons, using a photon-number quantum non-demolition (QND) measurement; (II) she blocks the single photon pulses, and for the multi-photon pulses she stores one photon in a quantum memory;she forwards the remaining photons to Bob using a perfectly transparent quantum channel, N B = 1 [7]; (III) she waits until Alice and Bob publicly reveal the used base sand correspondingly measures the photons stored in her quantum memory: she has to discriminate between two orthogonal states, and this can be done deterministically.
This way, Eve has obtained full information about Alice’s bits, thence no processing can distill secret keys for the legitimate users; moreover, Eve hasn't introduced any error on Bob’s side.
The unique constraint on PNS attack is that Eve’s presence should not be noticed; in particular, Eve must ensure that the rate of photons received by Bob (1) is not modified [8]. Thus, the PNS attack can be performed on all pulses only when the losses that Bob expects because of the fiber are equal to those introduced by Eve’s storing and blocking photons, that is, when the attenuation in the fiber is larger than a critical value B c BB84 defined by W For u = 0.1, we find B c BB84 = 13 dB, that is l c BB84 ~ 50 km. For shorter distances, Eve can optimize her attack, but won’t be able to obtain full information; Alice and Bob can therefore use a privacy amplification scheme to retrieve a shorter secret key from their data. In conclusion, for B >= B c BB84 , the weak-pulses implementation of the BB84 protocol becomes in principle insecure, even for zero quantum-bit error rate (QBER).
Encoding in non-orthogonal states. The extreme weakness of the BB84 protocol against PNS attacks is due to the fact that whenever Eve can keep one photon, she gets all the information, because after the sifting phase she has to discriminate between two eigen states of a known Hermitian operator. Intuition suggests then that the robustness against PNS attacks can be increased by using protocols that encode the classical bit into pairs of non orthogonal states, that cannot be discriminated deterministically. We prove that this intuition is correct.

To fix the ideas, consider the following protocol using four states: Alice encodes each bit in the state of a qubit,belonging  either to the set A = |0 a i, |1 a i or to the set B = |0 b i, |1 b i , with |h0 a |1 a i| = |h0 b |1 b i| = $\sigma$ 6 = 0(Fig. 1, left). In the absence of an eavesdropper, Bob can be perfectly correlated with Alice: in fact, although the two states are not orthogonal, one can construct a generalized measurement that unambiguously discriminates between the two. The price to pay is that sometimes one gets an inconclusive result [9]. Such a measurement can be realized by a selective filtering, that is filter whose effect is not the same on all states, followed by a von Neumann measurement on the photons that pass the filter [10]. In the example of Fig. 1, the filter that discriminates between the elements of A is given by | + xih1 a | + | - xih0 a  | , where |F  i is the F A =  1+$\sigma$ state orthogonal to |Fi. When the photons are prepared in a state of the pair A, a fraction 1 - $\sigma$ of them pass this filter, and in this case the von-Neumann measurement of O x achieves the discrimination. It is then clear how the cryptography protocol generalizes BB84: Bob randomly applies on each qubit one of the two filters F A or F B , and measures O x on the outcome. Later, Alice discloses for each bit the set A or B: Alice and Bob discard all the items in which Bob has chosen the wrong filter and all the inconclusive results.

Of course, since not all the qubits will pass the filter even when it was correctly chosen, there is a small nuisance on Bob’s side because the net key rate is decreased. This is compensated by increasing u by a factor 1/(1-$\sigma$).
However, the nuisance is by far bigger on Eve’s side, even when the increased mean number of photons u is taken into account. We shall give a detailed analysis of the PNS attacks below for a specific protocol, but a simple estimate shows the origin of the improved robustness. Eve can obtain full information only when (i) she can block all the pulses containing one and two photons, and (ii) on the pulses containing three or more photons, she performs a suitable unambiguous discrimination measurement (see below) and obtains a conclusive outcome, which happens only with probability p ok < 1. Consequently, the critical u attenuation is defined by R raw (B c ) = N det p 3 ( 1-$\sigma$) p ok ,and is determined by p 3 instead of p 2 as in the BB84, see(3). For typical values, B c - B c BB84 ~ 10 dB, which means an improvement of some 40km in the distance [11].

A specific protocol. Here is an astonishingly simple protocol using four non-orthogonal states. Alice sends randomly one of the four states | + xi or | + zi; Bob measures either O x or O z . Thus, at the ”quantum” level,the protocol is identical to BB84, and can be immediately implemented with the existing devices. However,we modify the classical sifting procedure: instead of revealing the basis, Alice announces publicly one of the four pairs of non-orthogonal states A $\omega$,$\omega$  = |$\omega$xi, |$\omega$  zi ,with $\omega$, $\omega$ {+, -}, and with the convention that | + xi code for 0 and | + zi code for 1. Within each set, the overlap of the two states is $\sigma$ =  1 2 . Because of the peculiar choice of states, the usual procedure of choosing randomly between O x or O z turns out to implement the most effective unambiguous discrimination. For definiteness, suppose that for a given qubit Alice has sent | + xi,and that she has announced the set A +,+ . If Bob has measured O x , which happens with probability 2 1 , he has certainly got the result +1; but since this result is possible for both states in the set A +,+ , he has to discard it.
If Bob has measured O z and got +1, again he cannot discriminate. But if he has measured O z and got -1, then he knows that Alice has sent | + xi and adds a 0 to his key.
By symmetry, we see that after this sifting procedure Bob is left with 14 of the raw list of bits, compared to the 12 of the original BB84 protocol. Thus, for a fair comparison with BB84 using u = 0.1, we shall take here u = 0.2, so that the net key rates without eavesdropper at a given distance are the same for both protocols. In spite of the fact that a larger u is used (that is, multi-photon pulses are more frequent), this new protocol is provably better than BB84 against PNS attacks at QBER= 0. This is our main claim, and is demonstrated in the following.

PNS attacks at QBER=0. First, let us prove something that we mentioned above, namely: for protocols using four states like the one under study, Eve can obtain full information from three-photon pulses by using strategies based on unambiguous state-discrimination.
Such strategies have also been considered for BB84, because (although worse than the storage attack for an all-powerful Eve) they don’t require a quantum memory[12], and in their simplest implementation the photon number QND measurement is not required either [13].
The most powerful of these attacks, against which any protocol using four states becomes completely insecure for the three-photon pulses, goes as follows [14]. A pulse containing three photons is necessarily in one of the four x3x3x3 states |f 1 i = | + zi , |F 2 i = | + xi , |F 3 i = | - zi ,x3|F 4 i = | - xi ; that is, in the symmetric subspace of 3 qubits. The dimension of this subspace is 4, and it can be shown that all the |F k i x3 are linearly independent[15]. Therefore, there exist a measurement M that distinguishes unambiguously among them, with some probability of success. In the present case, there exist even four orthogonal states of three qubits, |$\Phi$ k i, k = 1,  , 4,such that |h$\Phi$ i |F j i| =  1 2 B ij [16]. The measurement M is then any von-Neumann measurement discriminating the|$\Phi$ k i; it will give a conclusive outcome with probability p ok = 12 , which is optimal [15,17].
It is then clear that Eve can obtain full information if she can block all the one- and two-photons pulses and half of the three-photon pulses, by applying the following PNS attack: (I) she measures the number of photons;(II) she discards all pulses containing less than 3 photons;(III) on the pulses containing at least 3 photons, she performs M, and if the result is conclusive (which happens with probability p ok > 2 ) she sends a new photon prepared in the good state to Bob. We refer to this attack as to intercept-resend with unambiguous discrimination(IRUD) attack. Neither the quantum memory is needed,nor is the lossless channel, since the new state can be prepared by a friend of Eve located close to Bob.

The critical attenuation B c at which the IRUD attack becomes always possible is defined by N B c u = p ok p 3 (u);for u = 0.2, this gives B c = 25.6 dB ~ 2B c BB84 . Thus, the ultimate limit of robustness (in the case of zero errors) is shifted from - 50km up to - 100km by using our simpl emodification of the BB84 protocol. To further increase the limit of 100km, one can move to protocols using six or more non-orthogonal states [17].
Figure 2 plots Eve’s information for the best PNS attack at QBER= 0, as a function of the attenuation. Note that the new protocol is better than BB84 at any distance. For almost all B < B c , the best PNS attack is not the IRUD but a storage attack, in which Eve keeps one or two photons in a quantum memory and waits for the announcements of the sifting phase. Recall that in BB84,this kind of attack provides Eve with full information. In our protocol Alice announces sets of two non-orthogonal states, so storage attacks give Eve only a limited amount of information.
 If Eve keeps n photons and the overlap can obtain is $\sigma$ (here, 1/ 2), the largest information she p is I(n, $\sigma$) = 1 - H(P, 1 - P ) with P = 2 (1 + 1 - $\sigma$ 2n )[9]. In particular, Eve obtains I(1,  1 2 ) ~ 0.4 bits/pulse for the attenuation B 1 at which she can always keep one photon (B 1 = 11 dB for u = 0.2).

In conclusion: in the limiting case of QBER= 0, our protocol is always more secure than BB84 against PNS attacks, and can be made provably secure against such attacks in regions where BB84 is already provably insecure. Recall that the comparison is made by fixing the net key rates without eavesdropper at a given distance.
Attacks at QBER>0 on the new protocol. In real experiments, dark counts in the detectors and misalignement of optical elements always introduce some errors.
It is then important to show that the specific protocol we presented does not break down if a small amount of error on Bob’s side is allowed. Several attacks at nonzero QBER are described in detail in Ref. [17]. Here, we sketch the analysis of two individual attacks.
First, let us suppose that Eve uses the phase-covariant cloning machine that is the optimal individual attack against BB84 [18]. In the case of the present protocol, Eve can extract less information from her clones,again because Alice does not disclose a basis but a set of non-orthogonal states. As a consequence, the condition I Bob = I Eve is fulfilled up to QBER=15\% [17], a value which is slightly higher than the 14,67\% obtained for BB84. So our new protocol, designed to avoid PNS attacks in a weak-pulses implementation, seems to be robust also against individual eavesdropping in a single photon implementation. Incidentally note that, in the case of a single-photon implementation, our protocol is at least as secure as the B92 protocol in the sens of ”unconditional security” proofs [19]. This is because our protocol can be seen as a modified B92, where Alice chooses randomly between four sets of non-orthogonal states [20].
The second kind of individual attacks that we like to discuss, and that we call PNS+cloning attacks, are specific to imperfect sources. Focus on the range B =10 - 20 dB (see Fig. 2), where one-photon pulses can be blocked and the occurrence of three or more photons is still comparatively rare. Because for the BB84 Eve has already full information in this range, such attacks have never been considered before. Eve could take the two photons, apply an asymmetric 2 -> 3 cloning machine and send one of the clones to Bob; she keeps two clones and some information in the machine. By a suitable choice of the cloning machine, the QBER at which I Bob = I Eve is lowered down to - 9\% [17]. In Ref. [21], a successful qubit distribution over 67km with u = 0.2 and QBER= 5\% has been reported. Under the considered PNS attacks, such distribution is provably insecure using the sifting procedure of BB84, while it can yield a secret key if our sifting procedure is used.

In summary, we have shown that by encoding a classical bit in sets of non-orthogonal qubit states, quantum cryptography can be made significantly more robust against photon-number splitting attacks. We have presented a specific protocol, which is identical to the BB84protocol for all the manipulations at the quantum level and differs only in the classical sifting procedure. Under the studied attacks, our protocol is secure in a region where BB84 is provably insecure. Preliminary studies of more complex attacks suggest that it is at least as robust as BB84 in any situation, and could then replace it. Moreover, our encoding can easily be combined with more complex procedures on the quantum level, e.g. [22].

We thank Norbert Lütkenhaus and Daniel Collins for insightful comments. We acknowledge financial supports by the Swiss OFES and NSF within the European IST project EQUIP and the NCCR "Quantum Photonics".


\subsection{\trnas}
ss
\subsection{\review}
ss
\subsection{\dic}
ss