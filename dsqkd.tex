\subsection{Article}

\subsubsection*{Abstract}

There has been much interest in quantum key distribution. Experimentally, quantum key distribution over 150 km of commercial Telecom fibers has been successfully performed. The crucial is sue in quantum key distribution is its security. Unfortunately, all recent experiments are, in principle,insecure due to real-life imperfections. Here, we propose a method that can for the first time make most of those experiments secure by using essentially the same hardware. Our method is to use decoy states to detect eavesdropping attacks. As a consequence, we have the best of both worlds—enjoying unconditional security guaranteed by the fundamental laws of physics and yet dramatically surpassing even some of the best experimental performances reported in the literature.

\subsubsection{Introduction}

Quantum key distribution (QKD) allows two users, Alice and Bob, to communicate in absolute security in the presence of an eavesdropper, Eve. Unlike conventional cryptography, the security of QKD is based on the fundamental laws of physics, rather than unproven computational assumptions. The security of QKD has been rigorously proven in a number of recent papers [1]. See also[2]. There has been tremendous interest in experimental QKD [3, 4], with the current world record distance of 150km of Telecom fibers[4].

Unfortunately, all those exciting recent experiments are, in principle, insecure due to real-life imperfections.
More concretely, highly attenuated lasers are often used as sources. But, these sources sometimes produce signals that contain more than one photons. Those multi-photon signals open the door to powerful new eavesdropping attacks including photon splitting attack. For example,Eve can, in principle, measure the photon number of each signal emitted by Alice and selectively suppress single photon signals. She splits multi-photon signals, keeping one copy for herself and sending one copy to Bob. Now,since Eve has an identical copy of what Bob possesses,the unconditional security of QKD (in, for example, standard BB84 protocol[5]) is completely compromised.

In summary, in standard BB84 protocol, only signals originated from single photon pulses emitted by Alice are guaranteed to be secure. Consequently, paraphrasing GLLP [6], the secure key generation rate (per signal state emitted by Alice) can be shown to be given by:S  Q  {H 2 (E  ) + [1  H 2 (e 1 )]}(1)where Q  and E  are respectively the gain and quantum bit error rate (QBER) of the signal state [7], s and e 1 are respectively the fraction and QBER of detection events by Bob that have originated from single-photon signals emitted by Alice and H 2 is the binary Shannon entropy.

It is a priori very hard to obtain a good lower bound on sigma and a good upper bound on e 1 . Therefore, prior art methods (as in GLLP [6]) make the most pessimistic assumption that all multi-photon signals emitted by Alice will be received by Bob. For this reason, until now, it has been widely believed that the demand for unconditional security will severely reduce the performance of QKD systems [6, 8, 9, 10, 11].

In this paper, we present a simple method that will provide very good bounds to sigma and e 1 . Consequently, our method for the first time makes most of the long distance QKD experiments reported in the literature unconditionally secure. Our method has the advantage that it can be implemented with essentially the current hardware. So,unlike prior art solutions based on single-photon sources,our method does not require daunting experimental developments. Our method is based on the decoy state idea first proposed by Hwang [12]. While the idea of Hwang was highly innovative, his security analysis was heuristic. The key point of the decoy state idea is that Alice prepares a set of additional states—decoy states, in addition to standard BB84 states. Those decoy states are used for the purpose of detecting eavesdropping attacks only, whereas the standard BB84 states are used for key generation only. The only difference between the decoy state and the standard BB84 states is their intensities(i.e., their photon number distributions).

By measuring the yields and QBER of decoy states, we will show that Alice and Bob can obtain reliable bounds to s and e 1 , thus allowing them to surpass all prior art results substantially [13]. Here, we give for the first time a rigorous analysis of the security of decoy state QKD. Moreover, we show that the decoy state idea can be combined with the prior art GLLP [6] analysis.

Preliminary versions of our result in this paper have appeared in [14, 15], where we presented not only the general theory, but also proposed the idea of using only a few decoy states (for example, three states—the vacuum,a weak decoy state with u decoy << 1 and a signal state with u = O(1). We call this a Vacuum+Weak decoy state protocol). Subsequently, our protocols for decoy state QKD have been analyzed in [16] and more systematically in [17]. See also [18]. Recently, we have provided the first experimental demonstration of decoy state QKD in [19].

We now present the general theory of our new decoy state schemes. We will assume that Alice can prepare phase-randomized coherent states and can turn her powerup and down for each signal. This may be achieved by using standard commercial variable optical attenuators(VOAs) [20]. Let | ue i0 i denote a weak coherent state emitted by Alice. Assuming that the phase, 0, of all signals is totally randomized, the probability distribution for the number of photons of the signal state follows a Poisson distribution with some parameter u. That is to say that, with a probability p n = e -u u n /n!, Alice’s signal will have n photons. In summary, we have assumed that Alice can prepare any Poissonian (with parameter u) mixture of photon number states and, moreover, Alice can vary the parameter, u, for each individual signal.

Let us consider the gain Q u for a coherent state
| e i i. [Here and thereafter, we actually mean the ran
do m mixture of | ue io i over all values of 0 as the phase is assumed to be totally randomized].

Similarly, the QBER can depend on the photon number. Let us define e n as the QBER of an n-photon signal. The QBER E u for a coherent state | ue i0 i is given.

Essence of the decoy state idea Let us imagine that a decoy state and a signal state have the same characteristics (wavelength, timing information, etc). Therefore,Eve cannot distinguish a decoy state from a signal state and the only piece of information available to Eve is the number of photons in a signal. Therefore, the yield, Y n ,and QBER, e n , can depend on only the photon number,n, but not which distribution (decoy or signal) the state is from. We emphasize that the essence of the decoy state idea can be summarized by the following two equations:Y n (signal) = Y n (decoy) = Y ne n (signal) = e n (decoy) = e n .

While a few decoy states are sufficient, for ease of discussion, we will for the moment consider the case where Alice will pick an infinite number of possible intensities for decoy states. Let us imagine that Alice varies over all non-negative values of u randomly and independently for each signal, Alice and Bob can experimentally measure the yield Q u and the QBER E u . Since the relations between the variables Q u ’s and Y n ’s and between E u ’s and e n ’s are linear, given the set of variables Q u ’s and E u ’s measured from their experiments, Alice and Bob can deduce mathematically with high confidence the variables Y n ’s and e n ’s. This means that Alice and Bob can constrain simultaneously the yields, Y n and QBER e n simultaneously for all n. Suppose Alice and Bob know their channel property well. Then, they know what range of values of Y n ’s and e n ’s is acceptable. Any attack by Eve that will change the value of any one of the Y n ’s and e n ’s substantially will, in principle, be caught with high probability by our decoy state method. Therefore, in order to avoid being detected, the eavesdropper, Eve, has very limited options in her eavesdropping attack. In summary,the ability for Alice and Bob to verify experimentally the values of Y n and e n ’s in the decoy state method greatly strengthens their power in detecting eavesdropping, thus leading to a dramatic improvement in the performance of their QKD system.

The decoy state method allows Alice and Bob to detect deviations from the normal behavior due to eavesdropping attacks. Therefore, in what follows, we will consider normal behavior (i.e., the case of no eavesdropping). Details of QKD set-up model can be seen in [17].

Let us discuss the yields, Y n ’s, in a realistic set-up.
(a) The case n = 0.
In the absence of eavesdropping, Y 0 is simply given by the background detection event rate p dark of the system.
(b) The case n > 1. For n > 1, yield Y n comes from two sources, i) the detection of signal photons n n , and ii) the background event p dark . The combination gives,assuming the independence of background and signal detection event where in the second line we neglect the cross term because the background rate (typically 10 -5 ) and transmission efficiency (typically 10 -3 ) are both very small.

Suppose the overall transmission probability of each photon is n. In a normal channel, it is common to assume independence between the behaviors of the n photons. Therefore, the transmission efficiency for n-photon signals n n is given by:n n = 1 - (1 - n) n ,(7)[For a small n and ignore the dark count, Y n = nn.]QBER Let us discuss the QBERs, e n ’s, in a realistic experiment.
(a) If the signal is a vacuum, Bob’s detection is due to background including dark counts and stray light due to timing pulses. Assuming that the two detectors have equal background event rates, then the output is totally random and the error rate is 50\%. That is, the QBER for the vacuum e 0 = 1/2.
(b) If the signal has n > 1 photons, it also has some error rate, say e n . More concretely, e n comes from two parts, erroneous detections and background contribution, where e detector is independent of n.
The values of Y n and e n can be experimentally verified by Alice and Bob using our decoy state method. Any attempt by Eve to change them significantly will almost always be caught.
Combining decoy state idea with GLLP Suppose key generation is done on signal state | ue i0 i. In principle,Alice and Bob can isolate the single-photon signals and apply privacy amplification to them only. Therefore, generalizing the work in GLLP, Pwe find Eq. (1) where the gain of the signal state, Q u = k=0 Y k e -u u k /k! , [This comes directly from Eq. (2).] and the fraction of Bob’s detection events that have originated from single-photon signals emitted by Alice is the gain for the single photon state.

The derivation of Eq. (1) assumes that error correction protocols can achieve the fundamental (Shannon) limit.
However, practical error correction protocols are generally inefficient. As noted in [22], a simple way to take this inefficiency into account is to introduce a function,f (e) > 1, of the QBER, e. By doing so, we find that the key generation rate for practical protocols is given by:S > q{-Q u f (E u )H 2 (E u ) + Q 1 [1 - H 2 (e 1 )]},(11)where q depends on the implementation (1/2 for the BB84 protocol, because half the time Alice and Bob bases are not compatible, and if we use the efficient BB84 protocol [23], we can have q = 1. For simplicity, we will take q = 1 in this paper.), and f (e) is the error correction efficiency [22].

Let us now compare our result in Eq. (11) with the prior art GLLP result. In the prior art GLLP [6] method,secure key generation rate is shown to be at least where s, the fraction of “untagged” photons, (which is a pessimistic estimation of the fraction of detection events by Bob that have originated from single-photon signals emitted by Alice), is given by 1 - s = p multi /Q u ,(13)where p multi is the probability of Alice’s emitting a multi photon signal. Eq. (13) represents the worst situation where all the multi-photons emitted by Alice will be received by Bob.

Comparing our result (given in Eq. (11)) with the prior art GLLP result (given in Eq. (12)), we see that the main difference is that in our result, a much better lower bound on s and a much better upper bound on e 1can be obtained.

Implication of our result We obtain substantially higher key generation rate than in [6]. In more detail,note that, from Eq. (6), Y n for n > 2 is of similar order to Y 1 . Therefore, from Eq. (11) it is now advantageous for Alice to pick the average photon number in her signal state to be u = O(1). Therefore, the key generation rate in our new method is O(n) where n is the overall transmission probability of the channel. In comparison, in prior art methods for secure QKD, u is chosen to be of order O(n), thus giving a net key generation rate of O(n 2 ).
In summary, we have achieved a substantial increase in net key generation rate from O(n 2 ) to O(n). Moreover, as will be discussed below, our decoy state method allows secure QKD at much longer distances than previously thought possible.

More concretely, we [15] have applied our results to various experiments in the literature. The results are shown in Fig. 1 using the GYS [3] experiment as an example. We found that the optimal averaged number u in GYS that maximizes the key generation rate in our decoy state method in Eq. (11) is, indeed, of O(1) (roughly 0.5). Therefore, the key generation rate is of order O(n). We remark that the calculated optimal value of photon number of 0.5 is, in fact, higher than what experimentalists have been using. Experimentalists often liberally pick 0.1 as a convenient number for average photon number without any security justification. In other words, operating their equipment with the parameters proposed in the present paper will allow experimentalists to not only match, but also surpass their current experimental performance (by having at least five-fold the current experimental key generation rate). This demonstrates clearly the power of decoy state QKD. Moreover, Fig. 1 shows that with our decoy state idea, secure QKD can be done at distances over 140 km with only current technology. In summary, our result shows that we can have the best of both worlds: Enjoy both unconditional security and record-breaking experimental performance. The general principle of decoy state QKD developed here can have widespread applications in other set-ups (e.g. open-air QKD or QKD with other photon sources) and to multiparty quantum cryptographic protocols such as [24]. As demonstrated clearly in [17], one can achieve almost all the benefits of our decoy state method with only one or two decoy states. See also [16]. Recently, we have experimentally demonstrated decoy state QKD in [19].

We have benefitted greatly from enlightening discussions with many colleagues including particularly G. Brassard. Financial support from funding agencies such as CFI, CIPI, CRC program, NSERC, OIT, and PREAare gratefully acknowledged. H.-K. L also thanks travel support from the INI, Cambridge, UK and from the IQIat Caltech through NSF grant EIA-008603

\subsection{\trnas}
\subsubsection*{Аннотация}

Наблюдается большой интерес к квантовому распределению ключей. Экспериментально было успешно выполнено квантовое распределение ключей по 150 км коммерческого телекоммуникационного волокна. Решающим вопросом в квантовом распределении ключей является его безопасность. К сожалению, все последние эксперименты в принципе небезопасны из-за несовершенства реальной среды. Здесь мы предлагаем метод, который впервые может сделать большинство этих экспериментов безопасными, используя практически то же самое оборудование. Наш метод заключается в использовании ложных состояний для обнаружения атак подслушивания. В результате мы получаем лучшее из двух миров - безусловную безопасность, гарантированную фундаментальными законами физики, и при этом значительно превосходящую даже некоторые из лучших экспериментальных результатов, о которых сообщалось в литературе.

\subsubsection{Введение}

Квантовое распределение ключей (QKD) позволяет двум пользователям, Алисе и Бобу, общаться в абсолютной безопасности в присутствии подслушивающего устройства, Евы. В отличие от обычной криптографии, безопасность QKD основана на фундаментальных законах физики, а не на недоказанных вычислительных предположениях. Безопасность QKD была строго доказана в ряде недавних работ [1]. См. также[2]. В последнее время наблюдается огромный интерес к экспериментальному QKD [3, 4], с текущим мировым рекордом расстояния в 150 км по волокнам Telecom[4].

К сожалению, все эти захватывающие недавние эксперименты в принципе небезопасны из-за несовершенства реальной жизни. Более конкретно, в качестве источников часто используются сильно затухающие лазеры. Но эти источники иногда производят сигналы, содержащие более одного фотона. Эти многофотонные сигналы открывают дверь для новых мощных атак подслушивания, включая атаку с расщеплением фотонов. Например, Ева может, в принципе, измерить количество фотонов в каждом сигнале, испускаемом Алисой, и выборочно подавить однофотонные сигналы. Она разделяет многофотонные сигналы, сохраняя одну копию для себя и отправляя одну копию Бобу. Теперь, поскольку у Евы есть идентичная копия того, чем обладает Боб, безусловная безопасность QKD (например, в стандартном протоколе BB84[5]) полностью нарушена.

В целом, в стандартном протоколе BB84 гарантируется безопасность только тех сигналов, которые исходят от однофотонных импульсов, испускаемых Алисой. Следовательно, перефразируя GLLP [6], безопасная скорость генерации ключей (на состояние сигнала, испускаемого Алисой) может быть показана следующим образом:S Q {H 2 (E ) + [1 H 2 (e 1 )]}(1)где Q и E - соответственно коэффициент усиления и квантовая скорость битовой ошибки (QBER) состояния сигнала [7], s и e 1 - соответственно доля и QBER событий обнаружения Бобом, которые произошли от однофотонных сигналов, испускаемых Алисой, а H 2 - двоичная энтропия Шеннона.

Априори очень трудно получить хорошую нижнюю границу для сигмы и хорошую верхнюю границу для e 1 . Поэтому методы предшествующего уровня техники (как в GLLP [6]) делают самое пессимистичное предположение, что все многофотонные сигналы, испускаемые Алисой, будут получены Бобом. По этой причине до сих пор широко распространено мнение, что требование безусловной безопасности сильно снизит производительность систем QKD [6, 8, 9, 10, 11].

В этой статье мы представляем простой метод, который обеспечит очень хорошие границы сигмы и e 1 . Следовательно, наш метод впервые делает большинство экспериментов по QKD на больших расстояниях, о которых сообщалось в литературе, безусловно безопасными. Преимущество нашего метода заключается в том, что он может быть реализован с использованием практически всего современного оборудования. Таким образом, в отличие от предшествующих решений, основанных на однофотонных источниках, наш метод не требует огромных экспериментальных разработок. Наш метод основан на идее состояния приманки, впервые предложенной Хвангом [12]. Хотя идея Хванга была весьма инновационной, его анализ безопасности был эвристическим. Ключевой момент идеи ложных состояний заключается в том, что Алиса готовит набор дополнительных состояний - ложных состояний, в дополнение к стандартным состояниям BB84. Эти состояния-обманки используются только для обнаружения атак подслушивания, в то время как стандартные состояния BB84 используются только для генерации ключей. Единственное различие между состояниями-обманками и стандартными состояниями BB84 заключается в их интенсивности (т.е. в распределении числа фотонов).

Измеряя доходность и QBER состояний приманки, мы покажем, что Алиса и Боб могут получить надежные границы s и e 1 , что позволит им существенно превзойти все результаты предшествующей техники [13]. Здесь мы впервые даем строгий анализ безопасности QKD с ложными состояниями. Более того, мы показываем, что идея состояния приманки может быть объединена с предыдущим анализом GLLP [6].

Предварительные версии нашего результата в этой статье появились в [14, 15], где мы представили не только общую теорию, но и предложили идею использования только нескольких состояний приманки (например, три состояния - вакуум, слабое состояние приманки с u приманки << 1 и сигнальное состояние с u = O(1). Мы называем такой протокол протоколом с состоянием вакуум+слабая приманка). Впоследствии наши протоколы для QKD с состоянием приманки были проанализированы в [16] и более систематически в [17]. См. также [18]. Недавно мы представили первую экспериментальную демонстрацию QKD с ложным состоянием в [19].

Теперь мы представим общую теорию наших новых схем обманных состояний. Мы будем считать, что Алиса может готовить фазово-рандомизированные когерентные состояния и может увеличивать и уменьшать свою мощность для каждого сигнала. Это может быть достигнуто с помощью стандартных коммерческих переменных оптических аттенюаторов (VOA) [20]. Пусть | ue i0 i обозначает слабое когерентное состояние, излучаемое Алисой. Если предположить, что фаза 0 всех сигналов полностью случайна, то распределение вероятности для числа фотонов состояния сигнала соответствует распределению Пуассона с некоторым параметром u. То есть, с вероятностью p n = e -u u n /n!, сигнал Алисы будет содержать n фотонов. Таким образом, мы предположили, что Алиса может подготовить любую пуассоновскую (с параметром u) смесь состояний с числом фотонов и, более того, Алиса может варьировать параметр u для каждого отдельного сигнала.

Рассмотрим коэффициент усиления Q u для когерентного состояния | e i i. [Здесь и далее мы фактически имеем в виду, что сделать m смесь | ue io i по всем значениям 0, так как предполагается, что фаза полностью рандомизирована].

Аналогично, QBER может зависеть от числа фотонов. Определим e n как QBER n-фотонного сигнала. QBER E u для когерентного состояния | ue i0 i дано.

Представим, что состояние приманки и состояние сигнала имеют одинаковые характеристики (длина волны, информация о времени и т.д.). Поэтому Ева не может отличить состояние приманки от состояния сигнала, и единственная информация, доступная Еве, - это количество фотонов в сигнале. Поэтому выход, Y n , и QBER, e n , могут зависеть только от числа фотонов, n, но не от того, из какого распределения (приманка или сигнал) состояние. Подчеркнем, что суть идеи состояния приманки может быть обобщена следующими двумя уравнениями:Y n (сигнал) = Y n (приманка) = Y ne n (сигнал) = e n (приманка) = e n .

Хотя достаточно нескольких ложных состояний, для простоты обсуждения мы пока рассмотрим случай, когда Алиса выберет бесконечное число возможных интенсивностей для ложных состояний. Представим, что Алиса варьирует все неотрицательные значения u случайным образом и независимо для каждого сигнала, Алиса и Боб могут экспериментально измерить выход Q u и QBER E u . Поскольку отношения между переменными Q u и Y n и между E u и e n линейны, учитывая набор переменных Q u и E u, измеренных в ходе экспериментов, Алиса и Боб могут математически с высокой степенью достоверности вывести переменные Y n и e n . Это означает, что Алиса и Боб могут одновременно ограничить урожайность, Y n и QBER e n одновременно для всех n. Предположим, что Алиса и Боб хорошо знают свойство своего канала. Тогда они знают, какой диапазон значений Y n 's и e n 's является приемлемым. Любая атака Евы, которая существенно изменит значение любого из Y n и e n, в принципе, будет с высокой вероятностью поймана нашим методом ложных состояний. Поэтому, чтобы избежать обнаружения, подслушивающая Ева имеет очень ограниченные возможности в своей подслушивающей атаке. В итоге, возможность Алисы и Боба экспериментально проверить значения Y n и e n в методе обманного состояния значительно усиливает их возможности в обнаружении подслушивания, что приводит к значительному улучшению производительности их системы QKD.

Метод ложного состояния позволяет Алисе и Бобу обнаружить отклонения от нормального поведения из-за атак подслушивания. Поэтому в дальнейшем мы будем рассматривать нормальное поведение (т.е. случай отсутствия подслушивания). Подробности модели настройки QKD можно найти в [17].

Давайте обсудим доходность Y n в реалистичной ситуации.
(a) Случай n = 0.
В отсутствие подслушивания Y 0 просто определяется фоновой частотой событий обнаружения p системы.
(b) Случай n > 1. При n > 1 доход Y n поступает из двух источников, i) обнаружение сигнальных фотонов n n , и ii) фоновое событие p. Комбинация дает , предполагая независимость фона и события обнаружения сигнала, где во второй строке мы пренебрегаем перекрестным членом, потому что уровень фона (обычно 10 -5 ) и эффективность передачи (обычно 10 -3 ) очень малы.

Предположим, что общая вероятность передачи каждого фотона равна n. В нормальном канале принято считать, что поведение n фотонов независимо друг от друга. Поэтому эффективность передачи для n-фотонных сигналов n n дается следующим образом:n n = 1 - (1 - n) n ,(7)[Для малого n и игнорирования темнового счета, Y n = nn.]QBER Давайте обсудим QBERs, e n 's, в реалистичном эксперименте.
(a) Если сигналом является вакуум, то обнаружение Боба обусловлено фоном, включающим темные отсчеты и рассеянный свет из-за синхроимпульсов. Если предположить, что оба детектора имеют равную частоту фоновых событий, то выходной сигнал полностью случаен и коэффициент ошибок составляет 50\%. То есть, QBER для вакуума e 0 = 1/2.
(b) Если сигнал имеет n > 1 фотонов, он также имеет некоторый коэффициент ошибок, скажем e n . Более конкретно, e n складывается из двух частей, ошибочных обнаружений и фонового вклада, где e детектора не зависит от n.
Значения Y n и e n могут быть экспериментально проверены Алисой и Бобом с помощью нашего метода ложных состояний. Любая попытка Евы существенно изменить их почти всегда будет поймана.
Объединение идеи состояния приманки с GLLP Предположим, что генерация ключей осуществляется на сигнальном состоянии | ue i0 i. В принципе, Алиса и Боб могут изолировать однофотонные сигналы и применять усиление конфиденциальности только к ним. Поэтому, обобщая работу в GLLP, мы находим уравнение (1), где усиление состояния сигнала, Q u = k=0 Y k e -u u k /k! [Это следует непосредственно из уравнения (2)], а доля событий обнаружения Боба, которые произошли от однофотонных сигналов, испущенных Алисой, является усилением для однофотонного состояния.

Вывод уравнения (1) предполагает, что протоколы коррекции ошибок могут достичь фундаментального (шенноновского) предела.
Однако практические протоколы коррекции ошибок, как правило, неэффективны. Как отмечается в [22], простой способ учесть эту неэффективность - ввести функцию, f (e) > 1, от QBER, e. Сделав это, мы обнаружим, что скорость генерации ключей для практических протоколов дается следующим образом:S > q{-Q u f (E u )H 2 (E u ) + Q 1 [1 - H 2 (e 1 )]},(11)где q зависит от реализации (1/2 для протокола BB84, поскольку в половине случаев базы Алисы и Боба несовместимы, и если мы используем эффективный протокол BB84 [23], мы можем иметь q = 1. Для простоты в данной работе мы будем принимать q = 1.), а f (e) - эффективность коррекции ошибок [22].

Давайте теперь сравним наш результат в уравнении (11) с результатом, полученным в предшествующем методе GLLP. В известном методе GLLP [6] безопасная скорость генерации ключей, как показано, составляет, по меньшей мере, где s - доля "неотмеченных" фотонов (которая является пессимистической оценкой доли событий обнаружения Боба, которые произошли от однофотонных сигналов, испущенных Алисой), дается 1 - s = p multi /Q u ,(13)где p multi - вероятность того, что Алиса испустила многофотонный сигнал. Уравнение (13) представляет наихудшую ситуацию, когда все многофотонные сигналы, испущенные Алисой, будут получены Бобом.

Сравнивая наш результат (приведенный в уравнении (11)) с результатом GLLP (приведенным в уравнении (12)), мы видим, что основное различие заключается в том, что в нашем результате можно получить гораздо лучшее нижнее ограничение на s и гораздо лучшее верхнее ограничение на e 1.

Следствие нашего результата Мы получаем существенно более высокую скорость генерации ключей, чем в [6]. Более подробно, отметим, что из уравнения (6) следует, что Y n для n > 2 имеет тот же порядок, что и Y 1. Поэтому, исходя из уравнения (11), Алисе выгодно выбрать среднее число фотонов в сигнальном состоянии равным u = O(1). Таким образом, скорость генерации ключей в нашем новом методе составляет O(n), где n - общая вероятность передачи данных по каналу. Для сравнения, в известных методах безопасного QKD u выбирается порядка O(n), что дает чистую скорость генерации ключей O(n 2 ).
В итоге мы добились существенного увеличения чистой скорости генерации ключей с O(n 2 ) до O(n). Более того, как будет показано ниже, наш метод обманного состояния позволяет обеспечить безопасное QKD на гораздо больших расстояниях, чем считалось ранее возможным.

Более конкретно, мы [15] применили наши результаты к различным экспериментам в литературе. Результаты показаны на рис. 1 на примере эксперимента GYS [3]. Мы обнаружили, что оптимальное усредненное число u в GYS, которое максимизирует скорость генерации ключей в нашем методе состояния приманки в уравнении (11), действительно имеет порядок O(1) (приблизительно 0,5). Следовательно, скорость генерации ключей имеет порядок O(n). Заметим, что рассчитанное оптимальное значение числа фотонов 0,5 на самом деле выше, чем то, которое используют экспериментаторы. Экспериментаторы часто либерально выбирают 0,1 в качестве удобного числа для среднего числа фотонов без какого-либо обоснования безопасности. Другими словами, эксплуатация оборудования с параметрами, предложенными в настоящей статье, позволит экспериментаторам не только соответствовать, но и превзойти текущую экспериментальную производительность (по крайней мере, в пять раз превысив текущую экспериментальную скорость генерации ключей). Это наглядно демонстрирует возможности QKD с обманным состоянием. Более того, на рис. 1 показано, что с нашей идеей обманного состояния безопасное QKD может быть выполнено на расстоянии более 140 км с использованием только текущей технологии. В целом, наш результат показывает, что мы можем получить лучшее из двух миров: наслаждаться как безусловной безопасностью, так и рекордной экспериментальной производительностью. Разработанный здесь общий принцип QKD с обманным состоянием может иметь широкое применение в других установках (например, QKD под открытым небом или QKD с другими источниками фотонов) и в многосторонних квантовых криптографических протоколах, таких как [24]. Как было ясно продемонстрировано в [17], можно достичь почти всех преимуществ нашего метода ложных состояний только с одним или двумя ложными состояниями. См. также [16]. Недавно мы экспериментально продемонстрировали QKD с ложными состояниями в [19].

Мы получили большую пользу от просветительских дискуссий со многими коллегами, включая, в частности, Г. Брассарда. Финансовая поддержка от таких финансирующих организаций, как CFI, CIPI, CRC program, NSERC, OIT и PREA, выражает нам признательность. H.-K. L также благодарит за поддержку поездок от INI, Кембридж, Великобритания, и от IQI в Калтехе через грант NSF EIA-008603.

\subsection{\review}
Some scientists mistakenly believe that this [6] paper presented a new protocol, but in fact the authors showed a new quantum trap method. The authors of the paper used the method together with the BB84 protocol, but they also claim that it can be used with any quantum key distribution protocol.

The essence of the method is that along with photons containing confidential information, trap photons are sent through the communication channel. They do not contain any meaningful information. When an intruder reads a phototrap, he increases the amount of noise in the communication channel, thereby giving himself away, and does not receive any confidential information.

As a result, the authors were able to increase the distance of secure transmission of information over a quantum communication channel from 30 km to 150 km.


\subsection{\dic}
\begin{multicols}{2}
	\begin{itemize}
		
		\item algorithms - алгоритм
		\item analysis - анализ
		
		\item appropriate - подходящий
		\item approximately - примерно
		
		
		\item basis - основа
		\item beam - луч
		
		\item binary - двоичных
		\item bit - бит
		
		\item capacity - вместимость
		
		\item channel - канал
		
		\item coherent - связный
		\item combination - комбинации
		
		\item communication - связь
		\item compare - сравнить
		
		\item computation - вычисления
		\item computers - компьютеров
		
		\item condition - условие
		\item conjugate - спряжение
		\item considered - рассмотрено
		\item contain - содержат
		
		\item correlation - корреляция
		
		\item cryptography - криптография
		
		\item decode - декодировать
		\item decoy - ловушка
		\item density - плотность
		
		\item dependence - зависимость
		\item detect - обнаружить
		
		\item deterministic - детерминированный
		
		\item developing - разработка
		
		\item difference - разница
		
		\item differentiate - дифференцировать
		
		\item distribution - распределение
		
		\item eavesdropper - подслушиватель
		
		\item encoded - закодировано
		\item entaglement - запутанность
		\item equivalently - эквивалентно
		
		\item imperfections - недостаток
		\item implementation - реализация
		
		\item instances - экземпляров
		
		\item intensity - интенсивность
		\item intercept - перехват
		
		\item interferometer - интерферометр
		
		\item limitation - ограничение
		
		\item lossless - без потерь
		\item lossy - потери
		
		\item malicious - вредоносных
		\item managed - управляемых
		
		\item mathematical - математических
		\item matrix - матрица
		
		\item measurement - измерения
		
		\item method - метод
		\item mimic - имитировать
		
		\item modification - модификация
		
		\item neglected - пренебрегают
		
		\item normalized - нормализовано
		
		\item operator - оператор
		\item optical - оптический
		\item optimal - оптимальный
		
		\item optional - опционально
		
		\item orthogonal - ортогональный
		\item orthogonality - ортогональность
		\item orthonormal - ортонормированный
		
		\item phase - фаза
		\item photon - фотон
		
		\item polarization - поляризация
		
		
		\item probabilistic - вероятностный
		\item probability - вероятность
		\item problem - проблема
		
		\item projecting - проектирование
		
		\item property - свойство
		\item proportion - пропорция
		\item proposal - предложение
		
		\item protocol - протокол
		
		\item prove - доказательство
		
		\item provide - обеспечить
		\item provides - обеспечивает
		
		\item public - общедоступных
		\item pulse - импульс
		
		\item quantum - квантовый
		\item random - случайных
		\item randomize - рандомизировать
		
		\item signature - подпись
		
		\item strategy - стратегия
		
		\item string - строка
		
		\item symmetric - симметричный
		
		\item theoretical - теоретический
		\item theory - теория
		
		\item transmission - передача
		
		\item vacuum - вакуум
		\item value - значение
		
		\item variable - переменная
		
		
		
	\end{itemize}
\end{multicols}