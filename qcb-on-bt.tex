
\subsection{Article}

\subsubsection*{Abstract}

Practical application of the generalized Bell s theorem in the so-called key distribution process in cryptography is reported. The proposed scheme is based on the Bohm's version of the Einstein-Podolsky-Rosen gedanken experiment and Bell's theorem is used to test for eavesdropping.

\subsubsection{Main}

Cryptography, despite a colorful history that goes back to 400 B.C. , only became part of mathematics and information theory this century, in the late 1940s, mainly due to the seminal papers of Shannon [1]. Today, one can briefly define cryptography as a mathematical system of transforming information so that it is unintelligible and therefore useless to those who are not meant to have access to it. However, as the computational process associated with transforming the information is always performed by physical means, one cannot separate the mathematical structure from the underlying laws of physics that govern the process of computation [2]. Deutsch has shown that quantum physics enriches our computational possibilities far beyond classical Turing machines[2], and current work in quantum cryptography originated by Bennett and Brassard provides a good example of this fact [3].

In this paper I will present a method in which the security of the so-called key distribution process in cryptography depends on the completeness of quantum mechanics. Here completeness means that quantum description provides maximum possible information about any system under consideration. The proposed scheme is based onthe Bohm's well-known version of the Einstein-PodolskyRosen gedanken experiment [4]; the generalized Bell's theorem (Clauser-Horne-Shimony-Holt inequalities) [5] is used to test for eavesdropping. From a theoretical point of view the scheme provides an interesting and new extension of Bennett and Brassard's original idea, and from an experimental perspective offers a practical realization by a small modification of experiments that were set up to test Bell's theorem. Before I proceed any further let me first introduce some basic notions of cryptography.
Originally the security of a cryptotext depended on the secrecy of the entire encrypting and decrypting procedures; however, today we use ciphers for which the algorithm for encrypting and decrypting could be revealed to anybody without compromising the security of a particular cryptogram. In such ciphers a set of specific parameters, called a key, is supplied together with the plaintext as an input to the encrypting algorithm, and together with the cryptogram as an input to the decrypting algorithm. The encrypting and decrypting algorithms are publicly announced; the security of the cryptogram depends entirely on the secrecy of the key, and this key,which is very important, may consist of any randomly chosen, suSciently long string of bits. Once the key is established, subsequent communication involves sending cryptograms over a public channel which is vulnerable to total passive interception (e.g. , public announcement in mass media). However, in order to establish the key, two users, who share no secret information initially, must at a certain stage of communication use a reliable and a very secure channel. Since the interception is a set of measurements performed by the eavesdropper on this channel, however dificult this might be from a technological point of view, in principle any classical channel can always be passively monitored, without the legitimate being aware that any eavesdropping users has taken place. This is not so for quantum channels [3]. In the following I describe a quantum channel which distributes the key without any "element of reality" associated with the key and which is protected by the completeness of quantummechanics.

The channel consists of a source that emits pairs of spin- 2 particles, in a singlet state. The particles Ay apart along the z axis, towards the two legitimate users of the channel, say, Alice and Bob, who, after the particles have on spin components separated, perform measurements along one of three directions given by unit vectors a; and b, (i, =1, 2, 3), respectively, for Alice and Bob. For simplicity, both a; and bj vectors lie in the x-y plane, perpendicular to the trajectory of the particles, and are characterized by azimuthal angles: Pi'=0, Pz = 4 tr, P3 and pi = —, x, p2 = —, tr, p3 = —, tr. Superscripts "a" and "b" refer to Alice and Bob's analyzers, respectively, and the angle is measured from the vertical x axis. The users choose the orientation of the analyzers randomly and independently for each pair of incoming particles. Each in 2 6 units, can yield two results, +1 measurement,(spin up) and — 1 (spin down), and can potentially reveal one bit of information.
The quantity E(a, , b, ) =P++(a, , b, )+P(a, , b, )— P+ (a;, b~) — P — +(a;, b~)is the correlation coefficient of the measurements formed by Alice along a; and by Bob along b~. Here P+ ~ (a;, b~) denotes the probability that result +'1 has 1 along b~. According to been obtained along a; and the quantum rules~ E(a, , b, ) = — a b . (2)of the same orientation(a2, bi and a3, b2) quantum mechanics predicts total anticorrelation of the results obtained by Alice and Bob:E(a, , b, ) =E(a, , b, ) = — 1.

Let us also, following Clauser, Horne, Shimony, and Holt [5], define a quantity composed of the correlation coefficients for which Alice and Bob used analyzers of different orientation.

After the transmission has taken place, Alice and Bob can announce in public the orientations of the analyzers they have chosen for each particular measurement and divide the measurements into two separate groups: a first of group for which they used different orientation analyzers, and a second group for which they used the same orientation of their analyzers. They discard all measurements in which either or both of them failed to register a particle at all. Subsequently, Alice and Bob can reveal publicly the results they obtained but within the first group of measurements only. This allows them to establish the value of S, which, if the particles were not should reproduce the directly or indirectly "disturbed,result of Eq. (4). This assures the legitimate users that the results they obtained within the second group of measurements are anticorrelated and can be converted into a secret string of bits the key. This secret key may be then used in a conventional cryptographic communication between Alice and Bob.

The eavesdropper cannot elicit any information from the particles while in transit from the source to the legitimate users, simply because there is no information encoded there. The information "comes into being" only after the legitimate users perform measurements and communicate in public afterwards. The eavesdropper may try to substitute his own prepared data for Alice and Bob to misguide them, but as he does not know which orientation of the analyzers will be chosen for a given pair of particles, there is no good strategy to escape from being detected. In this case his intervention will be equivalent to introducing elements of physical reality to the measurements of the spin components. This can be easily modified (by the eaves seen if we put appropriately correlation coefficients dropper perfect measurement)into Eq. (3). We obtain formula where n, and nb are two unit vectors (for particles a and b, respectively), oriented along the directions of the quantization axes for which the eavesdropper acquired information about the spin component of a given particle. This information could be acquired either through a direct, "brute" measurement of the spin components or through a more subtle attack on the source, e. g. , substituting a source that produces a state of two spin- 2 particles correlated with another quantum system on which the actual measurement will be performed by the eavesdropper. The normalized probability measure p(n„nb)describes the eavesdropper strategy (probability of intercepting a spin component along a given direction for a particular measurement).

This way it has been shown that the generalized Bell's theorem can have a practical application in cryptography, namely, it can test the safety of the a key distribution. It is not a mathematical difficulty of a particular computation, but a fundamental physical law that protects the system, and as long as quantum theory is not refuted as a complete theory the system is secure.

Regarding more refined attacks associated with the faked source of three (or more) correlated particles, one may think, for example, about delayed measurement on the third particle which is correlated with the two spin- 2particles. By "delayed" I mean "after the orientation of the analyzers has been publicly revealed by Alice and Bob. However„as we want the two particles to be in pure, singlet state, and Alice and Bob test for it through Bell's theorem, then we cannot correlate the third particle with the other two without disturbing the purity of the singlet state. Therefore I conjecture that there is no universal (good for all orientations a;, bj. ) state of the faked source which will pass the statistical test of the legitimate users on the subsystem of the two correlated particles a and b. As Alice and Bob can also delay their public communication, the eavesdropper faces the problem of storing the third particle undisturbed for an appropriately long period of time.

I have already mentioned that the proposed channel can be realized as a modification of experiments that tested Bell's theorem. In particular, the celebrated experiment of Aspect and co-workers [6], in which polarized particles, would be photons were used instead of spin the most obvious choice. In the experiment, every 10 ns pairs of photons were emitted in a radiative atomic cascade of calcium. Acousto-optical switches were used to change the orientation of the analyzers in a time short compared with the photon transit time, and the detection efficiency was over 95\%. Apart from changing the main objective of the experiment, and some details in the setup,one will also need software to simulate Alice, Bob, and optionally the eavesdropper. The modifications are minor, so it raises hopes for experimental realization in the nearest future.

The authors thanks D. Deutsch, P. L. Knight, K. Burnett, S. M. Barnett, C. H. Bennett, A. Zeilinger, P. Grangier, G. M. Palma, and P. G. H. Sandars for interesting comments and discussion. This work was supported by Pirie-Reid Fund at Oxford University.

\subsection{\trnas}


\subsubsection*{Abstract}

Сообщается о практическом применении обобщенной теоремы Белла в так называемом процессе распределения ключей в криптографии. Предлагаемая схема основана на бомовской версии эксперимента Эйнштейна-Подольского-Розена с геданкеном, а теорема Белла используется для проверки на подслушивание.

\subsubsection{Main}

Криптография, несмотря на красочную историю, восходящую к 400 году до нашей эры, стала частью математики и теории информации только в этом веке, в конце 1940-х годов, в основном благодаря основополагающим работам Шеннона [1]. Сегодня криптографию можно кратко определить как математическую систему преобразования информации таким образом, что она становится неразборчивой и, следовательно, бесполезной для тех, кто не должен иметь к ней доступ. Однако, поскольку вычислительный процесс, связанный с преобразованием информации, всегда выполняется физическими средствами, нельзя отделить математическую структуру от лежащих в ее основе законов физики, которые управляют процессом вычислений [2]. Дойч показал, что квантовая физика обогащает наши вычислительные возможности намного больше, чем классические машины Тьюринга[2], а текущая работа в области квантовой криптографии, начатая Беннетом и Брассардом, служит хорошим примером этого факта [3].

В этой статье я представлю метод, в котором безопасность так называемого процесса распределения ключей в криптографии зависит от полноты квантовой механики. Здесь полнота означает, что квантовое описание предоставляет максимально возможную информацию о любой рассматриваемой системе. Предлагаемая схема основана на известной версии эксперимента Эйнштейна-Подольского-Розена с геданкеном [4]; для проверки на подслушивание используется обобщенная теорема Белла (неравенства Клаузера-Хорна-Шимони-Холта) [5]. С теоретической точки зрения схема представляет собой интересное и новое расширение оригинальной идеи Беннета и Брассарда, а с экспериментальной точки зрения предлагает практическую реализацию путем небольшой модификации экспериментов, которые были поставлены для проверки теоремы Белла. Прежде чем продолжить, позвольте мне сначала ввести некоторые основные понятия криптографии.
Изначально безопасность криптотекста зависела от секретности всей процедуры шифрования и дешифрования, однако сегодня используются шифры, для которых алгоритм шифрования и дешифрования может быть раскрыт кому угодно без ущерба для безопасности конкретной криптограммы. В таких шифрах набор определенных параметров, называемых ключом, подается вместе с открытым текстом на вход алгоритма шифрования, а вместе с криптограммой - на вход алгоритма дешифрования. Алгоритмы шифрования и дешифрования объявляются публично; безопасность криптограммы полностью зависит от секретности ключа, а этот ключ, что очень важно, может состоять из любой случайно выбранной, достаточно длинной строки битов. После установления ключа последующая коммуникация предполагает передачу криптограмм по публичному каналу, который уязвим для полного пассивного перехвата (например, публичное объявление в СМИ). Однако для того, чтобы установить ключ, два пользователя, которые изначально не делятся секретной информацией, должны на определенном этапе общения использовать надежный и очень защищенный канал. Поскольку перехват представляет собой набор измерений, выполняемых подслушивающим устройством на этом канале, как бы трудно это ни было с технологической точки зрения, в принципе, любой классический канал всегда может быть пассивно проконтролирован, при этом законный пользователь не будет знать, что произошло подслушивание. Это не так для квантовых каналов [3]. Далее я описываю квантовый канал, который распространяет ключ без какого-либо "элемента реальности", связанного с ключом, и который защищен полнотой квантовой механики.

Канал состоит из источника, который испускает пары частиц со спином 2, находящихся в синглетном состоянии. Частицы разлетаются вдоль оси z в направлении двух законных пользователей канала, скажем, Алисы и Боба, которые, после того как частицы разделили спиновые компоненты, проводят измерения в одном из трех направлений, заданных единичными векторами a; и b, (i, =1, 2, 3), соответственно, для Алисы и Боба. Для простоты, оба вектора a; и bj лежат в плоскости x-y, перпендикулярной траектории частиц, и характеризуются азимутальными углами: Pi'=0, Pz = 4 tr, P3 и pi = -, x, p2 = -, tr, p3 = -, tr. Суперскрипты "a" и "b" относятся к анализаторам Алисы и Боба, соответственно, а угол измеряется от вертикальной оси x. Пользователи выбирают ориентацию анализаторов случайным образом и независимо для каждой пары входящих частиц. Каждый из 2 6 блоков может дать два результата, +1 измерение, (спин вверх) и - 1 (спин вниз), и потенциально может раскрыть один бит информации.
Величина E(a, , b, ) =P++(a, , b, )+P(a, , b, )- P+ (a;, b~) - P - +(a;, b~)- это коэффициент корреляции измерений, сформированных Алисой вдоль a; и Бобом вдоль b~. Здесь P+ ~ (a;, b~) обозначает вероятность того, что результат +'1 имеет 1 вдоль b~. Согласно полученным вдоль a; и квантовым правилам~ E(a, , b, ) = - a b . (2)одинаковой ориентации (a2, bi и a3, b2) квантовая механика предсказывает полную антикорреляцию результатов, полученных Алисой и Бобом:E(a, , b, ) =E(a, , b, ) = - 1.

Давайте также, следуя Клаузеру, Хорну, Шимони и Холту [5], определим величину, состоящую из коэффициентов корреляции, для которых Алиса и Боб использовали анализаторы разной ориентации.

После того, как передача произошла, Алиса и Боб могут публично объявить ориентации анализаторов, которые они выбрали для каждого конкретного измерения, и разделить измерения на две отдельные группы: первая группа, для которой они использовали анализаторы разной ориентации, и вторая группа, для которой они использовали анализаторы одинаковой ориентации. Они отбрасывают все измерения, в которых один из них или оба не смогли зарегистрировать частицу. Впоследствии Алиса и Боб могут публично раскрыть полученные ими результаты, но только в рамках первой группы измерений. Это позволяет им установить значение S, которое, если частицы не было, должно воспроизвести прямо или косвенно "возмущенный" результат уравнения (4). Это гарантирует законным пользователям, что результаты, полученные ими во второй группе измерений, являются антикоррелированными и могут быть преобразованы в секретную строку битов - ключ. Этот секретный ключ может быть затем использован в обычной криптографической коммуникации между Алисой и Бобом.

Подслушивающее устройство не может извлечь никакой информации из частиц во время их передачи от источника к законным пользователям, просто потому, что в них не закодирована никакая информация. Информация "появляется" только после того, как законные пользователи проводят измерения и после этого публично общаются. Подслушивающий может попытаться подставить Алисе и Бобу свои собственные подготовленные данные, чтобы ввести их в заблуждение, но поскольку он не знает, какая ориентация анализаторов будет выбрана для данной пары частиц, у него нет хорошей стратегии, чтобы избежать обнаружения. В этом случае его вмешательство будет эквивалентно внесению элементов физической реальности в измерения спиновых компонент. Это может быть легко модифицировано (с помощью карниза, если мы подставим соответствующим образом коэффициенты корреляции капельницы идеального измерения) в уравнение (3). Мы получим формулу, где n, и nb - два единичных вектора (для частиц a и b, соответственно), ориентированных вдоль направлений осей квантования, для которых подслушивающий получил информацию о спиновой компоненте данной частицы. Эта информация может быть получена либо путем прямого, "грубого" измерения спиновых компонент, либо путем более тонкой атаки на источник, например, подстановкой источника, который производит состояние двух частиц со спином 2, коррелированное с другой квантовой системой, на которой подслушивающее устройство будет проводить фактическое измерение. Нормированная вероятностная мера p(n "nb)описывает стратегию подслушивающего устройства (вероятность перехвата спиновой компоненты вдоль заданного направления для конкретного измерения).

Таким образом, было показано, что обобщенная теорема Белла может иметь практическое применение в криптографии, а именно, она может проверить безопасность распределения ключей. Это не математическая трудность конкретного вычисления, а фундаментальный физический закон, который защищает систему, и до тех пор, пока квантовая теория не опровергнута как полная теория, система безопасна.

Что касается более тонких атак, связанных с поддельным источником трех (или более) коррелированных частиц, можно подумать, например, об отложенном измерении третьей частицы, которая коррелирована с двумя частицами со спином 2. Под "отложенным" я имею в виду "после того, как ориентация анализаторов будет публично раскрыта Алисой и Бобом". Однако "поскольку мы хотим, чтобы две частицы находились в чистом, синглетном состоянии, и Алиса и Боб проверяют это с помощью теоремы Белла, то мы не можем соотнести третью частицу с двумя другими без нарушения чистоты синглетного состояния. Поэтому я предполагаю, что не существует универсального (хорошего для всех ориентаций a;, bj. ) состояния поддельного источника, которое пройдет статистическую проверку легитимных пользователей на подсистеме двух коррелированных частиц a и b. Поскольку Алиса и Боб могут также задержать свое публичное сообщение, подслушивающий сталкивается с проблемой хранения третьей частицы в невозмущенном состоянии в течение достаточно длительного периода времени.

Я уже упоминал, что предложенный канал может быть реализован как модификация экспериментов, проверявших теорему Белла. В частности, наиболее очевидным выбором является знаменитый эксперимент Аспекта и соавторов [6], в котором вместо спина использовались поляризованные частицы, которые могли бы быть фотонами. В эксперименте каждые 10 нс пары фотонов испускались в радиационном атомном каскаде кальция. Акустооптические переключатели использовались для изменения ориентации анализаторов за время, малое по сравнению со временем прохождения фотонов, и эффективность обнаружения составила более 95\%. Помимо изменения основной цели эксперимента и некоторых деталей в установке, потребуется программное обеспечение для моделирования Алисы, Боба и, по желанию, подслушивающего устройства. Изменения незначительны, поэтому это позволяет надеяться на экспериментальную реализацию в ближайшем будущем.

Авторы благодарят Д. Дойча, П. Л. Найта, К. Бернетта, С. М. Барнетта, К. Х. Беннета, А. Цайлингера, П. Гранжье, Г. М. Пальму и П. Г. Х. Сандарса за интересные комментарии и обсуждение. Эта работа была поддержана фондом Pirie-Reid в Оксфордском университете.


\subsection{\review}
ss
\subsection{\dic}
ss