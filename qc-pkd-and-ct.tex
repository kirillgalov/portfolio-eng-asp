\subsection{Article}

\subsubsection*{Abstract}
When elementary quantum systems, such as polarized photons, are used to transmit digital information, the uncertainty principle gives rise to novel cryptographic phenomena unachievable with traditional transmission media, e.g. a communications channel on which it is impossible in principle to eavesdrop without a high probability of disturbing the transmission in such a way as to be detected. Such a quantum channel can be used in conjunction with ordinary insecure classical channels to distribute random key information between two users with the assurance that it remains unknown to anyone else, even when the users share no secret information initially. We also present a protocol for coin-tossing by exchange of quantum messages, which is secure against traditional kinds of cheating, even by an opponent with unlimited computing power, but ironically can be subverted by use of a still subtler quantum phenomenon, the Einstein-Podolsky-Rosen paradox.

\subsubsection{Introduction}

Conventional cryptosystems such as ENIGMA,DES, or even RSA, are based on a mixture of guess­work and mathematics. Infonu.tion theory shows that traditional secret-key cryptosystems car.not be to­tally secure unless the key, used once only, is at least as long as the clear text. On the other hand,the theory of computational complexity is not yet well enough understood to prove the computational security of public-key cryptosystems.

In this paper we use a radically different foundation for cryptography, viz. the uncertainty principle of quantum physics. In conventional information  theory and cryptography,· it is taken for granted that digital communications in principle can always be passively monitored or copied, even by someone ignorant of their meaning. However, when information is encoded in non-orthogonal quantum states, such as single photons with polarization directions 0, 45, 90, and 135 degrees, one obtains a communications channel whose transmissions in prin­ciple cannot be read or copied reliably by an eaves­dropper ignorant of certain key information used informing the transmission. The eavesdropper cannot even gain partial information about such a transmis­sion without altering it a random and uncontrollable way likely to be detected by the channel's legiti­mate user.

Quantum coding was first described in (W),along with two applications: making money that is in principle impossible to counterfeit, and multi plex­ing two or three messages in such a way that reading one destroys the others. More recently [BBBW],quantum coding has been used in conjunction with public key cryptographic techniques to yield several schemes,s for unforgeable subway tokens. Here we show that quantum coding by itself achieves one of the main advantages of public key cryptography by per­mitting secure distribution of random key informa­tion between parties who share no secret information initially, provided the parties have access, besides the quantum channel, to an ordinary channel suscep­tible to passive but not active eavesdropping. Even in the presence of active eavesdropping, the two parties can still distribute key securely if they share some secret information initially, provided the eavesdropping is not so active as to suppress communications completely. We also present a proto­col for coin tossing by exchange of quantum mes­sages. Except where otherwise noted the protocols are provably secure even against an opponent with superior technology and unlimited computing power,barring fundamental violations of accepted physical laws.

Offsetting these advantages is the practical disadvantage that quantum transmissions are neces­sarily very weak and cannot be amplified in transit. Moreover, quantum cryptography does not provide di­gital signatures, or applications such as certified mail or the ability to settle disputes before a judge.

\subsubsection{Essential Properties of Polarized Photons}

Polarized light can be produced by sending an ordinary light beam through a polarizing apparatus such as a Polaroid filter or calcite crystal; the beam's polarization axis is determined by the orien­tation of the polarizing apparatus in which the beam originates. Generating single polarized photons is also possible, in principle by picking them out of a polarized beam, and in practice by a variation of. an experinient [AGR] of Aspect, et. a

Although polarization is a continuous varia­ble, the uncertainty principle forbids measurements on any single photon from revealing more than one bit about its polarization. For example, if a light beam with polarization axis is a sent into a filter oriented at angle P, the individual photons behave dichotomously and probabilistically, being transmit­ted with probability cos 2 (a-P) and absorbed with the complementary probability sin 7 («-P). The photons behave deterministically only when the two axes arc parallel (certain transmission) or perpendicular (certain absorbtion).

If the two axes are not perpendicular, so that some photons are transmitted, one might hope to learn additional information about a by measuring the transmitted photons again with a polarizer ori­ented at some third angle; but this is to no avail,because the transmitted photons, in passing through polariza­polarizer, emerge with exactly polarization, having lost all memory of their previous po­larization a.

Another way one might hope to learn more than one bit from a single photon would be not to measure it directly, but rather somehow amplify it into a clone of identically polarized photons, then perform measurements on these; but this hope is also vain,because such cloning can be shown to be inconsistent with the foundations of quantum mechanics [WZ) .

Formally, quantum mechanics represents the internal state of a quantum system (e.g. the polarization ­of unit length df a photon) as a vector in a linear space H over the field of complex num­bers (Hilbert space). The inner product of two vec j+j where, is defined as complex conjugation. The dimensionality of the Hilbert space depends on the system, being larg­er (or even infinite) for more complicated systems. Each physical measurement H that might be performed on the system corresponds to resolution of its Hilbert space into orthogonal subspaces, one foreach possible outcome of the measurement. The num­ber of possible outcomes is thus limited to the dimensionality d of the Hilbert space, the most complete measurements being those that resolve the Hilbert space into d 1-dimensional subspaces.

The Hilbert space for a single polarized pho­ton is 2-dimensional; thus the state of a photon maybe completely described as a linear combination of,for example, the two unit vectors r1  (1,0) and r2 = (0,1), representing respectively horizontal and vertical. In particular, a photon  polarized at angle a to the horizontal is described (cosa, sina). When subjected to a measurement of vertical-vs.-horizontal polari­zation, such a photon in effect chooses to become horizontal with probability cos 2 a and vertical with probability sin 2 a. The two orthogonal vectors r1 and r2 thus a exemplify the resolution of a 2-dimensional Hilbert space into'2 orthogonal 1-dimensional subspaces; henceforth r1 and r2 will be said to comprise the 'rectilinear' basis for the Hilbert space.

An alternative basis for the same Hilbert space is provided by the two 'diagonal' basis vec­tors d1 (0. 707 ,0.'707), representing a 45-degree photon, and d2 (0.707,-0.707), representing a 135-degree photon. Two bases (e.g. rectilinear and diagonal) are said to be 'conjugate' (WI, if each vector of one basis has equal-length projections onto all vectors of the other basis: this means that a system prepared in a specific state of one basis will behave entirely randomly, and lose all its stored information, when subjected to a measurement corresponding to the other basis. Owing to the com­plex nature of its coefficients, the two-dimensional Hilbert space also admits a third basis conjugate to both the rectilinear and diagonal bases, comprising the two so-called 'circular' polarizations c1 (0.707,0.707i) and c2 (0.707i,0.70"1); but the rectilinear and diagonal bases are all that will be needed for the cryptographic applications in this paper.

The Hilbert space for a compound system is constructed by taking the tensor product of the Hil­bert spaces of its components; thus the state of a pair of photons is characterized by a-unit vector in the 4-dimensional Hilbert space spanned by the or­thogonal basis vectors r1r 1, r 1r2 , r2r1 , and r 2 r 2 . This formalism entails that the state of a compound system is not generally expressible as the cartesian product of the states of its parts: e.g. the·Einstei'n-Podolsky-Rosen state of two photons, 0.7071(r1r 2 -r2 r1), to be discussed later, is not equivalent to any product of one-photon state.

\subsubsection{Quantum Public Key Distribution}

In traditional public-key cryptography, trap ­door functions arc used to conceal the meaning of messages between two users from a passive eavesdrop­per, depite the lack of any initial shared secret information between the two users. In quantum pub­lic key distribution, the quantum channel is not used directly to send meaningful messages, but is rather used to transmit a supply of random bits be­tween two users who share no secret information ini­tially, in such a way that the users, by subsequent consultation over an ordinary non-quantum channel subject to passive eavesdropping, can tell with high,probability whether the original quantum transmis­sion has been disturbed in transit, as it would be by an eavesdrooper (it is the quantum channel's pe­culiar virtue to compel eavesdroppinq to be active). If the transmission has not been disturbed, they agree to use these shared secret bits in the well ­known way as a one-time pad to conceal the meaning of subsequent meaningful communications, or for oth­er cryptographic applications (e.g. authentication tags) requiring shared secret random information. If transmission has been disturbed, they discard it and try again, deferring any meaningful communica­tions until they have succeeded in transmitting enough random bits through the quantum channel to serve as a one-time pad.

In more detail one user ('Alice') chooses a random bit string and a random sequence of polariza­tion bases (rectilinear or diagonal). She then sends the other user (Bob) a train of photons, eachrepresenting one bit of the string in the basis cho­sen for that bit position, a horizontal or 45-degree photon standing for a binary zero and a vertical or 135-degree photon standing for a binary 1. As Bob receives the photons he decides, randomly for each photon and independently of Alice, whether to meas­ure the photon's rectilinear polarization or its diagonal polarization, and interprets the result of the measurement as a binary zero or one. As ex­plained in the previous section a random answer is produced and all information lost when one attempts to measure the rectilinear polarization of a diago­nal photon, or vice versa. Thus Bob obtains mean­ingful data from only half the photons he detects-­those for which he guessed the correct polarization basis. Bob's information is further degraded by the fact that, realistically, some of the photons would be lost in transit or would fail to be counted by Bob's imperfectly-efficient detectors.

Subsequent steps of the protocol take place over an ordinary public communications channel, as­sumed to be susceptible to eavesdropping but not to the injection or alteration of messages. Bob and Alice first determine, by public exchange of mes­sages, which photons were successfully received and of these which were received with the correct basis. If the quantum transmission has been undisturbed, Alice and Bob should agree on the bits encoded by these photons, even this data has never been dis­cussed over the public channel. Each of these pho­tons, in other words, presumably carries one bit of random information (e.g. whether a rectilinear pho­ton was vertical or horizontal) known to Alice and Bob but to no one else.

Because of the random mix of rectilinear and diagonal photons in the quantum transmission, any eavesdropping carries the risk of altering the transmission in such a way as to produce disagree­ment between Bob and Alice on some of the bits on which they think they should agree. Specifically,it can be shown that no measurement on a photon in transit, by an eavesdropper who is informed of the photon's original basis only after he has performed his measurement, can yield more than 1/2 expected bits of information about the key bit encoded by that photon; and that any such measurement yielding b bits of expected information (b s 1/2) must induce a disagreement with probability at least b/2 if the measured photon, or an attempted forgery of it,is later re-measured in its original basis. (This optimum trade off occurs, for example, when the ea­vesdropper measures and retransmits all intercepted photons in the rectilinear basis, thereby learning the correct polarizations of half the photons and inducing disagreements in 1/4 of those that are lat­er re-measured in the original basis.)

Alice and Bob can therefore test for eaves­dropping by publicly comparing some of the bits on which they think they should agree, though of course this sacrifices the secrecy of these bits. The bit positions used in this comparison should be a random subset (say one third) of the correctly received bits, so that eavesdropping on more than a few pho­tons is unlikely to escape detection. If all the comparisons agree, Alice and Bob can conclude that the quantum transmission has been free of signifi­cant eavesdropping; and those of the remaining bits that were sent and received with the same basis also agree, and can safely be used as a one time pad for subsequent secure communications over the public channel. When this one-time pad is used up, the protocol is repeated to send a new body of random information over the quantum channel.

The need for the public (non-quantum) channel in this scheme to be immune to active eavesdropping can be relaxed if the Alice and Bob have agreed be­fore hand on a small secret key, which they use ·to create Wegman-Carter authentication tags WCI for their messages over the public channel. In more detail the Weqman-Carter multiple-message authenti­cation scheme uses a small random key to produce a message-dependent 'tag' (rather like a check sum)for an arbitrary large message, in such a way that an eavesdropper ignorant of the key has only a small probability of being able to generate any other va­lid message-tag pairs. The tag thus provides evi­dence that the message is legitimate, and was not generated or altered by someone ignorant of the key. (Key bits are gradually used up in the Wegman-Carter scheme, and cannot be reused without compromising the system's provable security; however, in the present application, these key bits can be replaced by fresh random bits successfully transmitted through the quantum channel.) The eavesdropper can still prevent communication by suppressing messages in the public channel, as of course he can by sup­pressing or excessively perturbing the photons sent through the quantum channel. However, in either case, Alice and Bob will conclude with high proba­bility that their secret communications are being suppressed, and will not be fooled into thinking their communications are secure when in fact they're not.

\subsubsection{Quantum Coin Tossing}

'Coin Flipping by Telephone' was first dis­cussed by Blum [Bl]. The problem is for two dis­trustful parties, communicating at a distance with­out the help of a third party, to come to agree on a winner and a loser in such a way that each party has exactly 50 per cent chance of winning. Any attempt by either party to bias the outcome should be de­tected by the other party as cheating. Previous protocols for this problem are based on unproved assumptions in computational complexity theory, which makes them vulnerable to a breakthrough in algorithm design.

1. Alice chooses randomly one basis (say rectili­near) and a sequence of random bits (one thousand should be sufficient). She then encodes her bits as a sequence of photons in this same basis, using the same coding scheme as before. She sends the result­ing train of polarized photons to Bob.

2. Bob chooses, independently and randomly for each photon, a sequence of reading bases. He reads the photons accordingly, recording the results in two tables, one of rectilinearly received photons and one of diagonally received photons. Because of losses in his detectors and of the transmission channel, some of the photons may not be received a tall, resulting in holes in his tables. At this time, Bob makes his guess as to which basis Alice used, and announces it to Alice. He wins if heguessed correctly, loses otherwise.

3. Alice reports to Bob whether he won, by telling him which basis she had actually used. She certif­ies this information by sending Bob, over a classi­cal channel, her entire original bit sequence used in step 1.

4. Bob verifies that no cheating has occurred by comparing Alice's sequence with both his tables. There should be perfect agreement with the table corresponding to Alice's basis and no correlation with the other table. In our example, Bob can be confident that Alice's original basis was indeed rectilince  as claimed.

Alice could attempt cheating either at step 1 or step 2. Let us first assume that she follows step 1 honestly and finds herself losing at the end if step 2, because Bob made he correct guess, here rectiliniear. In order to pretend she has won, she would need to convince Bob that her photons were diagonally polarized, which she can only do by prod­ucing a sequence of bits in perfect agreement with Bob's diagonal table. This she cannot do reliably because this table is the result of probabilistic behavior of the photons after the left her hands. Suppose she goes ahead anyway and sends Bob a new 'original' sequence, different from the one that she used in step 1, in hopes that it will by luck agree perfectly with Bob's diaognal table. This attempt to cheat requires Alice to be not only lucky but daring, because in the vast majority of cases, the gamble would fail and would be detected as cheating. By contrast, in traditional coin-tossing schemes,analogous attempts to seize a lucky victory from the jaws of defeat, though unlikely to succeed, are unaccompanied by any danger of detection.

It is easy to see that things are even worse for Alice if she attempts to cheat in step 1, by sending a mixture of rectilinear and diagonal pho­tons, or photons which are polarized neither rectil­inearly or diagonally. In this case she will not be able to agree with either of Bob's tables in step 3,since both tables will record the results of proba­bilistic behavior not under her control.

In order to say how Alice can cheat using quantum mechanics it is necessary to describe the Einstein- Podolsky-Rosen (EPR) effect [Bo,AGR), often called a paradox because it contradicts the common ­sense notion that for two individually random events happening at distance from one another to be corre­lated, some physical influence must have propagated from the earlier event to the later, or else from some common random cause to both events.

The EPR effect occurs when certain types of atom or molecule decay with the emission of two pho­tons, and consists of the fact that the two photons are always found to have opposite polarization, re­gardless of the basis used to observe them, provided both are observed in the same basis. For example,if both Photons are measured rectilinearly, it will always be found that one is horizontal and the other vertical, though which is horizontal will vary ran­domly from one decay to the next. If both photons are measured diagonally, one will always be 135-degree and the other 45-degree. A moment's reflec­tion will show that this behavior cannot be explained by assuming the decay produce a distribution over a of oppositely polarized (a and a+90 1photons, since, in that case, if such a pair of photons were measured in an intermediate basis (say a+45), both would behave probabilistically so as to sometimes come out with the same polarization.

Probably the simplest, but paradoxical ­sounding, verbal explanation of the EPR effect is to say that the two photons are produced in an initial state of undefined polarization; and when one of them is measured, the measuring apparatus forces it to choose a polarization (choosing randomly and equiprobably between the two characteristic direc­tions offered by the apparatus) while simultaneously forcing the other unmeasured photon, no matter how far away, to choose the opposite polarization. This implausible-sounding explanation is supported by formal quantum mechanics, which represents the state of a pair of photons as a vector in a 4-dimensional Hilbert space obtained by taking the tensor product of two 2-dimensional Hilbert spaces. The EPR state produced by the decay is described by the vector 0.7071(r 1r2 - r2r1), and the EPR effect is explained by the fact that this vector has anticorrelated pro­jections into the 2-dimensional Hilbert spaces of the two photons no matter what basis is used to ex­press the tensor product.

In order to cheat, Alice produces a number of EPR photon-pairs instead of individual random pho­tons in step 1. In each case she sends Bob one mem­ber of the pair and stores the other herself, per­haps between perfectly reflecting mirrors. 


\subsection{\trnas}

\subsubsection*{Аннотация}
Когда элементарные квантовые системы, такие как поляризованные фотоны, используются для передачи цифровой информации, принцип неопределенности приводит к новым криптографическим явлениям, недостижимым с традиционными средствами передачи, например, к каналу связи, на котором в принципе невозможно подслушать без высокой вероятности нарушения передачи таким образом, чтобы быть обнаруженным. Такой квантовый канал может быть использован в сочетании с обычными небезопасными классическими каналами для распространения случайной ключевой информации между двумя пользователями с гарантией, что она останется неизвестной никому другому, даже если пользователи изначально не делятся никакой секретной информацией. Мы также представляем протокол для бросания монет путем обмена квантовыми сообщениями, который защищен от традиционных видов мошенничества, даже со стороны противника с неограниченной вычислительной мощностью, но, по иронии судьбы, может быть подменен с помощью еще более тонкого квантового явления, парадокса Эйнштейна-Подольского-Розена.

\subsubsection{Введение}

­­Традиционные криптосистемы, такие как ENIGMA, DES или даже RSA, основаны на смеси предположений и математики.Теория информатики показывает, что традиционные криптосистемы с секретным ключом не могут быть полностью безопасными, пока ключ, используемый только один раз, не будет, по крайней мере, таким же длинным, как и чистый текст. С другой стороны, теория сложности вычислений еще недостаточно хорошо изучена, чтобы доказать вычислительную безопасность криптосистем с открытым ключом.

­­­­В данной работе мы используем радикально иную основу для криптографии - принцип неопределенности квантовой физики. В обычной теории информации и криптографии считается само собой разумеющимся, что цифровые сообщения в принципе всегда могут быть пассивно отслежены или скопированы, даже тем, кто не знает их смысла.Однако, когда информация закодирована в неортогональных квантовых состояниях, таких как одиночные фотоны с направлениями поляризации 0, 45, 90 и 135 градусов, получается канал связи, передачи которого в принципе не могут быть надежно прочитаны или скопированы подслушивающим лицом не знающим определенной ключевой информации, используемой при передаче. Подслушивающее лицо не может получить даже частичную информацию о такой передаче не изменив ее случайным и неконтролируемым образом, который может быть обнаружен законным пользователем канала.

­­­­­­Квантовое кодирование было впервые описано в (W), вместе с двумя приложениями: изготовление денег, которые в принципе невозможно подделать, и мультиплексирование двух или трех сообщений таким образом, что чтение одного из них уничтожает остальные. Совсем недавно [BBBW] квантовое кодирование было использовано в сочетании с криптографическими методами с открытым ключом для создания нескольких схем, в том числе и для не подделываемых жетонов метро. Здесь мы показываем, что квантовое кодирование само по себе достигает одного из основных преимуществ криптографии с открытым ключом позволяя безопасно распределять случайную информацию между сторонами, которые изначально не делятся секретной информацией, при условии, что стороны имеют доступ, помимо квантового канала, к обычному каналу, восприимчивому к пассивному, но не активному подслушиванию. Даже при наличии активного подслушивания, две стороны могут безопасно распределять ключи, если они изначально делятся некоторой секретной информацией, при условии, что подслушивание не настолько активно, чтобы полностью подавить связь. Мы также представляем протокол для подбрасывания монеты путем обмена квантовыми сообщениями. Протоколы доказательно безопасны даже против противника с превосходящей технологией и неограниченной вычислительной мощностью, исключая фундаментальные нарушения принятых физических законов.

­­Эти преимущества нивелируются практическим недостатком: квантовые передачи неизбежно очень слабы и не могут быть усилены в пути.Более того, квантовая криптография не обеспечивает цифровых подписей, а также таких приложений, как сертифицированная почта или возможность разрешать споры перед судьей.

\subsubsection{Основные свойства поляризованных фотонов}

­Поляризованный свет может быть получен путем пропускания обычного светового пучка через поляризующее устройство, такое как фильтр Polaroid или кристалл кальцита; ось поляризации пучка определяется ориентацией поляризующего устройства, в котором пучок исходит. Генерация одиночных поляризованных фотонов также возможна, в принципе, путем выделения их из поляризованного пучка, а на практике с помощью вариации.

­­Хотя поляризация является непрерывной переменной, принцип неопределенности запрещает измерения любого отдельного фотона, чтобы выявить более одного бита о его поляризации. Например, если пучок света с осью поляризации a направить в фильтр, ориентированный под углом P, отдельные фотоны ведут себя дихотомически и вероятностно, передаваясь вероятностью cos 2 (a-P) и поглощаясь с дополнительной вероятностью sin 7 (-P). Фотоны ведут себя детерминированно только тогда, когда две оси дуги параллельны (определенная передача) или перпендикулярны (определенное поглощение).

­­­Если две оси не перпендикулярны, так что некоторые фотоны пропускаются, можно надеяться получить дополнительную информацию об a, измеряя пропущенные фотоны снова с поляризатором, ориентированным под некоторым третьим углом; но это бесполезно, потому что пропущенные фотоны, проходя через поляризатор, выходят с точной поляризацией, потеряв всю память о своей предыдущей поляризации.

Другой способ, с помощью которого можно было бы надеяться узнать больше одного бита от одного фотона - это не измерять его непосредственно, а каким-то образом усилить его в копии идентично поляризованных фотонов, а затем провести измерения на них; но эта надежда также тщетна, потому что такое клонирование может быть показано как несовместимое с основами квантовой механики [WZ] .

­­­­Формально квантовая механика представляет внутреннее состояние квантовой системы (например, поляризацию фотона единичной длины df) в виде вектора в линейном пространстве H над полем комплексных чисел (пространство Гильберта). Внутреннее произведение двух векторов j+j, определяется как комплексное сопряжение.Размерность гильбертова пространства зависит от системы, будучи большей (или даже бесконечной) для более сложных систем. Каждое физическое измерение H, которое может быть выполнено над системой, соответствует разрешению ее гильбертова пространства на ортогональные подпространства, по одному для каждого возможного результата измерения.Таким образом, число возможных исходов ограничено размерностью d гильбертова пространства, причем наиболее полными измерениями являются те, которые разрешают делить гильбертово пространство на d одномерных подпространств.

­­Гильбертово пространство для одного поляризованного фотона является двумерным; таким образом, состояние фотона может быть полностью описано как линейная комбинация, например, двух единичных векторов r11,0) и r2 = (0,1), представляющих соответственно горизонталь и вертикаль. В частности, фотон, поляризованный под углом a к горизонтали, описывается (cosa, sina).При измерении вертикальной и горизонтальной поляризации такой фотон фактически выбирает стать горизонтальным с вероятностью cos 2 a и вертикальным с вероятностью sin 2 a. Таким образом, два ортогональных вектора r1 и r2 служат примером разрешения двумерного гильбертова пространства на два ортогональных одномерных подпространства; далее будет сказано, что r1 и r2 составляют "прямолинейный" базис гильбертова пространства.

­­Альтернативным базисом для того же гильбертова пространства являются два "диагональных" базисных вектора d1 (0. 707 ,0.'707), представляющих 45-градусный фотон, и d2 (0.707,-0.707), представляющих 135-градусный фотон. Два базиса (например, прямолинейный и диагональный) считаются "сопряженными" (WI, если каждый вектор одного базиса имеет проекции равной длины на все векторы другого базиса: это означает, что система, подготовленная в определенном состоянии одного базиса, будет вести себя совершенно случайным образом и потеряет всю свою сохраненную информацию, если подвергнется измерению, соответствующему другому базису. Из-за сложной природы своих коэффициентов двумерное гильбертово пространство допускает также третий базис, сопряженный с прямолинейным и диагональным базисами, состоящий из двух так называемых "круговых" поляризаций c1 (0.707,0.707i) и c2 (0.707i,0.70"1); но прямолинейный и диагональный базисы - это все, что потребуется для криптографических приложений в данной работе.

­­Гильбертово пространство для составной системы строится путем тензорного произведения гильбертовых пространств ее компонентов; таким образом, состояние пары фотонов характеризуется единичным вектором в 4-мерном гильбертовом пространстве, охватываемом ортогональными базисными векторами r1r 1, r 1r2 , r2r1 , и r 2 r 2 . Этот формализм подразумевает, что состояние составной системы в общем случае не может быть выражено как картезианское произведение состояний ее частей: например, состояние двух фотонов по Эйнштейну-Подольскому-Розену, 0.7071(r1r 2 -r2 r1), которое будет обсуждаться позже, не эквивалентно произведению однофотонных состояний.

\subsubsection{Квантовое распределение открытых ключей}

­­­­­­­­­­В традиционной криптографии с открытым ключом функции-ловушки используются для сокрытия смысла сообщений между двумя пользователями от пассивного подслушивающего устройства, несмотря на отсутствие какой-либо первоначальной общей секретной информации между двумя пользователями.В квантовом распределении открытых ключей квантовый канал не используется непосредственно для передачи значимых сообщений, а скорее используется для передачи случайных битов между двумя пользователями, которые изначально не делятся секретной информацией, таким образом, что пользователи, при последующем общении по обычному неквантовому каналу, подверженному пассивному подслушиванию, могут с высокой вероятностью определить, была ли первоначальная квантовая передача нарушена в пути, как это сделал бы подслушивающий (это свойство квантового канала -заставлять подслушивающего быть активным).Если передача не была нарушена, они соглашаются использовать эти общие секретные биты хорошо известным способом в качестве одноразового блокнота для сокрытия смысла последующих значимых сообщений или для других криптографических приложений (например, аутентификационных меток), требующих общей секретной случайной информации.Если передача была нарушена, они отбрасывают ее и пробуют снова, откладывая любые значимые сообщения до тех пор, пока им не удастся передать достаточно случайных битов по квантовому каналу, чтобы они могли служить в качестве одноразового блокнота.

­­­­­­­Более подробно: один пользователь ("Алиса") выбирает случайную битовую строку и случайную последовательность базисов поляризации(прямолинейную или диагональную).Затем она посылает другому пользователю (Бобу) цепочку фотонов, каждый из которых представляет один бит строки в базисе выбранном для данной позиции бита: горизонтальный или 45-градусный фотон означает двоичный ноль, а вертикальный или 135-градусный - двоичную 1.По мере получения фотонов Боб решает, случайным образом для каждого фотона и независимо от Алисы, измерить ли прямолинейную поляризацию фотона или его диагональную поляризацию, и интерпретирует результат измерения как двоичный ноль или единицу.Как объяснялось в предыдущем разделе, при попытке измерить прямолинейную поляризацию диагонального фотона получается случайный ответ и вся информация теряется. Таким образом, Боб получает значимые данные только от половины обнаруженных им фотонов - тех для которых он угадал правильный базис поляризации. Информация Боба еще более ухудшается из-за того, что, в реальности, некоторые фотоны будут потеряны при прохождении или не будут подсчитаны несовершенными эффективными детекторами Боба.

­­­­­Последующие шаги протокола происходят по обычному публичному каналу связи как предполагается, восприимчив к подслушиванию, но не к вбросу или изменению сообщений.Сначала Боб и Алиса путем публичного обмена сообщениями определяют какие фотоны были успешно получены и какие из них были получены с правильным основанием.Если квантовая передача не была нарушена, Алиса и Боб должны прийти к согласию относительно битов, закодированных этими фотонами, даже если эти данные никогда не обсуждались по публичному каналу.Каждый из этих фотонов, другими словами, предположительно несет один бит случайной информации (например, является ли прямолинейный фотон вертикальным или горизонтальным), известной Алисе и Бобу, но никому другому.

­­­Из-за случайного смешения прямолинейных и диагональных фотонов в квантовой передаче, любое подслушивание несет риск изменения передачи таким образом, чтобы вызвать разногласия между Бобом и Алисой по некоторым битам, по которым, как они думают, они должны договориться. В частности, можно показать, что никакое измерение фотона в пути подслушивающим лицом, которое информировано об исходном базисе фотона только после проведения измерения, не может дать более 1/2 ожидаемых бит информации о ключевом бите, закодированном этим фотоном; и что любое такое измерение, дающее b бит ожидаемой информации (b s 1/2), должно вызвать разногласия с вероятностью не менее b/2, если измеренный фотон или попытка его подделки будут позже повторно измерены в его исходном базисе.(Этот оптимальный компромисс имеет место, например, когда подслушивающее устройство измеряет и повторно передает все перехваченные фотоны в прямолинейном базисе, тем самым узнавая правильную поляризацию половины фотонов и вызывая разногласия в 1/4 случаях, которые позже повторно измерены в исходном базисе).

­­­Поэтому Алиса и Боб могут проверить себя на подслушивание, публично сравнивая некоторые биты, по которым, по их мнению, они должны договориться, хотя, конечно, это жертвует секретностью этих битов.Позиции битов, используемые в этом сравнении, должны быть случайным подмножеством (скажем, одной третью) правильно принятых битов, так что подслушивание более чем нескольких фотонов вряд ли ускользнет от обнаружения.Если все сравнения совпадают, Алиса и Боб могут сделать вывод, что квантовая передача была свободна от значительного подслушивания; и те из оставшихся битов, которые были отправлены и получены с той же основой, также совпадают, и могут быть безопасно использованы в качестве одноразового блокнота для последующих безопасных коммуникаций по общедоступному каналу. Когда этот одноразовый блокнот израсходован, протокол повторяется для передачи нового массива случайной информации по квантовому каналу.

­­­­­­Необходимость в том, чтобы открытый (неквантовый) канал в этой схеме был защищен от активного подслушивания, может быть ослаблена, если Алиса и Боб заранее договорились о небольшом секретном ключе, который они используют для создания аутентификационных меток Вегмана-Картера WCI для своих сообщений по открытому каналу. Более подробно схема аутентификации Векмана-Картера для нескольких сообщений использует небольшой случайный ключ для создания зависящей от сообщения "метки" (скорее как контрольная сумма) для произвольного большого сообщения таким образом, что подслушивающее лицо, не знающее ключа, имеет лишь небольшую вероятность того, что оно сможет создать любые другие правильные пары "сообщение-метка".Таким образом, метка служит доказательством того, что сообщение легитимно и не было сгенерировано или изменено кем-то, не знающим ключа. (Биты ключа постепенно расходуются в схеме Вегмана-Картера и не могут быть использованы повторно без ущерба для доказанной безопасности системы; однако в данном случае эти биты ключа могут быть заменены свежими случайными битами, успешно переданными по квантовому каналу). Подслушивающий все еще может предотвратить коммуникацию, подавляя сообщения в открытом канале или чрезмерно возмущая фотоны, передаваемые по квантовому каналу. Однако в любом случае Алиса и Боб с большой вероятностью придут к выводу, что их секретные сообщения подавляются, и не будут обмануты, думая, что их сообщения безопасны, хотя на самом деле это не так.

\subsubsection{Квантовое бросание монет}

"Квантовое бросание ­­­­монет по телефону" было впервые рассмотрено Блюмом [Bl].Проблема заключается в том, чтобы две недоверчивые стороны, общающиеся на расстоянии без помощи третьей стороны, пришли к соглашению о победителе и проигравшем таким образом, чтобы у каждой стороны было ровно 50 процентов шансов на победу.Любая попытка одной из сторон предвзято оценить результат должна быть распознана другой стороной как жульничество. Предыдущие протоколы для этой проблемы основаны на недоказанных предположениях теории сложности вычислений, что делает их уязвимыми для прорыва в разработке алгоритмов.

­­1. Алиса выбирает случайным образом один базис (скажем, прямолинейный) и последовательность случайных битов (одной тысячи должно быть достаточно). Затем она кодирует свои биты в виде последовательности фотонов в этом же базисе, используя ту же схему кодирования, что и раньше.Она посылает Бобу полученную последовательность поляризованных фотонов.

2. Боб выбирает, независимо и случайно для каждого фотона, последовательность баз считывания. Он считывает фотоны соответствующим образом, записывая результаты в две таблицы, одну из прямолинейно принятых фотонов и одну из диагонально принятых фотонов. Из-за потерь в его детекторах и в канале передачи некоторые фотоны могут быть не приняты, что приведет к появлению дыр в его таблицах. В это время Боб делает предположение о том, какой базис использовала Алиса, и объявляет его Алисе. Он выигрывает, если угадал правильно, и проигрывает в противном случае.

­­3. Алиса сообщает Бобу, выиграл ли он, сообщая ему, какой базис она использовала на самом деле.Она подтверждает эту информацию, посылая Бобу по классическому каналу всю свою исходную последовательность битов, использованную на шаге 1.

4. Боб проверяет, не произошло ли обмана, сравнивая последовательность Алисы с обеими своими таблицами. Должно быть полное совпадение с таблицей, соответствующей базису Алисы, и отсутствие корреляции с другой таблицей. В нашем примере Боб может быть уверен, что первоначальный базис Алисы действительно был выпрямлен, как и было заявлено.

­Алиса может попытаться обмануть либо на шаге 1, либо на шаге 2. Сначала предположим, что она честно выполнила шаг 1 и проиграла в конце шага 2, потому что Боб сделал правильную догадку. Чтобы сделать вид, что она выиграла, ей нужно убедить Боба, что ее фотоны были диагонально поляризованы, что она может сделать, только произведя последовательность битов в полном согласии с диагональной таблицей Боба. Этого она не может сделать надежно, потому что эта таблица является результатом вероятностного поведения фотонов после того, как они покинули ее руки. Предположим, что она все равно идет вперед и посылает Бобу новую "оригинальную" последовательность, отличную от той, которую она использовала на шаге 1, в надежде, что она по счастливой случайности будет идеально согласована с диагональной таблицей Боба. Эта попытка обмана требует от Алисы не только везения, но и смелости, потому что в подавляющем большинстве случаев игра проваливается и обнаруживается как обман. В отличие от этого, в традиционных схемах бросания монет, аналогичные попытки вырвать удачу из пасти поражения, хотя и маловероятны, но не сопровождаются опасностью обнаружения.

­­­Легко видеть, что дела обстоят еще хуже для Алисы, если она попытается обмануть на шаге 1, посылая смесь прямолинейных и диагональных фотонов, или фотонов, которые поляризованы не прямолинейно и не диагонально.В этом случае она не сможет согласиться ни с одной из таблиц Боба на шаге 3, поскольку обе таблицы будут записывать результаты вероятностного поведения, не контролируемого ею.

Чтобы ­­сказать, как Алиса может обмануть квантовую механику, необходимо описать эффект Эйнштейна-Подольского-Розена (ЭПР) [Bo,AGR], который часто называют парадоксом, поскольку он противоречит здравому смыслу, согласно которому для того, чтобы два индивидуально случайных события, происходящих на расстоянии друг от друга, коррелировали между, какое-то физическое влияние должно было распространиться от более раннего события к более позднему, или же от какой-то общей случайной причины к обоим событиям.

­­­­Эффект ЭПР возникает при распаде некоторых типов атомов или молекул с испусканием двух фотонов и заключается в том, что оба фотона всегда имеют противоположную поляризацию, независимо от основы, используемой для их наблюдения, при условии, что оба наблюдаются в одной и той же основе.Например, если оба фотона измерены прямолинейно, то всегда будет обнаружено, что один из них горизонтален, а другой вертикален, хотя горизонтальность будет меняться случайным образом от одного распада к другому. Если оба фотона измерены по диагонали, то один всегда будет 135-градусным, а другой 45-градусным. Минутное размышление покажет, что такое поведение нельзя объяснить, если предположить, что распад производит распределение по a противоположно поляризованных (a и a+90 1фотонов, поскольку в этом случае, если бы такая пара фотонов была измерена в промежуточном базисе (скажем, a+45), оба вели бы себя вероятностно так, что иногда выходили бы с одинаковой поляризацией.

­­­­Возможно, самое простое, но парадоксально звучащее словесное объяснение эффекта ЭПР состоит в том, что два фотона производятся в начальном состоянии неопределенной поляризации; и когда один из них измеряется, измерительная аппаратура заставляет его выбрать поляризацию (выбирая случайным образом и равновероятно между двумя характерными направлениями предлагаемыми аппаратурой), одновременно заставляя другой неизмеренный фотон, независимо от расстояния, выбрать противоположную поляризацию. Это неправдоподобно звучащее объяснение поддерживается формальной квантовой механикой, которая представляет состояние пары фотонов как вектор в 4-мерном гильбертовом пространстве, полученном путем тензорного произведения двух 2-мерных гильбертовых пространств.ЭПР-состояние, возникающее при распаде, описывается вектором 0.7071(r 1r2 - r2r1), а эффект ЭПР объясняется тем, что этот вектор имеет антикоррелированные проекции в двумерные гильбертовы пространства двух фотонов независимо от того, какой базис используется для выражения тензорного произведения.

­­­Чтобы обмануть, Алиса производит несколько пар фотонов ЭПР вместо отдельных случайных фотонов на шаге 1.В каждом случае она посылает Бобу один член пары, а другой хранит сама, возможно, между идеально отражающими зеркалами.

\subsection{\review}

At a conference in 1984 in [2] the first quantum key distribution protocol was presented.

In their work, the authors used two communication channels: a classical public channel and a quantum channel, which transmitted photons of light polarized at angles of 0, 45, 90 and 135 degrees. When trying to listen to quantum channel, an intruder inevitably introduced noise into the communication channel, as it is impossible to reliably distinguish photons in non-orthogonal states.

Thus, the authors of the paper showed how with the help of properties of quantum objects:
\begin{itemize}
	\item determine the presence of an intruder in the channel;
	\item generate a random number between interlocutors, which can later be used to create an encryption key or as a one-time cipher-key.
\end{itemize}



\subsection{\dic}
