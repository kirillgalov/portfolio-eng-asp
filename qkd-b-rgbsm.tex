\subsection{Article}

\subsubsection*{Abstract}

We propose a four-qubit quantum key distribution protocol via two Bell states which constitute a transmitted unit from the sender to the receiver in each communication.
An encryption here is designed by randomly grouping four qubits of a unit into two new couples, which is a way to increase the possibility of detecting the eavesdropper. Ultimately,the receiver randomly measures this grouped unit with two Bell state measurements. From the comparison of grouping information of these four qubits, we find that the two sides in a valid communication can discover the illegal party in the channel. In the proposed protocol, the receiver measures the unit when he receives it instantly, which is an efficient way to overcome the ultrashort storage time of quantum state.
Index Terms— Quantum communication, quantum key distribution, Bell state measurement.

\subsubsection{Introduction}
Quantum key distribution (QKD) is a promising technology to protect the security of classical information in quantum epoch [1]. It enables two parties to share a secret key with unconditional security, which can then be used to encrypt and decrypt messages. Many works have been conducted to promote the development of QKD protocols. In 1984, it was Bennett and Brassard who first introduced the QKD protocol by using two mutually unbiased bases of photon’s polarization degree of freedom [2]. Later Ekert proposed another QKD protocol which was called E91 based on Einstein-Podolsky Rosen (EPR) pairs [3]. After that, various QKD protocol sare theoretically proposed and experimentally realized, such as QKD protocols designed with single-photon [4], multiple states [5] and Bell state [8]. Among these works, photons are extensively used to carry information because they are easy to manipulate and they transmit at light speed.
As the quantum channel, Bell state was firstly proposed by [6] and verified to be the maximally entangled state of a two-qubit quantum system. Compared with other multi-qubit states, such as W state, GHZ state and cluster state, Bell state is easier to prepare via nonlinear process [7]. In reference [8],two parties share the secret key by comparing the form of initial Bell state and the outcome of entanglement swapping.
Then, [9] improved the total efficiency of the communication to 100\% compared with the former 50\% in [8]. Reference [10]presented the first authenticated semi-quantum key distribution protocol without using authenticated classical channels based on Bell states. In this letter, we propose a protocol to prevent the eavesdropper with lower qubit error rate and shorter detecting key bits based on a four-qubit state which consists of two couples of Bell states.
In our protocol, a group of four-qubit state are prepared each time and sent from the sender to the receiver. The receiver performs quantum state measurement immediately after he receives the qubits. Compared with the two-way protocols where the quantum state needs to be preserved until the tranmission is finished in [8] and [9], our protocol can overcome the ultrashort coherence time of quantum states.
Every four qubits form a unit to transfer information from the sender to the receiver in each communication. The four qubits are sent in random order by the sender and received by the receiver in a randomly grouping measurement. Our calculation shows that the qubit error rate is 4.17\%, which is lower than 46.875\% in [11]. Furthermore, only 11 bits are needed to detect the eavesdropper in our QKD protocol which is smaller than 72 bits in BB84 protocol [2] with the same security.
\subsubsection{QKD protocol based on four-qubit states}

Quantum Channel With Bell State References [8] and [9] proposed two QKD protocols, both used Bell states distributed from the sender to the receiver. In their protocols, two pairs of Bell states are shared between two legal parties of communication. The sender and the receiver both keep two qubits entangled with each other. After the simultaneous Bell state measurement (BSM) of the two parties, there exists an entanglement swapping among these four qubits.
To be more specific, let’s denote the four qubits of two Bell states as P 1 , P 2 , P 3 and P 4 . Entanglement exists between P 1and P 2 , P 3 and P 4 . After BSM of the two sides, P 1 and P 3 ,P 2 and P 4 become entangled, respectively. However, the two sides still need to keep their qubits during communication, and it is challenging to store qubits in the state of the art.
Consider the ultrashort storage time of qubits, a novel QKD protocol is proposed with a four-qubit state consisting of two couples of Bell states expressed as equation (1). A group of four-qubit state is prepared each time to send to Bob for measurement immediately. What is different from [8] and [9]is, the sender sends all the qubits to the receiver and the reciever performs quantum state measurement immediately when he receives the qubits. This is a one way process. The two parties don’t need to store the qubits thus the qubits are measured before they decoherence.

From (1) we can see that the state becomes superposition of four states which means we can obtain four different outcomes by combinations of BSMs. Note that equation (1)represents only one special case, other two forms are shownfor comparison in Table II with different groupings of qubits.
In brief, there exists three forms of random grouping of these four qubits, of which only equation (1) is defined as the right one. This is the primary technique to encrypt the information during communication. The proposed protocol is illustrated in Fig. 1, where Alice and Bob are the legitimate sender and receiver, respectively.
B. Preliminaries In this section, we show the details of the protocol in Fig. 1.
a) Step 1 state preparation: Alice prepares one group of the four-qubit state in equation (1). Each group consists of four qubits P y , y $\in$ {1, 2, 3, 4}. With such dense coding, the key information G of each group of the two Bell states is shown in Table I. Alice needs to record the information of qubits and the corresponding information G.
b) Step 2 qubit distribution: According to equation (1),Alice knows that the group order of these four qubits is{(P 1 , P 3 ), (P 2 , P 4 )}. Then, Alice rearranges these four qubits randomly and sends them to Bob via quantum channel.
c) Step 3 grouping measurement: Bob receives the qubits and divides them into two parts randomly. Then he performs BSMs on the two parts then sends the grouping information and the measurement results to Alice via classical channel.
d) Step 4 results comparison: Alice compares the results from Bob with her reserved information of P y . If Alice gets a coincident comparison, she announces ‘1’, and then the communication can proceed to step 5 or returns to step 1.
Otherwise, she announces ‘0’ and the communication returns to step 1 or ends. The different groupings of the four qubits by Bob are shown in Table II.
e) Step 5 raw key acquirement: Alice and Bob obtain a long binary sequence as the raw key R after multiple communications. If we define R A and R B as the raw keys belonging to Alice and Bob, respectively. Alice randomly selects parts of R A in different positions as her agreement key C A and announces the corresponding positions. Then, Bob chooses the agreed key C B in the same location in R B .
f) Step 6 privacy amplification: Bob chooses a set of C B bits as the parity bits D B and announces D B along with its corresponding positions. Alice selects her D A in the same way and compares it with D B . If the bit error rate is smaller than the threshold, the communication is secure and they can proceed to step 7; if not, they need to return to step 1 or terminate this communication.
g) Step 7 final key acquirement: The final keys R A and R Bare used to encrypt the secret message through the communication. Theoretically, R A= R B, R A is the raw key R A without C A under ideal condition, the same as R BC. Analyses of the Qubit Error Rate With Different Groupings We can see that there are three different kinds of groupings from Table II. One is right, another two are wrong. Therefore,the probability of correctly grouping four qubits by Bob is 13 .
If there is no eavesdropper in the communication channel,the qubit error rate $\epsilon$ 0 , which is the threshold during one communication, can be calculated as following.
(1) Bob divides the four qubits into the wrong grouping{(P 1 , P 2 ), (P 3 , P 4 )}.
(2)|C 1234 =In this case, he will only get |$\Phi$ -  12 |$\Phi$ +  34 . Bob will get the right binary random sequence G = 01 with the probability of 1.

By BSM, he can obtain four correct measurement results {|$\Phi$ + 1, |$\Phi$ - 1}, {|$\Phi$ - 1, |$\Phi$ + 1}, {|$\Psi$ + 1, |$\Psi$ - 1} and {|$\Psi$ - 1, |$\Psi$ + 1}. As a result, Bob will get the right binary random sequence G with the probability of 44 = 1 even he makes a wrong grouping.
In summary, the results of the BSM can only be partially right in the first case or totally right in the second case.
Therefore, Bob can acquire the right binary random sequence with the probability of 12 x 14 + 12 x 44 = 58 when grouping wrongly. Hence, the qubit error rate $\epsilon$ 0 is, 11 5= 0.0417. (3)$\epsilon$ 0 = 1 - ( + ) = 3 824
Note that if Bob chooses the right grouping with probability 1, he can obtain the right binary random sequence with the probability of 1. Ultimately, the threshold in our protocol can be set to be 4.17\%.

\subsubsection{Security Analyses}
In this section, we analyze the security of our protocol under two major attacks: the intercept-resend attack and the Trojan Horse attack. Meanwhile, we assume the existence ofan eavesdropper Eve in the communication channel.
A. The Intercept-Resend Attack The eavesdropper Eve can interact and resend a new four qubit state to Bob so that he can acquire the information of the state. In this case, Eve plays the same role as Bob does in the communication. He can group the four-qubit state and measure it with Bell bases.
After step 2 of our protocol, the four qubits sent by Alice will transmit through the communication channel. Let’s assume that Eve intercepts these four qubits and processes them in the same way as Bob does. Then, Eve resends his decoy four qubits to Bob. Let us discuss the different cases of the communication between Eve and Bob.
(1) Eve chooses the right grouping and acquires the right measurement results. Bob simultaneously chooses the right grouping. Now, there are three parties, Alice, Bob and Eve in the communication. There is a probability of p 1 that Eve can successfully filch the information, p 1 = ( x 1)(4)x ( x 1).

Bob(2) Similarly, Eve can divide the qubits and obtain the measurement results correctly. Then, he sends the right decoy four-qubit state to Bob. After receiving them,Bob chooses a wrong grouping but yields a right measurement result. However, this group of state will be abandoned since in the comparison stage Alice and Bob can detect the wrong grouping.
Eve chooses a wrong grouping but he gets a right measurement result while Bob chooses a right grouping.
Bob(4) In the same way, Eve transmits the decoy state with a wrong grouping but Bob gets the right results. Meanwhile, Bob divides this decoy four-qubit state into wrong groups but obtain the right measurement result. In this case, the qubits will be abandoned because of the wrong grouping of Bob.
In conclusion, there is a probability that Bob gets a wrong measurement result, i.e., the qubit error rate with existence of eavesdropper in our protocol can be calculated as (8).
Suppose Alice and Bob need to compare n bits of binary random sequence to detect Eve with the probability ofp d = 0.999999999 with
92 n). We can find a minimum value of n = 11 from equation (9).
On the other hand, in BB84 protocol, Alice and Bob need to compare n = 72 bits to detect Eve with the same probability. From the aforementioned analyses, we can calculate the mutual information between Alice, Bob and Eve. The mutual information between Alice and Bob is p d = 1 - (1 - $\epsilon$ e ) = 1 - (I(A; B) = 1 - [--(1 - $\epsilon$ 0 ) log(1 - $\epsilon$ 0 ) - $\epsilon$ 0 log $\epsilon$ 0 ]
The mutual information between Alice and Eve is I(A; E) = 1 - [-(1 - $\epsilon$ e ) log(1 - $\epsilon$ e ) - $\epsilon$ e log $\epsilon$ e ] = 0.3664.
Therefore, the communication is secure since I(A; B) >I(A; E). Furthermore, I(A; B) in our protocol is greater than I (A; B) = 0.1887 bit in BB84 protocol.
The theoretical secret key rate is R = I(A; B) - I(A; E) = 0.7501 - 0.3664 = 0.3837 > 0.(11)So, the binary random bits can be distilled after Alice and Bob perform key reconciliation and privacy amplification the final key R A and R B. The transmission distance of the secret key is given by the qubit error rate $\epsilon$ [12], which is:$\epsilon$ =u10 -$\alpha$/10$\eta$, u10 -$\alpha$/10 $\eta$ + 2P e(12)where u = 4 is the averaged qubit flux leaving from Alice, 10 -$\alpha$·/10 represents the fiber attenuation through distance, $\eta$ is the detection efficiency. Here, we set the channel loss to be $\alpha$ = 0.2 and $\alpha$ = 0.5. Referring to [12], P e = 8.5 x 10 -7is the probability of an error count per clock cycle. We set the values of $\eta$ to be 0.2, 0.5 and 0.99. Figure 2 shows the relation between (km) and the qubit error rate $\epsilon$. We can see that decreases with the increase of the channel loss. Meanwhile,the higher the detection efficiency of the communication $\eta$ is,the larger the qubit error rate $\epsilon$ is. From Fig. 2, we can infer that there exists a further distance when $\epsilon$ e = 0.0417 which is the qubit error rate in our protocol.

B. Trojan Horse Attack. A QKD system may be probed by Eve by sending a bright light into the quantum channel and analyzing the back reflection in a Trojan-horse attack. Eve occupies part of the quantum channel to probe the laser sent by Alice with an auxiliary source. Note that his detection scheme relies on the feature of the auxiliary source. Eve needs to remove part of the legitimate signal, compensating the introduced loss by an improved quantum channel. Hence, Eve needs to prepare a better channel which has less attenuation than the legitimate channel. Then, he can measure the intercepted state with a quantum memory.
With the Trojan Horse attack, the measurement that maximizes Eve’s information gain is known [13],(13)I E (|$\nu$| 2 ) = 1 - H(p), where p = 12 (1 + 1 - |$\nu$, 0|0, $\nu$| 2 ) = 1+ 2 2|$\nu$| and H(·) is the binary entropy. |$\nu$| 2 is the mean photon number of Eve. Hence, (13) can be rewritten as.

From [13], if Alice’s monitoring detector sets a limit to Eve’s backscattered signal of 0.1 photon, then 0.095 < I E (|$\nu$| 2 ) < 0.135. The lower bound is I E (|$\nu$| 2 ) = 1 - exp(-|$\nu$| 2 ).

In our protocol, the maximum mean photon number of Eve can be 0.4, so, 0.329 M I E (|$\nu$| 2 ) < 0.448. We can see that I E(max) (|$\nu$| 2 ) < I(A; B). Hence, the Trojan Horse attack can be prevented in our protocol.

\subsubsection{Conclusion}
We have analyzed a four-qubit QKD protocol theoretically. Our protocol can provide the encryption without inserting decoy qubits in the qubit sequence to detect the eavesdropper. Reference [14] designed a QKD protocol based on decoy-state,which is more applicable in practical, with longer transmission distance in the state of the art. This is follow-up work since we analyse our protocol in practical implementation. With the increase of the randomness of grouping of four qubits in each unit by Bob, it will be more difficult for Eve to decode the classical information. Our results show that the security can be guaranteed and the storage of the qubits during the communication is not required. Furthermore, our protocol can be extended to distributing multiple qubits.

\subsection{\trnas}
\subsubsection*{Аннотация}

Мы предлагаем протокол квантового распределения ключей на четыре кубита через два состояния Белла, которые составляют передаваемую единицу от отправителя к получателю в каждой коммуникации.
Шифрование здесь разрабатывается путем случайной группировки четырех кубитов единицы в две новые пары, что является способом увеличения возможности обнаружения подслушивающего устройства. В конечном итоге, приемник случайным образом измеряет эту сгруппированную единицу с помощью двух измерений состояния Белла. Из сравнения информации о группировке этих четырех кубитов следует, что обе стороны в действительной коммуникации могут обнаружить нелегальную сторону в канале. В предложенном протоколе приемник измеряет кубит, когда получает его мгновенно, что является эффективным способом преодоления сверхкороткого времени хранения квантового состояния.
Индексные термины - квантовая связь, квантовое распределение ключей, измерение состояния Белла.

\subsubsection{Введение}
Квантовое распределение ключей (QKD) является перспективной технологией для защиты безопасности классической информации в квантовую эпоху [1]. Она позволяет двум сторонам обмениваться секретным ключом с безусловной безопасностью, который затем может быть использован для шифрования и дешифрования сообщений. Было проведено много работ, направленных на развитие протоколов QKD. В 1984 году Беннетт и Брассард впервые представили протокол QKD, используя два взаимно несмещенных основания степени свободы поляризации фотона [2]. Позже Экерт предложил другой протокол QKD, который был назван E91 на основе пар Эйнштейн-Подольский-Розен (ЭПР) [3]. После этого были теоретически предложены и экспериментально реализованы различные протоколы QKD, такие как протоколы QKD, разработанные с использованием одного фотона [4], нескольких состояний [5] и состояния Белла [8]. Среди этих работ фотоны широко используются для переноса информации, поскольку ими легко манипулировать и они передают информацию со скоростью света.
В качестве квантового канала состояние Белла было впервые предложено в [6] и подтверждено как максимально запутанное состояние двухквантовой квантовой системы. По сравнению с другими многоквантовыми состояниями, такими как W-состояние, GHZ-состояние и кластерное состояние, состояние Белла легче подготовить с помощью нелинейного процесса [7]. В работе [8] две стороны делятся секретным ключом, сравнивая форму начального состояния Белла и результат обмена запутанностью.
Затем в [9] общая эффективность коммуникации была повышена до 100\% по сравнению с прежними 50\% в [8]. В [10]представлен первый аутентифицированный полуквантовый протокол распределения ключей без использования аутентифицированных классических каналов на основе состояний Белла. В этом письме мы предлагаем протокол для защиты от подслушивающего устройства с более низкой частотой ошибок на кубитах и более коротким обнаружением ключевых битов на основе четырехквантового состояния, которое состоит из двух пар состояний Белла.
В нашем протоколе каждый раз подготавливается группа из четырех кубитов и отправляется от отправителя к получателю. Приемник выполняет измерение квантового состояния сразу после получения кубитов. По сравнению с двусторонними протоколами, в которых квантовое состояние должно быть сохранено до завершения передачи в [8] и [9], наш протокол может преодолеть сверхкороткое время когерентности квантовых состояний.
Каждые четыре кубита образуют единое целое для передачи информации от отправителя к получателю в каждом сообщении. Четыре кубита отправляются в случайном порядке отправителем и принимаются получателем в случайном порядке группировки. Наши расчеты показывают, что коэффициент ошибок в кубитах составляет 4,17\%, что ниже, чем 46,875\% в [11]. Более того, для обнаружения подслушивающего устройства в нашем QKD протоколе требуется всего 11 бит, что меньше, чем 72 бита в протоколе BB84 [2] при той же безопасности.
\subsubsection{Протокол QKD, основанный на четырехкубитном состоянии}

Квантовый канал с состояниями Белла [8] и [9] предложили два протокола QKD, оба использовали состояния Белла, распределенные от отправителя к получателю. В их протоколах две пары состояний Белла разделяются между двумя законными сторонами коммуникации. Отправитель и получатель хранят два кубита, запутанные друг с другом. После одновременного измерения состояния Белла (BSM) двух сторон, существует обмен запутанностью между этими четырьмя кубитами.
Более конкретно, обозначим четыре кубита двух состояний Белла как P 1 , P 2 , P 3 и P 4 . Между P 1 и P 2 , P 3 и P 4 существует запутанность. После BSM двух сторон, P 1 и P 3 , P 2 и P 4 становятся запутанными, соответственно. Однако обеим сторонам все еще необходимо хранить свои кубиты во время коммуникации, а хранение кубитов на современном уровне техники является сложной задачей.
Учитывая сверхкороткое время хранения кубитов, предлагается новый протокол QKD с четырехкбитными состояниями, состоящими из двух пар состояний Белла, выраженных уравнением (1). Каждый раз готовится группа из четырех кубитов, чтобы немедленно отправить Бобу для измерения. Отличие от [8] и [9]заключается в том, что отправитель посылает все кубиты получателю, а получатель выполняет измерение квантового состояния сразу после получения кубитов. Это односторонний процесс. Обеим сторонам не нужно хранить кубиты, таким образом, кубиты измеряются до того, как они декогерентны.

Из (1) видно, что состояние становится суперпозицией четырех состояний, что означает, что мы можем получить четыре различных исхода путем комбинаций BSMs. Заметим, что уравнение (1) представляет только один частный случай, две другие формы показаны для сравнения в таблице II с различными группировками кубитов.
Вкратце, существует три формы случайной группировки этих четырех кубитов, из которых только уравнение (1) определяется как правильное. Это основная техника для шифрования информации во время коммуникации. Предлагаемый протокол показан на рис. 1, где Алиса и Боб являются законными отправителем и получателем, соответственно.
B. Предварительные сведения В этом разделе мы покажем детали протокола на рис. 1.
a) Шаг 1 подготовки состояния: Алиса готовит одну группу из четырех кубитов в уравнении (1). Каждая группа состоит из четырех кубитов P y , y $\in$ {1, 2, 3, 4}. При таком плотном кодировании ключевая информация G каждой группы двух состояний Белла показана в таблице I. Алисе необходимо записать информацию о кубитах и соответствующую информацию G.
b) Распределение кубитов на этапе 2: Согласно уравнению (1), Алиса знает, что групповой порядок этих четырех кубитов таков{(P 1 , P 3 ), (P 2 , P 4 )}. Затем Алиса переставляет эти четыре кубита случайным образом и отправляет их Бобу по квантовому каналу.
c) Шаг 3 измерения группировки: Боб получает кубиты и делит их на две части случайным образом. Затем он выполняет BSM на этих двух частях и отправляет информацию о группировке и результаты измерений Алисе по классическому каналу.
d) Сравнение результатов на этапе 4: Алиса сравнивает результаты, полученные от Боба, со своей зарезервированной информацией о P y . Если Алиса получает совпадающее сравнение, она объявляет '1', после чего коммуникация может перейти к шагу 5 или вернуться к шагу 1.
В противном случае она объявляет '0', и коммуникация возвращается к шагу 1 или завершается. Различные группировки четырех кубитов Бобом показаны в таблице II.
e) Шаг 5 - получение необработанного ключа: Алиса и Боб получают длинную двоичную последовательность в качестве необработанного ключа R после многократного обмена данными. Если мы определим R A и R B как необработанные ключи, принадлежащие Алисе и Бобу, соответственно. Алиса случайным образом выбирает части R A в различных позициях в качестве своего ключа согласия C A и объявляет соответствующие позиции. Затем Боб выбирает согласованный ключ C B в том же месте в R B .
f) Шаг 6 усиление конфиденциальности: Боб выбирает набор битов C B в качестве битов четности D B и объявляет D B вместе с соответствующими позициями. Алиса выбирает свой D A таким же образом и сравнивает его с D B . Если коэффициент битовых ошибок меньше порогового значения, связь безопасна и они могут перейти к шагу 7; если нет, им нужно вернуться к шагу 1 или прервать эту связь.
g) Шаг 7. Получение окончательного ключа: Окончательные ключи R A и R B используются для шифрования секретного сообщения посредством связи. Теоретически, R A= R B, R A - это необработанный ключ R A без C A при идеальных условиях, такой же, как и R BC. Анализ коэффициента ошибок Кубита при различных группировках Из таблицы II видно, что существует три различных вида группировок. Одна из них правильная, две другие - неправильные. Таким образом, вероятность того, что Боб правильно сгруппирует четыре кубита, равна 13.
Если в канале связи нет подслушивающего устройства, то коэффициент ошибок на кубитах $\epsilon$ 0, который является пороговым для одной связи, может быть рассчитан следующим образом.
(1) Боб делит четыре кубита на неправильные группы{(P 1 , P 2 ), (P 3 , P 4 )}.
(2)|C 1234 =В этом случае он получит только |$\Phi$ - 12 |$\Phi$ + 34 . Боб получит правильную двоичную случайную последовательность G = 01 с вероятностью 1.

С помощью BSM он может получить четыре правильных результата измерений {|$\Phi$ + 1, |$\Phi$ - 1}, {|$\Phi$ - 1, |$\Phi$ + 1}, {|$\Psi$ + 1, |$\Psi$ - 1} и {|$\Psi$ - 1, |$\Psi$ + 1}. В результате Боб получит правильную двоичную случайную последовательность G с вероятностью 44 = 1, даже если он сделает неправильную группировку.
Таким образом, результаты BSM могут быть только частично верными в первом случае или полностью верными во втором.
Поэтому Боб может получить правильную двоичную случайную последовательность с вероятностью 12 x 14 + 12 x 44 = 58 при неправильной группировке. Следовательно, коэффициент ошибок кубита $\epsilon$ 0 составляет, 11 5 = 0,0417. (3)$\epsilon$ 0 = 1 - ( + ) = 3 824
Обратите внимание, что если Боб выбирает правильную группировку с вероятностью 1, он может получить правильную двоичную случайную последовательность с вероятностью 1. В конечном итоге, порог в нашем протоколе может быть установлен равным 4,17\%.

\subsubsection{Анализ безопасности}
В этом разделе мы анализируем безопасность нашего протокола при двух основных атаках: атаке перехвата-отправки и атаке "Троянский конь". При этом мы предполагаем существование подслушивающего устройства Ева в канале связи.
A. Атака перехвата-передачи Подслушивающая Ева может взаимодействовать и повторно отправить Бобу новое состояние из четырех кубитов, чтобы он мог получить информацию об этом состоянии. В этом случае Ева играет ту же роль, что и Боб в коммуникации. Он может сгруппировать состояние четырех кубитов и измерить его с помощью базиса Белла.
После шага 2 нашего протокола четыре кубита, отправленные Алисой, пройдут по каналу связи. Предположим, что Ева перехватывает эти четыре кубита и обрабатывает их так же, как и Боб. Затем Ева повторно посылает Бобу свои ложные четыре кубита. Давайте обсудим различные случаи коммуникации между Евой и Бобом.
(1) Ева выбирает правильную группировку и получает правильные результаты измерений. Боб одновременно выбирает правильную группировку. Теперь в коммуникации участвуют три стороны, Алиса, Боб и Ева. Существует вероятность p 1 того, что Ева сможет успешно перехватить информацию, p 1 = ( x 1)(4)x ( x 1).

Боб(2) Аналогично, Ева может разделить кубиты и получить правильные результаты измерений. Затем он посылает Бобу правильное состояние четырех кубитов. Получив их, Боб выбирает неправильную группировку, но получает правильный результат измерения. Однако эта группа состояний будет отброшена, так как на этапе сравнения Алиса и Боб могут обнаружить неправильную группировку.
Ева выбирает неправильную группировку, но получает правильный результат измерения, в то время как Боб выбирает правильную группировку.
Боб(4) Таким же образом, Ева передает ложное состояние с неправильной группировкой, но Боб получает правильные результаты. Между тем, Боб делит это ложное состояние из четырех кубитов на неправильные группы, но получает правильный результат измерения. В этом случае кубиты будут отброшены из-за неправильной группировки Боба.
В заключение, существует вероятность того, что Боб получит неверный результат измерения, т.е. коэффициент ошибок кубитов при наличии подслушивающего устройства в нашем протоколе может быть рассчитан как (8).
Предположим, что Алисе и Бобу нужно сравнить n битов двоичной случайной последовательности, чтобы обнаружить Еву с вероятностью p d = 0,999999999 с 92 n). Из уравнения (9) можно найти минимальное значение n = 11.
С другой стороны, в протоколе BB84 Алисе и Бобу нужно сравнить n = 72 бита, чтобы обнаружить Еву с одинаковой вероятностью. Исходя из вышеупомянутого анализа, мы можем рассчитать взаимную информацию между Алисой, Бобом и Евой. Взаимная информация между Алисой и Бобом равна p d = 1 - (1 - $\epsilon$ e ) = 1 - (I(A; B) = 1 - [--(1 - $\epsilon$ 0 ) log(1 - $\epsilon$ 0 ) - $\epsilon$ 0 log $\epsilon$ 0 ].
Взаимная информация между Алисой и Евой равна I(A; E) = 1 - [-(1 - $\epsilon$ e ) log(1 - $\epsilon$ e ) - $\epsilon$ e log $\epsilon$ e ] = 0,3664.
Следовательно, коммуникация безопасна, поскольку I(A; B) >I(A; E). Более того, I(A; B) в нашем протоколе больше, чем I (A; B) = 0,1887 бит в протоколе BB84.
Теоретическая скорость передачи секретного ключа равна R = I(A; B) - I(A; E) = 0.7501 - 0.3664 = 0.3837 > 0.(11)Таким образом, двоичные случайные биты могут быть дистиллированы после того, как Алиса и Боб выполнят согласование ключей и усиление секретности окончательного ключа R A и R B. Расстояние передачи секретного ключа задается коэффициентом ошибок $\epsilon$ [12], который имеет вид:$\epsilon$ =u10 -$\alpha$/10$\eta$, u10 -$\alpha$/10$\eta$ + 2P e(12)где u = 4 - усредненный поток кубитов, уходящих от Алисы, 10 -$\alpha$-/10 - затухание волокна через расстояние, $\eta$ - эффективность обнаружения. Здесь мы задаем потери в канале $\alpha$ = 0.2 и $\alpha$ = 0.5. Согласно [12], P e = 8.5 x 10 -7 - это вероятность ошибки за тактовый цикл. Мы установили значения $\eta$ равными 0.2, 0.5 и 0.99. На рисунке 2 показана зависимость между (km) и коэффициентом ошибок на кубитах $\epsilon$. Видно, что с увеличением потери канала она уменьшается. При этом, чем выше эффективность обнаружения связи $\eta$, тем больше коэффициент ошибок на кубитах $\epsilon$. Из рис. 2 мы можем сделать вывод, что существует дальнейшее расстояние, когда $\epsilon$ e = 0.0417, что является коэффициентом ошибок кубитов в нашем протоколе.

B. Атака "Троянский конь". Система QKD может быть прозондирована Евой путем посылки яркого света в квантовый канал и анализа обратного отражения в атаке "троянский конь". Ева занимает часть квантового канала, чтобы прозондировать лазер, посланный Алисой, с помощью вспомогательного источника. Обратите внимание, что его схема обнаружения полагается на особенность вспомогательного источника. Еве нужно удалить часть легитимного сигнала, компенсируя вносимые потери улучшенным квантовым каналом. Следовательно, Еве нужно подготовить лучший канал, который имеет меньшее затухание, чем легитимный канал. Затем он может измерить перехваченное состояние с помощью квантовой памяти.
При атаке "Троянский конь" известно измерение, которое максимизирует информационный выигрыш Евы [13], (13)I E (|$\nu$| 2 ) = 1 - H(p), где p = 12 (1 + 1 - |$\nu$, 0|0, $\nu$| 2 ) = 1 + 2 2|$\nu$| и H(-) - бинарная энтропия. |$\nu$| 2 - среднее число фотонов Евы. Следовательно, (13) может быть переписано как.

Из [13] следует, что если следящий детектор Алисы устанавливает предел для обратно рассеянного сигнала Евы в 0,1 фотона, то 0,095 < I E (|$\nu$| 2 ) < 0,135. Нижняя граница равна I E (|$\nu$| 2 ) = 1 - exp(-|$\nu$| 2 ).

В нашем протоколе максимальное среднее число фотонов Евы может быть 0,4, поэтому 0,329 M I E (|$\nu$| 2 ) < 0,448. Мы видим, что I E(max) (|$\nu$| 2 ) < I(A; B). Следовательно, атака "Троянский конь" может быть предотвращена в нашем протоколе.

\subsubsection{Вывод}
Мы теоретически проанализировали четырехкбитную QKD-программу. Наш протокол может обеспечить шифрование без вставки ложных кубитов в последовательность кубитов для обнаружения подслушивающего устройства. В работе [14] разработан протокол QKD на основе ложного состояния, который более применим в практических условиях при большом расстоянии передачи. Данная работа является продолжением, поскольку мы анализируем наш протокол в практической реализации. С увеличением случайности группировки четырех кубитов в каждом блоке Бобом, Еве будет сложнее декодировать классическую информацию. Наши результаты показывают, что безопасность может быть гарантирована, а хранение кубитов во время коммуникации не требуется. Более того, наш протокол может быть расширен для распределения нескольких кубитов.


\subsection{\review}
ss
\subsection{\dic}
ss