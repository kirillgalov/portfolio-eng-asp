\subsection{Article}

\subsubsection*{Abstract}

Quantum key distribution (QKD) provides a way for distribution of secure key in at least two parties which they initially share. And there are many protocols for providing a secure key i.e. BB84 protocol,SARG04 protocol, E91 protocol and many more. In this paper all the concerned protocols that share a secret key is explained and comparative study of all protocols shown.
Keywords : Quantum key distribution, BB84 protocol, BB92 protocol, SARG04 protocol, E91 protocol, COW protocol, DPS protocol, KMB09 Protocol, S09 protocol, S13 protocol.

\subsubsection{Introduction}

Quantum key distribution (QKD) [1] [2] provides a way for two parties to expand a secure key that they initially share. The best known QKD is the BB84 protocol published by Bennett and Brassard in 1984 [1]. The security of BB84 was not proved until many years after its introduction. Among the proofs [3] [4] [5] [6],the one by Shor and Preskill [6] is relevant to this paper. Their simple proof essentially converts an entanglement distillation protocol (EDP) based QKD proposed by Lo and Chau [5] to the BB84 Protocol. The EDP-based QKD has already been shown to be secure by [5] and the conversion successively leads to thesecurity of BB84 protocol.

Security proofs of QKD protocols were further extended to explicitly accommodate the imperfection in practical devices [7] [8]. One important imperfection is that the laser sources used in practice and coherent sources that occasionally emit more than one photon in each signal. Thus they are not single –photon sources that the other security proofs [3] [4] [6] of BB84 assumed. In particular, BB84 may become insecure when coherent sources with strong intensity are used. For instance Eve can launch a photon-number-splitting (PNS)attack puts severe limits on the distance and the key generation rate of unconditionally secure QKD. A novel solution to the problem of imperfect devices in BB84 protocol was proposed by Hwang [9]. Which uses extra test states called the decoy states to learn the properties of the channel and/or eavesdropping on the key generating signal states. An unconditional security proof of decoy-state QKD [10] [11] is presented. Another method to combat PNS attack was by Scarani et.al. [12], who introduced a new protocol called SARG04, which is very similar to the BB84 protocol. The quantum state transmission phase and the measurement phase of SARG04 are the same as that of BB84, as both use the same four quantum state and the same experimental measurement. The only difference between the two protocols is the classical post-processing phase, the protocol becomes secure even when Alice emits two photon, a situation under which BB84 is insecure .This protocol was proved by [13] who also proved the security of SARG04 with a single-photon source. They also proposed a modified SARG04 protocol that uses same six states as the original six state protocols [14] [15]. The security of SAG04 with a single-photon source was also proved by Branciard et.al [16]. They considered SARG04 protocol implemented with single-sources and with realistic sources. For the single photon source case, they provided upper and lower bounds of the bit error rate with one-way classical communications. For the realistic source case they considered only incoherent attack by Eve and showed that SARG04 can achieve higher secret key rate and greater source distance than BB84. Another protocol that is similar to SARG04 is the B92 Protocol [17] which uses two non orthogonal quantum states. The security of B92 with a single-photon source was proved by Tamaki et al [18] [19]. On the other hand Koashi [20] proposed an implementation of B92 with strong phase-reference coherent light that was proved secure.

The Focus of this paper is to survey the most prominent quantum key distribution protocols and their security. In this paper we briefly describe the necessary principles of quantum mechanics from which the protocols are divided in to two categories those based on the Heisenberg Uncertainty Principles and others are based on quantum entanglement Rest of the paper is organised as: In section II description of quantum cryptography and there mechanism is explained. Section III depicts all the Quantum key distribution protocols used Heisenberg's uncertainty principles. IV depicts all the Quantum key distribution protocols used quantum entanglement principles and section V other protocols that both prepare and measure and entanglement based is shown and in Section VI observation table of all the protocols with their applications is depicted. Finally Conclusion is shown in Section VII.

\subsubsection{QUANTUM CRYPTOGRAPHY}
Quantum cryptography is a relatively recent arrival in the information security world. It harnesses the laws of quantum Mechanics to create new cryptographic primitives. There is however, one quantum cryptographic primitive which is achievable with today's technology i.e. Quantum key distribution. By using the quantum properties of light, current lasers, fibre-optics and free space transmission technology can be used for QKD, so that many observers claim security can be based on the law of quantum physics only.

Quantum key distribution is a key establishment protocol which creates symmetric key material by using quantum properties of light to transfer information from Client A to Client B in a manner which, through the incontrovertible results of quantum mechanics, will highlight any eavesdropping by an adversary.
2.1 The Heisenberg Uncertainty Principle. According the Heisenberg Uncertainty principle, it is not possible to measure the quantum state of any system without disturbing that system. Thus the polarization of photon or light particle can only be known at the point when it is measured. This principle play a critical role in thwarting the attempts of eavesdroppers in a cryptosystem based on quantum cryptography.
For any two observable properties linked together like mass and momentum Where A = A and B = B And where= AB – BA

According to the principle two interrelated properties cannot be measured individually without affecting the others. The principle is that since you cannot partition the photon in to two halves measuring the state of photon will affect it value. So if someone tries to detect the state of photons being send to the receiver the error can be detected [21]

2.2 Quantum Entanglement. The other important principle on which QKD can be based is the principle of quantum entanglement. It is possible for two particles to become entangled such that when a particular property is measured in one particle, the opposite state will be observed on the entangled particle instantaneously. This is true regardless of the distance between the entangled particles. It is impossible, however to predict prior to measurement what state will be observed thus it is not possible to communicate via entangled particles without discussing the observation over classical channel. The process of communicating using entangled states, aided by a classical information channel is known as quantum teleportation and is the basis of Eckert's protocol [22].

\subsubsection{Qkd Protocols Using Heisenberg’s Uncertainity Prinicples}
Quantum cryptography exploits the quantum mechanical property that a qubit cannot be copied or amplified without disturbing its original state i.e. No-Cloning Theorem [23] [24]. Key distribution using quantum cryptography would be almost impossible to steal because Quantum key distribution (QKD) [25] [26][27] systems continually and randomly generate new private keys that both parties shares automatically. A compromised key in a QKD system is able to decrypt only a small amount of encoded information because of continuously changes in private key. A secret key can be build from a stream of a single photon where each photon is encoded with a bit value of 0 or 1, typically by a photon superposition state such as polarization. These photons are emitted by a conventional laser as pulses of dim light so that most pulses do not emit a photon. This approach ensures that few pulses contain more than one photon travel through the fiber-optic line. In the end only a small fraction of the received pulses actually contains a photon [28]. The photons that are reached to the receiver are used. The key is generally encoded in either the polarization or the relative phase ofthe photon.
3.1 BB84 protocol. Quantum cryptography is based upon conventional cryptographic methods and extends these through the use of quantum effects. Quantum key Distribution (QKD) is used in quantum cryptography for generating a secret key shared between two parties using a quantum channel and an authenticated classical channel as show in figure 6. The private key obtained then used to encrypt message that are sent over an insecure channel (such as a conventional internet connection).

The BB84 protocol described using Photon polarization state to transmit the information. It was originally developed by Charles Bennett and Gilles Brassard in 1984 [1].

Below are the steps of the BB84 protocol for exchange the secret key in the BB84 protocol [29], client A and client B must do as follow:STAGE 1 PROTOCOL Communication over quantum channel
Client A prepare photon randomly with either rectilinear (+) or diagonal polarization (x) therefore Client A transmit photons in the four polarization states (0, 45, 90,135 degree).
Client A records the polarization of each photon and sends it to Client B.
Client B receives a photon and randomly records its polarization according to the rectilinear or diagonal basis. The Client B records the measurement type (basis used) and the resulting polarization measured.
Client B doesn't know which of the measurement are deterministic, i.e. measured in the same basis as the one used by client A. Half the time Client B will be lucky and chose the same quantum alphabet as the third person. In this case, the bit resulting from his measurement will agree with the bit sent by Client A. However the other half time he will be unlucky and choose the alphabet not used by client A. In this case,the bit resulting from his measurement will agree with the bit sent by client A only 50\% of the time. Afterall these measurement, client B now has in hand a binary sequence 
Client A and Client B now proceed to communicate over the public two-way channel using the following stage 2 protocol.

STAGE 2 PROTOCOL: Communication over a public channel
Phase 1. Raw Key extraction

Over the public channel, client B communicates to client A which quantum alphabet he used for each of his measurements.
In response client A communicate to client B over the public channel which of his measurement were made correct alphabet.

Client A and Client B then delete all bits for which they used incompatible quantum alphabet to produce their resulting raw keys. If the third person has not eavesdropped, then their resulting keys will be the same. If the third person has eavesdropped their resulting key will not be in total agreement.
Phase 2. Error estimation. Over the public channel, Client A and client B compare small portion of their raw keys to estimate the error-rate R, and then delete the disclosed bits from their raw keys to produce their tentative final keys. If through their public disclosures Client A and Client B find no errors (i.e., R=0), then they know that the third person was not eavesdropping and that their tentative keys must be the same final key. If they discover at least one error during their public disclosures (i.e., R>0), then they know that the third person has been eavesdropping. In this case,they discard their tentative final keys and start all over again.
3.2 BB92 protocol. Soon after BB84 protocol was published, Charles Bennett realized that it was not necessary to use two orthogonal basis for encoding and decoding. It turns out that a single non-orthogonal basis can be used instead,without affecting the security of the protocol against eavesdropping. This idea is used in the BB92 protocol [30],which is otherwise identical to BB84 protocol.
The key difference in BB92 is that only two states are necessary rather than the possible 4 polarization states in BB84 protocol.

As shown in figure 3, 0 can be encoded as 0 degrees in the rectilinear basis and 1 can be encoded by 45 degrees in the diagonal basis. Like the BB84 protocol, Client A transmit to Client B a string of photons encoded with randomly chosen bits but this time the bits Client A chooses dictates which bases Client B must use. Client B still randomly chooses a basis by which to measure but if he chooses the wrong basis, he will not measure anything; a condition in quantum mechanics which is known as an erasure. Client B can simply tell Client A after each bit Client B sends whether or not he measured it correctly [31].
3.3 SARG04 protocol. The SARG04 protocol is built when researcher noticed that by using the four states of BB84 with different information encoding they could develop a new protocol which would more robust when attenuated laser pulses are used instead of single- photon sources. SARG04 protocol was proposed in 2004 by Scarani et.al[32].

The SARG04 protocol shares the exact same first phase as BB84. In the second Phase when Client A and Client B determine for which bits their bases matched, Client A does not directly announce her bases rather than Client A announces a pair of non-orthogonal states one of which she used to encode her bit. If Client Bused the correct basis, he will measure the correct state. If he chose incorrectly he will not measure either Client A states and will not be able to determine the bit. If there are no errors, then the length of the key remaining after the sifting stage is 1/4 of the raw key. The SARG04 protocol provides almost identical security to BB84 in perfect single-photon implementations: If the quantum channel is of a given visibility (i.e. with losses) then the QBER of SARG04 is twice that of BB84protocol, and is more sensitive to losses.
However SARG04 protocol provides more security than BB84 in the presence of PNS attack, in both the secret key rate and distance the signal can be carried (limiting distance).
3.4 Six-State protocol. (SSP)The 6-state or 3 bases cryptographic is nothing but the well-known BB84 4-state scheme with an additional basis [33]. Six-State Protocol (SSP) is proposed by Pasquinucci and Gisin in 1999 [34].
When represented on the Poincare sphere the BB84 protocol makes use of four spin-1/2 states corresponding to-x and -y direction. In brief summary Client A sends of the four states to Client B, who measures the qubits he receives in either the X or Y basis. A priori this gives a probability 1/2 that Client A and Client B use the same basis. On an average Client A and Client B have to discard half of the qubits even before they can start extracting their cryptographic key.

In the 6 state protocols the two extra states correspond to -z, i.e. the 6 states are -x, -y, and -z on the Poincare sphere. In this case Client A sends a state chosen freely among the 6 and Client B measures either in the x, y or z-basis. Here the prior probability that Client A and Client B use the same basis is reduced to 1/3,which means that they have to discard 2/3 of the transmitted qubits before they can extract a cryptographic key.
However, this scheme does hold an advantage compared to the BB84 protocol – higher symmetry. As it will be seen this fact together with the use of symmetric eavesdropping strategies dramatically reduced the number of free variables in the problem under investigation.
\subsubsection{Qkd Protocols Using Quantum Entanglement}
A new approach to quantum key distribution where the key is distributed using quantum teleportation
4.1 E91 protocol. The Ekert scheme uses entangled pairs of photons [2]. These can be created by created by Client A, by Client B, or by some source separate from both of them, including eavesdropper Eve. The photons are distributed so that Client A and Client B each up with one photon from each pair.
The Scheme relies on two properties of entanglement. First the entangled states are perfectly correlated in the sense that if Client A and Client B both measure whether their particles have vertical or horizontal polarizations,they will always get the same answer with 100\% probability. The same is true if they both measure any other pair of complementary (orthogonal) polarization However the particular results are completely random, it is impossible for Client A to predict if and Client B will get vertical polarization or horizontal polarization.
Second any attempt at eavesdropping by Eve will destroy these correlations in a way that Client A and Client B can detect.
A typical physical set-up is shown in figure 5, using active polarization rotators (PR), polarizing beam-splitters(PBS) and avalanche photodiodes (APD)

The measurement in the figure 5 is divided into two groups; the first is when different orientations of thea nalyser were used and the second when the same analyser orientation was employed. Any photon which was not registered is discarded. Alice and Bob then reveal the result of the first group only, and check that they correspond to the value expected from Bell's inequality. If this is so then Alice and Bob can be sure that the results they obtained in the second group are anti-correlated and can be used to produce a secret key string. Eve cannot obtain any information from the photons when they are transit as there is simple no information there. Information is only present once the authorized user performs their analyser measurements and key sifting. Eve's only hope is to inject her own data for Alice and Bob, but as she doesn't know their analyser orientations,she will always be detected (the Bell's inequality value will be too low).

4.2 COW protocol. Coherent One-Way protocol (COW protocol) is a new protocol for Quantum cryptography elaborated by Nicolas Gisin et al in 2004 [37].

A new protocol for QKD tailored to work with weak coherent pulses at high bit rates [36]. The advantage of this system are that the setup is experimentally simple and it is tolerant to reduced interference visibility and to photon numbers splitting attacks, thus resulting in a high efficiency in terms of distilled secret bits per qubit. The figure 6 presents the COW protocol. The information is encoded in time. Alice sends Coherent pulses that are either empty or have a mean photon number u < 1. Each logical bit of information is encoded by sequences of two pulses, u-0 for a logical “0” or 0-u for a logical “1”.

For security reason, Alice can also send decoy sequences u-u. To obtain the key, Bob measure the time-of-arrival of the photon on his data-line, detector D B . To ensure the security Bob randomly measures the coherence between successive non-empty pulses, bit sequence “1 -0” or decoy sequence, with the interferometer and detectors D M1 and D M2 . If wavelength of the laser and the phase in the interferometer are well aligned, we have all detection on D M1 and no detection on D M2. A loss of coherence and therefore a reduction of the visibility reveal the presence of an eavesdropper, in which case the key is simply discarded, hence no information will be lost.

4.3 DPS protocol. Differential –phase-shift QKD (DPS-QKD) is a new quantum key distribution scheme that was proposed by K.Inoue et al. [38]. Figure 7 shows the setup of the DPS-QKD scheme.

Alice randomly phase-modulates a pulse train of weak coherent states by {0,$\pi$} for each pulse and sends it to Bob with an average photon number of less than one per pulse. Bob measure the Phase difference between two sequential pulses using a 1-bit delay. Mach-Zehnder interferometer and photon detectors, and records the photon arrival time and which detector clicked. After transmission of the optical pulse train, Bob tells Alice the time instances at which a photon was counted. From this time information and her modulation data. Alice knows which detector clicked at Bob's site. Under an agreement that a click by detector 1 denotes “0” and click by detector 2 denotes “1”, for example Alice and Bob obtain an identical bit string.
The DPS-QKD scheme has certain advantageous features including a simple configuration, efficient time domain use, and robustness against photon number splitting attack [38] [39].

\subsubsection{Qther Protocols}

There are many other protocols in existence, both prepare-and-measures and entanglement based. They are as follows:
5.1. KMB09 protocol. KMB09 protocol is an alternative quantum key distribution protocol [40]. Where Alice and Bob use two mutually unbiased bases with one of them encoding a „0' and the other one encoding a „1'. The security of the scheme is due to a minimum index transmission error rate (ITER) and quantum bit error rate (QBER)introduced by an eavesdropper.
The ITER increase significantly for higher dimensional photon states. This allows for more Noise in the transmission line, thereby increasing the possible distance between Alice and Bob Without the need for intermediate nodes

5.2 S09 protocol. S09 protocol is quantum protocol based on public private key cryptography for secure transmission of data over a public channel [41]. The security of the protocol derives from the fact that Alice and Bob each use secret keys in multiple exchange of the qubit. Unlike the BB84 protocol [1] and its many variants. Bob Know the key to transmit, the qubits are transmitted in only one direction and classical information exchanged thereafter, the communication in the proposed protocol remains quantum in each stage. In the BB84 protocol,each transmitted qubit is in one of four different states in this protocol transmitted qubit can be in any arbitrary states
5.3 S13 protocol. S13 protocol is a new quantum protocol [42] that is identical to the BB84 protocol for all the quantum manipulation, but differs from it by using Private Reconciliation from a Random Seed and Asymmetric Cryptography. Thus allowing the generation of larger secure key

\subsubsection{Conclusions}
QKD Protocols are based on principles from quantum physics and information theory. Quantum key distribution is clearly an unconditionally secure means of establishing secret keys. Combined with unconditionally secure authentication, and an unconditionally secure cryptosystem.
The current commercial systems are aimed mainly at governments and corporations with high security requirements. The major difference of quantum key distribution is the ability to detect any interception of the key, whereas with courier the key security cannot be proven or tested. QKD system has the advantage of being automatic, with greater reliability and lower operating costs than a secure human courier network.

\subsection{\trnas}
\subsubsection*{Аннотация}

Квантовое распределение ключей (QKD) обеспечивает способ распределения безопасного ключа, по крайней мере, между двумя сторонами, который они первоначально разделяют. Существует множество протоколов для обеспечения безопасного ключа, например, протокол BB84, протокол SARG04, протокол E91 и многие другие. В этой статье объясняются все протоколы, которые разделяют секретный ключ, и проводится сравнительное исследование всех протоколов.
Ключевые слова : Квантовое распределение ключей, протокол BB84, протокол BB92, протокол SARG04, протокол E91, протокол COW, протокол DPS, протокол KMB09, протокол S09, протокол S13.

\subsubsection{Введение}

Квантовое распределение ключей (QKD) [1] [2] предоставляет двум сторонам возможность расширить защищенный ключ, который они первоначально разделяют. Наиболее известным QKD является протокол BB84, опубликованный Беннетом и Брассардом в 1984 году [1]. Безопасность BB84 была доказана только через много лет после его появления. Среди доказательств [3] [4] [5] [6], доказательство Шора и Прескилла [6] имеет отношение к данной работе. Их простое доказательство, по сути, преобразует QKD на основе протокола дистилляции запутанности (EDP), предложенного Ло и Чау [5], в протокол BB84. В [5] уже было показано, что QKD на основе EDP является безопасным, и преобразование последовательно приводит к безопасности протокола BB84.

Доказательства безопасности протоколов QKD были в дальнейшем расширены для явного учета несовершенства практических устройств [7] [8]. Одним из важных несовершенств является то, что лазерные источники, используемые на практике, являются когерентными источниками, которые иногда испускают более одного фотона в каждом сигнале. Таким образом, они не являются однофотонными источниками, что предполагалось в других доказательствах безопасности [3] [4] [6] BB84. В частности, BB84 может стать небезопасным, если используются когерентные источники с сильной интенсивностью. Например, Ева может запустить атаку с расщеплением фотонного числа (PNS), что накладывает серьезные ограничения на расстояние и скорость генерации ключей безусловно безопасного QKD. Новое решение проблемы несовершенных устройств в протоколе BB84 было предложено Хвангом [9]. В нем используются дополнительные тестовые состояния, называемые состояниями-приманками, для изучения свойств канала и/или подслушивания состояний сигнала, генерирующего ключ. Представлено безусловное доказательство безопасности QKD с ложными состояниями [10] [11]. Другой метод борьбы с PNS-атакой был предложен Scarani et.al. [12], которые представили новый протокол под названием SARG04, очень похожий на протокол BB84. Фаза передачи квантового состояния и фаза измерения в SARG04 такие же, как и в BB84, так как оба используют одинаковые четыре квантовых состояния и одинаковое экспериментальное измерение. Единственное различие между двумя протоколами заключается в классической фазе постобработки, протокол становится безопасным, даже когда Алиса испускает два фотона, ситуация, при которой BB84 небезопасен. Этот протокол был доказан в [13], которые также доказали безопасность SARG04 с однофотонным источником. Они также предложили модифицированный протокол SARG04, который использует те же шесть состояний, что и оригинальный протокол с шестью состояниями [14] [15]. Безопасность SAG04 с однофотонным источником была также доказана Branciard et.al [16]. Они рассмотрели протокол SARG04, реализованный с однофотонным источником и с реалистичными источниками. Для случая с однофотонным источником они предоставили верхнюю и нижнюю границы частоты битовых ошибок при односторонней классической связи. Для случая реалистичного источника они рассмотрели только некогерентную атаку Евы и показали, что SARG04 может достичь более высокой скорости секретного ключа и большего расстояния до источника, чем BB84. Другим протоколом, похожим на SARG04, является протокол B92 [17], который использует два неортогональных квантовых состояния. Безопасность протокола B92 с однофотонным источником была доказана Тамаки и другими [18] [19]. С другой стороны, Коаши [20] предложил реализацию B92 с когерентным светом с сильной фазовой референцией, безопасность которого была доказана.

Целью данной статьи является обзор наиболее известных протоколов квантового распределения ключей и их безопасности. В этой статье мы кратко описываем необходимые принципы квантовой механики, на основе которых протоколы делятся на две категории: основанные на принципах неопределенности Гейзенберга и основанные на квантовой запутанности: В разделе II дается описание квантовой криптографии и ее механизма. В разделе III представлены все протоколы квантового распределения ключей, использующие принципы неопределенности Гейзенберга. В разделе IV показаны все протоколы распределения квантовых ключей, использующие принципы квантовой запутанности, а в разделе V показаны другие протоколы, основанные на принципах подготовки и измерения и запутанности, а в разделе VI приведена таблица наблюдений всех протоколов с их приложениями. Наконец, в разделе VII приведено заключение.

\subsubsection{Квантовая криптография}
Квантовая криптография относительно недавно появилась в мире информационной безопасности. Она использует законы квантовой механики для создания новых криптографических примитивов. Однако существует один квантовый криптографический примитив, который достижим с помощью сегодняшней технологии - квантовое распределение ключей. Используя квантовые свойства света, современные лазеры, оптоволокно и технологии передачи данных в свободном пространстве могут быть использованы для QKD, так что многие наблюдатели утверждают, что безопасность может быть основана только на законах квантовой физики.

Квантовое распределение ключей - это протокол установления ключей, который создает симметричный ключевой материал, используя квантовые свойства света для передачи информации от клиента A к клиенту B таким образом, который, благодаря неопровержимым результатам квантовой механики, выявляет любое подслушивание противником.
2.1 Принцип неопределенности Гейзенберга. Согласно принципу неопределенности Гейзенберга, невозможно измерить квантовое состояние любой системы, не нарушив эту систему. Таким образом, поляризация фотона или частицы света может быть известна только в тот момент, когда она измеряется. Этот принцип играет важную роль в пресечении попыток подслушивания в криптосистеме, основанной на квантовой криптографии.
Для любых двух наблюдаемых свойств, связанных между собой, таких как масса и импульс, где A = A и B = B и где = AB - BA

Согласно этому принципу, два взаимосвязанных свойства не могут быть измерены по отдельности без влияния на другие. Принцип заключается в том, что поскольку вы не можете разделить фотон на две половины, измерение состояния фотона повлияет на его значение. Поэтому если кто-то попытается определить состояние фотонов, передаваемых приемнику, ошибка может быть обнаружена [21].

2.2 Квантовая запутанность. Другим важным принципом, на котором может быть основана QKD, является принцип квантовой запутанности. Две частицы могут стать запутанными таким образом, что когда измеряется определенное свойство одной частицы, противоположное состояние будет наблюдаться на запутанной частице мгновенно. Это происходит независимо от расстояния между запутанными частицами. Однако невозможно предсказать до измерения, какое состояние будет наблюдаться, поэтому невозможно общаться через запутанные частицы, не обсуждая наблюдение по классическому каналу. Процесс коммуникации с использованием запутанных состояний при помощи классического информационного канала известен как квантовая телепортация и лежит в основе протокола Эккерта [22].

\subsubsection{Qkd протоколы, использующие принципы неопределенности Гейзенберга}
Квантовая криптография использует квантово-механическое свойство, согласно которому кубиты не могут быть скопированы или усилены без нарушения их исходного состояния, т.е. теорему об отсутствии клонирования [23] [24]. Распределение ключей с помощью квантовой криптографии практически невозможно украсть, поскольку системы квантового распределения ключей (QKD) [25] [26] [27] постоянно и случайно генерируют новые закрытые ключи, которые автоматически передаются обеим сторонам. Взломанный ключ в системе QKD способен расшифровать лишь небольшое количество закодированной информации из-за постоянного изменения закрытого ключа. Секретный ключ может быть создан из потока одиночных фотонов, где каждый фотон закодирован битовым значением 0 или 1, обычно с помощью состояния суперпозиции фотонов, например, поляризации. Эти фотоны испускаются обычным лазером в виде импульсов тусклого света, так что большинство импульсов не испускают фотонов. Такой подход гарантирует, что лишь немногие импульсы, содержащие более одного фотона, проходят через волоконно-оптическую линию. В итоге только небольшая часть полученных импульсов действительно содержит фотон [28]. Используются те фотоны, которые дошли до приемника. Ключ обычно кодируется либо в поляризации, либо в относительной фазе фотона.
3.1 Протокол BB84. Квантовая криптография основана на обычных криптографических методах и расширяет их за счет использования квантовых эффектов. Квантовое распределение ключей (QKD) используется в квантовой криптографии для генерации секретного ключа, разделяемого между двумя сторонами с использованием квантового канала и аутентифицированного классического канала, как показано на рисунке 6. Полученный секретный ключ затем используется для шифрования сообщений, передаваемых по незащищенному каналу (например, через обычное интернет-соединение).

Протокол BB84 описывает использование состояния поляризации фотонов для передачи информации. Первоначально он был разработан Чарльзом Беннетом и Жилем Брассаром в 1984 году [1].

Ниже приведены шаги протокола BB84 для обмена секретным ключом в протоколе BB84 [29], клиент A и клиент B должны сделать следующее:

1 ПРОТОКОЛ Общение по квантовому каналу
Клиент A готовит фотоны случайным образом с прямолинейной (+) или диагональной поляризацией (x), поэтому клиент A передает фотоны в четырех состояниях поляризации (0, 45, 90, 135 градусов).
Клиент A регистрирует поляризацию каждого фотона и отправляет ее клиенту B.
Клиент B получает фотон и случайным образом регистрирует его поляризацию в соответствии с прямолинейным или диагональным базисом. Клиент B записывает тип измерения (используемый базис) и результирующую измеренную поляризацию.
Клиент B не знает, какие из измерений являются детерминированными, т.е. измеренными в том же базисе, что и тот, который использовал клиент A. В половине случаев клиенту B повезет, и он выберет тот же квантовый алфавит, что и третий человек. В этом случае бит, полученный в результате его измерения, будет совпадать с битом, посланным клиентом А. Однако в другой половине случаев ему не повезет, и он выберет алфавит, не используемый клиентом А. В этом случае бит, полученный в результате его измерения, будет совпадать с битом, посланным клиентом А, только в 50\% случаев. После всех этих измерений, клиент B теперь имеет в руках двоичную последовательность 
Теперь клиент A и клиент B переходят к общению по публичному двустороннему каналу, используя следующий протокол стадии 2.

ПРОТОКОЛ ЭТАПА 2: Общение по общедоступному каналу
Этап 1. Добыча сырого ключа

По публичному каналу клиент B сообщает клиенту A, какой квантовый алфавит он использовал для каждого из своих измерений.
В ответ клиент A сообщает клиенту B по общедоступному каналу, какие из его измерений были сделаны правильным алфавитом.

Затем клиент A и клиент B удаляют все биты, для которых они использовали несовместимый квантовый алфавит, чтобы получить свои результирующие необработанные ключи. Если третий человек не подслушивал, то их результирующие ключи будут одинаковыми. Если третий человек подслушивал, то их результирующие ключи не будут полностью совпадать.
Фаза 2. Оценка ошибок. По публичному каналу клиент А и клиент Б сравнивают небольшие части своих необработанных ключей, чтобы оценить коэффициент ошибок R, а затем удаляют раскрытые биты из своих необработанных ключей, чтобы получить предварительные окончательные ключи. Если в результате публичного раскрытия клиент А и клиент Б не обнаруживают ошибок (т.е. R=0), то они знают, что третий человек не подслушивал и что их предварительные ключи должны быть одним и тем же окончательным ключом. Если они обнаруживают хотя бы одну ошибку во время публичного раскрытия (т.е. R>0), то они знают, что третий человек подслушивал. В этом случае они отбрасывают свои предварительные окончательные ключи и начинают все сначала.
3.2 Протокол BB92. Вскоре после публикации протокола BB84 Чарльз Беннетт понял, что нет необходимости использовать два ортогональных базиса для кодирования и декодирования. Оказалось, что вместо них можно использовать один неортогональный базис без ущерба для безопасности протокола от подслушивания. Эта идея используется в протоколе BB92 [30], который в остальном идентичен протоколу BB84.
Ключевым отличием BB92 является то, что необходимо только два состояния, а не возможные 4 состояния поляризации в протоколе BB84.

Как показано на рисунке 3, 0 может быть закодирован как 0 градусов в прямолинейном базисе, а 1 может быть закодирована как 45 градусов в диагональном базисе. Как и в протоколе BB84, клиент A передает клиенту B строку фотонов, закодированных случайно выбранными битами, но на этот раз биты, выбранные клиентом A, диктуют клиенту B, какие базисы он должен использовать. Клиент В по-прежнему произвольно выбирает базис для измерения, но если он выберет неправильный базис, он ничего не измерит; это условие в квантовой механике известно как стирание. Клиент B может просто сообщить клиенту A после каждого переданного клиентом B бита, правильно ли он его измерил [31].
3.3 Протокол SARG04. Протокол SARG04 был создан, когда исследователи заметили, что, используя четыре состояния BB84 с различным кодированием информации, они могут разработать новый протокол, который будет более устойчивым, когда вместо однофотонных источников используются ослабленные лазерные импульсы. Протокол SARG04 был предложен в 2004 году Скарани и другими[32].

Протокол SARG04 имеет точно такую же первую фазу, как и BB84. Во второй фазе, когда Клиент А и Клиент Б определяют, для каких битов их базисы совпали, Клиент А не объявляет напрямую свои базисы, вместо этого Клиент А объявляет пару неортогональных состояний, одно из которых он использовал для кодирования своего бита. Если клиент B использовал правильное основание, он измерит правильное состояние. Если он выбрал неверно, он не измерит ни одно из состояний клиента А и не сможет определить бит. Если ошибок нет, то длина ключа, оставшегося после этапа просеивания, равна 1/4 длины необработанного ключа. Протокол SARG04 обеспечивает почти такую же безопасность, как и BB84 в совершенных однофотонных реализациях: Если квантовый канал имеет заданную видимость (т.е. с потерями), то QBER протокола SARG04 в два раза больше, чем у протокола BB84, и более чувствителен к потерям.
Однако протокол SARG04 обеспечивает большую безопасность, чем BB84 при наличии атаки PNS, как по скорости передачи секретного ключа, так и по расстоянию, на которое может быть перенесен сигнал (предельное расстояние).
3.4 Шестисоставной протокол. (SSP)6-статусный или 3-базовый криптографический протокол - это не что иное, как хорошо известная 4-статусная схема BB84 с дополнительным базисом [33]. Шестисоставной протокол (SSP) был предложен Паскинуччи и Гизином в 1999 году [34].
При представлении на сфере Пуанкаре протокол BB84 использует четыре состояния спин-1/2, соответствующие направлениям x и -y. Вкратце, клиент A посылает четыре состояния клиенту B, который измеряет полученные им кубиты в базисе X или Y. Априори это дает вероятность 1/2, что Клиент А и Клиент Б используют один и тот же базис. В среднем Клиент А и Клиент Б должны отбросить половину кубитов еще до того, как они смогут начать добывать свой криптографический ключ.

В протоколах с 6 состояниями два дополнительных состояния соответствуют -z, т.е. 6 состояний - это -x, -y и -z на сфере Пуанкаре. В этом случае клиент A посылает состояние, выбранное произвольно из 6, а клиент B измеряет либо в x, y, либо в z-базисе. Здесь предварительная вероятность того, что клиент A и клиент B используют один и тот же базис, уменьшается до 1/3, что означает, что они должны отбросить 2/3 переданных кубитов, прежде чем смогут извлечь криптографический ключ.
Однако у этой схемы есть преимущество по сравнению с протоколом BB84 - более высокая симметрия. Как будет показано далее, этот факт вместе с использованием симметричных стратегий подслушивания резко сократил количество свободных переменных в исследуемой задаче.
\subsubsection{Qkd протоколы, использующие квантовую запутанность}
Новый подход к квантовому распределению ключей, где ключ распределяется с помощью квантовой телепортации
4.1 Протокол E91. Схема Ekert использует запутанные пары фотонов [2]. Они могут быть созданы клиентом A, клиентом B или каким-либо источником, отдельным от них обоих, включая подслушивающую Еву. Фотоны распределяются таким образом, что клиент А и клиент Б получают по одному фотону из каждой пары.
Схема опирается на два свойства запутанности. Во-первых, запутанные состояния идеально коррелированы в том смысле, что если клиент А и клиент Б измерят, имеют ли их частицы вертикальную или горизонтальную поляризацию, они всегда получат один и тот же ответ с вероятностью 100\%. То же самое верно, если они оба измерят любую другую пару комплементарных (ортогональных) поляризаций. Однако конкретные результаты совершенно случайны, клиент A не может предсказать, получит ли клиент B вертикальную или горизонтальную поляризацию.
Во-вторых, любая попытка подслушивания со стороны Евы разрушит эти корреляции таким образом, что Клиент А и Клиент Б смогут их обнаружить.
Типичная физическая установка показана на рисунке 5, в ней используются активные поляризационные вращатели (PR), поляризационные сплиттеры луча (PBS) и лавинные фотодиоды (APD).

Измерения на рисунке 5 разделены на две группы; первая - когда использовались различные ориентации анализатора, вторая - когда использовалась одна и та же ориентация анализатора. Любой фотон, который не был зарегистрирован, отбрасывается. Затем Алиса и Боб раскрывают результаты только первой группы и проверяют, соответствуют ли они значению, ожидаемому из неравенства Белла. Если это так, то Алиса и Боб могут быть уверены, что результаты, полученные ими во второй группе, антикоррелированы и могут быть использованы для создания строки секретного ключа. Ева не может получить никакой информации от фотонов при их транзите, поскольку там просто нет никакой информации. Информация появляется только после того, как авторизованный пользователь выполнит измерения анализатора и отбор ключей. Единственная надежда Евы - ввести свои собственные данные для Алисы и Боба, но поскольку она не знает ориентации их анализаторов, она всегда будет обнаружена (значение неравенства Белла будет слишком мало).

4.2 Протокол COW. Когерентный односторонний протокол (протокол COW) - это новый протокол для квантовой криптографии, разработанный Николя Гизином и др. в 2004 году [37].

Новый протокол для QKD, приспособленный для работы со слабыми когерентными импульсами при высокой скорости передачи данных [36]. Преимущество этой системы в том, что установка экспериментально проста, и она устойчива к снижению видимости помех и атакам с расщеплением числа фотонов, что приводит к высокой эффективности с точки зрения количества дистиллированных секретных битов на один кубит. На рисунке 6 представлен протокол COW. Информация кодируется во времени. Алиса посылает когерентные импульсы, которые либо пусты, либо имеют среднее число фотонов u < 1. Каждый логический бит информации кодируется последовательностями из двух импульсов, u-0 для логического "0" или 0-u для логической "1".

В целях безопасности Алиса также может посылать ложные последовательности u-u. Чтобы получить ключ, Боб измеряет время прибытия фотона на своей линии передачи данных, детектор D B . Для обеспечения безопасности Боб случайным образом измеряет когерентность между последовательными непустыми импульсами, битовой последовательностью "1 -0" или ложной последовательностью, с помощью интерферометра и детекторов D M1 и D M2. Если длина волны лазера и фаза в интерферометре хорошо согласованы, мы имеем полное обнаружение на D M1 и отсутствие обнаружения на D M2. Потеря когерентности и, следовательно, уменьшение видимости свидетельствуют о наличии подслушивающего устройства, в этом случае ключ просто отбрасывается, следовательно, информация не теряется.

4.3 Протокол DPS - это новая схема квантового распределения ключей, предложенная K.Inoue и др [38]. На рисунке 7 показана схема DPS-QKD.

Алиса случайным образом фазово-модулирует последовательность импульсов слабых когерентных состояний на {0,$\pi$} для каждого импульса и посылает их Бобу со средним числом фотонов менее одного на импульс. Боб измеряет разность фаз между двумя последовательными импульсами с задержкой в 1 бит. Маха-Цендера интерферометр и детекторы фотонов, и записывает время прихода фотонов и какой детектор сработал. После передачи последовательности оптических импульсов Боб сообщает Алисе моменты времени, в которые был засчитан фотон. Из этой информации о времени и данных о модуляции. Алиса знает, какой детектор щелкнул на участке Боба. По соглашению о том, что щелчок детектора 1 обозначает "0", а щелчок детектора 2 обозначает "1", например, Алиса и Боб получают идентичную битовую строку.
Схема DPS-QKD обладает некоторыми преимуществами, включая простую конфигурацию, эффективное использование временной области и устойчивость к атакам с расщеплением числа фотонов [38] [39].

\subsubsection{Qther Protocols}

Существует множество других протоколов, как подготовительно-измерительных, так и основанных на запутывании. Они следующие:
5.1. Протокол KMB09. Протокол KMB09 - это альтернативный протокол квантового распределения ключей [40]. В нем Алиса и Боб используют две взаимно несмещенные базы, одна из которых кодирует "0", а другая - "1". Безопасность схемы обусловлена минимальным коэффициентом ошибок передачи индекса (ITER) и коэффициентом ошибок квантового бита (QBER), вносимых подслушивающим устройством.
ITER значительно увеличивается для фотонных состояний более высокой размерности. Это позволяет увеличить количество шумов в линии передачи, тем самым увеличивая возможное расстояние между Алисой и Бобом без необходимости промежуточных узлов

5.2 Протокол S09. Протокол S09 - это квантовый протокол, основанный на криптографии с открытым закрытым ключом для безопасной передачи данных по публичному каналу [41]. Безопасность протокола вытекает из того факта, что Алиса и Боб используют секретные ключи при многократном обмене кубитами. В отличие от протокола BB84 [1] и его многочисленных вариантов. Боб знает ключ для передачи, кубиты передаются только в одном направлении, и после этого происходит обмен классической информацией, коммуникация в предлагаемом протоколе остается квантовой на каждом этапе. В протоколе BB84 каждый передаваемый кубита находится в одном из четырех различных состояний, в данном протоколе передаваемый кубита может находиться в любом произвольном состоянии
5.3 Протокол S13. Протокол S13 - это новый квантовый протокол [42], который идентичен протоколу BB84 по всем квантовым манипуляциям, но отличается от него использованием приватной свертки от случайного семени и асимметричной криптографии. Это позволяет генерировать большие безопасные ключи

\subsubsection{Выводы}
Протоколы QKD основаны на принципах квантовой физики и теории информации. Квантовое распределение ключей является безусловно безопасным средством создания секретных ключей. В сочетании с безусловно безопасной аутентификацией и безусловно безопасной криптосистемой.
Современные коммерческие системы предназначены в основном для правительств и корпораций с высокими требованиями к безопасности. Основным отличием квантового распределения ключей является возможность обнаружить любой перехват ключа, в то время как при курьерской передаче безопасность ключа не может быть доказана или проверена. Преимущество QKD-системы в том, что она автоматизирована, обладает большей надежностью и меньшими эксплуатационными расходами, чем защищенная сеть курьеров-людей.
\subsection{\review}
This [7] article is a review. The authors describe the main physical phenomena underlying the protocols and divide them into two categories:
\begin{itemize}
	\item based on the Heisenberg uncertainty principle;
	\item based on the phenomenon of quantum entanglement.
\end{itemize}

The authors then consider protocols from the earliest (BB84, E91) to the newest (S13). This work shows the interest of the scientific environment in the study of quantum key distribution protocols.

The result of the work is a comparative table of all protocols, in which the authors highlight the main features and differences.


\subsection{\dic}
\begin{multicols}{2}
	\begin{itemize}
		
		\item algorithms - алгоритм
		\item analysis - анализ
		
		\item appropriate - подходящий
		\item approximately - примерно
		
		
		\item basis - основа
		\item beam - луч
		
		\item binary - двоичных
		\item bit - бит
		
		\item capacity - вместимость
		
		\item channel - канал
		
		\item coherent - связный
		\item combination - комбинации
		
		\item communication - связь
		\item compare - сравнить
		
		\item computation - вычисления
		\item computers - компьютеров
		
		\item condition - условие
		\item conjugate - спряжение
		\item considered - рассмотрено
		\item contain - содержат
		
		\item correlation - корреляция
		
		\item cryptography - криптография
		
		\item decode - декодировать
		\item decoy - ловушка
		\item density - плотность
		
		\item dependence - зависимость
		\item detect - обнаружить
		
		\item deterministic - детерминированный
		
		\item developing - разработка
		
		\item difference - разница
		
		\item differentiate - дифференцировать
		
		\item distribution - распределение
		
		\item eavesdropper - подслушиватель
		
		\item encoded - закодировано
		\item entaglement - запутанность
		\item equivalently - эквивалентно
		
		\item imperfections - недостаток
		\item implementation - реализация
		
		\item instances - экземпляров
		
		\item intensity - интенсивность
		\item intercept - перехват
		
		\item interferometer - интерферометр
		
		\item limitation - ограничение
		
		\item lossless - без потерь
		\item lossy - потери
		
		\item malicious - вредоносных
		\item managed - управляемых
		
		\item mathematical - математических
		\item matrix - матрица
		
		\item measurement - измерения
		
		\item method - метод
		\item mimic - имитировать
		
		\item modification - модификация
		
		\item neglected - пренебрегают
		
		\item normalized - нормализовано
		
		\item operator - оператор
		\item optical - оптический
		\item optimal - оптимальный
		
		\item optional - опционально
		
		\item orthogonal - ортогональный
		\item orthogonality - ортогональность
		\item orthonormal - ортонормированный
		
		\item phase - фаза
		\item photon - фотон
		
		\item polarization - поляризация
		
		
		\item probabilistic - вероятностный
		\item probability - вероятность
		\item problem - проблема
		
		\item projecting - проектирование
		
		\item property - свойство
		\item proportion - пропорция
		\item proposal - предложение
		
		\item protocol - протокол
		
		\item prove - доказательство
		
		\item provide - обеспечить
		\item provides - обеспечивает
		
		\item public - общедоступных
		\item pulse - импульс
		
		\item quantum - квантовый
		\item random - случайных
		\item randomize - рандомизировать
		
		\item signature - подпись
		
		\item strategy - стратегия
		
		\item string - строка
		
		\item symmetric - симметричный
		
		\item theoretical - теоретический
		\item theory - теория
		
		\item transmission - передача
		
		\item vacuum - вакуум
		\item value - значение
		
		\item variable - переменная
		
		
		
	\end{itemize}
\end{multicols}