%\documentclass[a4pape, 14pt]{extarticle} % расширенный класс статьи с возможностью указать 14 шрифт
\documentclass[a4pape, 11pt]{extarticle}

\usepackage{graphicx}  % поддержка .eps-графики
\usepackage[utf8]{inputenc} % кодировка в которой набран текст 
\usepackage[T2C]{fontenc} % поддержка кирилицы % ещё можно T1 T2B T2A T2
\usepackage[english,russian]{babel} % переносы и типографские правила для русского
\usepackage{indentfirst}  % красная строка
\usepackage{listings}  % оформление листингов программ
\usepackage{amssymb,amsfonts,amsmath,mathtext} % ядро для научной статьи
\usepackage{physics}
\usepackage{mathtools}
\usepackage{braket}
\usepackage{float} % для рисунков
\usepackage{array} % для m{2cm} в таблицах
\usepackage{blindtext} % случайный текст для заглушки

\usepackage{pdfpages} % для вставки пдф файлов

\usepackage{pgfplots} % построение графиков
% We will externalize the figures
%\usepgfplotslibrary{external}
%\tikzexternalize
%\pgfplotsset{width=15cm,compat=1.5}

\setcounter{tocdepth}{2} % Для того, чтобы были видны сабсекции в содержании

\usepackage[left=2cm, right=2cm, top=1.5cm, bottom=1.5cm]{geometry} % установка полей

\newcommand{\hup}{Принцип неопределенности Гейзенберга}
\newcommand{\qe}{Квантовая запутанность}

%\usepackage{citehack}  % https://www.opennet.ru/docs/RUS/cyr_howto/ch08s02.html
%\usepackage[bibencoding=auto,backend=biber,babel=other]{biblatex}
%\usepackage{csquotes} 


% https://ru.overleaf.com/learn/latex/Pgfplots_package  - для построения графиков
% https://ru.overleaf.com/learn/latex/Tables - построение таблиц
% https://ru.overleaf.com/learn/latex/Commands - обьявление коман
\graphicspath{{./images/}} %путь к рисункам
\frenchspacing % длина пробелов после пунктуации
\pagestyle{plain}
\selectlanguage{russian}
\bibliographystyle{gost780} % gost780 - сортировка по порядку следования, gost780s - по алфавиту кажется. gost780u не завелся