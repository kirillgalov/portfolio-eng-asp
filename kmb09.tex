\subsection{Article}

\subsubsection*{Abstract}

In the original BB84 protocol by Bennett and Brassard, an eavesdropper is detected because his attempts to intercept information result in a quantum bit error rate (QBER) of at least 25\%. Here we design an alternative quantum key distribution protocol, where Alice and Bob use two mutually unbiased bases with one of them encoding a ‘0’ and the other one encoding a ‘1’. The security of the scheme is due to a minimum index transmission error rate (ITER) introduced by an eavesdropper that increases significantly for higher dimensional photon states. This allows for more noise in the transmission line,thereby increasing the possible distance between Alice and Bob without the need for intermediate nodes.

\subsubsection{Introduction}

The aim of quantum cryptography is to establish a shared, secret and random sequence of bits between a sender, called Alice and a receiver, called Bob [1]. This sequence constitutes a perfect cryptographic key, a so-called one-time pad, and allows Alice and Bob to securely encrypt a message of the same length. The cryptographic key is obtained solely via the transmission of photons and classical communication. Each bit is encoded in the state of a single photon and read out by Bob upon arrival via a quantum measurement. Random switching between different bases makes it impossible for an eavesdropper, called Evan, to predict the states used in the protocol. All his attempts to intercept photons result in a significant quantum bit error rate(QBER). This guarantees a high level of security, since Evan’s presence is detected easily when Alice and Bob compare a number of test bits.

Under ideal conditions, the exchange of single photons allows Alice and Bob to establish a cryptographic key over an arbitrarily long distance. In practice, cryptographic setups consist of imperfect single-photon sources, lossy transmission lines, and photon detectors with dark count rates. Alice and Bob must hence apply classical information processing tools like error correction and privacy amplification [2, 3] to their data in order to obtain identical secret keys. However, cryptographic protocols are only secure as long as it is possible to detect the presence of an eavesdropper. System errors could cloud Evan’s presence; especially since he could simply replace parts of the equipment with high-quality components. This makes it impossible to tolerate large system errors and limits the possible distance between Alice and Bob.

Recently, Rosenberg et al [4, 5] reported the creation of a secure cryptographic key over a distance of 144.3 km of optical fiber. Their scheme is based on a decoy state protocol [6]–[8]which is immune to photon number splitting attacks and highly resistant to Trojan horse attacks [9]. It is expected that improvements in filtering of blackbody photons might allow for an extension of the fiber to 250 km. In the mean time, Takesue et al [10] and Stuckiet al [11] created a secure cryptographic key over a distance of 200 km of optical fiber. These experimental setups are believed to be the longest terrestrial quantum key distribution fiber-links yet demonstrated. Comparable distances have been achieved in free space. For example, Schmitt-Manderbach et al [12] securely distributed a cryptographic key over a
144 km free-space link.

In this paper, we design a novel quantum key distribution protocol whose efficiency and minimum error rate in the case of eavesdropping increase with the dimension of the photon states used by Alice and Bob. In this way, we increase the threshold for tolerable system errors without sacrificing the security of the protocol and hence increase the possible distance between Alice and Bob. In principle, single photons could be purified with the help of quantum repeaters [13]. Proposals for their implementation (see e.g. [14, 15]) and other noise reducing links [16]–[18] have been made but their experimental implementation and their practical integration into cryptographic networks remains to be seen.

The above-mentioned long-distance quantum key distribution schemes [4, 5, 10, 12] are all based on the BB84 protocol by Bennett and Brassard [19]. In BB84, Alice encodes her bits in two-dimensional photon states. These can be obtained using polarization encoding. However, a more natural choice is time-bin encoding, which affords better protection of the photons against decoherence [20]. Alice and Bob independently vary their bases between two possibilities. A key bit is obtained whenever both use the same basis. This means, on average, every second photon contributes a bit to the cryptographic key. Using a simple intercept–resend strategy, an eavesdropper introduces a QBER of at least 25\% into the communication.

In the following, we assume that Alice and Bob use time-bin or path encoded N -dimensional photon states. As in BB84, Alice and Bob randomly vary their bases between two mutually unbiased bases [21]. However, contrary to BB84, Alice and Bob detect the presence of an eavesdropper by calculating the index transmission error rate (ITER). As we shall see below, for N = 2, Evan causes a minimum ITER of 25\%. In the case of four dimensional photon states, this error rate becomes 37.5\%. When increasing N further, the minimum ITER approaches 50\%. The efficiency of the protocol in units of transmitted bits per photon is the same as the minimum ITER in case of eavesdropping. The states required by the proposed key distribution scheme can be realized using a symmetric Bell multiport beam splitter [22]. Before the transmission, the path encoding of the output states of the Bell multiport beam splitter should be switched to the above-mentioned time-bin encoding [20].

Several generalizations of quantum cryptographic schemes to higher dimensions have already been proposed. The papers [23]–[29] are generalizations of the original BB84protocol [19] based on the encoding of information in higher-dimensional alphabets. However,recently it has been shown [31]–[33] that the security of the BB84 protocol is entirely compromised if Alice and Bob share, for example, four-dimensional photon states in this way [34]. Beige et al [35, 36] propose two alternative quantum cryptographic protocols using four-dimensional photon states, which, under ideal conditions, allow Alice and Bob to communicate directly but whose minimum error rates in the case of eavesdropping are relatively low.

Here we show that there are other possible cryptographic schemes in higher dimensions. A sin [35, 36], Alice and Bob use two bases e and f with all states of e encoding a ‘0’ and all states of f encoding a ‘1,’ even when N is larger than two. This means, contrary to [19], [23]–[29],all vectors of the same basis encode the same bit. Moreover, a bit can be transmitted only when Alice and Bob use different bases. The quantum key distribution scheme considered in this paper is designed such that no conditions have to be posed on the states of e and f , thereby giving us a lot of flexibility when maximizing the relevant minimum error rate introduced by Evan.

For simplicity, we consider only intercept–resend eavesdropping attacks. This is not the only possible eavesdropping attack, but security against this is considered a strong indication for the general security of a quantum cryptographic protocol.

In the special case of N = 2, the cryptographic scheme proposed in this paper is essentially equivalent to the SARG quantum key distribution protocol [30] with the parameter x chosen equal to 1/ 2. This protocol is tailored to be robust against photon number splitting attacks. In the SARG protocol, Alice publicly announces which one of the four different sets of states A +,+ ,A +,- , A -,+ and A -,- she used, while our protocol requires her only to announce either ‘i = 1’or ‘i = 2’. This difference is due to a redundancy in the SARG protocol.
There are five sections in this paper. In section 2, we introduce the notations used throughout this paper. In section 3, we calculate the ITER and the QBER for the quantum key distribution protocol introduced in section 2 as a function of the states used by Alice, Bob and Evan analytically. Afterwards, we determine their minima in the presence of intercept–resend eavesdropping attacks for different N s numerically. Geometrical considerations suggest tha tAlice and Bob should use two mutually unbiased bases. Section 4 analyses a concrete protocol based on this idea and shows that mutually unbiased bases indeed guarantee a high ITER and a high QBER in case of eavesdropping. Finally, we summarize our results in section 5.

\subsubsection{Alternative design}

In quantum cryptography there are conventionally three parties, Alice, Bob and Evan. Alice wants to transmit a sequence of secret bits to Bob. To do so, she prepares single photons in certain states and sends them to Bob. Bob measures the state of each incoming photon. Afterwards, Alice and Bob exchange information via classical communication. At the same time, Evan tries to catch the secret bits without revealing his presence. For example, he measures the state of every transmitted photon and listens in to the classical communication between Alice and Bob. The cryptographic protocol is secure as long as Evan’s attempts to obtain information result in an error rate which can be detected easily.

Let us start by introducing sufficient conditions for such a protocol to work:

1. Bob should measure the incoming photons in a randomly chosen basis. Otherwise, Evan simply uses the same measurement basis and the bit error rate remains zero. This means Bob should randomly switch between at least two sets of basis states. In the following weas sume that this is the case and denote these bases e = {|e i i: i = 1, . . . , N }and f ] {| f i i: i = 1, . . . , N }.

Here N is the dimension of the photon states. The only condition imposed on e and f is that they form a basis.

2. Similarly, Alice should encode the information that she wants to transmit to Bob such that it cannot be deduced easily by Evan. To obtain a non zero error rate in case of eavesdropping,she should either use non-orthogonal states (as in B92 [37]) or randomly switch between at least two sets of basis states (as in BB84 [19]). For simplicity, we assume in the following that Alice prepares each photon randomly in one of the basis states of e and f .

3. In order to establish strong correlations between Alice’s input state and Bob’s measurement outcome, Alice needs to reveal some information via classical communication. This information should be enough for Alice and Bob to obtain a shared secret key bit but not enough for the eavesdropper to deduce it. One possibility is that Alice announces which basis, e or f , she used (as in BB84 [19]). Another possibility is that Alice reveals the index i of the respective basis state (as in [35, 36]). This does not reveal any information about the key as long as the states |e i i and | f i i with the same index i encode different bits. In this paper, we consider this second approach and show that it can guarantee relatively high error rates in the presence of an eavesdropper.

4. We now have a closer look at how Bob should interpret his measurement outcomes after Alice told him the index i of her basis state. He can do this by using a table like table 1. If Alice announces that she prepared the photon in a state with index i, Bob obtains ‘n i j ’when he measures |e j i and he obtains ‘u i j ’ when he measures | f j i. The parameters n i j and u i j in the table assume three different values, ‘0,’ ‘1’, or ‘x’, depending on whether Bob obtains a ‘0’, a ‘1’, or no key bit is transmitted.
Suppose Alice sends a photon prepared in |e 1 i in order to transmit a ‘0’. This implies n 11 = ‘0’ or ‘x’(2)in order to avoid that Bob obtains a wrong key bit. The state | f 1 i has to encode a ‘1’ in this case. Otherwise, Evan knows that a ‘0’ is transmitted, when ‘i = 1’ is announced.
Consequently,u 11 = ‘1’ or ‘x’. Moreover, if Alice announces ‘i = 1’ and Bob measures | f j i with j 6= 1, then he knows for sure that she prepared her photon in |e 1 i. Analogously, if Alice announces ‘i = 1’ and Bob measures |e j i with j 6= 1, then he knows that Alice prepared | f 1 i. Alice and Bob should therefore choose n 1 j = ‘1’ and u 1 j = ‘0’, for all j 6= 1.
(4)There is no need for Bob to ignore a measurement outcome with j 6= 1 since he always knows which key bit the photon encodes in this case.

5. It is indeed possible (cf equations (2)–(4)) that table 1 contains no crosses and that every detected photon transmits one bit of the cryptographic key. However, the minimum error rate in the case of eavesdropping is already known to be relatively low in this case [35, 36]. We therefore assume here that Bob ignores the cases where his measured state has the index i announced by Alice and choose n ii = u ii = ‘x’.

This means a key bit can only be obtained when Bob’s measurement basis is different from the one used by Alice to prepare the photon. One can easily check that Alice and Bob always obtain the same secret key bit under ideal conditions.

6. For symmetry reasons, Alice should have equally many states to encode a ‘0’ as she has to encode a ‘1’. Without restrictions we therefore assume in the following that all the states o fe encode a ‘0’ while all states of f encode a ‘1’. This means, Bob obtains a ‘1’ whenever he measures a state |e j i with j different from Alice’s index i. Analogously, he obtains a‘0’ when he measures a state | f j i with j 6= i.

The final protocol is summarized in table 2 for the case where Alice and Bob communicate using four-dimensional photon states. For arbitrary N , the entire scheme works as follows:

1. Alice generates a random key sequence of classical bits and randomly assigns each bit value a random index i = 1, 2, . . . , N .
2. Alice then uses this sequence and sends single photons prepared accordingly either in |e i i or | f i i to Bob.
3. Bob measures the state of every incoming photon, thereby randomly switching the measurement basis between e and f .
4. Alice publicly announces the random sequence of indices i used to establish the cryptographic key.
5. Bob interprets his measurement outcomes accordingly, using, for example, table 2, if N = 4. He obtains a key bit whenever his index is different from the index announced by Alice.
6. Bob tells Alice which photon measurements have been successful and provide a bit of the secret key.
7. Finally, Alice and Bob determine whether an eavesdropper introduced an error into their communication. Whenever this error rate is sufficiently small, Alice and Bob can assume that no eavesdropping has occurred.
Note that no conditions are imposed on e and f in this section other than them being bases. This gives us a lot of flexibility when maximizing the security of the corresponding cryptographic protocol. In fact, the only difference between the above protocol and the direct communication scheme introduced in [35, 36] is that we avoid the assumption of he i | f i i being zero. Alice and Bob therefore have to discard their measurement outcomes when both their states have the same index. As we shall see below, the payoff for the corresponding loss inefficiency is a relatively high error rate in the presence of an eavesdropper.


\subsubsection{Eavesdropping}

In the quantum key distribution protocol proposed here, the index i of the photon state in transmission is the only publicly announced information. This index does not reveal any information about the corresponding key bit since it equally likely encodes a ‘0’ a sit encodes ‘1’. An eavesdropper can therefore only learn about the cryptographic key by performing quantum measurements on the transmitted photons. In the following, we assume that Evan measures the state of every photon using a basis g which is optimal for his purpose. Afterwards, he forwards his measurement outcome to Bob. This eavesdropping strategy is known as an intercept–resend attack. Although it is not the most general eavesdropping attack,the corresponding error rate is a strong indication for the security of a cryptographic protocol. Our aim is to increase the minimum error rate introduced by an eavesdropper above the 25\% of the original BB84 protocol [19]. As already mentioned above, there are different types of errors that Alice and Bob can consider.

In the following, i denotes again the index of the photon state prepared by Alice and j is the index of the basis vector measured by Bob. When Alice and Bob use different bases, j can assume any value between 1 and N , even in the absence of any eavesdropping. However, when Alice and Bob use the same basis, i and j should be the same. To detect Evan’s presence,Alice and Bob could therefore do the following: Alice should randomly select some photons that should not be used to obtain key bits. For these photons, she tells Bob exactly which states she prepared. Comparing this information with his own measurement outcomes, Bob can then easily calculate the ITER.

An index transmission error occurs when a photon prepared in |e i i (| f i i) is measured at Bob’s end as |e j i (| f j i) with i 6= j. Assuming that Alice prepares the 2N basis states of e and f with the same frequency and that Bob measures e and f with the same frequency, we find that the ITER of the proposed protocol equals N for a given set of bases e, f and g. The states |g k i denote Evan’s possible measurement outcomes, which are forwarded to Bob without alteration. In principle, Evan could change the state of the transmitted photon by guessing which state Alice prepared. However, this strategy is not expected to reduce the above error rate significantly. A non zero overlap between the basis states of e and f ensures that there is always a certain probability to guess incorrectly.

The same expression is obtained when calculating this error rate by subtracting the probability of not making an error under the condition that Alice and Bob use the same basis from unity.

While Alice and Bob want the error rate in equation (8) to be as large as possible, Evan wants it to be as small as possible. Both parties, with Alice and Bob on one side and Evan on the other side, try to optimize the choice of the bases e, f and g accordingly. Table 3shows the results of a numerical solution of this double-optimization problem for different dimensions N . To obtain this table, the basis e is kept fixed and a large number of bases f is generated randomly. For each f we then generate another large set of random bases g and determine the minima of the corresponding error rates using equation (8). This is illustrated in figure 1 for the N = 4 case. The ITER in table 3 is the maximum of all the obtained minimum error rates.
For N = 2, we find that the minimum ITER introduced by an eavesdropper equals 25\% when Alice and Bob use an optimal choice of e and f , as in the original BB84 protocol [19]. However, when Alice and Bob increase the dimension N of their photon states, the minimum ITER increases. One can easily see that 50\% constitutes an upper bound for the minimum ITER by considering the case where Evan measures the incoming photons either in the e or in the f basis. Using this eavesdropping strategy, the states of at least half of the transmitted photons remain unaffected.

Alternatively, Alice and Bob can detect a potential eavesdropper by calculating the usual QBER. To do so, both randomly select a certain number of control bits from the obtained key sequence and compare them openly. Note that a key bit is obtained when the index j of the state measured by Bob and the index i of the state prepared by Alice are different. Bob interprets his measurement result correctly only when both states belong to a different bases. A quantum bit error hence occurs when Bob measures |e j i (| f j i), while Alice prepared |e i i (| f i i) with i 6= j. Using equation (6), the QBER for a given set of bases e, f and g can be written asP QBER =P ITER,
2P IC(9)where the index-change (IC) probability P IC ,N. The third column in table 3 shows the minimum QBER introduced by an eavesdropper, when Alice and Bob use optimal bases e and f , for different dimensions N . As in section 3.1,these probabilities have been obtained by comparing probabilities for a large set of randomly generated bases f and g.

Even for N = 2, the minimum QBER can be as high as 50\%. A more detailed analysis of the corresponding protocol shows that Alice and Bob can realize this scenario by choosing e and f almost identical, independent of Evan’s choice of measurement basis g. However, the price they pay for this very high QBER is a steep drop in the efficiency of their quantum key distribution. In the extreme case, where |e 1 i = | f 1 i and |e 2 i = | f 2 i, it becomes impossible to generate secret key bits, since Alice’s and Bob’s state always have the same index i, at least in the absence of any eavesdropping. In the following, we assume therefore that Alice and Bob use the ITER in order to detect the presence of an eavesdropper.

\subsubsection{Optimal choices of e , f and g}

We now address the question of how Alice and Bob can take advantage of the high ITERs shown in table 3 by having a look at possible realizations of the proposed quantum key distribution protocol. First, we consider the N = 2 case, which suggests an optimal strategy for Alice and Bob in higher dimensions. Moreover, we discuss in this section what Evan can do to best cloud his presence.

The problems that Alice, Bob and Evan have to solve in the N = 2 case in order to optimize their strategies are exactly the same as in BB84 [19]. Suppose Alice and Bob choose (cf figure 2)

In order to maximize the minimum of this probability with respect to fi 2 , Alice and Bob should choose fi 1 = 14 pi. In this case, sin 2 (2(fi 1 - fi 2 )) becomes the same as cos 2 (2fi 2 ). This error rate equals 25\% independent of Evan’s choice of fi 2 . For the eavesdropper, every possible strategy is hence an optimal one.

In other words, for N = 2, Alice and Bob’s optimal choice for e and f are two mutually unbiased bases [21]. This means, upon measurement, a photon prepared in any of the basis states of e is found with equal probability in any of the basis states of f and vice versa. As shown in figure 2, e and f should be as far away from each other as possible. A straightforward generalization of this result to higher dimensions suggests that Alice and Bob should always us etwo mutually unbiased bases e and f .

Let us now have a closer look at the case where Alice and Bob communicate with four dimensional photon states. 

Since the second term in the brackets is always positive, one can easily see that P ITER > 37.5 \% .
Indeed, the best strategy for Evan is to choose sin(2a) = 0. This means, Evan should measure either e or f .
Figure 3 shows the error rate introduced by Evan for the above choice of e and f and fora large set of randomly generated gs with N = 4. It confirms that P ITER is always above 37.5\% if Alice and Bob use two mutually unbiased bases. This applies even when no assumption on the form of the states of g is made, as we do for our analytical calculations in equation (18). A comparison of P ITER = 37.5\% with the result in table 3 for N = 4 confirms that this error rate corresponds to (or is at least very close to) an optimal strategy of Alice, Bob and Evan. Both results agree within the error limits of the underlying numerical calculation.

Let us now have a look at the optimal choice of e, f and g for the general case where Alice and Bob use N -dimensional photon states. As suggested at the end of section 4.1, we assume that Alice and Bob use two mutually unbiased bases. More concretely, we assume that their states are given by 2. One can easily check that e and f are orthonormal and mutually unbiased. We then generate a large set of random gs and calculate the corresponding error rates P ITER using equation (8). Table 4 shows the maxima of these rates as a function of N . The given error rates hence correspond to Evan’s optimal intercept–resend eavesdropping strategy.

A comparison of the ITERs in table 4 with the ITERs in table 3 confirms that using two mutually unbiased bases e and f is (at least close to) an optimal strategy for Alice and Bob. For N = 2, we find again that the minimum error rate introduced by Evan equals 25\%. For N = 3this rate equals 33\%, and for higher-dimensional photon states, the values in the third column of table 4 approach their predicted maximum of 50\% (cf section 3). The second column has been obtained from a numerical optimization of Evan’s strategy. Since it is a relatively hard computational problem to find the best eavesdropping measurement basis g numerically, the errors in this column are relatively large, especially for large N .


Let us now have a closer look at the best intercept–resend eavesdropping strategy for Evan. The discussion of the N = 4 case in section 4.2 suggests that Evan should measure either e or f in order to minimize the bit transmission error rate. If e and f are mutually unbiased,then the probability of detecting a photon in |e i i equals 1/N when Alice prepares an f -state. Analogously, the probability of detecting a photon in | f i i equals 1/N when Alice prepares an e-state. Substituting this into equation (8) yields N - 1P ITER.

For completeness we mention that the corresponding QBER (cf equation (11)) equals 33\% independent of N . A comparison with a numerical evaluation of the ITER confirms that measuring either e or f and forwarding the respective measurement outcome to Bob is indeed an optimal (or at least a close to optimal) strategy for Evan, if Alice and Bob test his presence by calculating this error rate.
Let us conclude this subsection by commenting on the efficiency of the described quantum key distribution scheme. As in BB84, Alice and Bob randomly switch between two sets of basis states. Here a key bit can only be obtained when both use a different basis. Moreover, the index of the state measured by Bob should be different from the index of Alice’s state. The probability for this to happen and hence the mean number of bits per transmitted photon equals N - 1P success.

This expression is exactly the same as the ITER in equation (21).

In order to implement the above protocol, Alice needs a single-photon source. As in BB84, the photon can come from a parametric down conversion crystal, a very weak laser pulse, or an on-demand single-photon source. Using path encoding, the states of f can be prepared easily with the help of a Bell multiport beam splitter. Such a beam splitter may consist of a network of beam splitters and phase plates [38, 39], which have to be interferometrically stable. It can also be made by splicing N optical fibers [40]. Spliced fibre constructions are commercially available and can include between three and thirty input and output ports.

The main feature of a symmetric N x N Bell multiport beam splitter is that a photon entering any of its input ports is redirected with equal likelihood to any of its N possible out putports. One way for Alice to prepare the state |e i i in equation (20) is to bypass the beam splitter and to send a single photon directly to output port i. In this case, preparing the state | f i i inequation (20) only requires to send a single photon into input port i [22]. Bob can use the same setup as Alice to decode the key bit. To measure f , he should send the incoming photon first through a Bell multiport beam splitter and then detect it in one of the N output ports. To measure e, he can simply bypass this step. During the transmission, the path encoding should be switched to time-bin encoding, which promises a better protection of the photons against decoherence [20].

\subsubsection{Conclusions}

In this paper, we propose a quantum key distribution protocol where Alice and Bob use higher-dimensional photon states. The scheme does not encode information in a higher dimensional alphabet [23]–[29] and is not a straightforward generalization of the original BB84protocol [19]. Instead, Alice and Bob use two bases e and f with all e-states encoding a ‘0’and all f -states encoding a ‘1’. Under ideal conditions, a key bit is obtained when Alice and Bob use different bases. In section 2, no conditions are imposed on the states of e and f , which gives us the flexibility to maximize the error rate introduced by an eavesdropper in the case of an intercept–resend attack. This is not the only possible eavesdropping strategy but security against this is a strong indication for the general security of a cryptographic protocol.

In section 3, we generate large sets of random basis states and determine the minimum ITER introduced by an eavesdropper numerically. For N = 2, this error rate equals the 25\% QBER of the BB84 protocol [19]. However, the minimum ITER rapidly approaches 50\% in higher dimensions (cf table 3). A detailed analysis of the N = 2 and 4 cases suggests that Alice and Bob should use two mutually unbiased bases e and f . The best an eavesdropper can do to hide his presence is to measure the transmitted photons either in e or f . This hypothesis is consistent with the numerical results in section 4 (cf table 4). Finally, we point out that the proposed quantum key distribution protocol can be implemented for example with the help of a symmetric Bell multiport beam splitter [22] and switching from path to time-bin encoding during the transmission. The mean number of key bits per transmitted photon turns out to be exactly the same as the minimum error rate introduced by Evan (cf equation (22)).
In section 3, we point out that it is in principle possible to obtain a minimum QBER close to 50\%, even for N = 2. This requires Alice and Bob to use two bases e and f , which are almost identical. Unfortunately, this strategy corresponds to a very low efficiency of the proposed cryptographic protocol. For e = f , the key transmission rate drops to zero. Analytic expressions for the QBER and the efficiency of the bit transmission in the presence of an eavesdropper for a given set of bases can be found in equations (10) and (11).

We thank H Zbinden for helpful comments. MMK thanks J Xu for encouraging discussions and acknowledges funding from the NED University of Engineering and Technology, Karachi,Pakistan. AB thanks the project students S So and P Woodward for their initial participation in this project and acknowledges a James Ellis University Research Fellowship from the Royal Society and the GCHQ. This project was supported in part by the UK Engineering and Physical Sciences Research Council through the QIP IRC and the EU Research and Training Network EMALI.

\subsection{\trnas}
\subsubsection*{Аннотация}

В оригинальном протоколе BB84 Беннета и Брассарда подслушивающий обнаруживается, поскольку его попытки перехватить информацию приводят к высокому квантовому коэффициенту битовых ошибок (QBER) не менее 25\%. Здесь мы разрабатываем альтернативный протокол квантового распределения ключей, в котором Алиса и Боб используют два взаимно несмещенных базиса, один из которых кодирует '0', а другой '1'. Безопасность схемы обусловлена минимальным индексом частоты ошибок передачи (ITER), вносимых подслушивающим устройством, который значительно увеличивается для состояний фотонов более высокой размерности. Это позволяет увеличить шум в линии передачи, тем самым увеличивая возможное расстояние между Алисой и Бобом без необходимости в промежуточных узлах.

\subsubsection{Введение}

Цель квантовой криптографии - установить общую, секретную и случайную последовательность битов между отправителем по имени Алиса и получателем по имени Боб [1]. Эта последовательность представляет собой идеальный криптографический ключ, так называемый одноразовый блокнот, и позволяет Алисе и Бобу надежно зашифровать сообщение одинаковой длины. Криптографический ключ получается исключительно с помощью передачи фотонов и классической коммуникации. Каждый бит кодируется в состоянии одного фотона и считывается Бобом по прибытии с помощью квантового измерения. Случайное переключение между различными базами делает невозможным для подслушивающего, которого зовут Эван, предсказать состояния, используемые в протоколе. Все его попытки перехватить фотоны приводят к значительному коэффициенту квантовых битовых ошибок (QBER). Это гарантирует высокий уровень безопасности, поскольку присутствие Эвана легко обнаруживается, когда Алиса и Боб сравнивают несколько тестовых битов.

В идеальных условиях обмен одиночными фотонами позволяет Алисе и Бобу установить криптографический ключ на произвольно большом расстоянии. На практике криптографические установки состоят из несовершенных однофотонных источников, линий передачи с потерями и детекторов фотонов с темной скоростью счета. Следовательно, Алиса и Боб должны применять классические средства обработки информации, такие как коррекция ошибок и усиление конфиденциальности [2, 3], к своим данным, чтобы получить идентичные секретные ключи. Однако криптографические протоколы безопасны только до тех пор, пока можно обнаружить присутствие подслушивающего. Системные ошибки могут затуманить присутствие Эвана; тем более что он может просто заменить части оборудования на высококачественные компоненты. Это делает невозможным переносить большие системные ошибки и ограничивает возможное расстояние между Алисой и Бобом.

Недавно Розенберг и другие [4, 5] сообщили о создании защищенного криптографического ключа на расстоянии 144,3 км по оптическому волокну. Их схема основана на протоколе обманного состояния [6]-[8], который невосприимчив к атакам с расщеплением числа фотонов и очень устойчив к атакам "троянского коня" [9]. Ожидается, что усовершенствования в фильтрации фотонов черного позволят расширить оптоволокно до 250 км. Тем временем, Takesue et al [10] и Stuckiet al [11] создали защищенный криптографический ключ на расстоянии 200 км оптического волокна. Считается, что эти экспериментальные установки являются самыми длинными наземными оптоволоконными линиями квантового распределения ключей. Сопоставимые расстояния были достигнуты в свободном пространстве. Например, Шмитт-Мандербах и другие [12] обеспечили безопасное распространение криптографического ключа через
144-километровая линия связи в свободном пространстве.

В данной работе мы разработали новый протокол квантового распределения ключей, эффективность которого и минимальная частота ошибок в случае подслушивания увеличиваются с ростом размерности фотонных состояний, используемых Алисой и Бобом. Таким образом, мы увеличиваем порог допустимых системных ошибок без ущерба для безопасности протокола и, следовательно, увеличиваем возможное расстояние между Алисой и Бобом. В принципе, одиночные фотоны могут быть очищены с помощью квантовых повторителей [13]. Были сделаны предложения по их реализации (см., например, [14, 15]) и другие шумопонижающие звенья [16]-[18], но их экспериментальная реализация и практическая интеграция в криптографические сети еще предстоит.

Все вышеупомянутые схемы квантового распределения ключей на большие расстояния [4, 5, 10, 12] основаны на протоколе BB84 Беннета и Брассарда [19]. В BB84 Алиса кодирует свои биты в двумерных состояниях фотонов. Они могут быть получены с помощью поляризационного кодирования. Однако более естественным выбором является кодирование по временному интервалу, которое обеспечивает лучшую защиту фотонов от декогеренции [20]. Алиса и Боб независимо меняют свои базы между двумя вариантами. Ключевой бит получается всякий раз, когда оба используют один и тот же базис. Это означает, что в среднем каждый второй фотон вносит бит в криптографический ключ. Используя простую стратегию перехвата-передачи, подслушивающее лицо вносит в коммуникацию QBER не менее 25\%.

В дальнейшем мы предполагаем, что Алиса и Боб используют закодированные во времени или пути N -мерные состояния фотонов. Как и в BB84, Алиса и Боб случайным образом изменяют свои базы между двумя взаимно несмещенными базами [21]. Однако, в отличие от BB84, Алиса и Боб обнаруживают присутствие подслушивающего устройства путем вычисления коэффициента ошибок передачи индекса (ITER). Как мы увидим ниже, для N = 2, Эван приводит к минимальному ITER в 25\%. В случае четырехмерных фотонных состояний этот коэффициент ошибок становится равным 37,5\%. При дальнейшем увеличении N минимальный ITER приближается к 50\%. Эффективность протокола в единицах переданных бит на фотон совпадает с минимальным ITER в случае подслушивания. Состояния, необходимые для предлагаемой схемы распределения ключей, могут быть реализованы с помощью симметричного многопортового делителя луча Белла [22]. Перед передачей кодирование путей выходных состояний многопортового делителя луча Белла должно быть переключено на вышеупомянутое кодирование с временным бином [20].

Уже было предложено несколько обобщений квантовых криптографических схем на более высокие измерения. Работы [23]-[29] являются обобщением оригинального протокола BB84 [19], основанного на кодировании информации в более высокоразмерных алфавитах. Однако недавно было показано [31]-[33], что безопасность протокола BB84 полностью нарушается, если Алиса и Боб таким образом обмениваются, например, четырехмерными состояниями фотонов [34]. Бейдж и другие [35, 36] предлагают два альтернативных квантовых криптографических протокола с использованием четырехмерных фотонных состояний, которые в идеальных условиях позволяют Алисе и Бобу общаться напрямую, но чьи минимальные коэффициенты ошибок в случае подслушивания относительно низки.

Здесь мы покажем, что существуют и другие возможные криптографические схемы в более высоких измерениях. A sin [35, 36], Алиса и Боб используют два базиса e и f, причем все состояния e кодируют "0", а все состояния f кодируют "1", даже если N больше двух. Это означает, что, в отличие от [19], [23]-[29], все векторы одного базиса кодируют один и тот же бит. Более того, бит может быть передан только тогда, когда Алиса и Боб используют разные базисы. Схема квантового распределения ключей, рассматриваемая в данной работе, разработана таким образом, что на состояния e и f не нужно накладывать никаких условий, что дает нам большую гибкость при максимизации соответствующего минимального коэффициента ошибок, введенного Эваном.

Для простоты мы рассматриваем только атаки перехвата-передачи. Это не единственная возможная атака подслушивания, но защита от нее считается сильным показателем общей безопасности квантового криптографического протокола.

В частном случае N = 2 криптографическая схема, предложенная в данной работе, по существу эквивалентна протоколу квантового распределения ключей SARG [30] с параметром x, выбранным равным 1/ 2. Этот протокол приспособлен для устойчивости к атакам с расщеплением числа фотонов. В протоколе SARG Алиса публично объявляет, какой из четырех различных наборов состояний A +,+ , A +,- , A -,+ и A -,- она использовала, в то время как наш протокол требует, чтобы она объявила только 'i = 1' или 'i = 2'. Это различие объясняется избыточностью протокола SARG.
В данной работе пять разделов. В разделе 2 мы вводим обозначения, используемые в данной работе. В разделе 3 мы рассчитываем ITER и QBER для протокола квантового распределения ключей, представленного в разделе 2, как функцию состояний, используемых Алисой, Бобом и Эваном, аналитически. Затем мы численно определяем их минимумы в присутствии атак перехвата-передачи-подслушивания для различных N s. Геометрические соображения показывают, что Алиса и Боб должны использовать две взаимно несмещенные базы. В разделе 4 анализируется конкретный протокол, основанный на этой идее, и показывается, что взаимно несмещенные базы действительно гарантируют высокий ITER и высокий QBER в случае подслушивания. Наконец, в разделе 5 мы обобщаем наши результаты.

\subsubsection{Альтернативный дизайн}

В квантовой криптографии условно есть три стороны, Алиса, Боб и Эван. Алиса хочет передать Бобу последовательность секретных битов. Для этого она готовит отдельные фотоны в определенных состояниях и посылает их Бобу. Боб измеряет состояние каждого входящего фотона. После этого Алиса и Боб обмениваются информацией посредством классической связи. В то же время Эван пытается перехватить секретные биты, не раскрывая своего присутствия. Например, он измеряет состояние каждого передаваемого фотона и прослушивает классический обмен информацией между Алисой и Бобом. Криптографический протокол безопасен до тех пор, пока попытки Эвана получить информацию приводят к частоте ошибок, которую можно легко обнаружить.

Давайте начнем с введения достаточных условий для работы такого протокола:

1. Боб должен измерять входящие фотоны в случайно выбранном базисе. В противном случае Эван просто использует один и тот же базис измерения, и коэффициент битовых ошибок остается нулевым. Это означает, что Боб должен случайным образом переключаться между как минимум двумя наборами состояний базиса. В дальнейшем мы будем считать, что это так, и обозначим эти базисы e = {|e i i: i = 1, . . . , N } и f ] {| f i i: i = 1, . . . . , N }.

Здесь N - размерность состояний фотона. Единственное условие, налагаемое на e и f, состоит в том, что они образуют базис.

2. Аналогично, Алиса должна закодировать информацию, которую она хочет передать Бобу, так, чтобы она не могла быть легко вычитана Эваном. Чтобы получить ненулевой коэффициент ошибок в случае подслушивания, она должна либо использовать неортогональные состояния (как в B92 [37]), либо случайным образом переключаться между как минимум двумя наборами базисных состояний (как в BB84 [19]). Для простоты мы предположим в дальнейшем, что Алиса готовит каждый фотон случайным образом в одном из базисных состояний e и f .

3. Чтобы установить сильные корреляции между входным состоянием Алисы и результатом измерения Боба, Алиса должна раскрыть некоторую информацию посредством классической связи. Этой информации должно быть достаточно, чтобы Алиса и Боб получили общий бит секретного ключа, но недостаточно, чтобы подслушивающее лицо могло его вычислить. Одна из возможностей состоит в том, что Алиса сообщает, какой базис, e или f, она использовала (как в BB84 [19]). Другая возможность заключается в том, что Алиса раскрывает индекс i соответствующего состояния базиса (как в [35, 36]). Это не раскрывает никакой информации о ключе до тех пор, пока состояния |e i i и | f i i с одинаковым индексом i кодируют разные биты. В данной работе мы рассматриваем этот второй подход и показываем, что он может гарантировать относительно высокую частоту ошибок в присутствии подслушивающего устройства.

4. Теперь мы более подробно рассмотрим, как Боб должен интерпретировать результаты своих измерений после того, как Алиса сообщила ему индекс i своего базисного состояния. Он может сделать это с помощью таблицы, подобной таблице 1. Если Алиса сообщает, что она приготовила фотон в состоянии с индексом i, Боб получает 'n i j' при измерении |e j i и 'u i j ' при измерении | f j i. Параметры n i j и u i j в таблице принимают три различных значения, '0', '1' или 'x', в зависимости от того, получает ли Боб '0', '1' или ключевой бит не передается.
Предположим, что Алиса посылает фотон, подготовленный |e 1 i, чтобы передать '0'. Это подразумевает n 11 = '0' или 'x'(2), чтобы Боб не получил неправильный бит ключа. Состояние | f 1 i должно кодировать '1' в этом случае. В противном случае Эван знает, что передается '0', когда объявляется 'i = 1'.
Следовательно, u 11 = '1' или 'x'. Более того, если Алиса объявит "i = 1" и Боб измерит | f j i при j 6= 1, то он точно знает, что она приготовила свой фотон в |e 1 i. Аналогично, если Алиса объявит "i = 1" и Боб измерит |e j i при j 6= 1, то он знает, что Алиса приготовила | f 1 i. Поэтому Алиса и Боб должны выбрать n 1 j = '1' и u 1 j = '0', для всех j 6= 1.
(4)Бобу нет необходимости игнорировать результат измерения с j 6= 1, поскольку он всегда знает, какой ключевой бит кодирует фотон в этом случае.

5. Действительно, возможно (ср. уравнения (2)-(4)), что таблица 1 не содержит крестиков и что каждый обнаруженный фотон передает один бит криптографического ключа. Однако уже известно, что минимальный коэффициент ошибок в случае подслушивания в этом случае относительно мал [35, 36]. Поэтому мы предполагаем, что Боб игнорирует случаи, когда его измеренное состояние имеет индекс i, объявленный Алисой, и выбирает n ii = u ii = 'x'.

Это означает, что бит ключа может быть получен только тогда, когда измерительная база Боба отличается от той, которую использовала Алиса для подготовки фотона. Можно легко проверить, что в идеальных условиях Алиса и Боб всегда получают один и тот же бит секретного ключа.

6. Из соображений симметрии Алиса должна иметь столько же состояний для кодирования "0", сколько и для кодирования "1". Поэтому без ограничений мы предполагаем в дальнейшем, что все состояния fe кодируют "0", а все состояния f кодируют "1". Это означает, что Боб получает "1" всякий раз, когда он измеряет состояние |e j i с j, отличным от индекса i Алисы. Аналогично, он получает "0", когда он измеряет состояние | f j i с j 6= i.

Окончательный протокол обобщен в таблице 2 для случая, когда Алиса и Боб общаются, используя четырехмерные состояния фотонов. Для произвольного N вся схема работает следующим образом:

1. Алиса генерирует случайную последовательность ключей из классических битов и случайным образом присваивает каждому значению бита случайный индекс i = 1, 2, . . . , N .
2. Затем Алиса использует эту последовательность и посылает Бобу одиночные фотоны, подготовленные соответствующим образом либо в |e i i, либо в | f i i.
3. Боб измеряет состояние каждого входящего фотона, тем самым случайным образом переключая базу измерения между e и f .
4. Алиса публично объявляет случайную последовательность индексов i, использованную для создания криптографического ключа.
5. Боб интерпретирует результаты своих измерений соответствующим образом, используя, например, таблицу 2, если N = 4. Он получает бит ключа всякий раз, когда его индекс отличается от индекса, объявленного Алисой.
6. Боб сообщает Алисе, какие измерения фотонов были успешными, и предоставляет бит секретного ключа.
7. Наконец, Алиса и Боб определяют, внесла ли подслушивающая сторона ошибку в их сообщение. Если процент ошибок достаточно мал, Алиса и Боб могут считать, что подслушивания не было.
Обратите внимание, что в этом разделе на e и f не накладывается никаких условий, кроме того, что они являются базисами. Это дает нам большую гибкость при максимизации безопасности соответствующего криптографического протокола. Фактически, единственное различие между приведенным выше протоколом и схемой прямой связи, представленной в [35, 36], заключается в том, что мы избегаем предположения о том, что he i | f i i равно нулю. Поэтому Алиса и Боб должны отбрасывать результаты своих измерений, когда оба их состояния имеют одинаковый индекс. Как мы увидим ниже, платой за соответствующую неэффективность потерь является относительно высокий коэффициент ошибок в присутствии подслушивающего устройства.


\subsubsection{Подслушивание}

В предлагаемом здесь протоколе квантового распределения ключей индекс i состояния фотона при передаче является единственной публично объявленной информацией. Этот индекс не раскрывает никакой информации о соответствующем ключевом бите, поскольку он с равной вероятностью кодирует '0', а сид кодирует '1'. Поэтому подслушивающее устройство может узнать о криптографическом ключе, только выполнив квантовые измерения передаваемых фотонов. Далее мы предположим, что Эван измеряет состояние каждого фотона, используя базис g, который оптимален для его цели. После этого он передает результаты измерений Бобу. Эта стратегия подслушивания известна как атака "перехват - передача". Хотя это не самая общая атака подслушивания, соответствующая частота ошибок является сильным показателем безопасности криптографического протокола. Наша цель - увеличить минимальный коэффициент ошибок, вносимых подслушивающим устройством, выше 25\% оригинального протокола BB84 [19]. Как уже упоминалось выше, существуют различные типы ошибок, которые могут учитывать Алиса и Боб.

В дальнейшем i снова обозначает индекс состояния фотона, подготовленного Алисой, а j - индекс вектора базиса, измеренного Бобом. Когда Алиса и Боб используют разные базисы, j может принимать любое значение между 1 и N, даже в отсутствие подслушивания. Однако, когда Алиса и Боб используют один и тот же базис, i и j должны быть одинаковыми. Чтобы обнаружить присутствие Эвана, Алиса и Боб могут поступить следующим образом: Алиса должна случайным образом выбрать несколько фотонов, которые не должны быть использованы для получения ключевых битов. Для этих фотонов она сообщает Бобу, какие именно состояния она подготовила. Сравнивая эту информацию с результатами собственных измерений, Боб может затем легко вычислить ITER.

Ошибка передачи индекса возникает, когда фотон, подготовленный в |e i i (| f i i), измеряется на стороне Боба как |e j i (| f j i) с i 6= j. Предполагая, что Алиса готовит 2N базисных состояний e и f с одинаковой частотой и что Боб измеряет e и f с одинаковой частотой, мы находим, что ITER предлагаемого протокола равен N для данного набора базисов e, f и g. Состояния |g k i обозначают возможные результаты измерений Эвана, которые передаются Бобу без изменений. В принципе, Эван может изменить состояние передаваемого фотона, угадав, какое состояние приготовила Алиса. Однако ожидается, что эта стратегия не приведет к значительному снижению приведенного выше коэффициента ошибок. Ненулевое перекрытие между базисными состояниями e и f гарантирует, что всегда существует определенная вероятность ошибочного угадывания.

Такое же выражение получается при вычислении коэффициента ошибок путем вычитания из вероятности, что Алиса и Боб используют один и тот же базис.

В то время как Алиса и Боб хотят, чтобы коэффициент ошибок в уравнении (8) был как можно больше, Эван хочет, чтобы он был как можно меньше. Обе стороны, Алиса и Боб с одной стороны и Эван с другой, пытаются оптимизировать выбор базисов e, f и g соответственно. В таблице 3 показаны результаты численного решения этой задачи двойной оптимизации для различных размерностей N . Для получения этой таблицы базис e остается фиксированным, а большое количество базисов f генерируется случайным образом. Для каждого f мы генерируем другой большой набор случайных базисов g и определяем минимумы соответствующих коэффициентов ошибок, используя уравнение (8). Это показано на рисунке 1 для случая N = 4. Показатель ITER в таблице 3 является максимальным из всех полученных минимальных коэффициентов ошибок.
Для N = 2 мы обнаружили, что минимальный ITER, вносимый подслушивающим устройством, равен 25\%, когда Алиса и Боб используют оптимальный выбор e и f, как в оригинальном протоколе BB84 [19]. Однако, когда Алиса и Боб увеличивают размерность N своих фотонных состояний, минимальный ITER увеличивается. Можно легко убедиться, что 50\% составляет верхнюю границу для минимального ITER, рассмотрев случай, когда Эван измеряет входящие фотоны либо в базисе e, либо в базисе f. При использовании этой стратегии подслушивания состояния по крайней мере половины передаваемых фотонов остаются незатронутыми.

В качестве альтернативы Алиса и Боб могут обнаружить потенциального подслушивающего устройства, вычислив обычный QBER. Для этого оба случайным образом выбирают определенное количество контрольных битов из полученной ключевой последовательности и открыто сравнивают их. Обратите внимание, что ключевой бит получается, когда индекс j состояния, измеренного Бобом, и индекс i состояния, подготовленного Алисой, различны. Боб правильно интерпретирует результат своего измерения только тогда, когда оба состояния принадлежат разным базам. Следовательно, ошибка квантового бита возникает, когда Боб измеряет |e j i (| f j i), а Алиса подготовила |e i i (| f i i) с i 6= j. Используя уравнение (6), QBER для заданного набора базисов e, f и g можно записать как P QBER =P ITER, 2P IC(9)где вероятность смены индекса (IC) P IC ,N. В третьем столбце таблицы 3 показано минимальное QBER, вносимое подслушивающим лицом, когда Алиса и Боб используют оптимальные базы e и f , для различных размеров N . Как и в разделе 3.1, эти вероятности были получены путем сравнения вероятностей для большого набора случайно сгенерированных баз f и g.

Даже для N = 2 минимальное QBER может достигать 50\%. Более детальный анализ соответствующего протокола показывает, что Алиса и Боб могут реализовать этот сценарий, выбрав e и f почти одинаковыми, независимо от выбора Эваном базиса измерения g. Однако ценой, которую они платят за этот очень высокий QBER, является резкое падение эффективности квантового распределения ключей. В предельном случае, когда |e 1 i = | f 1 i и |e 2 i = | f 2 i, становится невозможным генерировать биты секретного ключа, поскольку состояния Алисы и Боба всегда имеют один и тот же индекс i, по крайней мере, в отсутствие подслушивания. В дальнейшем мы предполагаем, что Алиса и Боб используют ITER для обнаружения присутствия подслушивающего лица.

\subsubsection{Оптимальный выбор e, f и g}

Теперь мы обратимся к вопросу о том, как Алиса и Боб могут воспользоваться преимуществами высоких ITER, показанных в таблице 3, рассмотрев возможные реализации предложенного протокола квантового распределения ключей. Во-первых, мы рассматриваем случай N = 2, который предполагает оптимальную стратегию для Алисы и Боба в более высоких измерениях. Более того, в этом разделе мы обсуждаем, что может сделать Эван, чтобы наилучшим образом скрыть свое присутствие.

Задачи, которые Алиса, Боб и Эван должны решить в случае N = 2, чтобы оптимизировать свои стратегии, точно такие же, как и в BB84 [19]. Предположим, что Алиса и Боб выбирают (см. рисунок 2)

Чтобы максимизировать минимум этой вероятности относительно fi 2 , Алиса и Боб должны выбрать fi 1 = 14 pi. В этом случае sin 2 (2(fi 1 - fi 2 )) становится равным cos 2 (2fi 2 ). Этот коэффициент ошибок равен 25\% независимо от выбора Эваном значения fi 2 . Для подслушивающего устройства каждая возможная стратегия, следовательно, является оптимальной.

Другими словами, для N = 2 оптимальный выбор Алисы и Боба для e и f - это два взаимно несмещенных базиса [21]. Это означает, что при измерении фотон, подготовленный в любом из состояний базиса e, с равной вероятностью будет найден в любом из состояний базиса f, и наоборот. Как показано на рисунке 2, e и f должны быть как можно дальше друг от друга. Прямое обобщение этого результата на более высокие измерения предполагает, что Алиса и Боб всегда должны использовать два взаимно несмещенных базиса e и f .

Давайте теперь более подробно рассмотрим случай, когда Алиса и Боб общаются с помощью четырехмерных состояний фотонов. 

Поскольку второй член в скобках всегда положительный, можно легко убедиться, что P ITER > 37,5 \% .
Действительно, лучшей стратегией для Эвана является выбор sin(2a) = 0. Это означает, что Эван должен измерить либо e, либо f .
На рисунке 3 показан коэффициент ошибок, вносимых Эваном для вышеуказанного выбора e и f и для большого набора случайно сгенерированных gs с N = 4. Это подтверждает, что P ITER всегда выше 37,5\%, если Алиса и Боб используют две взаимно несмещенные базы. Это справедливо даже тогда, когда не делается никаких предположений о форме состояний g, как мы делаем для наших аналитических расчетов в уравнении (18). Сравнение P ITER = 37,5\% с результатом в таблице 3 для N = 4 подтверждает, что этот коэффициент ошибок соответствует (или, по крайней мере, очень близок к нему) оптимальной стратегии Алисы, Боба и Эвана. Оба результата согласуются в пределах погрешности основополагающего численного расчета.

Давайте теперь посмотрим на оптимальный выбор e, f и g для общего случая, когда Алиса и Боб используют N -мерные состояния фотонов. Как было предложено в конце раздела 4.1, мы предполагаем, что Алиса и Боб используют две взаимно несмещенные базы. Более конкретно, мы предполагаем, что их состояния заданы 2. Можно легко проверить, что e и f ортонормальны и взаимно несмещены. Затем мы генерируем большой набор случайных gs и вычисляем соответствующие коэффициенты ошибок P ITER, используя уравнение (8). В таблице 4 показаны максимумы этих коэффициентов как функция от N. Следовательно, данные коэффициенты ошибок соответствуют оптимальной стратегии подслушивания Эвана "перехват - передача".

Сравнение ИТЭР в таблице 4 с ИТЭР в таблице 3 подтверждает, что использование двух взаимно несмещенных баз e и f является (по крайней мере, близкой к оптимальной) стратегией для Алисы и Боба. Для N = 2 мы снова находим, что минимальная ошибка, вносимая Эваном, равна 25\%. Для N = 3 этот показатель равен 33\%, а для более высокоразмерных фотонных состояний значения в третьей колонке таблицы 4 приближаются к предсказанному максимуму в 50\% (см. раздел 3). Вторая колонка была получена в результате численной оптимизации стратегии Эвана. Поскольку численный поиск наилучшего базиса измерения g для подслушивания является относительно трудной вычислительной задачей, ошибки в этом столбце относительно велики, особенно для больших N .


Теперь давайте более подробно рассмотрим наилучшую стратегию перехвата-передачи для Эвана. Обсуждение случая N = 4 в разделе 4.2 предполагает, что Эван должен измерять либо e, либо f, чтобы минимизировать коэффициент ошибок при передаче битов. Если e и f взаимно несмещены, то вероятность обнаружения фотона в |e i i равна 1/N, когда Алиса готовит f -состояние. Аналогично, вероятность обнаружения фотона в | f i i равна 1/N, когда Алиса готовит e-состояние. Подстановка этих данных в уравнение (8) дает N - 1P ITER.

Для полноты картины отметим, что соответствующий QBER (ср. уравнение (11)) равен 33\% независимо от N. Сравнение с численной оценкой ITER подтверждает, что измерение либо e, либо f и пересылка соответствующего результата измерения Бобу действительно является оптимальной (или, по крайней мере, близкой к оптимальной) стратегией для Эвана, если Алиса и Боб проверят его присутствие, вычислив этот коэффициент ошибок.
В заключение этого подраздела прокомментируем эффективность описанной схемы квантового распределения ключей. Как и в BB84, Алиса и Боб случайным образом переключаются между двумя наборами базисных состояний. Здесь ключевой бит может быть получен только тогда, когда оба используют разные базисы. Более того, индекс состояния, измеренного Бобом, должен отличаться от индекса состояния Алисы. Вероятность этого и, следовательно, среднее число бит на передаваемый фотон равны N - 1P успехам.

Это выражение в точности совпадает с выражением ITER в уравнении (21).

Для реализации вышеописанного протокола Алисе необходим однофотонный источник. Как и в BB84, фотон может исходить от кристалла параметрического преобразования вниз, очень слабого лазерного импульса или однофотонного источника по требованию. Используя кодирование пути, состояния f могут быть легко подготовлены с помощью многопортового делителя луча Белла. Такой светоделитель может состоять из сети светоделителей и фазовых пластин [38, 39], которые должны быть интерферометрически стабильными. Он также может быть изготовлен путем сращивания N оптических волокон [40]. Конструкции из сращиваемых волокон являются коммерчески доступными и могут включать от трех до тридцати входных и выходных портов.

Главная особенность симметричного многопортового делителя луча Белла N x N заключается в том, что фотон, входящий в любой из его входных портов, с равной вероятностью перенаправляется в любой из N возможных выходных портов. Один из способов для Алисы подготовить состояние |e i i в уравнении (20) - это обойти делитель луча и послать один фотон непосредственно в выходной порт i. В этом случае для подготовки состояния | f i i в уравнении (20) требуется только послать один фотон во входной порт i [22]. Боб может использовать ту же установку, что и Алиса, для декодирования ключевого бита. Чтобы измерить f, он должен послать входящий фотон сначала через многопортовый делитель луча Белла, а затем обнаружить его в одном из N выходных портов. Чтобы измерить e, он может просто обойти этот шаг. Во время передачи кодирование пути должно быть переключено на кодирование с временным интервалом, что обещает лучшую защиту фотонов от декогеренции [20].

\subsubsection{Выводы}

В этой статье мы предлагаем протокол квантового распределения ключей, в котором Алиса и Боб используют состояния фотонов более высокой размерности. Схема не кодирует информацию в алфавите более высокой размерности [23]-[29] и не является прямым обобщением оригинального протокола BB84 [19]. Вместо этого Алиса и Боб используют две базы e и f, причем все e-состояния кодируют '0', а все f-состояния - '1'. В идеальных условиях бит ключа получается, когда Алиса и Боб используют разные базы. В разделе 2 на состояния e и f не накладывается никаких условий, что дает нам возможность максимизировать коэффициент ошибок, вносимых подслушивающим устройством в случае атаки перехвата-передачи. Это не единственная возможная стратегия подслушивания, но защита от нее является сильным показателем общей безопасности криптографического протокола.

В разделе 3 мы генерируем большие наборы случайных базисных состояний и численно определяем минимальный ITER, вносимый подслушивающим устройством. Для N = 2 эта частота ошибок равна 25\% QBER протокола BB84 [19]. Однако, минимальный ITER быстро приближается к 50\% в более высоких измерениях (см. таблицу 3). Детальный анализ случаев N = 2 и 4 показывает, что Алиса и Боб должны использовать две взаимно несмещенные базы e и f . Лучшее, что может сделать подслушивающий, чтобы скрыть свое присутствие, это измерить переданные фотоны либо в e, либо в f . Эта гипотеза согласуется с численными результатами в разделе 4 (см. таблицу 4). Наконец, отметим, что предложенный протокол квантового распределения ключей может быть реализован, например, с помощью симметричного многопортового делителя луча Белла [22] и переключения с кодирования по пути на кодирование по временной шкале во время передачи. Среднее число ключевых битов на передаваемый фотон оказывается точно таким же, как минимальный коэффициент ошибок, введенный Эваном (ср. уравнение (22)).
В разделе 3 мы указываем, что в принципе возможно получить минимальный QBER, близкий к 50\%, даже для N = 2. Для этого Алисе и Бобу необходимо использовать две базы e и f, которые практически идентичны. К сожалению, такая стратегия соответствует очень низкой эффективности предлагаемого криптографического протокола. При e = f скорость передачи ключа падает до нуля. Аналитические выражения для QBER и эффективности передачи битов в присутствии подслушивающего устройства для заданного набора баз можно найти в уравнениях (10) и (11).

Мы благодарим Х. Збиндена за полезные комментарии. MMK благодарит J Xu за поощрительные обсуждения и выражает признательность за финансирование Университета инженерии и технологии NED, Карачи, Пакистан. AB благодарит студентов проекта S So и P Woodward за их первоначальное участие в этом проекте и выражает признательность за стипендию James Ellis University Research Fellowship от Королевского общества и GCHQ. Этот проект был частично поддержан Исследовательским советом по инженерным и физическим наукам Великобритании через QIP IRC и Исследовательской и учебной сетью ЕС EMALI.
\subsection{\review}
The author gives a full theoretical coverage of quantum cryptography.
\begin{itemize}
	\item quantum information transfer;
	\item quantum sensorics;
	\item quantum computer;
	\item quantum software.
\end{itemize}
The author describes recent advances in these areas and draws conclusions about the further development of quantum technologies.


\subsection{\dic}
\begin{multicols}{2}
	\begin{itemize}
		
		\item algorithms - алгоритм
		\item analysis - анализ
		
		\item appropriate - подходящий
		\item approximately - примерно
		
		
		\item basis - основа
		\item beam - луч
		
		\item binary - двоичных
		\item bit - бит
		
		\item capacity - вместимость
		
		\item channel - канал
		
		\item coherent - связный
		\item combination - комбинации
		
		\item communication - связь
		\item compare - сравнить
		
		\item computation - вычисления
		\item computers - компьютеров
		
		\item condition - условие
		\item conjugate - спряжение
		\item considered - рассмотрено
		\item contain - содержат
		
		\item correlation - корреляция
		
		\item cryptography - криптография
		
		\item decode - декодировать
		\item decoy - ловушка
		\item density - плотность
		
		\item dependence - зависимость
		\item detect - обнаружить
		
		\item deterministic - детерминированный
		
		\item developing - разработка
		
		\item difference - разница
		
		\item differentiate - дифференцировать
		
		\item distribution - распределение
		
		\item eavesdropper - подслушиватель
		
		\item encoded - закодировано
		\item entaglement - запутанность
		\item equivalently - эквивалентно
		
		\item imperfections - недостаток
		\item implementation - реализация
		
		\item instances - экземпляров
		
		\item intensity - интенсивность
		\item intercept - перехват
		
		\item interferometer - интерферометр
		
		\item limitation - ограничение
		
		\item lossless - без потерь
		\item lossy - потери
		
		\item malicious - вредоносных
		\item managed - управляемых
		
		\item mathematical - математических
		\item matrix - матрица
		
		\item measurement - измерения
		
		\item method - метод
		\item mimic - имитировать
		
		\item modification - модификация
		
		\item neglected - пренебрегают
		
		\item normalized - нормализовано
		
		\item operator - оператор
		\item optical - оптический
		\item optimal - оптимальный
		
		\item optional - опционально
		
		\item orthogonal - ортогональный
		\item orthogonality - ортогональность
		\item orthonormal - ортонормированный
		
		\item phase - фаза
		\item photon - фотон
		
		\item polarization - поляризация
		
		
		\item probabilistic - вероятностный
		\item probability - вероятность
		\item problem - проблема
		
		\item projecting - проектирование
		
		\item property - свойство
		\item proportion - пропорция
		\item proposal - предложение
		
		\item protocol - протокол
		
		\item prove - доказательство
		
		\item provide - обеспечить
		\item provides - обеспечивает
		
		\item public - общедоступных
		\item pulse - импульс
		
		\item quantum - квантовый
		\item random - случайных
		\item randomize - рандомизировать
		
		\item signature - подпись
		
		\item strategy - стратегия
		
		\item string - строка
		
		\item symmetric - симметричный
		
		\item theoretical - теоретический
		\item theory - теория
		
		\item transmission - передача
		
		\item vacuum - вакуум
		\item value - значение
		
		\item variable - переменная
		
		
		
	\end{itemize}
\end{multicols}