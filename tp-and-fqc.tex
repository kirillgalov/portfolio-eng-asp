\subsection{Article}

\subsubsection*{Abstract}

We present a new protocol for practical quantum cryptography, tailored for an implementation with weak coherent pulses. The key is obtained by a very simple time-of-arrival measurement on the data line; an interferometer is built on an additional monitoring line, allowing to monitor the presence of a spy (who would break coherence by her intervention). Against zero-error attacks (the analog of photon-number-splitting attacks), this protocol performs as well as standard protocols with strong reference pulses: the key rate decreases only as the transmission t of the quantum channel. We present also two attacks that introduce errors on the monitoring line: the intercept-resend, and a coherent attack on two subsequent pulses. Finally, we sketch several possible variations of this protocol.

\subsubsection{INTRODUCTION}

Quantum cryptography [1], or quantum key distribution (QKD), is probably the most mature field in quantum information, both in theoretical and in experimental advances. On the theoretical side, almost all QKD protocols have been proven to provide unconditional security in some regime; on the practical side, QKD has already reached the stage of commercial prototypes. Still,much work is needed. A big task consists in bringing both theory and applications in contact again: practical QKD systems do not fulfill all the requirements of unconditional security proofs (or, if you prefer, these proofs are still too abstract to cope with a practical system). Here, we address a different question: we aim for the most practical QKD system. Instead of looking for a new implementations of known protocols, we choose to start from scratch by inventing a new protocol. There are two basic requirement.

1. The protocol must be easily implementable, say with the smallest number of standard telecom devices. Note that this requirement, as a side benefit, may simplify security studies: we have learnt in the recent years that any optical component can be regarded as a ”Trojan horse” because of its imperfections [2].

2. The security of the system must be guaranteed by quantum physics, thence in some way quantum coherence must play a role.

The goal of this paper is to illustrate this program by presenting such a system. the key is created in a data line that is probably the simplest one can think of —just measure the time of arrival of weak pulses. The intervention of a spy is checked interferometrically in a monitoring line. In Section II, we define precisely the protocol and stress its advantages. In Section III, we present a quantitative study of security. Finally, Section 4 presents a number of possible variations on the main idea.


\subsubsection{THE PROTOCOL}

Alice uses a mode-locked laser, producing pulses of mean photon-number u that are separated by a fixed and well-defined time $\tau$ ; with a variable attenuator, she can blocks some of the pulses (note that a more economical source would just consist of a cw laser followed by the variable attenuator). Each logical bit is encoded in a two-pulse sequence according to the rule.

For instance, the eight-pulse sequence drawn in Fig. 1 codes for the four-bit string 0100 (read in temporal order, that is, from right to left). For small $\mu$, the states|0 A i and |1 A i have a large overlap because of their vacuum component. Since the laser is mode-locked, there is a phase coherence between any two non-empty pulses.

The pulses now propagate to Bob, on a quantum channel characterized by a transmission t = 10 -$\alpha$ d/10(a typical value for a in optical fibers is 0.2 dB/km).
Bob’s setup first splits the pulses using a non-equilibrated beam-splitter with transmission coefficient t B. The pulses that are transmitted are used to establish the raw key (data line). To obtain the bit value, Bob has to distinguish unambiguously between the two non-orthogonal states.

As well-known, unambiguous discrimination between two pure states can succeed with probability p ok = 1 -|h0 B |1 B i|; in the present case, the overlap is |h0 B |1 B i| = 2 e -|a| , and consequently p ok = 1 - e -u t t B . Now, there is an obvious way to achieve this result: photon counting with a perfect detector, because p ok is just the probability that the detector will detect something. The realistic situation where the detector has a finite efficiency n can be modelled by an additional beam-splitter with transmittivity n followed by a perfect detector; in this case, n appears in the exponent as well. In conclusion, the optimal unambiguous discrimination between |0 B i and |1 B i is achieved by the most elementary strategy, simply try to detect where the photons are. Later, Bob must announce Alice which items he has detected: this is how Alice and Bob establish their raw key. Note that no error is expected on this line, if the switch is perfect and in the absence of dark counts of the detector: a bit-flip is impossible because it would correspond to a photon jumping from a time-bin to another. Note that the simplicity of Bob’s data line has concrete practical advantages. There are no lossy and active elements. Hence, the transmission range can be increased and no random number generator is needed. As for the data line, our protocol is similar to the one of Debuisschert and Boucher [3]. However there, the security was obtained by the overlap in time between the pulses coding for different bits. Here, we use rather the monitoring line described in the next paragraph.

The pulses that are reflected at Bob’s beam-splitter go to an interferometer that is used for monitoring Eve’s presence (monitoring line). Here is where quantum coherence plays a role. Let a j be the amplitude of pulse j entering the interferometer: in particular, |a j | 2 is either 0 or u t (1 - t B ); and if both a j and a j+1 are non-zero,then a j+1 = a j . After the interferometer, the pulses that reach the detectors at time j + 1.


Now, if either a j or a j+1 are zero, then |D M1 | 2 =|D M2 | 2 = 12 u t (1 - t B ); i.e., conditioned to the fact that a photon takes the monitoring line, the probabilities of detecting it in either detector is 12 . However, if both a j and a j+1 are non-zero, then |D M1 | 2 = u t (1 - t B ) and|D M2 | 2 = 0: only detector D M1 can fire. Consider then again the case where bit number k is 1 and bit number k + 1 is 0: as we said above, in this case the two consecutive pulses 2k and 2k + 1 are non-empty. This means that, if coherence is not broken, detector D M2 cannot fire at time 2k + 1. If Eve happens to break the coherence by reading the channel, it could be detected this way.

Actually, it turns out that, as just described, the protocol is insecure: Eve can make a coherent measurement of the number of photons in the two pulses across the bit-separation. With such an attack, she would not break the coherence, thus introduce no errors in the monitoring line, and obtain almost full information (see next Section for more details). There are several ways of countering this attack: here, we make use of decoy sequences, inspired by the idea of ”decoy states” introduced by Hwang[4] and by Lo and co-workers [5], but different in its implementation. The principle is the following: with probability f , Alice leaves both the (2k - 1)-th and the 2k-the pulses non-empty. A decoy sequence does not encode a bit value (in contrast to the decoy states of [4,5] that still encode a state, but in a different way): thence, if the item is detected in the data line, it will be discarded in public discussion. However, if a detection takes place in the monitoring line at time 2k, then it must be in detector D 1M because of coherence. Now Eve can no longer pass unnoticed: if she attacks coherently across the bit separation, then she breaks the coherence of the decoy sequences; if she attacks coherently within each bit, then she breaks the coherence across the separation; finally, if she makes a coherent attack on a larger number of pulses,then she breaks the coherence in fewer positions but gets much less information.

Thus, errors are rare: they appear only in the monitoring line, and just for a fraction of the whole cases. Still,one can estimate the error (thence, the coherence of the channel) in a reasonable time, if the bit rate is high.

Should one say in one sentence where the improvement lies, here it is: one can define a very simple data line and protect it quantum-mechanically. At this point, two important remarks can be done. First, this protocol cannot be analyzed in terms of qubits. This is obvious, because any bits and coherence are checked on differently defined pairs. In particular, there is not a ”natural” single-photon version of the protocol (simply replace non-empty coherent state with one photon Fock states would be dramatic, since all the sequences would become orthogonal). The second remark is the answer to a possible question. With the idea of a simple data line for key creation, and a ”complementary” line for monitoring, one may implement a version of the BB84 protocol: Alice and Bob agree to produce the key using only the Z basis; sometimes Alice prepares one of the eigenstates of the X basis that acts as a decoy state. Which are the advantages of our protocol? We are going to see that our protocol is much more robust against attacks at zero errors (the analog of photon number-splitting attacks).

\subsubsection{QUANTITATIVE ANALYSIS OF SECURITY}

For a reasonable comparison with experiment, we must introduce the following parameters

1. The visibility V of the monitoring interferometer, whence the probability that D 2M fires in a time corresponding to a coherence is 1-V instead of zero. We suppose that Eve can take advantage of these imperfections: for instance, if the reduced visibility is due to fi 6 = 0 in the interferometer, Eve can systematically correct for this error by displacing the pulses, and then reproduce V by adding errors in a way that is profitable for her.

2. The imperfections of the three Bob’s detectors, supposed to be identical for simplicity: the quantum efficiency n and the probability per gate of a dark count p d . Typical values are n = 10\%, p d = 10 -5 . These imperfections are not given to Eve (see Section IV on the possibility that Eve forces a detection, thus effectively setting n = 1 for some pulses).

For simplicity in writing, we make all the quantitative analysis in the limit of small mean photon-number in Bob’s channel, that is ut << 1

First, we compute the parameters of Alice-Bob on the data line. Bob’s detection rate in D B , once decoy sequences are removed, is where T = t t B n. In other words, R B times the number of two-pulse sequences sent by Alice is the length of the raw key.
If we assume that the switch prepares really empty pulses when it is closed, the error expected in this line is only due to the dark counts of the detectors: Q=R because dark counts may make the detector fire at both times with equal probability. The mutual information Alice-Bob in bits per sent photon is thence. In what follows, we shall concentrate on attacks by Eve that do not modify Q. Before moving to that, let’s have a look at the monitoring line as well.

In the presence of dark counts and reduced visibility, the meaningful detection probabilities in D M1 and D M2 , neglecting double counts are the following [6]: R2= 1/V.

Contrary to the errors due to dark counts, the departure from perfect visibility will be entirely attributed to Eve. This is why we consider a priori different values V d and V 10 for the visibility in the two cases: as we shall see,Eve’s attacks may be different.

If Bob’s detector has dark counts, I(B : E) is smaller than I(A : E) for a prepare-and-measure scheme, becaus eeven if Eve knows perfectly what Alice has sent, she cannot know whether Bob has detected a photon or has had a dark count. Thus in our case, the Csiszar-Korner bound[7] that gives an estimate of the extractable secret key rate becomes.

Therefore, we have to compute the mutual information Bob-Eve.

The kind of attacks by Eve that we consider is sketched in Fig. 2. We can give Eve all the losses in the line, that is, we can suppose that Eve removes a fraction 1-t of the photons, and forwards the remaining fraction t to Bob on a lossless line. We are going to study:

1. An attack in which Eve can gain information without introducing errors. This is related to the losses on the line; it is the analog of the usual photon number-splitting attack [8,9], but is a different attack and less powerful.


2. Eve can immediately know if the previous attack was successful or not; in the case it wasn’t, we consider further the possibility of attacks that introduce errors in the monitoring line (but still no errors in the data line). Specifically, we study a usual intercept-resend strategy, and a more clever attack which is performed coherently on two subsequent pulses across the bit-separation.

In the case of BB84 and many other protocols, Eve can exploit multi-photon pulses in a lossy line to perform the photon-number-splitting attack [8,9]: she counts the photons in each pulse, and whenever this number is larger than one, she keeps one photon in a quantum memory and forwards the remaining photons to Bob on a lossless line. As such, this attack is not error-free in the present protocol: counting the photons in each pulse breaks the coherence between successive pulses, thus introducing errors in the monitoring line — actually, because of the peculiar encoding of the bits, this attack reduces here to the intercept-resend, see below.

More subtle is the analysis of a practical version of the attack using cascaded beam-splitters [10]: Eve uses a highly unbalanced BS, with transmission 1 - e and reflection e; if she has a detection, she forwards the remaining photons to Bob; otherwise, she begins anew, and so on until the losses that she introduces reach the transmission t of the quantum channel. The advantage of this strategy is that, in the presence of two or more photons, it is very rare that more than one photon is coupled into Eve’s detector. Indeed, this beam-splitting attack approximates a photon-counting. In our case, this strategy will introduce errors in the monitoring line as well: it does not modify the relative phase, but the relative intensity between subsequent pulses, thus leading to an unbalancing of the interferometer. The full analysis of such a strategy will be studied in a further work.

In summary, both the ideal photon-number-splitting and its approximate implementation through cascaded beam-splitters do not rank among the zero-error attacks against our protocol. In fact, Eve can only perform the basic beam-splitting attack: she removes a fraction 1 - t of the photons, and transmits the remaining fraction t to Bob on a lossless line. With the fraction that she has kept, the best thing Eve can do is just to measure them(recall the argument about optimal unambiguous state discrimination). This way, she detects u(1 - t) photons per pulse. When Eve has a detection in D E , she knows the bit that Alice has sent. Then she lets the remaining part of the pulse travel to Bob on the lossless line. Bob detects something exactly as if Eve had not been there.

In summary, Eve knows a fraction m(1 - t) of the key just because of the losses in the quantum channel: this fraction must always be subtracted in privacy amplification.


This is an important improvement over BB84: u opt is large and is basically constant with decreasing t (long quantum channels); as a consequence, the secret-key rate decreases only linearly (and not quadratically) with t. This is the same improvement that can be obtained by using decoy states [5] or a strong reference pulse [12];note however that the hardware is much simpler here.

When Eve’s detector D E does not fire, which happens with probability 1 - u(1 - t), Eve must perform some attack on the pulses flying to Bob if she wants to gainsome information. These attacks will certainly introduce errors, either in the data line or in the monitoring line. In the following, we present two such attacks (Fig. 3): a basic intercept-resend (I-R), and a photon-number-counting attack performed coherently on two subsequent pulses across the bit-separation (2c-PNC).

FIG. 3. Comparison of two attacks that introduce errors. In the I-R attack, Eve prepares a sequence of localized Fock states, thus breaking the coherence everywhere. In the 2c-PNC attack, Eve prepares a sequence of Fock states that are delocalized across the separation of bits: only the coherence of decoy sequences is broken. Note that arrows denote only coherence between subsequent pulses, the one checked by the interferometer; however, on the original sequence, all the non-empty pulses are coherent with one another, while in the sequences after Eve’s attack only the indicated coherence remains.

Let’s begin with the intercept-resend (I-R) strategy. Eve simply detects the pulse flying to Bob: her detector will fire with probability $\mu$ t, and in this case she prepares a single-photon in the good time-bin and forwards it to Bob. Obviously, both R B and Q are unchanged under this strategy.

Note that R B will be the sum of three terms: Eve has detected and Bob detects too; Eve has detected and Bob has a dark count; Eve has not detected and Bob has a dark count. Now, Eve can distinguish the last one from the two first, and she knows that in the last case she has no information on Alice’s and Bob’s bits.

In the absence of decoy sequences, Eve may obtain information without introducing errors in the monitoring line, by counting the number of photons coherently between two pulses, not within each bit but across the separation line (see Fig. 3). This attack does not break the phase between these pulses. Of course, if Eve finds n > 0 photons, on the spot she does not know to which bit the photon belongs; but she will learn it later, by listening to Bob’s list of accepted bits [13]. Actually, in some very rare case Eve still does not get any information: if Eve prepares n > 0 photons in two successive two-pulse sequences, and Bob accepts a detection in the bit common to both sequences, Eve has no idea of his result. However, since such cases are rare, we make the conservative assumption that Eve always gets full information.

Note in particular that, if Alice and Bob find V d = V 10 ,they can conclude that Eve has not used the 2c-PNC attack — by the way, this is why we presented the analysis of the I-R strategy, obviously worse than 2c-PNC from Eve’s standpoint: in a practical experiment, V d = V 10 is very likely to hold (after all, Eve is not there...). Therefore, formulae for the I-R attack will be useful in the analysis of experimental data.

\subsubsection{VARIATIONS AND OPEN QUESTIONS}

Here are a few ideas of variations in the protocol, that may have some additional benefit and require further study:

1. Alice may change during the protocol the definition of the pulses that define a bit. If there i sa convenient fraction of decoy sequences, Eve has no way of distinguishing a priori which pairs of successive pulses encode a bit. This way, the effect of the 2c-PNC attack becomes equally distributed among decoy and ”1,0” sequences, i.e. V d = V 10 = 1 - p IR - 12 p 2c . Moreover, whenever Eve attacks the bit instead of attacking across the bit-separation, she cannot gain any information.

2. In fact, nothing forces to define bits by subsequent pulses: Alice and Bob can decide later, adding a sort of ”sifting” phase to the protocol. This means that Alice can now send whatever pulse sequence,she is no longer restricted to those that define a bit or a decoy sequence. It is not clear whether this modification helps, if the hardware is kept unchanged: Alice and Bob still check only the coherence between subsequent pulses; moreover, sifting means additional losses and additional information revealed publicly.

3. Instead of introducing decoy sequences as we did above, one may study the effect of decoy pulses with different intensities, as proposed by Hwang and by Lo and co-workers to protect the BB84 protocol. against the photon-number-splitting attack [4,5].

The difficulty in assessing the security of practical QKD, is the huge number of imperfections that may hide loopholes for security. These imperfections exist in all implementations and for all protocols, but their effect and the corresponding protection may vary. Here we present some of these.

1.About Trojan-horse and similar realistic attacks [2]:Alice’s setup must be protected against Trojan horse attacks, with the suitable filters, isolators etc. In Bob’s setup nothing is variable; however,one must prevent the possible light emission from avalanche photodiodes to become available to Eve:if the firing of a detector can be seen from outside Bob, the protocol becomes immediately insecure.

2. After a detection, Bob’s detectors are blind during some time. In particular, if Eve happens to know when Bob’s detector D M2 has fired, she can attack strongly the subsequent pulses because no error will be detected then, and gain one bit (just one, because when D B has fired, then it has a dead time as well). For the security of the protocol, Eve must have no way of assessing the firing of a detector, and Bob must announce publicly this information only after the detector is ready again. This may imply some suitable synchronization in the software,or more simply, to shut D B as long as D M2 has not recovered. The nuisance depends of course on the ratio between the raw bit rate and the dead time.


3. In all this paper, we have considered only the case where Eve does not change Bob’s detection rates in the data line and in the monitoring line. By sending out stronger pulses, Eve might force the detection of those items on which she has full information; but in turn, she would increase the rate of double counts among Bob’s detectors. This effect must be quantified, and the number of double,or even triple, counts must be monitored during the experiment.

\subsubsection{CONCLUSION}

In conclusion, we have presented a new protocol for quantum cryptography whose realization is much simpler than that of previously described ones. Specifically,Bob’s station is such that losses are minimized and no dynamical component is needed.

We thank H.-K. Lo for stimulating comments. We acknowledge financial support from idQuantique and from the Swiss NCCR ”Quantum photonics".

\subsection{\trnas}

\subsubsection*{Аннотация}

Мы представляем новый протокол для практической квантовой криптографии, адаптированный для реализации со слабыми когерентными импульсами. Ключ получается очень простым измерением времени прихода на линии данных; на дополнительной линии контроля построен интерферометр, позволяющий отслеживать присутствие шпиона (который своим вмешательством нарушит когерентность). Против атак с нулевой ошибкой (аналог атак с расщеплением фотонного числа) этот протокол работает так же хорошо, как и стандартные протоколы с сильными опорными импульсами: ключевая скорость уменьшается только с ростом t передачи квантового канала. Мы также представляем две атаки, которые вносят ошибки на линии контроля: перехват-передача и когерентная атака на два последующих импульса. Наконец, мы предложили несколько возможных вариаций этого протокола.

\subsubsection{Введение}

Квантовая криптография [1], или квантовое распределение ключей (КРК), вероятно, является наиболее изученной областью в квантовой информации, как в теоретическом, так и в экспериментальном плане. С теоретической точки зрения, почти все протоколы QKD были доказаны для обеспечения безусловной безопасности в некотором режиме; с практической точки зрения, QKD уже достигли стадии коммерческих прототипов. Тем не менее, предстоит еще много работы. Большая задача состоит в том, чтобы снова свести теорию и практику: реальные системы QKD не удовлетворяют всем требованиям доказательств безусловной безопасности (или, если хотите, эти доказательства все еще слишком абстрактны, чтобы справиться с практической системой). Здесь мы решаем другой вопрос: мы стремимся найти наиболее практичную систему QKD. Вместо того чтобы искать новые реализации известных протоколов, мы решили начать с нуля, придумав новый протокол. Есть два основных требования.

1. Протокол должен быть легко реализуемым, например, с использованием наименьшего числа стандартных телекоммуникационных устройств. Отметим, что это требование, как побочное преимущество, может упростить исследования безопасности: в последние годы мы узнали, что любой оптический компонент может рассматриваться как "троянский конь" из-за его несовершенства [2].

2. Безопасность системы должна быть гарантирована квантовой физикой, следовательно, каким-то образом квантовая когерентность должна играть роль.

Цель данной статьи - проиллюстрировать эту программу, представив такую систему. Ключ создается в линии данных, которая, вероятно, является самой простой, какую только можно придумать - просто измерьте время прихода слабых импульсов. Вмешательство шпиона проверяется интерферометрически в линии мониторинга. В разделе II мы даем точное определение протокола и подчеркиваем его преимущества. В разделе III мы представляем количественное исследование безопасности. Наконец, в разделе 4 представлен ряд возможных вариаций основной идеи.


\subsubsection{Протокол}

Алиса использует лазер с модовой блокировкой, производящий импульсы со средним числом фотонов u, которые разделены фиксированным и четко определенным временем $\tau$; с помощью переменного аттенюатора она может блокировать некоторые из импульсов (заметим, что более экономичный источник состоял бы из коротковолнового лазера и переменного аттенюатора). Каждый логический бит кодируется в последовательности из двух импульсов в соответствии с правилом.

Например, последовательность из восьми импульсов, показанная на рис. 1, кодирует четырех-битовую строку 0100 (читается во временном порядке, то есть справа налево). Для малых $\mu$ состояния|0 A i и |1 A i имеют большое перекрытие из-за их вакуумной составляющей. Поскольку лазер имеет модовую синхронизацию, между любыми двумя непустыми импульсами существует фазовая когерентность.

Теперь импульсы распространяются к Бобу по квантовому каналу, характеризующемуся передачей t = 10 -$\alpha$ d/10 (типичное значение a в оптических волокнах составляет 0,2 дБ/км).
Установка Боба сначала разделяет импульсы с помощью неэквивалентного сплиттера луча с коэффициентом передачи t B. Передаваемые импульсы используются для создания необработанного ключа (линии данных). Чтобы получить значение бита, Боб должен однозначно различить два неортогональных состояния.

Как известно, однозначное различение двух чистых состояний может быть успешным с вероятностью p ok = 1 -|h0 B |1 B i|; в данном случае перекрытие равно |h0 B |1 B i| = 2 e -|a| , и, следовательно, p ok = 1 - e -u t t B . Теперь существует очевидный способ достижения этого результата: подсчет фотонов с идеальным детектором, поскольку p ok - это просто вероятность того, что детектор что-то обнаружит. Реалистичная ситуация, когда детектор имеет конечную эффективность n, может быть смоделирована дополнительным светоделителем с пропусканием n, за которым следует совершенный детектор; в этом случае n также появляется в экспоненте. В заключение, оптимальная однозначная дискриминация между |0 B i и |1 B i достигается с помощью самой элементарной стратегии: просто попытаться обнаружить, где находятся фотоны. Позже Боб должен сообщить Алисе, какие объекты он обнаружил: так Алиса и Боб устанавливают свой необработанный ключ. Обратите внимание, что на этой линии не ожидается никакой ошибки, если коммутатор совершенен и в отсутствие отсчетов детектора: перестановка битов невозможна, поскольку она соответствовала бы переходу фотона из временного состояния в другое. Отметим, что простота линии данных Боба имеет конкретные практические преимущества. В ней нет активных элементов с потерями. Следовательно, дальность передачи может быть увеличена, и нет необходимости в генераторе случайных чисел. Что касается линии передачи данных, то наш протокол похож на протокол Дебуисшерта и Буше [3]. Однако там безопасность достигалась за счет времени импульсов, кодирующих разные биты. Здесь мы используем скорее линию контроля, описанную в следующем параграфе.

Импульсы, которые отражаются от луча-расщепителя Боба, поступают на интерферометр, который используется для контроля присутствия Евы (линия контроля). Здесь квантовая когерентность играет свою роль. Пусть a j - амплитуда импульса j, поступающего в интерферометр: в частности, |a j | 2 либо 0, либо u t (1 - t B ); и если a j и a j+1 ненулевые, то a j+1 = a j . После интерферометра импульсы, которые достигают детекторов в момент времени j + 1.


Теперь, если a j или a j+1 равны нулю, то |D M1 | 2 =|D M2 | 2 = 12 u t (1 - t B ); т.е. при условии, что фотон проходит линию наблюдения, вероятность его обнаружения в любом детекторе равна 12. Однако, если a j и a j+1 ненулевые, то |D M1 | 2 = u t (1 - t B ) и|D M2 | 2 = 0: сработать может только детектор D M1. Рассмотрим еще раз случай, когда номер бита k равен 1, а номер бита k + 1 равен 0: как мы говорили выше, в этом случае два последовательных импульса 2k и 2k + 1 непустые. Это означает, что, если когерентность не нарушена, детектор D M2 не может сработать в момент времени 2k + 1. Если Ева нарушит когерентность, прочитав канал, это можно будет обнаружить таким образом.

На самом деле, оказывается, что, как только что было описано, протокол небезопасен: Ева может провести когерентное измерение количества фотонов в двух импульсах через разделение битов. При такой атаке она не нарушит когерентность, не внесет ошибок в линию мониторинга и получит почти полную информацию (подробнее см. следующий раздел). Существует несколько способов противостоять этой атаке: здесь мы используем ложные последовательности, вдохновленные идеей "ложных состояний", представленной Хвангом [4] и Ло и соавторами [5], но отличающейся своей реализацией. Принцип заключается в следующем: с вероятностью f Алиса оставляет непустыми как (2k - 1)-й, так и 2k-ый импульсы. Ложная последовательность не кодирует битовое значение (в отличие от ложных состояний из [4,5], которые все же кодируют состояние, но другим способом): следовательно, если элемент будет обнаружен в линии данных, он будет отброшен в публичном обсуждении. Однако если обнаружение происходит в линии контроля в момент времени 2k, то из-за когерентности оно должно быть в детекторе D 1M. Теперь Ева уже не может пройти незамеченной: если она атакует когерентно через разделение битов, то она нарушает когерентность последовательностей-приманок; если она атакует когерентно в пределах каждого бита, то она нарушает когерентность через разделение; наконец, если она проводит когерентную атаку на большее число импульсов, то она нарушает когерентность в меньшем количестве позиций, но получает гораздо меньше информации.

Таким образом, ошибки редки: они появляются только в линии мониторинга и лишь для части всех случаев. Тем не менее, можно оценить ошибку (а значит, и когерентность канала) за разумное время, если скорость передачи данных высока.

Если сказать одним предложением, в чем заключается улучшение, то вот оно: можно определить очень простую линию данных и защитить ее квантово-механически. На этом этапе можно сделать два важных замечания. Во-первых, этот протокол нельзя анализировать в терминах кубитов. Это очевидно, поскольку любые биты и когерентность проверяются на различных определенных парах. В частности, не существует "естественной" однофотонной версии протокола (простая замена непустого когерентного состояния на однофотонные фоковские состояния была бы драматичной, поскольку все последовательности стали бы ортогональными). Второе замечание является ответом на возможный вопрос. Используя идею простой линии передачи данных для создания ключа и "дополнительной" линии для мониторинга, можно реализовать версию протокола BB84: Алиса и Боб договариваются о создании ключа, используя только базис Z; иногда Алиса подготавливает одно из собственных состояний базиса X, которое действует как ложное состояние. Каковы преимущества нашего протокола? Мы увидим, что наш протокол гораздо более устойчив к атакам при нулевой ошибке (аналог атак с расщеплением числа фотонов).

\subsubsection{Анализ безопасности}

Для разумного сравнения с экспериментом мы должны ввести следующие параметры

1. Видимость V интерферометра мониторинга, поэтому вероятность того, что D 2M сработает за время, соответствующее когерентности, равна 1-V, а не нулю. Мы предполагаем, что Ева может воспользоваться этими несовершенствами: например, если снижение видимости связано с fi 6 = 0 в интерферометре, Ева может систематически исправлять эту ошибку путем смещения импульсов, а затем воспроизводить V путем добавления ошибок выгодным для нее способом.

2. Несовершенства трех детекторов Боба, которые для простоты предполагаются одинаковыми: квантовая эффективность n и вероятность на ворота темного счета p d . Типичные значения: n = 10\%, p d = 10 -5. Эти несовершенства не передаются Еве (см. раздел IV о возможности того, что Ева принудительно производит обнаружение, тем самым эффективно устанавливая n = 1 для некоторых импульсов).

Для простоты изложения мы проводим весь количественный анализ в пределе малого среднего числа фотонов в канале Боба, то есть ut << 1

Сначала мы вычисляем параметры Алисы-Боба на линии данных. Скорость обнаружения Боба в D B , после удаления последовательностей-приманок, равна где T = t t B n. Другими словами, R B , умноженное на количество двухимпульсных последовательностей, посланных Алисой, равно длине необработанного ключа.
Если мы предположим, что переключатель готовит действительно пустые импульсы, когда он закрыт, то ошибка, ожидаемая в этой строке, обусловлена только отсчетами детекторов: Q=R, поскольку отсчеты могут заставить детектор сработать в оба момента времени с равной вероятностью. Взаимная информация Алиса-Боб в битах на каждый переданный фотон равна тогда СЕ. Далее мы сосредоточимся на атаках Евы, которые не изменяют Q. Прежде чем перейти к этому, давайте посмотрим и на линию мониторинга.

При наличии темных отсчетов и сниженной видимости значимые вероятности обнаружения в D M1 и D M2, пренебрегая двойными отсчетами, следующие [6]: R2= 1/V.

В отличие от ошибок, связанных с теневыми отсчетами, отклонение от идеальной видимости будет полностью приписываться Еве. Именно поэтому мы априори считаем разные значения V d и V 10 для видимости в двух случаях: как мы увидим, атаки Евы могут быть разными.

Если детектор Боба имеет теневые отсчеты, I(B : E) меньше, чем I(A : E) для схемы подготовки и измерения, потому что даже если Ева прекрасно знает, что послала Алиса, она не может знать, обнаружил ли Боб фотон или имел теневые отсчеты. Таким образом, в нашем случае ограничение Циссара-Корнера[7], которое дает оценку скорости извлекаемого секретного ключа, становится меньше.

Поэтому мы должны вычислить взаимную информацию Боб-Ева.

Вид атак Евы, который мы рассматриваем, показан на рис. 2. Мы можем предоставить Еве все потери в линии, то есть предположить, что Ева удаляет долю 1-t фотонов, а оставшуюся долю t пересылает Бобу по линии без потерь. Мы собираемся изучить:

1. Атака, при которой Ева может получить информацию без внесения ошибок. Это связано с потерями на линии; это аналог обычной атаки с расщеплением числа фотонов [8,9], но это другая атака и менее мощная.


2. Ева может сразу узнать, была ли предыдущая атака успешной или нет; в случае, если нет, мы рассматриваем далее возможность атак, которые вносят ошибки в линию контроля (но при этом не вносят ошибок в линию данных). В частности, мы изучаем обычную стратегию перехвата-передачи и более хитрую атаку, которая выполняется когерентно на двух последующих импульсах через разделение битов.

В случае BB84 и многих других протоколов, Ева может использовать многофотонные импульсы в линии с потерями для выполнения атаки с разделением числа фотонов [8,9]: она считает фотоны в каждом импульсе, и всякий раз, когда это число больше единицы, она сохраняет один фотон в квантовой памяти и пересылает остальные фотоны Бобу по линии без потерь. Как таковая, эта атака не является безошибочной в данном протоколе: подсчет фотонов в каждом импульсе нарушает когерентность между последовательными импульсами, тем самым внося ошибки в линию мониторинга - фактически, из-за особенностей кодирования битов, эта атака сводится здесь к перехвату-передаче, см. ниже.

Более тонким является анализ практической версии атаки с использованием каскадных рассеивателей луча [10]: Ева использует сильно несбалансированную БС, с передачей 1 - e и отражением e; если у нее есть обнаружение, она пересылает оставшиеся фотоны Бобу; в противном случае она начинает заново, и так далее, пока потери, которые она вводит, не достигнут передачи t квантового канала. Преимущество этой стратегии заключается в том, что при наличии двух или более фотонов очень редко более одного фотона попадает в детектор Евы. Действительно, эта атака с расщеплением луча приближается к подсчету фотонов. В нашем случае эта стратегия внесет ошибки и в линию мониторинга: она изменяет не относительную фазу, а относительную интенсивность между последующими импульсами, что приводит к разбалансировке интерферометра. Полный анализ такой стратегии будет изучен в дальнейшей работе.

В итоге, как идеальное расщепление числа фотонов, так и его приблизительная реализация с помощью каскадных расщепителей лучей не входят в число атак с нулевой ошибкой против нашего протокола. Фактически, Ева может выполнить только базовую атаку с расщеплением луча: она удаляет часть 1 - t фотонов, а оставшуюся часть t передает Бобу по линии без потерь. С той долей, которую она сохранила, лучшее, что Ева может сделать, это просто измерить их (вспомните аргумент об оптимальной однозначной дискриминации состояний). Таким образом, она обнаруживает u(1 - t) фотонов за импульс. Когда Ева обнаруживает фотон в D E , она знает бит, который послала Алиса. Затем она пропускает оставшуюся часть импульса к Бобу по линии без потерь. Боб обнаруживает что-то точно так же, как если бы Евы там не было.

В итоге, Ева знает часть m(1 - t) ключа только из-за потерь в квантовом канале: эта часть всегда должна быть вычтена при усилении секретности.


Это важное улучшение по сравнению с BB84: U велико и в основном постоянно при уменьшении t (длинные квантовые каналы); как следствие, скорость секретного ключа уменьшается только линейно (а не квадратично) с t. Это такое же улучшение, которое может быть получено при использовании ложных состояний [5] или сильного опорного импульса [12]; отметим, однако, что аппаратура здесь намного проще.

Когда детектор Евы D E не срабатывает, что происходит с вероятностью 1 - u(1 - t), Ева должна провести некоторую атаку на импульсы, летящие к Бобу, если она хочет получить информацию. Эти атаки обязательно внесут ошибки либо в линию данных, либо в линию контроля. Далее мы представим две такие атаки (рис. 3): базовый перехват-передача (I-R) и атака с подсчетом числа фотонов, выполняемая когерентно на двух последующих импульсах через битовую развязку (2c-PNC).

РИС. 3. Сравнение двух атак, вносящих ошибки. В атаке I-R Ева подготавливает последовательность локализованных состояний Фока, нарушая таким образом когерентность повсюду. В атаке 2c-PNC Ева подготавливает последовательность состояний Фока, которые являются делокализованными через разделение битов: нарушается только когерентность последовательностей обманок. Обратите внимание, что стрелки обозначают только когерентность между последующими импульсами, проверяемую интерферометром; однако в исходной последовательности все непустые импульсы когерентны друг с другом, тогда как в последовательностях после атаки Евы остается только указанная когерентность.

Начнем со стратегии перехвата-передачи (I-R). Ева просто обнаруживает импульс, летящий к Бобу: ее детектор срабатывает с вероятностью $\mu$ t, и в этом случае она готовит однофотонный импульс в хорошем временном отрезке и пересылает его Бобу. Очевидно, что при такой стратегии R B и Q остаются неизменными.

Обратите внимание, что R B будет суммой трех членов: Ева обнаружила и Боб тоже обнаружил; Ева обнаружила и Боб имеет теневой счет; Ева не обнаружила и Боб имеет теневой счет. Теперь Ева может отличить последнее от двух первых, и она знает, что в последнем случае у нее нет информации о битах Алисы и Боба.

В отсутствие последовательностей-приманок Ева может получить информацию, не внося ошибок в линию контроля, путем подсчета количества фотонов, когерентных между двумя импульсами, не в пределах каждого бита, а по всей линии разделения (см. рис. 3). Эта атака не нарушает фазу между этими импульсами. Конечно, если Ева обнаружит n > 0 фотонов, то на месте она не будет знать, к какому биту принадлежит фотон; но она узнает это позже, прослушав список принятых битов Боба [13]. На самом деле, в некоторых очень редких случаях Ева все равно не получит никакой информации: если Ева подготовит n > 0 фотонов в двух последовательных двухимпульсных последовательностях, а Боб примет обнаружение в бите, общем для обеих последовательностей, Ева не будет иметь представления о его результате. Однако, поскольку такие случаи редки, мы делаем консервативное предположение, что Ева всегда получает полную информацию.

Заметим, в частности, что если Алиса и Боб найдут V d = V 10, то они могут сделать вывод, что Ева не использовала атаку 2c-PNC - кстати, именно поэтому мы представили анализ стратегии I-R, очевидно худшей, чем 2c-PNC с точки зрения Евы: в практическом эксперименте V d = V 10 имеет большую вероятность (ведь Евы там нет). Поэтому формулы для атаки I-R будут полезны при анализе экспериментальных данных.

\subsubsection{ВАРИАЦИИ И ОТКРЫТЫЕ ВОПРОСЫ}

Вот несколько идей вариаций протокола, которые могут принести дополнительную пользу и требуют дальнейшего изучения:

1. Алиса может изменить в ходе протокола определение импульсов, определяющих бит. Если существует i удобная доля ложных последовательностей, Ева не имеет возможности априори отличить, какие пары последовательных импульсов кодируют бит. Таким образом, эффект от атаки 2c-PNC равномерно распределяется между приманками и последовательностями "1,0", т.е. V d = V 10 = 1 - p IR - 12 p 2c . Более того, когда Ева атакует бит вместо того, чтобы атаковать через разделение битов, она не может получить никакой информации.

2. На самом деле, ничто не заставляет определять биты по последующим импульсам: Алиса и Боб могут решить это позже, добавив в протокол своего рода фазу "просеивания". Это означает, что Алиса теперь может посылать любую последовательность импульсов, она больше не ограничена теми, которые определяют бит или последовательность обманки. Неясно, помогает ли эта модификация, если аппаратное обеспечение остается неизменным: Алиса и Боб по-прежнему проверяют только когерентность между последующими импульсами; кроме того, просеивание означает дополнительные потери и дополнительную информацию, раскрываемую публично.

3. Вместо введения ложных последовательностей, как мы сделали выше, можно изучить эффект ложных импульсов с различной интенсивностью, как это было предложено Хвангом и Ло с соавторами для защиты протокола BB84 от атаки с расщеплением фотонного числа [4,5].

Сложность в оценке безопасности практического QKD заключается в огромном количестве несовершенств, которые могут скрывать лазейки для безопасности. Эти недостатки существуют во всех реализациях и для всех протоколов, но их влияние и соответствующая защита могут быть разными. Здесь мы представляем некоторые из них.

1.О троянском коне и подобных реалистичных атаках [2]:Установка Алисы должна быть защищена от атак троянского коня, с помощью соответствующих фильтров, изоляторов и т.д. В установке Боба ничего не меняется; однако необходимо предотвратить возможное излучение света от лавинных фотодиодов, чтобы оно стало доступно Еве: если срабатывание детектора можно увидеть извне Боба, протокол немедленно становится небезопасным.

2. После обнаружения детекторы Боба остаются слепыми в течение некоторого времени. В частности, если Ева случайно узнает, когда сработал детектор Боба D M2, она может атаковать сильно последующие импульсы, потому что тогда ошибка не будет обнаружена, и выиграть один бит. Для безопасности протокола Ева не должна иметь возможности оценить срабатывание детектора, а Боб должен публично объявить эту информацию только после того, как детектор снова будет готов. Это может подразумевать некоторую подходящую синхронизацию в программном обеспечении, или, более просто, отключение D B до тех пор, пока D M2 не восстановится. Конечно, неприятность зависит от соотношения между необработанной скоростью передачи данных и мертвым временем.


3. Во всей этой работе мы рассматривали только случай, когда Ева не изменяет скорости обнаружения Боба в линии данных и в линии контроля. Посылая более сильные импульсы, Ева может заставить обнаружить те элементы, о которых у нее есть полная информация; но в свою очередь, она увеличит количество двойных счетов среди детекторов Боба. Этот эффект должен быть оценен количественно, и количество двойных, или даже тройных, отсчетов должно отслеживаться в ходе эксперимента.

\subsubsection{Заключение}

В заключение, мы представили новый протокол для квантовой криптографии, реализация которого намного проще, чем у ранее описанных. В частности, станция Боба такова, что потери сводятся к минимуму, и нет необходимости в динамическом компоненте.

Мы благодарим H.-K. Lo за стимулирующие комментарии. Мы признательны за финансовую поддержку от компании idQuantique и швейцарского NCCR "Квантовая фотоника".

\subsection{\review}

The authors in [3] presented a new protocol for practical quantum cryptography, adapted for implementation with weak coherent pulses. This protocol has significant differences from previous ones.

The key is obtained by measuring the arrival time of the data. By means of an interferometer on an additional control line, it is possible to trace the presence of a spy who would interfere with the coherence of the photons. Against attacks with zero error (analogue of photon number splitting attacks) this protocol works as well as the standard protocols.

The paper presents two attacks that introduce errors on the line: the intercept-send attack and the coherent attack on two subsequent pulses.

The authors have also provided several possible practical implementations of this protocol.


\subsection{\dic}
\begin{multicols}{2}
	\begin{itemize}
		
		\item algorithms - алгоритм
		\item analysis - анализ
		
		\item appropriate - подходящий
		\item approximately - примерно
		
		
		\item basis - основа
		\item beam - луч
		
		\item binary - двоичных
		\item bit - бит
		
		\item capacity - вместимость
		
		\item channel - канал
		
		\item coherent - связный
		\item combination - комбинации
		
		\item communication - связь
		\item compare - сравнить
		
		\item computation - вычисления
		\item computers - компьютеров
		
		\item condition - условие
		\item conjugate - спряжение
		\item considered - рассмотрено
		\item contain - содержат
		
		\item correlation - корреляция
		
		\item cryptography - криптография
		
		\item decode - декодировать
		\item decoy - ловушка
		\item density - плотность
		
		\item dependence - зависимость
		\item detect - обнаружить
		
		\item deterministic - детерминированный
		
		\item developing - разработка
		
		\item difference - разница
		
		\item differentiate - дифференцировать
		
		\item distribution - распределение
		
		\item eavesdropper - подслушиватель
		
		\item encoded - закодировано
		\item entaglement - запутанность
		\item equivalently - эквивалентно
		
		\item imperfections - недостаток
		\item implementation - реализация
		
		\item instances - экземпляров
		
		\item intensity - интенсивность
		\item intercept - перехват
		
		\item interferometer - интерферометр
		
		\item limitation - ограничение
		
		\item lossless - без потерь
		\item lossy - потери
		
		\item malicious - вредоносных
		\item managed - управляемых
		
		\item mathematical - математических
		\item matrix - матрица
		
		\item measurement - измерения
		
		\item method - метод
		\item mimic - имитировать
		
		\item modification - модификация
		
		\item neglected - пренебрегают
		
		\item normalized - нормализовано
		
		\item operator - оператор
		\item optical - оптический
		\item optimal - оптимальный
		
		\item optional - опционально
		
		\item orthogonal - ортогональный
		\item orthogonality - ортогональность
		\item orthonormal - ортонормированный
		
		\item phase - фаза
		\item photon - фотон
		
		\item polarization - поляризация
		
		
		\item probabilistic - вероятностный
		\item probability - вероятность
		\item problem - проблема
		
		\item projecting - проектирование
		
		\item property - свойство
		\item proportion - пропорция
		\item proposal - предложение
		
		\item protocol - протокол
		
		\item prove - доказательство
		
		\item provide - обеспечить
		\item provides - обеспечивает
		
		\item public - общедоступных
		\item pulse - импульс
		
		\item quantum - квантовый
		\item random - случайных
		\item randomize - рандомизировать
		
		\item signature - подпись
		
		\item strategy - стратегия
		
		\item string - строка
		
		\item symmetric - симметричный
		
		\item theoretical - теоретический
		\item theory - теория
		
		\item transmission - передача
		
		\item vacuum - вакуум
		\item value - значение
		
		\item variable - переменная
		
		
		
	\end{itemize}
\end{multicols}