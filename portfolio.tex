%\documentclass[a4pape, 14pt]{extarticle} % расширенный класс статьи с возможностью указать 14 шрифт
\documentclass[a4pape, 11pt]{extarticle}

\usepackage{graphicx}  % поддержка .eps-графики
\usepackage[utf8]{inputenc} % кодировка в которой набран текст 
\usepackage[T2C]{fontenc} % поддержка кирилицы % ещё можно T1 T2B T2A T2
\usepackage[english,russian]{babel} % переносы и типографские правила для русского
\usepackage{indentfirst}  % красная строка
\usepackage{listings}  % оформление листингов программ
\usepackage{amssymb,amsfonts,amsmath,mathtext} % ядро для научной статьи
\usepackage{physics}
\usepackage{mathtools}
\usepackage{braket}
\usepackage{float} % для рисунков
\usepackage{array} % для m{2cm} в таблицах
\usepackage{blindtext} % случайный текст для заглушки

\usepackage{pdfpages} % для вставки пдф файлов

\usepackage{pgfplots} % построение графиков
% We will externalize the figures
%\usepgfplotslibrary{external}
%\tikzexternalize
%\pgfplotsset{width=15cm,compat=1.5}


\usepackage[left=2cm, right=2cm, top=1.5cm, bottom=1.5cm]{geometry} % установка полей

\newcommand{\hup}{Принцип неопределенности Гейзенберга}
\newcommand{\qe}{Квантовая запутанность}

%\usepackage{citehack}  % https://www.opennet.ru/docs/RUS/cyr_howto/ch08s02.html
%\usepackage[bibencoding=auto,backend=biber,babel=other]{biblatex}
%\usepackage{csquotes} 


% https://ru.overleaf.com/learn/latex/Pgfplots_package  - для построения графиков
% https://ru.overleaf.com/learn/latex/Tables - построение таблиц
% https://ru.overleaf.com/learn/latex/Commands - обьявление коман
\graphicspath{{./images/}} %путь к рисункам
\frenchspacing % длина пробелов после пунктуации
\pagestyle{plain}
\selectlanguage{russian}
\bibliographystyle{gost780} % gost780 - сортировка по порядку следования, gost780s - по алфавиту кажется. gost780u не завелся

\newcommand{\trnas}{Translation}
\newcommand{\dic}{Dictionary}
\newcommand{\review}{Review}

\begin{document}

\begin{titlepage}
\centerline{\includegraphics[width=2cm]{dstu-logo}}
\vfill
\Large
\centerline{МИНЕСТЕРСТВО НАУКИ И ВЫСШЕГО ОБРАЗОВАНИЯ}
\centerline{РОССИЙСКОЙ ФЕДЕРАЦИИ}
\vfill
\centerline{\bf ФЕДЕРАЛЬНОЕ ГОСУДАРСТВЕННОЕ БЮДЖЕТНОЕ}
\centerline{\bf ОБРАЗОВАТЕЛЬНОЕ УЧРЕЖДЕНИЕ ВЫСШЕГО ОБРАЗОВАНИЯ}
\centerline{\bf «ДОНСКОЙ ГОСУДАРСТВЕННЫЙ ТЕХНИЧЕСКИЙ УНИВЕРСИТЕТ»}
\centerline{\bf (ДГТУ)}
\normalsize
\vfill\vfill
\leftline{Факультет «Отдел Аспирантуры и Докторантуры»}
\leftline{Кафедра «Кибербезопасность информационных систем»}
\vfill
%\rightline{Место для подписи завкафедрой}
\vfill
\vfill
\Large{\centerline{\bf ПОРТФОЛИО АСПИРАНТА}}
\centerline{\bf По дисциплине «Иностранный язык»}
\vfill
\leftline{Автор работы: \underline{\hspace{2cm}} \hspace{0.2cm} \underline{\hspace{2cm}}  \hspace{0.2cm} К.А. Галов}
\leftline{\hspace{3.8cm} \small{(дата)} \hspace{1cm} \small{(подпись)}}
\vfill

\leftline{Направление 09.06.01 «Информатика и вычислительная техника»}
\leftline{Профиль «Информационные системы и процессы»}
\leftline{Руководитель работы \underline{\hspace{4cm}} д.ф.н., профессор Т.Б. Астен}
\leftline{\hspace{5.7cm} \small{(подпись,дата)}}
\vfill
\leftline{Работа защищена \underline{\hspace{2cm}} \hfill \underline{\hspace{4cm}} \hfill \underline{\hspace{4cm}}}
\leftline{\hspace{4.4cm} \small{(дата)} \hspace{3.3cm} \small{(оценка)} \hspace{4cm} \small{(подпись)}}
\vfill
\vfill
%\begin{flushright}
%Научный руководитель:\\
%д.ф.-м.н. Черкесова Л. В.
%\end{flushright}
\vfill
\vfill
\centerline{г. Ростов-на-Дону}
\centerline{2021 г.}

\end{titlepage}
\setcounter{page}{2}
\tableofcontents
\clearpage
\section{Quantum cryptography with coherent states}
\subsubsection*{Abstract}
The safety of a quantum key distribution system relies on the fact that any eavesdropping attempt on the quantum channel creates errors in the transmission. For a given error rate, the amount of information that may have leaked to the eavesdropper depends on both the particular system and the eavesdropping strategy. In this work, we discuss quantum cryptographic protocols based on the transmission of weak coherent states and present anew system, based on a symbiosis of two existing ones, and for which the information available to the eavesdropper is significantly reduced. This system is therefore safer than the two previous ones. We also suggest a poss experimental implementation.

\subsubsection*{Introduction}
The only known method to exchange secret information through a communication channel in a proven secure way, is to use the so-called one-time pad (for a good review of both classical and quantum cryptography, see [1]). In this technique, the data, which is represented by a string of bits, is combined with a random string of bits of equal length called the key, and is then sent through the communication channel. The randomness of the key ensures that the encoded message is also completely random and as such totally unintelligible to a potential eavesdropper. The safety of the transmission is thus entirely dependent on the safety of the key, which has to be secret and shared just by both legitimate users. Moreover,safety can be guaranteed only if the key is used once, and then discarded. The problem is therefore how to distribute the random key between users in a secure way. Classically,the only possibility is either through personal meetings, or through a trusted courier, which makes the technique rather expensive, and not practical for many applications. Therefore,most practical cryptographic systems nowadays rely on different principles [1]. However,these cannot really guarantee the safety of the transmission, but rely on a weaker property of the system, namely that it is computationally safe. This means that the system can be broken in principle, but that the computation time required to do so is too long to pose a real threat. The main problem with this approach is that its safety could be destroyed by technological progress (faster computers) or mathematical advances (faster algorithms or future theoretical progress in computation theory). Another technique, whose safety does not rely on computing abilities, and which was only recently developed, is quantum cryptography(for an introduction, see [2]).

In quantum cryptography, the two users, generally referred to as Alice (the sender) and Bob (the receiver), have two kinds of communication channels at their disposal. One is a classical public channel, which can be overheard by anybody, but cannot be modified;and the second is a quantum channel, whose main characteristic is that any attempt at eavesdropping will create errors in the transmission. The quantum channel will be used to transmit the secret key, and the classical public channel will be used to exchange information and to send the encoded message. In principle, this is sufficient to ensure the safety of a transmission: Alice and Bob exchange a series of bits over the quantum channel, and then use part of the transmission to test for eavesdropping. If they find any discrepancy between their strings, they can infer that an eavesdropper, usually referred to as Eve, was listening and that their transmission is not secret. If they detect no errors, they can assume that the key is safe. By testing a large proportion of their initial string, they can attain any safety level they wish. Unfortunately, quantum channels are very sensitive devices, and due to the imperfections of the channels and of the detectors, some errors will always be unavoidable.

The problem facing Alice and Bob is therefore, for a given error rate, to estimate the amount of information that may have leaked to Eve, and decide on the safety of the transmission.

This of course depends on both the particular system used by Alice and Bob, and on the eavesdropping strategy adopted by Eve. A safer system is a system for which the amount of information that may have leaked to Eve is lower. If the information leaked to Eve is not too high, Alice and Bob can use classical information processing techniques to reduce it to approximately zero, at the expense of shortening their strings [3–5].
At present, there exist three different quantum cryptographic systems. The first one relies on the transmission of single photons randomly polarized along four directions [6].
As single photons are difficult to produce experimentally, a slight modification of this system, using weak pulses instead of single photons, was the first one to be implemented in practice [3,7,8]. The second system, which is conceptually the simplest, uses only two non orthogonal quantum states [9]. Its implementation relies on weak coherent pulses, with a phase difference between them [9,10]. The third system is based on the creation of pairs of EPR correlated photons [11]. One of its potential advantages is that the correlations are between single photons and not weak pulses, which can be a great advantage, as we shall emphasize later. However, creation and transmission over long distances of EPR correlated pairs is technologically more difficult, and it is not clear yet whether this will prove practical [12]. In this work, we focus on quantum cryptographic schemes implemented with weak pulses of coherent light. We compare the safety of the first and the second of these quantum cryptographic systems, and present a new system, which is a symbiosis of both, and for which the safety can be significantly increased.

In Section II, we analyze the first system, referred to as 4-states system. In Section III,we turn to the second one, named 2-states system, and present a new implementation.
We introduce our new 4+2 system in Section IV, and show that it is more sensitive to eavesdropping than the two previous ones. In Section V, we show the dangers associated with a lossy transmission line, and conclude in Section V

\subsubsection*{4-STATES PROTOCOL}

This protocol was developed by Bennett et al. [6]. The sender, Alice, chooses at random one out of four states, e.g. for polarized photons: |, <->, / or /, and sends it to the receiver, Bob. The two states - and | stand for bit value ‘0’, while the other two, <-> and /, stand for ‘1’. Bob chooses, also at random, a basis in which he measures the polarization. When his basis corresponds to Alice’s, his bit should be perfectly correlated with hers, whereas when his basis is the conjugate, there is no correlation between his result and Alice’s original choice. By discussing over the public channel, Alice and Bob agree to discard all the instances where they did not use the same basis (half of the total on average).

The result is what we call the sifted key, which should be two perfectly correlated strings,but which may contain errors. The two fundamental properties of this protocol are:(i) the choice of basis is completely hidden from the other protagonist (the two bases correspond to the same density matrix), as well as from any mischievous eavesdropper,Eve;(ii) when Alice and Bob use different bases, there is no correlation between their bits.

The first one, (i), ensures that, as the eavesdropper Eve cannot know which basis was used,she will unavoidably introduce errors. The second one, (ii), is not really necessary, but is preferable, as it reduces the information available to Eve to a minimum [13] (25 \% for each photon on which she eavesdropped). There have been various works analyzing eavesdropping strategies, calculating the information available to Eve as a function of the error rate and developing information processing techniques to reduce it to any required level [3–5,14].

One technical difficulty with this scheme is that in principle it should be implemented by means of single photons [6]. As these are difficult to generate experimentally, existing schemes rely on weak pulses of coherent or thermal light, with much less than one photon per pulse on average [3,7,8]. This ensures that the probability of having two or more photons in a pulse remains very small. This strategy reduces the transmission rate (recent experiments use about one tenth of a photon per pulse), while providing no advantage to an honest participant. More precisely, if Alice and Bob use coherent pulses, the transmission rate t, where the factor 1/2 comes from the fact that half of the transmissions had to be discarded(when Alice and Bob used different bases). In fact, even for these weak pulses, the probability of having two or more photons per pulse may not always be neglected (for the above pulses,one out of twenty non-zero pulses will have two photons). We shall show in Section V how to take this into account.

In order to get quantitative results, we shall assume that Eve uses the intercept resend strategy: she intercepts the pulses, attempts to gain as much information as possible, and sends to Bob a new pulse, prepared according to the information she obtained. Moreover,we shall assume that she eavesdrops in the bases used by Alice and Bob. This is the intercept/resend strategy which provides her with the most information on the sifted key [5]. However, it is not yet known whether this is the optimal strategy. It is easy to see that, when Eve eavesdrops on a fraction n of the transmissions, the error rate created is n/4 (when Eve uses the correct basis, she does not introduce any error, while she creates a 50\% error rate when she uses the wrong basis), and that the information she obtained is n/2 (she has total information when she used the correct basis, and none when she used the wrong one). Moreover, the scheme is completely symmetric, so that Eve shares the same information with Alice and with Bob. Therefore, we can write the mutual information shared by Alice and Eve and shared by Eve and Bob as a function of the error rate Q:I
(4)AE(Q) = I(4)EB(Q) = 2Q . (2)In this system, the intensity of the weak pulses, or equivalently the transmission rate t(4)defined in (1) has no influence. In the following, we will compare (2) to the information obtained by Eve for the other two system.

\subsubsection*{2-STATES PROTOCOL}
In this protocol [9], Alice chooses between only two non-orthogonal states, and sends one to Bob. As these are not orthogonal, there is no way for Bob to decode them deterministically. 

However, by means of a generalized measurement, also known as positive operator valued measure or POVM [15,16], he can perform a test which will sometimes fail to give an answer, and at all other times give the correct one. In essence, instead of having a binary test (with results 0 or 1), which will unavoidably create errors when the two states are not orthogonal, Bob has a ternary system, with possible results: 0, 1, or ? (with ? corresponding to inconclusive results). For example, if Alice sends a 0, Bob may get either a 0 or an inconclusive result, but he will never get a 1. We present a practical implementation of such a POVM in the next Section (for a theoretical description see [15,16]). As with the previous system, Alice and Bob use their public channel to discard all the inconclusive results. They should now have two perfectly correlated strings, except for possible errors. The safety of the protocol against eavesdropping is ensured by the fact that Eve cannot get deterministic results either. She may attempt to get as much information as possible by projecting the states onto an orthogonal basis. This is discussed in Section III C, where we show that this will unavoidably create errors. Another possibility for Eve is to mimic Bob’s measurement,and obtain deterministic results on a fraction of the bits. 

However, in this case, she will have to guess Alice’s choice on the remaining bits, and this will provide her with less information than the previous strategy. Let us emphasize that, as we want Eve’s guesses to result in errors on Bob’s side, this scheme should not be implemented with weak pulses only. In this case, Eve could simply intercept the transmission, resend a pulse to Bob only when she managed to obtain the bit sent by Alice (i.e. 0 or 1), and send nothing when she obtained an inconclusive result. The signature of eavesdropping would then just be a reduction in the transmission rate. In the original proposal, this is overcome by using phase encoding of a weak pulse (the two states are, with overlap e 2a ), which is sent together with a strong pulse, used also as a phase reference [9,10]. When Eve fails to obtain information,she still has to send the strong pulse, which will create errors in the reception. Ekert etal. [16] recently analyzed various eavesdropping strategies related to this system. Here, we shall restrict ourselves to the intercept/resend strategy and compare the safety of various schemes.

Let us now suggest a new implementation of the 2 states system, which will be later extended to our new 4 + 2 protocol. We shall call it parallel reference implementation, to distinguish it from the sequential reference implementation of [9]. Alice uses weak coherent states, with phase encoding 0 or pi with respect to a strong coherent state. We denote the weak states  and the strong state.

Instead of sending the two states one after the other, Alice uses two orthogonal polarizations: |, say, will have vertical polarization, and b will have horizontal polarization.

The decoding on Bob’s side is schematized in Fig. 1. The two states are separated by a polarization beam splitter (PBS). b is rotated to vertical polarization and sent through a mainly transmitting beam splitter (BS1) to detector D1. A small fraction of b, equal to 'a', is sent to interfere with 'a' at BS2, and towards two detectors D2 and D3. A count in D2, say, corresponds to phase zero, while a count in D3 corresponds to pi. No count in both D2 and D3 is of course an inconclusive result. It is easy to calculate the probability of such a result: which is equal to the overlap between the two states. It was shown that (4) represents the optimum for separating deterministically two non-orthogonal states [17]. Since all the cases corresponding to inconclusive results have to be discarded, the transmission rate of the channel is given by:

Detector D1 should always fire, and can therefore be used as a trigger for the other two,enabling them only for a short time corresponding to the length of the pulse (this will reduce the dark counts). Moreover, an eavesdropper with an inconclusive result would still have to send b, which will result in random counts in D2 and D3. A small modification of the above will be used for our 4 + 2 protocol.

As in Section II, we assume that Eve performs an intercept/resend strategy, and attempts to get as much information as possible on the state sent by Alice. This is obtained by projecting it onto the orthogonal basis B sym = (i, j) as shown in Fig. 2 [15], and provides Eve with probabilistic information only. Another method for Eve would be to use the same POVM as Bob. This would provide her with a smaller amount of deterministic information.
The error rate introduced by Eve is equal to the probability of making the wrong reading,e.g. the probability of obtaining j when the input was in fact . With this particular eavesdropping strategy,the transmission channel Alice-Eve is known as a binary symmetric channel, and is fully characterized by the error rate q [18]. The mutual information shared by Alice and Eve is equal to the channel capacity and is given by [16]:
b is the maximum information that can be extracted from two states with overlap cos b. On the other hand, after Bob discards all his inconclusive results, the transmission channel Eve-Bob is perfect. Therefore the mutual information Eve-Bob is:i EB = 1 .
(8)This strategy, where Eve eavesdrops on all the transmitted photons, will create a high error rate. In order to reduce it, Eve shall only eavesdrop on a fraction n of the transmissions,thus creating an error rate Q = nq = n

It is now easy to obtain the mutual information between Alice and Eve, I AE and between(2)Eve and Bob, I EB , as a function of the error rate Q and the angle b between the states: we can also write the mutual information as a function of the transmission rate for each value of the error rate. These functions are plotted in Fig 3, where the information gained for the 4-states system is used as a references 






\subsection*{\trnas}
\blindtext
\subsection*{\review}
\blindtext
\subsection*{\dic}
\blindtext

\section{Quantum cryptography: public key distribution and coin tossing}
\subsection{Article}

\subsubsection*{Abstract}
When elementary quantum systems, such as polarized photons, are used to transmit digital information, the uncertainty principle gives rise to novel cryptographic phenomena unachievable with traditional transmission media, e.g. a communications channel on which it is impossible in principle to eavesdrop without a high probability of disturbing the transmission in such a way as to be detected. Such a quantum channel can be used in conjunction with ordinary insecure classical channels to distribute random key information between two users with the assurance that it remains unknown to anyone else, even when the users share no secret information initially. We also present a protocol for coin-tossing by exchange of quantum messages, which is secure against traditional kinds of cheating, even by an opponent with unlimited computing power, but ironically can be subverted by use of a still subtler quantum phenomenon, the Einstein-Podolsky-Rosen paradox.

\subsubsection{Introduction}

Conventional cryptosystems such as ENIGMA,DES, or even RSA, are based on a mixture of guess­work and mathematics. Infonu.tion theory shows that traditional secret-key cryptosystems car.not be to­tally secure unless the key, used once only, is at least as long as the clear text. On the other hand,the theory of computational complexity is not yet well enough understood to prove the computational security of public-key cryptosystems.

In this paper we use a radically different foundation for cryptography, viz. the uncertainty principle of quantum physics. In conventional information  theory and cryptography,· it is taken for granted that digital communications in principle can always be passively monitored or copied, even by someone ignorant of their meaning. However, when information is encoded in non-orthogonal quantum states, such as single photons with polarization directions 0, 45, 90, and 135 degrees, one obtains a communications channel whose transmissions in prin­ciple cannot be read or copied reliably by an eaves­dropper ignorant of certain key information used informing the transmission. The eavesdropper cannot even gain partial information about such a transmis­sion without altering it a random and uncontrollable way likely to be detected by the channel's legiti­mate user.

Quantum coding was first described in (W),along with two applications: making money that is in principle impossible to counterfeit, and multi plex­ing two or three messages in such a way that reading one destroys the others. More recently [BBBW],quantum coding has been used in conjunction with public key cryptographic techniques to yield several schemes,s for unforgeable subway tokens. Here we show that quantum coding by itself achieves one of the main advantages of public key cryptography by per­mitting secure distribution of random key informa­tion between parties who share no secret information initially, provided the parties have access, besides the quantum channel, to an ordinary channel suscep­tible to passive but not active eavesdropping. Even in the presence of active eavesdropping, the two parties can still distribute key securely if they share some secret information initially, provided the eavesdropping is not so active as to suppress communications completely. We also present a proto­col for coin tossing by exchange of quantum mes­sages. Except where otherwise noted the protocols are provably secure even against an opponent with superior technology and unlimited computing power,barring fundamental violations of accepted physical laws.

Offsetting these advantages is the practical disadvantage that quantum transmissions are neces­sarily very weak and cannot be amplified in transit. Moreover, quantum cryptography does not provide di­gital signatures, or applications such as certified mail or the ability to settle disputes before a judge.

\subsubsection{Essential Properties of Polarized Photons}

Polarized light can be produced by sending an ordinary light beam through a polarizing apparatus such as a Polaroid filter or calcite crystal; the beam's polarization axis is determined by the orien­tation of the polarizing apparatus in which the beam originates. Generating single polarized photons is also possible, in principle by picking them out of a polarized beam, and in practice by a variation of. an experinient [AGR] of Aspect, et. a

Although polarization is a continuous varia­ble, the uncertainty principle forbids measurements on any single photon from revealing more than one bit about its polarization. For example, if a light beam with polarization axis is a sent into a filter oriented at angle P, the individual photons behave dichotomously and probabilistically, being transmit­ted with probability cos 2 (a-P) and absorbed with the complementary probability sin 7 («-P). The photons behave deterministically only when the two axes arc parallel (certain transmission) or perpendicular (certain absorbtion).

If the two axes are not perpendicular, so that some photons are transmitted, one might hope to learn additional information about a by measuring the transmitted photons again with a polarizer ori­ented at some third angle; but this is to no avail,because the transmitted photons, in passing through polariza­polarizer, emerge with exactly polarization, having lost all memory of their previous po­larization a.

Another way one might hope to learn more than one bit from a single photon would be not to measure it directly, but rather somehow amplify it into a clone of identically polarized photons, then perform measurements on these; but this hope is also vain,because such cloning can be shown to be inconsistent with the foundations of quantum mechanics [WZ) .

Formally, quantum mechanics represents the internal state of a quantum system (e.g. the polarization ­of unit length df a photon) as a vector in a linear space H over the field of complex num­bers (Hilbert space). The inner product of two vec j+j where, is defined as complex conjugation. The dimensionality of the Hilbert space depends on the system, being larg­er (or even infinite) for more complicated systems. Each physical measurement H that might be performed on the system corresponds to resolution of its Hilbert space into orthogonal subspaces, one foreach possible outcome of the measurement. The num­ber of possible outcomes is thus limited to the dimensionality d of the Hilbert space, the most complete measurements being those that resolve the Hilbert space into d 1-dimensional subspaces.

The Hilbert space for a single polarized pho­ton is 2-dimensional; thus the state of a photon maybe completely described as a linear combination of,for example, the two unit vectors r1  (1,0) and r2 = (0,1), representing respectively horizontal and vertical. In particular, a photon  polarized at angle a to the horizontal is described (cosa, sina). When subjected to a measurement of vertical-vs.-horizontal polari­zation, such a photon in effect chooses to become horizontal with probability cos 2 a and vertical with probability sin 2 a. The two orthogonal vectors r1 and r2 thus a exemplify the resolution of a 2-dimensional Hilbert space into'2 orthogonal 1-dimensional subspaces; henceforth r1 and r2 will be said to comprise the 'rectilinear' basis for the Hilbert space.

An alternative basis for the same Hilbert space is provided by the two 'diagonal' basis vec­tors d1 (0. 707 ,0.'707), representing a 45-degree photon, and d2 (0.707,-0.707), representing a 135-degree photon. Two bases (e.g. rectilinear and diagonal) are said to be 'conjugate' (WI, if each vector of one basis has equal-length projections onto all vectors of the other basis: this means that a system prepared in a specific state of one basis will behave entirely randomly, and lose all its stored information, when subjected to a measurement corresponding to the other basis. Owing to the com­plex nature of its coefficients, the two-dimensional Hilbert space also admits a third basis conjugate to both the rectilinear and diagonal bases, comprising the two so-called 'circular' polarizations c1 (0.707,0.707i) and c2 (0.707i,0.70"1); but the rectilinear and diagonal bases are all that will be needed for the cryptographic applications in this paper.

The Hilbert space for a compound system is constructed by taking the tensor product of the Hil­bert spaces of its components; thus the state of a pair of photons is characterized by a-unit vector in the 4-dimensional Hilbert space spanned by the or­thogonal basis vectors r1r 1, r 1r2 , r2r1 , and r 2 r 2 . This formalism entails that the state of a compound system is not generally expressible as the cartesian product of the states of its parts: e.g. the·Einstei'n-Podolsky-Rosen state of two photons, 0.7071(r1r 2 -r2 r1), to be discussed later, is not equivalent to any product of one-photon state.

\subsubsection{Quantum Public Key Distribution}

In traditional public-key cryptography, trap ­door functions arc used to conceal the meaning of messages between two users from a passive eavesdrop­per, depite the lack of any initial shared secret information between the two users. In quantum pub­lic key distribution, the quantum channel is not used directly to send meaningful messages, but is rather used to transmit a supply of random bits be­tween two users who share no secret information ini­tially, in such a way that the users, by subsequent consultation over an ordinary non-quantum channel subject to passive eavesdropping, can tell with high,probability whether the original quantum transmis­sion has been disturbed in transit, as it would be by an eavesdrooper (it is the quantum channel's pe­culiar virtue to compel eavesdroppinq to be active). If the transmission has not been disturbed, they agree to use these shared secret bits in the well ­known way as a one-time pad to conceal the meaning of subsequent meaningful communications, or for oth­er cryptographic applications (e.g. authentication tags) requiring shared secret random information. If transmission has been disturbed, they discard it and try again, deferring any meaningful communica­tions until they have succeeded in transmitting enough random bits through the quantum channel to serve as a one-time pad.

In more detail one user ('Alice') chooses a random bit string and a random sequence of polariza­tion bases (rectilinear or diagonal). She then sends the other user (Bob) a train of photons, eachrepresenting one bit of the string in the basis cho­sen for that bit position, a horizontal or 45-degree photon standing for a binary zero and a vertical or 135-degree photon standing for a binary 1. As Bob receives the photons he decides, randomly for each photon and independently of Alice, whether to meas­ure the photon's rectilinear polarization or its diagonal polarization, and interprets the result of the measurement as a binary zero or one. As ex­plained in the previous section a random answer is produced and all information lost when one attempts to measure the rectilinear polarization of a diago­nal photon, or vice versa. Thus Bob obtains mean­ingful data from only half the photons he detects-­those for which he guessed the correct polarization basis. Bob's information is further degraded by the fact that, realistically, some of the photons would be lost in transit or would fail to be counted by Bob's imperfectly-efficient detectors.

Subsequent steps of the protocol take place over an ordinary public communications channel, as­sumed to be susceptible to eavesdropping but not to the injection or alteration of messages. Bob and Alice first determine, by public exchange of mes­sages, which photons were successfully received and of these which were received with the correct basis. If the quantum transmission has been undisturbed, Alice and Bob should agree on the bits encoded by these photons, even this data has never been dis­cussed over the public channel. Each of these pho­tons, in other words, presumably carries one bit of random information (e.g. whether a rectilinear pho­ton was vertical or horizontal) known to Alice and Bob but to no one else.

Because of the random mix of rectilinear and diagonal photons in the quantum transmission, any eavesdropping carries the risk of altering the transmission in such a way as to produce disagree­ment between Bob and Alice on some of the bits on which they think they should agree. Specifically,it can be shown that no measurement on a photon in transit, by an eavesdropper who is informed of the photon's original basis only after he has performed his measurement, can yield more than 1/2 expected bits of information about the key bit encoded by that photon; and that any such measurement yielding b bits of expected information (b s 1/2) must induce a disagreement with probability at least b/2 if the measured photon, or an attempted forgery of it,is later re-measured in its original basis. (This optimum trade off occurs, for example, when the ea­vesdropper measures and retransmits all intercepted photons in the rectilinear basis, thereby learning the correct polarizations of half the photons and inducing disagreements in 1/4 of those that are lat­er re-measured in the original basis.)

Alice and Bob can therefore test for eaves­dropping by publicly comparing some of the bits on which they think they should agree, though of course this sacrifices the secrecy of these bits. The bit positions used in this comparison should be a random subset (say one third) of the correctly received bits, so that eavesdropping on more than a few pho­tons is unlikely to escape detection. If all the comparisons agree, Alice and Bob can conclude that the quantum transmission has been free of signifi­cant eavesdropping; and those of the remaining bits that were sent and received with the same basis also agree, and can safely be used as a one time pad for subsequent secure communications over the public channel. When this one-time pad is used up, the protocol is repeated to send a new body of random information over the quantum channel.

The need for the public (non-quantum) channel in this scheme to be immune to active eavesdropping can be relaxed if the Alice and Bob have agreed be­fore hand on a small secret key, which they use ·to create Wegman-Carter authentication tags WCI for their messages over the public channel. In more detail the Weqman-Carter multiple-message authenti­cation scheme uses a small random key to produce a message-dependent 'tag' (rather like a check sum)for an arbitrary large message, in such a way that an eavesdropper ignorant of the key has only a small probability of being able to generate any other va­lid message-tag pairs. The tag thus provides evi­dence that the message is legitimate, and was not generated or altered by someone ignorant of the key. (Key bits are gradually used up in the Wegman-Carter scheme, and cannot be reused without compromising the system's provable security; however, in the present application, these key bits can be replaced by fresh random bits successfully transmitted through the quantum channel.) The eavesdropper can still prevent communication by suppressing messages in the public channel, as of course he can by sup­pressing or excessively perturbing the photons sent through the quantum channel. However, in either case, Alice and Bob will conclude with high proba­bility that their secret communications are being suppressed, and will not be fooled into thinking their communications are secure when in fact they're not.

\subsubsection{Quantum Coin Tossing}

'Coin Flipping by Telephone' was first dis­cussed by Blum [Bl]. The problem is for two dis­trustful parties, communicating at a distance with­out the help of a third party, to come to agree on a winner and a loser in such a way that each party has exactly 50 per cent chance of winning. Any attempt by either party to bias the outcome should be de­tected by the other party as cheating. Previous protocols for this problem are based on unproved assumptions in computational complexity theory, which makes them vulnerable to a breakthrough in algorithm design.

1. Alice chooses randomly one basis (say rectili­near) and a sequence of random bits (one thousand should be sufficient). She then encodes her bits as a sequence of photons in this same basis, using the same coding scheme as before. She sends the result­ing train of polarized photons to Bob.

2. Bob chooses, independently and randomly for each photon, a sequence of reading bases. He reads the photons accordingly, recording the results in two tables, one of rectilinearly received photons and one of diagonally received photons. Because of losses in his detectors and of the transmission channel, some of the photons may not be received a tall, resulting in holes in his tables. At this time, Bob makes his guess as to which basis Alice used, and announces it to Alice. He wins if heguessed correctly, loses otherwise.

3. Alice reports to Bob whether he won, by telling him which basis she had actually used. She certif­ies this information by sending Bob, over a classi­cal channel, her entire original bit sequence used in step 1.

4. Bob verifies that no cheating has occurred by comparing Alice's sequence with both his tables. There should be perfect agreement with the table corresponding to Alice's basis and no correlation with the other table. In our example, Bob can be confident that Alice's original basis was indeed rectilince  as claimed.

Alice could attempt cheating either at step 1 or step 2. Let us first assume that she follows step 1 honestly and finds herself losing at the end if step 2, because Bob made he correct guess, here rectiliniear. In order to pretend she has won, she would need to convince Bob that her photons were diagonally polarized, which she can only do by prod­ucing a sequence of bits in perfect agreement with Bob's diagonal table. This she cannot do reliably because this table is the result of probabilistic behavior of the photons after the left her hands. Suppose she goes ahead anyway and sends Bob a new 'original' sequence, different from the one that she used in step 1, in hopes that it will by luck agree perfectly with Bob's diaognal table. This attempt to cheat requires Alice to be not only lucky but daring, because in the vast majority of cases, the gamble would fail and would be detected as cheating. By contrast, in traditional coin-tossing schemes,analogous attempts to seize a lucky victory from the jaws of defeat, though unlikely to succeed, are unaccompanied by any danger of detection.

It is easy to see that things are even worse for Alice if she attempts to cheat in step 1, by sending a mixture of rectilinear and diagonal pho­tons, or photons which are polarized neither rectil­inearly or diagonally. In this case she will not be able to agree with either of Bob's tables in step 3,since both tables will record the results of proba­bilistic behavior not under her control.

In order to say how Alice can cheat using quantum mechanics it is necessary to describe the Einstein- Podolsky-Rosen (EPR) effect [Bo,AGR), often called a paradox because it contradicts the common ­sense notion that for two individually random events happening at distance from one another to be corre­lated, some physical influence must have propagated from the earlier event to the later, or else from some common random cause to both events.

The EPR effect occurs when certain types of atom or molecule decay with the emission of two pho­tons, and consists of the fact that the two photons are always found to have opposite polarization, re­gardless of the basis used to observe them, provided both are observed in the same basis. For example,if both Photons are measured rectilinearly, it will always be found that one is horizontal and the other vertical, though which is horizontal will vary ran­domly from one decay to the next. If both photons are measured diagonally, one will always be 135-degree and the other 45-degree. A moment's reflec­tion will show that this behavior cannot be explained by assuming the decay produce a distribution over a of oppositely polarized (a and a+90 1photons, since, in that case, if such a pair of photons were measured in an intermediate basis (say a+45), both would behave probabilistically so as to sometimes come out with the same polarization.

Probably the simplest, but paradoxical ­sounding, verbal explanation of the EPR effect is to say that the two photons are produced in an initial state of undefined polarization; and when one of them is measured, the measuring apparatus forces it to choose a polarization (choosing randomly and equiprobably between the two characteristic direc­tions offered by the apparatus) while simultaneously forcing the other unmeasured photon, no matter how far away, to choose the opposite polarization. This implausible-sounding explanation is supported by formal quantum mechanics, which represents the state of a pair of photons as a vector in a 4-dimensional Hilbert space obtained by taking the tensor product of two 2-dimensional Hilbert spaces. The EPR state produced by the decay is described by the vector 0.7071(r 1r2 - r2r1), and the EPR effect is explained by the fact that this vector has anticorrelated pro­jections into the 2-dimensional Hilbert spaces of the two photons no matter what basis is used to ex­press the tensor product.

In order to cheat, Alice produces a number of EPR photon-pairs instead of individual random pho­tons in step 1. In each case she sends Bob one mem­ber of the pair and stores the other herself, per­haps between perfectly reflecting mirrors. 


\subsection{\trnas}

\subsection{\review}

\subsection{\dic}



\section{Towards practical and fast Quantum Cryptography}
\subsection{Article}

\subsubsection*{Abstract}

We present a new protocol for practical quantum cryptography, tailored for an implementation with weak coherent pulses. The key is obtained by a very simple time-of-arrival measurement on the data line; an interferometer is built on an additional monitoring line, allowing to monitor the presence of a spy (who would break coherence by her intervention). Against zero-error attacks (the analog of photon-number-splitting attacks), this protocol performs as well as standard protocols with strong reference pulses: the key rate decreases only as the transmission t of the quantum channel. We present also two attacks that introduce errors on the monitoring line: the intercept-resend, and a coherent attack on two subsequent pulses. Finally, we sketch several possible variations of this protocol.

\subsubsection{INTRODUCTION}

Quantum cryptography [1], or quantum key distribution (QKD), is probably the most mature field in quantum information, both in theoretical and in experimental advances. On the theoretical side, almost all QKD protocols have been proven to provide unconditional security in some regime; on the practical side, QKD has already reached the stage of commercial prototypes. Still,much work is needed. A big task consists in bringing both theory and applications in contact again: practical QKD systems do not fulfill all the requirements of unconditional security proofs (or, if you prefer, these proofs are still too abstract to cope with a practical system). Here, we address a different question: we aim for the most practical QKD system. Instead of looking for a new implementations of known protocols, we choose to start from scratch by inventing a new protocol. There are two basic requirement.

1. The protocol must be easily implementable, say with the smallest number of standard telecom devices. Note that this requirement, as a side benefit, may simplify security studies: we have learnt in the recent years that any optical component can be regarded as a ”Trojan horse” because of its imperfections [2].

2. The security of the system must be guaranteed by quantum physics, thence in some way quantum coherence must play a role.

The goal of this paper is to illustrate this program by presenting such a system. the key is created in a data line that is probably the simplest one can think of —just measure the time of arrival of weak pulses. The intervention of a spy is checked interferometrically in a monitoring line. In Section II, we define precisely the protocol and stress its advantages. In Section III, we present a quantitative study of security. Finally, Section 4 presents a number of possible variations on the main idea.


\subsubsection{THE PROTOCOL}

Alice uses a mode-locked laser, producing pulses of mean photon-number u that are separated by a fixed and well-defined time $\tau$ ; with a variable attenuator, she can blocks some of the pulses (note that a more economical source would just consist of a cw laser followed by the variable attenuator). Each logical bit is encoded in a two-pulse sequence according to the rule.

For instance, the eight-pulse sequence drawn in Fig. 1 codes for the four-bit string 0100 (read in temporal order, that is, from right to left). For small $\mu$, the states|0 A i and |1 A i have a large overlap because of their vacuum component. Since the laser is mode-locked, there is a phase coherence between any two non-empty pulses.

The pulses now propagate to Bob, on a quantum channel characterized by a transmission t = 10 -$\alpha$ d/10(a typical value for a in optical fibers is 0.2 dB/km).
Bob’s setup first splits the pulses using a non-equilibrated beam-splitter with transmission coefficient t B. The pulses that are transmitted are used to establish the raw key (data line). To obtain the bit value, Bob has to distinguish unambiguously between the two non-orthogonal states.

As well-known, unambiguous discrimination between two pure states can succeed with probability p ok = 1 -|h0 B |1 B i|; in the present case, the overlap is |h0 B |1 B i| = 2 e -|a| , and consequently p ok = 1 - e -u t t B . Now, there is an obvious way to achieve this result: photon counting with a perfect detector, because p ok is just the probability that the detector will detect something. The realistic situation where the detector has a finite efficiency n can be modelled by an additional beam-splitter with transmittivity n followed by a perfect detector; in this case, n appears in the exponent as well. In conclusion, the optimal unambiguous discrimination between |0 B i and |1 B i is achieved by the most elementary strategy, simply try to detect where the photons are. Later, Bob must announce Alice which items he has detected: this is how Alice and Bob establish their raw key. Note that no error is expected on this line, if the switch is perfect and in the absence of dark counts of the detector: a bit-flip is impossible because it would correspond to a photon jumping from a time-bin to another. Note that the simplicity of Bob’s data line has concrete practical advantages. There are no lossy and active elements. Hence, the transmission range can be increased and no random number generator is needed. As for the data line, our protocol is similar to the one of Debuisschert and Boucher [3]. However there, the security was obtained by the overlap in time between the pulses coding for different bits. Here, we use rather the monitoring line described in the next paragraph.

The pulses that are reflected at Bob’s beam-splitter go to an interferometer that is used for monitoring Eve’s presence (monitoring line). Here is where quantum coherence plays a role. Let a j be the amplitude of pulse j entering the interferometer: in particular, |a j | 2 is either 0 or u t (1 - t B ); and if both a j and a j+1 are non-zero,then a j+1 = a j . After the interferometer, the pulses that reach the detectors at time j + 1.


Now, if either a j or a j+1 are zero, then |D M1 | 2 =|D M2 | 2 = 12 u t (1 - t B ); i.e., conditioned to the fact that a photon takes the monitoring line, the probabilities of detecting it in either detector is 12 . However, if both a j and a j+1 are non-zero, then |D M1 | 2 = u t (1 - t B ) and|D M2 | 2 = 0: only detector D M1 can fire. Consider then again the case where bit number k is 1 and bit number k + 1 is 0: as we said above, in this case the two consecutive pulses 2k and 2k + 1 are non-empty. This means that, if coherence is not broken, detector D M2 cannot fire at time 2k + 1. If Eve happens to break the coherence by reading the channel, it could be detected this way.

Actually, it turns out that, as just described, the protocol is insecure: Eve can make a coherent measurement of the number of photons in the two pulses across the bit-separation. With such an attack, she would not break the coherence, thus introduce no errors in the monitoring line, and obtain almost full information (see next Section for more details). There are several ways of countering this attack: here, we make use of decoy sequences, inspired by the idea of ”decoy states” introduced by Hwang[4] and by Lo and co-workers [5], but different in its implementation. The principle is the following: with probability f , Alice leaves both the (2k - 1)-th and the 2k-the pulses non-empty. A decoy sequence does not encode a bit value (in contrast to the decoy states of [4,5] that still encode a state, but in a different way): thence, if the item is detected in the data line, it will be discarded in public discussion. However, if a detection takes place in the monitoring line at time 2k, then it must be in detector D 1M because of coherence. Now Eve can no longer pass unnoticed: if she attacks coherently across the bit separation, then she breaks the coherence of the decoy sequences; if she attacks coherently within each bit, then she breaks the coherence across the separation; finally, if she makes a coherent attack on a larger number of pulses,then she breaks the coherence in fewer positions but gets much less information.

Thus, errors are rare: they appear only in the monitoring line, and just for a fraction of the whole cases. Still,one can estimate the error (thence, the coherence of the channel) in a reasonable time, if the bit rate is high.

Should one say in one sentence where the improvement lies, here it is: one can define a very simple data line and protect it quantum-mechanically. At this point, two important remarks can be done. First, this protocol cannot be analyzed in terms of qubits. This is obvious, because any bits and coherence are checked on differently defined pairs. In particular, there is not a ”natural” single-photon version of the protocol (simply replace non-empty coherent state with one photon Fock states would be dramatic, since all the sequences would become orthogonal). The second remark is the answer to a possible question. With the idea of a simple data line for key creation, and a ”complementary” line for monitoring, one may implement a version of the BB84 protocol: Alice and Bob agree to produce the key using only the Z basis; sometimes Alice prepares one of the eigenstates of the X basis that acts as a decoy state. Which are the advantages of our protocol? We are going to see that our protocol is much more robust against attacks at zero errors (the analog of photon number-splitting attacks).

\subsubsection{QUANTITATIVE ANALYSIS OF SECURITY}

For a reasonable comparison with experiment, we must introduce the following parameters

1. The visibility V of the monitoring interferometer, whence the probability that D 2M fires in a time corresponding to a coherence is 1-V instead of zero. We suppose that Eve can take advantage of these imperfections: for instance, if the reduced visibility is due to fi 6 = 0 in the interferometer, Eve can systematically correct for this error by displacing the pulses, and then reproduce V by adding errors in a way that is profitable for her.

2. The imperfections of the three Bob’s detectors, supposed to be identical for simplicity: the quantum efficiency n and the probability per gate of a dark count p d . Typical values are n = 10\%, p d = 10 -5 . These imperfections are not given to Eve (see Section IV on the possibility that Eve forces a detection, thus effectively setting n = 1 for some pulses).

For simplicity in writing, we make all the quantitative analysis in the limit of small mean photon-number in Bob’s channel, that is ut << 1

First, we compute the parameters of Alice-Bob on the data line. Bob’s detection rate in D B , once decoy sequences are removed, is where T = t t B n. In other words, R B times the number of two-pulse sequences sent by Alice is the length of the raw key.
If we assume that the switch prepares really empty pulses when it is closed, the error expected in this line is only due to the dark counts of the detectors: Q=R because dark counts may make the detector fire at both times with equal probability. The mutual information Alice-Bob in bits per sent photon is thence. In what follows, we shall concentrate on attacks by Eve that do not modify Q. Before moving to that, let’s have a look at the monitoring line as well.

In the presence of dark counts and reduced visibility, the meaningful detection probabilities in D M1 and D M2 , neglecting double counts are the following [6]: R2= 1/V.

Contrary to the errors due to dark counts, the departure from perfect visibility will be entirely attributed to Eve. This is why we consider a priori different values V d and V 10 for the visibility in the two cases: as we shall see,Eve’s attacks may be different.

If Bob’s detector has dark counts, I(B : E) is smaller than I(A : E) for a prepare-and-measure scheme, becaus eeven if Eve knows perfectly what Alice has sent, she cannot know whether Bob has detected a photon or has had a dark count. Thus in our case, the Csiszar-Korner bound[7] that gives an estimate of the extractable secret key rate becomes.

Therefore, we have to compute the mutual information Bob-Eve.

The kind of attacks by Eve that we consider is sketched in Fig. 2. We can give Eve all the losses in the line, that is, we can suppose that Eve removes a fraction 1-t of the photons, and forwards the remaining fraction t to Bob on a lossless line. We are going to study:

1. An attack in which Eve can gain information without introducing errors. This is related to the losses on the line; it is the analog of the usual photon number-splitting attack [8,9], but is a different attack and less powerful.


2. Eve can immediately know if the previous attack was successful or not; in the case it wasn’t, we consider further the possibility of attacks that introduce errors in the monitoring line (but still no errors in the data line). Specifically, we study a usual intercept-resend strategy, and a more clever attack which is performed coherently on two subsequent pulses across the bit-separation.

In the case of BB84 and many other protocols, Eve can exploit multi-photon pulses in a lossy line to perform the photon-number-splitting attack [8,9]: she counts the photons in each pulse, and whenever this number is larger than one, she keeps one photon in a quantum memory and forwards the remaining photons to Bob on a lossless line. As such, this attack is not error-free in the present protocol: counting the photons in each pulse breaks the coherence between successive pulses, thus introducing errors in the monitoring line — actually, because of the peculiar encoding of the bits, this attack reduces here to the intercept-resend, see below.

More subtle is the analysis of a practical version of the attack using cascaded beam-splitters [10]: Eve uses a highly unbalanced BS, with transmission 1 - e and reflection e; if she has a detection, she forwards the remaining photons to Bob; otherwise, she begins anew, and so on until the losses that she introduces reach the transmission t of the quantum channel. The advantage of this strategy is that, in the presence of two or more photons, it is very rare that more than one photon is coupled into Eve’s detector. Indeed, this beam-splitting attack approximates a photon-counting. In our case, this strategy will introduce errors in the monitoring line as well: it does not modify the relative phase, but the relative intensity between subsequent pulses, thus leading to an unbalancing of the interferometer. The full analysis of such a strategy will be studied in a further work.

In summary, both the ideal photon-number-splitting and its approximate implementation through cascaded beam-splitters do not rank among the zero-error attacks against our protocol. In fact, Eve can only perform the basic beam-splitting attack: she removes a fraction 1 - t of the photons, and transmits the remaining fraction t to Bob on a lossless line. With the fraction that she has kept, the best thing Eve can do is just to measure them(recall the argument about optimal unambiguous state discrimination). This way, she detects u(1 - t) photons per pulse. When Eve has a detection in D E , she knows the bit that Alice has sent. Then she lets the remaining part of the pulse travel to Bob on the lossless line. Bob detects something exactly as if Eve had not been there.

In summary, Eve knows a fraction m(1 - t) of the key just because of the losses in the quantum channel: this fraction must always be subtracted in privacy amplification.


This is an important improvement over BB84: u opt is large and is basically constant with decreasing t (long quantum channels); as a consequence, the secret-key rate decreases only linearly (and not quadratically) with t. This is the same improvement that can be obtained by using decoy states [5] or a strong reference pulse [12];note however that the hardware is much simpler here.

When Eve’s detector D E does not fire, which happens with probability 1 - u(1 - t), Eve must perform some attack on the pulses flying to Bob if she wants to gainsome information. These attacks will certainly introduce errors, either in the data line or in the monitoring line. In the following, we present two such attacks (Fig. 3): a basic intercept-resend (I-R), and a photon-number-counting attack performed coherently on two subsequent pulses across the bit-separation (2c-PNC).

FIG. 3. Comparison of two attacks that introduce errors. In the I-R attack, Eve prepares a sequence of localized Fock states, thus breaking the coherence everywhere. In the 2c-PNC attack, Eve prepares a sequence of Fock states that are delocalized across the separation of bits: only the coherence of decoy sequences is broken. Note that arrows denote only coherence between subsequent pulses, the one checked by the interferometer; however, on the original sequence, all the non-empty pulses are coherent with one another, while in the sequences after Eve’s attack only the indicated coherence remains.

Let’s begin with the intercept-resend (I-R) strategy. Eve simply detects the pulse flying to Bob: her detector will fire with probability $\mu$ t, and in this case she prepares a single-photon in the good time-bin and forwards it to Bob. Obviously, both R B and Q are unchanged under this strategy.

Note that R B will be the sum of three terms: Eve has detected and Bob detects too; Eve has detected and Bob has a dark count; Eve has not detected and Bob has a dark count. Now, Eve can distinguish the last one from the two first, and she knows that in the last case she has no information on Alice’s and Bob’s bits.

In the absence of decoy sequences, Eve may obtain information without introducing errors in the monitoring line, by counting the number of photons coherently between two pulses, not within each bit but across the separation line (see Fig. 3). This attack does not break the phase between these pulses. Of course, if Eve finds n > 0 photons, on the spot she does not know to which bit the photon belongs; but she will learn it later, by listening to Bob’s list of accepted bits [13]. Actually, in some very rare case Eve still does not get any information: if Eve prepares n > 0 photons in two successive two-pulse sequences, and Bob accepts a detection in the bit common to both sequences, Eve has no idea of his result. However, since such cases are rare, we make the conservative assumption that Eve always gets full information.

Note in particular that, if Alice and Bob find V d = V 10 ,they can conclude that Eve has not used the 2c-PNC attack — by the way, this is why we presented the analysis of the I-R strategy, obviously worse than 2c-PNC from Eve’s standpoint: in a practical experiment, V d = V 10 is very likely to hold (after all, Eve is not there...). Therefore, formulae for the I-R attack will be useful in the analysis of experimental data.

\subsubsection{VARIATIONS AND OPEN QUESTIONS}

Here are a few ideas of variations in the protocol, that may have some additional benefit and require further study:

1. Alice may change during the protocol the definition of the pulses that define a bit. If there i sa convenient fraction of decoy sequences, Eve has no way of distinguishing a priori which pairs of successive pulses encode a bit. This way, the effect of the 2c-PNC attack becomes equally distributed among decoy and ”1,0” sequences, i.e. V d = V 10 = 1 - p IR - 12 p 2c . Moreover, whenever Eve attacks the bit instead of attacking across the bit-separation, she cannot gain any information.

2. In fact, nothing forces to define bits by subsequent pulses: Alice and Bob can decide later, adding a sort of ”sifting” phase to the protocol. This means that Alice can now send whatever pulse sequence,she is no longer restricted to those that define a bit or a decoy sequence. It is not clear whether this modification helps, if the hardware is kept unchanged: Alice and Bob still check only the coherence between subsequent pulses; moreover, sifting means additional losses and additional information revealed publicly.

3. Instead of introducing decoy sequences as we did above, one may study the effect of decoy pulses with different intensities, as proposed by Hwang and by Lo and co-workers to protect the BB84 protocol. against the photon-number-splitting attack [4,5].

The difficulty in assessing the security of practical QKD, is the huge number of imperfections that may hide loopholes for security. These imperfections exist in all implementations and for all protocols, but their effect and the corresponding protection may vary. Here we present some of these.

1.About Trojan-horse and similar realistic attacks [2]:Alice’s setup must be protected against Trojan horse attacks, with the suitable filters, isolators etc. In Bob’s setup nothing is variable; however,one must prevent the possible light emission from avalanche photodiodes to become available to Eve:if the firing of a detector can be seen from outside Bob, the protocol becomes immediately insecure.

2. After a detection, Bob’s detectors are blind during some time. In particular, if Eve happens to know when Bob’s detector D M2 has fired, she can attack strongly the subsequent pulses because no error will be detected then, and gain one bit (just one, because when D B has fired, then it has a dead time as well). For the security of the protocol, Eve must have no way of assessing the firing of a detector, and Bob must announce publicly this information only after the detector is ready again. This may imply some suitable synchronization in the software,or more simply, to shut D B as long as D M2 has not recovered. The nuisance depends of course on the ratio between the raw bit rate and the dead time.


3. In all this paper, we have considered only the case where Eve does not change Bob’s detection rates in the data line and in the monitoring line. By sending out stronger pulses, Eve might force the detection of those items on which she has full information; but in turn, she would increase the rate of double counts among Bob’s detectors. This effect must be quantified, and the number of double,or even triple, counts must be monitored during the experiment.

\subsubsection{CONCLUSION}

In conclusion, we have presented a new protocol for quantum cryptography whose realization is much simpler than that of previously described ones. Specifically,Bob’s station is such that losses are minimized and no dynamical component is needed.

We thank H.-K. Lo for stimulating comments. We acknowledge financial support from idQuantique and from the Swiss NCCR ”Quantum photonics".

\subsection{\trnas}
sss
\subsection{\review}
ss
\subsection{\dic}
ss


\section{Quantum cryptography based on Bell's theorem}

\subsection{Article}

\subsubsection*{Abstract}

Practical application of the generalized Bell s theorem in the so-called key distribution process in cryptography is reported. The proposed scheme is based on the Bohm's version of the Einstein-Podolsky-Rosen gedanken experiment and Bell's theorem is used to test for eavesdropping.

\subsubsection{Main}

Cryptography, despite a colorful history that goes back to 400 B.C. , only became part of mathematics and information theory this century, in the late 1940s, mainly due to the seminal papers of Shannon [1]. Today, one can briefly define cryptography as a mathematical system of transforming information so that it is unintelligible and therefore useless to those who are not meant to have access to it. However, as the computational process associated with transforming the information is always performed by physical means, one cannot separate the mathematical structure from the underlying laws of physics that govern the process of computation [2]. Deutsch has shown that quantum physics enriches our computational possibilities far beyond classical Turing machines[2], and current work in quantum cryptography originated by Bennett and Brassard provides a good example of this fact [3].

In this paper I will present a method in which the security of the so-called key distribution process in cryptography depends on the completeness of quantum mechanics. Here completeness means that quantum description provides maximum possible information about any system under consideration. The proposed scheme is based onthe Bohm's well-known version of the Einstein-PodolskyRosen gedanken experiment [4]; the generalized Bell's theorem (Clauser-Horne-Shimony-Holt inequalities) [5] is used to test for eavesdropping. From a theoretical point of view the scheme provides an interesting and new extension of Bennett and Brassard's original idea, and from an experimental perspective offers a practical realization by a small modification of experiments that were set up to test Bell's theorem. Before I proceed any further let me first introduce some basic notions of cryptography.
Originally the security of a cryptotext depended on the secrecy of the entire encrypting and decrypting procedures; however, today we use ciphers for which the algorithm for encrypting and decrypting could be revealed to anybody without compromising the security of a particular cryptogram. In such ciphers a set of specific parameters, called a key, is supplied together with the plaintext as an input to the encrypting algorithm, and together with the cryptogram as an input to the decrypting algorithm. The encrypting and decrypting algorithms are publicly announced; the security of the cryptogram depends entirely on the secrecy of the key, and this key,which is very important, may consist of any randomly chosen, suSciently long string of bits. Once the key is established, subsequent communication involves sending cryptograms over a public channel which is vulnerable to total passive interception (e.g. , public announcement in mass media). However, in order to establish the key, two users, who share no secret information initially, must at a certain stage of communication use a reliable and a very secure channel. Since the interception is a set of measurements performed by the eavesdropper on this channel, however dificult this might be from a technological point of view, in principle any classical channel can always be passively monitored, without the legitimate being aware that any eavesdropping users has taken place. This is not so for quantum channels [3]. In the following I describe a quantum channel which distributes the key without any "element of reality" associated with the key and which is protected by the completeness of quantummechanics.

The channel consists of a source that emits pairs of spin- 2 particles, in a singlet state. The particles Ay apart along the z axis, towards the two legitimate users of the channel, say, Alice and Bob, who, after the particles have on spin components separated, perform measurements along one of three directions given by unit vectors a; and b, (i, =1, 2, 3), respectively, for Alice and Bob. For simplicity, both a; and bj vectors lie in the x-y plane, perpendicular to the trajectory of the particles, and are characterized by azimuthal angles: Pi'=0, Pz = 4 tr, P3 and pi = —, x, p2 = —, tr, p3 = —, tr. Superscripts "a" and "b" refer to Alice and Bob's analyzers, respectively, and the angle is measured from the vertical x axis. The users choose the orientation of the analyzers randomly and independently for each pair of incoming particles. Each in 2 6 units, can yield two results, +1 measurement,(spin up) and — 1 (spin down), and can potentially reveal one bit of information.
The quantity E(a, , b, ) =P++(a, , b, )+P(a, , b, )— P+ (a;, b~) — P — +(a;, b~)is the correlation coefficient of the measurements formed by Alice along a; and by Bob along b~. Here P+ ~ (a;, b~) denotes the probability that result +'1 has 1 along b~. According to been obtained along a; and the quantum rules~ E(a, , b, ) = — a b . (2)of the same orientation(a2, bi and a3, b2) quantum mechanics predicts total anticorrelation of the results obtained by Alice and Bob:E(a, , b, ) =E(a, , b, ) = — 1.

Let us also, following Clauser, Horne, Shimony, and Holt [5], define a quantity composed of the correlation coefficients for which Alice and Bob used analyzers of different orientation.

After the transmission has taken place, Alice and Bob can announce in public the orientations of the analyzers they have chosen for each particular measurement and divide the measurements into two separate groups: a first of group for which they used different orientation analyzers, and a second group for which they used the same orientation of their analyzers. They discard all measurements in which either or both of them failed to register a particle at all. Subsequently, Alice and Bob can reveal publicly the results they obtained but within the first group of measurements only. This allows them to establish the value of S, which, if the particles were not should reproduce the directly or indirectly "disturbed,result of Eq. (4). This assures the legitimate users that the results they obtained within the second group of measurements are anticorrelated and can be converted into a secret string of bits the key. This secret key may be then used in a conventional cryptographic communication between Alice and Bob.

The eavesdropper cannot elicit any information from the particles while in transit from the source to the legitimate users, simply because there is no information encoded there. The information "comes into being" only after the legitimate users perform measurements and communicate in public afterwards. The eavesdropper may try to substitute his own prepared data for Alice and Bob to misguide them, but as he does not know which orientation of the analyzers will be chosen for a given pair of particles, there is no good strategy to escape from being detected. In this case his intervention will be equivalent to introducing elements of physical reality to the measurements of the spin components. This can be easily modified (by the eaves seen if we put appropriately correlation coefficients dropper perfect measurement)into Eq. (3). We obtain formula where n, and nb are two unit vectors (for particles a and b, respectively), oriented along the directions of the quantization axes for which the eavesdropper acquired information about the spin component of a given particle. This information could be acquired either through a direct, "brute" measurement of the spin components or through a more subtle attack on the source, e. g. , substituting a source that produces a state of two spin- 2 particles correlated with another quantum system on which the actual measurement will be performed by the eavesdropper. The normalized probability measure p(n„nb)describes the eavesdropper strategy (probability of intercepting a spin component along a given direction for a particular measurement).

This way it has been shown that the generalized Bell's theorem can have a practical application in cryptography, namely, it can test the safety of the a key distribution. It is not a mathematical difficulty of a particular computation, but a fundamental physical law that protects the system, and as long as quantum theory is not refuted as a complete theory the system is secure.

Regarding more refined attacks associated with the faked source of three (or more) correlated particles, one may think, for example, about delayed measurement on the third particle which is correlated with the two spin- 2particles. By "delayed" I mean "after the orientation of the analyzers has been publicly revealed by Alice and Bob. However„as we want the two particles to be in pure, singlet state, and Alice and Bob test for it through Bell's theorem, then we cannot correlate the third particle with the other two without disturbing the purity of the singlet state. Therefore I conjecture that there is no universal (good for all orientations a;, bj. ) state of the faked source which will pass the statistical test of the legitimate users on the subsystem of the two correlated particles a and b. As Alice and Bob can also delay their public communication, the eavesdropper faces the problem of storing the third particle undisturbed for an appropriately long period of time.

I have already mentioned that the proposed channel can be realized as a modification of experiments that tested Bell's theorem. In particular, the celebrated experiment of Aspect and co-workers [6], in which polarized particles, would be photons were used instead of spin the most obvious choice. In the experiment, every 10 ns pairs of photons were emitted in a radiative atomic cascade of calcium. Acousto-optical switches were used to change the orientation of the analyzers in a time short compared with the photon transit time, and the detection efficiency was over 95\%. Apart from changing the main objective of the experiment, and some details in the setup,one will also need software to simulate Alice, Bob, and optionally the eavesdropper. The modifications are minor, so it raises hopes for experimental realization in the nearest future.

The authors thanks D. Deutsch, P. L. Knight, K. Burnett, S. M. Barnett, C. H. Bennett, A. Zeilinger, P. Grangier, G. M. Palma, and P. G. H. Sandars for interesting comments and discussion. This work was supported by Pirie-Reid Fund at Oxford University.

\subsection{\trnas}


\subsubsection*{Аннотация}

Сообщается о практическом применении обобщенной теоремы Белла в так называемом процессе распределения ключей в криптографии. Предлагаемая схема основана на бомовской версии мысленного эксперимента Эйнштейна-Подольского-Розена, а теорема Белла используется для проверки на подслушивание.

\subsubsection{Основная часть}

Криптография, несмотря на красочную историю, восходящую к 400 году до нашей эры, стала частью математики и теории информации только в этом веке, в конце 1940-х годов, в основном благодаря основополагающим работам Шеннона [1]. Сегодня криптографию можно кратко определить как математическую систему преобразования информации таким образом, что она становится неразборчивой и, следовательно, бесполезной для тех, кто не должен иметь к ней доступ. Однако, поскольку вычислительный процесс, связанный с преобразованием информации, всегда выполняется физическими средствами, нельзя отделить математическую структуру от лежащих в ее основе законов физики, которые управляют процессом вычислений [2]. Дойч показал, что квантовая физика обогащает наши вычислительные возможности намного больше, чем классические машины Тьюринга[2], а текущая работа в области квантовой криптографии, начатая Беннетом и Брассардом, служит хорошим примером этого факта [3].

В этой статье я представлю метод, в котором безопасность так называемого процесса распределения ключей в криптографии зависит от полноты квантовой механики. Здесь полнота означает, что квантовое описание предоставляет максимально возможную информацию о любой рассматриваемой системе. Предлагаемая схема основана на известной версии эксперимента Эйнштейна-Подольского-Розена с геданкеном [4]; для проверки на подслушивание используется обобщенная теорема Белла (неравенства Клаузера-Хорна-Шимони-Холта) [5]. С теоретической точки зрения схема представляет собой интересное и новое расширение оригинальной идеи Беннета и Брассарда, а с экспериментальной точки зрения предлагает практическую реализацию путем небольшой модификации экспериментов, которые были поставлены для проверки теоремы Белла. Прежде чем продолжить, позвольте мне сначала ввести некоторые основные понятия криптографии.
Изначально безопасность криптотекста зависела от секретности всей процедуры шифрования и дешифрования, однако сегодня используются шифры, для которых алгоритм шифрования и дешифрования может быть раскрыт кому угодно без ущерба для безопасности конкретной криптограммы. В таких шифрах набор определенных параметров, называемых ключом, подается вместе с открытым текстом на вход алгоритма шифрования, а вместе с криптограммой - на вход алгоритма дешифрования. Алгоритмы шифрования и дешифрования объявляются публично; безопасность криптограммы полностью зависит от секретности ключа, а этот ключ, что очень важно, может состоять из любой случайно выбранной, достаточно длинной строки битов. После установления ключа последующая коммуникация предполагает передачу криптограмм по публичному каналу, который уязвим для полного пассивного перехвата (например, публичное объявление в СМИ). Однако для того, чтобы установить ключ, два пользователя, которые изначально не делятся секретной информацией, должны на определенном этапе общения использовать надежный и очень защищенный канал. Поскольку перехват представляет собой набор измерений, выполняемых подслушивающим устройством на этом канале, как бы трудно это ни было с технологической точки зрения, в принципе, любой классический канал всегда может быть пассивно проконтролирован, при этом законный пользователь не будет знать, что произошло подслушивание. Это не так для квантовых каналов [3]. Далее я описываю квантовый канал, который распространяет ключ без какого-либо "элемента реальности", связанного с ключом, и который защищен полнотой квантовой механики.

Канал состоит из источника, который испускает пары частиц со спином 2, находящихся в синглетном состоянии. Частицы разлетаются вдоль оси z в направлении двух законных пользователей канала, скажем, Алисы и Боба, которые, после того как частицы разделили спиновые компоненты, проводят измерения в одном из трех направлений, заданных единичными векторами a; и b, (i, =1, 2, 3), соответственно, для Алисы и Боба. Для простоты, оба вектора a; и bj лежат в плоскости x-y, перпендикулярной траектории частиц, и характеризуются азимутальными углами: Pi'=0, Pz = 4 tr, P3 и pi = -, x, p2 = -, tr, p3 = -, tr. Супер программы "a" и "b" относятся к анализаторам Алисы и Боба, соответственно, а угол измеряется от вертикальной оси x. Пользователи выбирают ориентацию анализаторов случайным образом и независимо для каждой пары входящих частиц. Каждый из 2 6 блоков может дать два результата, +1 измерение, (спин вверх) и - 1 (спин вниз), и потенциально может раскрыть один бит информации.
Величина E(a, , b, ) =P++(a, , b, )+P(a, , b, )- P+ (a;, b~) - P - +(a;, b~)- это коэффициент корреляции измерений, сформированных Алисой вдоль a; и Бобом вдоль b~. Здесь P+ ~ (a;, b~) обозначает вероятность того, что результат +'1 имеет 1 вдоль b~. Согласно полученным вдоль a; и квантовым правилам~ E(a, , b, ) = - a b . (2)одинаковой ориентации (a2, bi и a3, b2) квантовая механика предсказывает полную антикорреляцию результатов, полученных Алисой и Бобом:E(a, , b, ) =E(a, , b, ) = - 1.

Давайте также, следуя Клаузеру, Хорну, Шимони и Холту [5], определим величину, состоящую из коэффициентов корреляции, для которых Алиса и Боб использовали анализаторы разной ориентации.

После того, как передача произошла, Алиса и Боб могут публично объявить ориентации анализаторов, которые они выбрали для каждого конкретного измерения, и разделить измерения на две отдельные группы: первая группа, для которой они использовали анализаторы разной ориентации, и вторая группа, для которой они использовали анализаторы одинаковой ориентации. Они отбрасывают все измерения, в которых один из них или оба не смогли зарегистрировать частицу. Впоследствии Алиса и Боб могут публично раскрыть полученные ими результаты, но только в рамках первой группы измерений. Это позволяет им установить значение S, которое, если частицы не было, должно воспроизвести прямо или косвенно "возмущенный" результат уравнения (4). Это гарантирует законным пользователям, что результаты, полученные ими во второй группе измерений, являются антикоррелированными и могут быть преобразованы в секретную строку битов - ключ. Этот секретный ключ может быть затем использован в обычной криптографической коммуникации между Алисой и Бобом.

Подслушивающее устройство не может извлечь никакой информации из частиц во время их передачи от источника к законным пользователям, просто потому, что в них не закодирована никакая информация. Информация "появляется" только после того, как законные пользователи проводят измерения и после этого публично общаются. Подслушивающий может попытаться подставить Алисе и Бобу свои собственные подготовленные данные, чтобы ввести их в заблуждение, но поскольку он не знает, какая ориентация анализаторов будет выбрана для данной пары частиц, у него нет хорошей стратегии, чтобы избежать обнаружения. В этом случае его вмешательство будет эквивалентно внесению элементов физической реальности в измерения спиновых компонент. Это может быть легко модифицировано (с помощью карниза, если мы подставим соответствующим образом коэффициенты корреляции капельницы идеального измерения) в уравнение (3). Мы получим формулу, где n, и nb - два единичных вектора (для частиц a и b, соответственно), ориентированных вдоль направлений осей квантования, для которых подслушивающий получил информацию о спиновой компоненте данной частицы. Эта информация может быть получена либо путем прямого, "грубого" измерения спиновых компонент, либо путем более тонкой атаки на источник, например, подстановкой источника, который производит состояние двух частиц со спином 2, коррелированное с другой квантовой системой, на которой подслушивающее устройство будет проводить фактическое измерение. Нормированная вероятностная мера p(n "nb)описывает стратегию подслушивающего устройства (вероятность перехвата спиновой компоненты вдоль заданного направления для конкретного измерения).

Таким образом, было показано, что обобщенная теорема Белла может иметь практическое применение в криптографии, а именно, она может проверить безопасность распределения ключей. Это не математическая трудность конкретного вычисления, а фундаментальный физический закон, который защищает систему, и до тех пор, пока квантовая теория не опровергнута как полная теория, система безопасна.

Что касается более тонких атак, связанных с поддельным источником трех (или более) коррелированных частиц, можно подумать, например, об отложенном измерении третьей частицы, которая коррелирована с двумя частицами со спином 2. Под "отложенным" я имею в виду "после того, как ориентация анализаторов будет публично раскрыта Алисой и Бобом". Однако "поскольку мы хотим, чтобы две частицы находились в чистом, синглетном состоянии, и Алиса и Боб проверяют это с помощью теоремы Белла, то мы не можем соотнести третью частицу с двумя другими без нарушения чистоты синглетного состояния. Поэтому я предполагаю, что не существует универсального (хорошего для всех ориентаций a;, bj. ) состояния поддельного источника, которое пройдет статистическую проверку легитимных пользователей на подсистеме двух коррелированных частиц a и b. Поскольку Алиса и Боб могут также задержать свое публичное сообщение, подслушивающий сталкивается с проблемой хранения третьей частицы в невозмущенном состоянии в течение достаточно длительного периода времени.

Я уже упоминал, что предложенный канал может быть реализован как модификация экспериментов, проверявших теорему Белла. В частности, наиболее очевидным выбором является знаменитый эксперимент Аспекта и соавторов [6], в котором вместо спина использовались поляризованные частицы, которые могли бы быть фотонами. В эксперименте каждые 10 нс пары фотонов испускались в радиационном атомном каскаде кальция. Акустооптические переключатели использовались для изменения ориентации анализаторов за время, малое по сравнению со временем прохождения фотонов, и эффективность обнаружения составила более 95\%. Помимо изменения основной цели эксперимента и некоторых деталей в установке, потребуется программное обеспечение для моделирования Алисы, Боба и, по желанию, подслушивающего устройства. Изменения незначительны, поэтому это позволяет надеяться на экспериментальную реализацию в ближайшем будущем.

Авторы благодарят Д. Дойча, П. Л. Найта, К. Бернетта, С. М. Барнетта, К. Х. Беннета, А. Цайлингера, П. Гранжье, Г. М. Пальму и П. Г. Х. Сандарса за интересные комментарии и обсуждение. Эта работа была поддержана фондом Pirie-Reid в Оксфордском университете.


\subsection{\review}
The E91 protocol, published in [4], represents the second group of quantum key distribution protocols.

This protocol is based on the phenomenon of quantum entanglement. The essence of the phenomenon is that when measuring a state of entangled quantum objects, the results of measurements turn out to be directly opposite. Measurement destroys the state of quantum objects and there is no way to learn about the state of the object before its measurement.

The authors of the paper proposed using a special laser as a source of entangled photons. When trying to read an entangled photon, the intruder destroys the connection between photons and the recipient, with the help of Bell's inequality, can determine the presence of the intruder in the communication channel.


\subsection{\dic}
\begin{multicols}{2}
	\begin{itemize}
		
		\item algorithms - алгоритм
		\item analysis - анализ
		
		\item appropriate - подходящий
		\item approximately - примерно
		
		
		\item basis - основа
		\item beam - луч
		
		\item binary - двоичных
		\item bit - бит
		
		\item capacity - вместимость
		
		\item channel - канал
		
		\item coherent - связный
		\item combination - комбинации
		
		\item communication - связь
		\item compare - сравнить
		
		\item computation - вычисления
		\item computers - компьютеров
		
		\item condition - условие
		\item conjugate - спряжение
		\item considered - рассмотрено
		\item contain - содержат
		
		\item correlation - корреляция
		
		\item cryptography - криптография
		
		\item decode - декодировать
		\item decoy - ловушка
		\item density - плотность
		
		\item dependence - зависимость
		\item detect - обнаружить
		
		\item deterministic - детерминированный
		
		\item developing - разработка
		
		\item difference - разница
		
		\item differentiate - дифференцировать
		
		\item distribution - распределение
		
		\item eavesdropper - подслушиватель
		
		\item encoded - закодировано
		\item entaglement - запутанность
		\item equivalently - эквивалентно
		
		\item imperfections - недостаток
		\item implementation - реализация
		
		\item instances - экземпляров
		
		\item intensity - интенсивность
		\item intercept - перехват
		
		\item interferometer - интерферометр
		
		\item limitation - ограничение
		
		\item lossless - без потерь
		\item lossy - потери
		
		\item malicious - вредоносных
		\item managed - управляемых
		
		\item mathematical - математических
		\item matrix - матрица
		
		\item measurement - измерения
		
		\item method - метод
		\item mimic - имитировать
		
		\item modification - модификация
		
		\item neglected - пренебрегают
		
		\item normalized - нормализовано
		
		\item operator - оператор
		\item optical - оптический
		\item optimal - оптимальный
		
		\item optional - опционально
		
		\item orthogonal - ортогональный
		\item orthogonality - ортогональность
		\item orthonormal - ортонормированный
		
		\item phase - фаза
		\item photon - фотон
		
		\item polarization - поляризация
		
		
		\item probabilistic - вероятностный
		\item probability - вероятность
		\item problem - проблема
		
		\item projecting - проектирование
		
		\item property - свойство
		\item proportion - пропорция
		\item proposal - предложение
		
		\item protocol - протокол
		
		\item prove - доказательство
		
		\item provide - обеспечить
		\item provides - обеспечивает
		
		\item public - общедоступных
		\item pulse - импульс
		
		\item quantum - квантовый
		\item random - случайных
		\item randomize - рандомизировать
		
		\item signature - подпись
		
		\item strategy - стратегия
		
		\item string - строка
		
		\item symmetric - симметричный
		
		\item theoretical - теоретический
		\item theory - теория
		
		\item transmission - передача
		
		\item vacuum - вакуум
		\item value - значение
		
		\item variable - переменная
		
		
		
	\end{itemize}
\end{multicols}

\section{High error-rate quantum key distribution for long-distance communication }

\subsection{Article}

\subsubsection*{Abstract}

In the original BB84 protocol by Bennett and Brassard, an eavesdropper is detected because his attempts to intercept information result in a quantum bit error rate (QBER) of at least 25\%. Here we design an alternative quantum key distribution protocol, where Alice and Bob use two mutually unbiased bases with one of them encoding a ‘0’ and the other one encoding a ‘1’. The security of the scheme is due to a minimum index transmission error rate (ITER) introduced by an eavesdropper that increases significantly for higher dimensional photon states. This allows for more noise in the transmission line,thereby increasing the possible distance between Alice and Bob without the need for intermediate nodes.

\subsubsection{Introduction}

The aim of quantum cryptography is to establish a shared, secret and random sequence of bits between a sender, called Alice and a receiver, called Bob [1]. This sequence constitutes a perfect cryptographic key, a so-called one-time pad, and allows Alice and Bob to securely encrypt a message of the same length. The cryptographic key is obtained solely via the transmission of photons and classical communication. Each bit is encoded in the state of a single photon and read out by Bob upon arrival via a quantum measurement. Random switching between different bases makes it impossible for an eavesdropper, called Evan, to predict the states used in the protocol. All his attempts to intercept photons result in a significant quantum bit error rate(QBER). This guarantees a high level of security, since Evan’s presence is detected easily when Alice and Bob compare a number of test bits.

Under ideal conditions, the exchange of single photons allows Alice and Bob to establish a cryptographic key over an arbitrarily long distance. In practice, cryptographic setups consist of imperfect single-photon sources, lossy transmission lines, and photon detectors with dark count rates. Alice and Bob must hence apply classical information processing tools like error correction and privacy amplification [2, 3] to their data in order to obtain identical secret keys. However, cryptographic protocols are only secure as long as it is possible to detect the presence of an eavesdropper. System errors could cloud Evan’s presence; especially since he could simply replace parts of the equipment with high-quality components. This makes it impossible to tolerate large system errors and limits the possible distance between Alice and Bob.

Recently, Rosenberg et al [4, 5] reported the creation of a secure cryptographic key over a distance of 144.3 km of optical fiber. Their scheme is based on a decoy state protocol [6]–[8]which is immune to photon number splitting attacks and highly resistant to Trojan horse attacks [9]. It is expected that improvements in filtering of blackbody photons might allow for an extension of the fiber to 250 km. In the mean time, Takesue et al [10] and Stuckiet al [11] created a secure cryptographic key over a distance of 200 km of optical fiber. These experimental setups are believed to be the longest terrestrial quantum key distribution fiber-links yet demonstrated. Comparable distances have been achieved in free space. For example, Schmitt-Manderbach et al [12] securely distributed a cryptographic key over a
144 km free-space link.

In this paper, we design a novel quantum key distribution protocol whose efficiency and minimum error rate in the case of eavesdropping increase with the dimension of the photon states used by Alice and Bob. In this way, we increase the threshold for tolerable system errors without sacrificing the security of the protocol and hence increase the possible distance between Alice and Bob. In principle, single photons could be purified with the help of quantum repeaters [13]. Proposals for their implementation (see e.g. [14, 15]) and other noise reducing links [16]–[18] have been made but their experimental implementation and their practical integration into cryptographic networks remains to be seen.

The above-mentioned long-distance quantum key distribution schemes [4, 5, 10, 12] are all based on the BB84 protocol by Bennett and Brassard [19]. In BB84, Alice encodes her bits in two-dimensional photon states. These can be obtained using polarization encoding. However, a more natural choice is time-bin encoding, which affords better protection of the photons against decoherence [20]. Alice and Bob independently vary their bases between two possibilities. A key bit is obtained whenever both use the same basis. This means, on average, every second photon contributes a bit to the cryptographic key. Using a simple intercept–resend strategy, an eavesdropper introduces a QBER of at least 25\% into the communication.

In the following, we assume that Alice and Bob use time-bin or path encoded N -dimensional photon states. As in BB84, Alice and Bob randomly vary their bases between two mutually unbiased bases [21]. However, contrary to BB84, Alice and Bob detect the presence of an eavesdropper by calculating the index transmission error rate (ITER). As we shall see below, for N = 2, Evan causes a minimum ITER of 25\%. In the case of four dimensional photon states, this error rate becomes 37.5\%. When increasing N further, the minimum ITER approaches 50\%. The efficiency of the protocol in units of transmitted bits per photon is the same as the minimum ITER in case of eavesdropping. The states required by the proposed key distribution scheme can be realized using a symmetric Bell multiport beam splitter [22]. Before the transmission, the path encoding of the output states of the Bell multiport beam splitter should be switched to the above-mentioned time-bin encoding [20].

Several generalizations of quantum cryptographic schemes to higher dimensions have already been proposed. The papers [23]–[29] are generalizations of the original BB84protocol [19] based on the encoding of information in higher-dimensional alphabets. However,recently it has been shown [31]–[33] that the security of the BB84 protocol is entirely compromised if Alice and Bob share, for example, four-dimensional photon states in this way [34]. Beige et al [35, 36] propose two alternative quantum cryptographic protocols using four-dimensional photon states, which, under ideal conditions, allow Alice and Bob to communicate directly but whose minimum error rates in the case of eavesdropping are relatively low.

Here we show that there are other possible cryptographic schemes in higher dimensions. A sin [35, 36], Alice and Bob use two bases e and f with all states of e encoding a ‘0’ and all states of f encoding a ‘1,’ even when N is larger than two. This means, contrary to [19], [23]–[29],all vectors of the same basis encode the same bit. Moreover, a bit can be transmitted only when Alice and Bob use different bases. The quantum key distribution scheme considered in this paper is designed such that no conditions have to be posed on the states of e and f , thereby giving us a lot of flexibility when maximizing the relevant minimum error rate introduced by Evan.

For simplicity, we consider only intercept–resend eavesdropping attacks. This is not the only possible eavesdropping attack, but security against this is considered a strong indication for the general security of a quantum cryptographic protocol.

In the special case of N = 2, the cryptographic scheme proposed in this paper is essentially equivalent to the SARG quantum key distribution protocol [30] with the parameter x chosen equal to 1/ 2. This protocol is tailored to be robust against photon number splitting attacks. In the SARG protocol, Alice publicly announces which one of the four different sets of states A +,+ ,A +,- , A -,+ and A -,- she used, while our protocol requires her only to announce either ‘i = 1’or ‘i = 2’. This difference is due to a redundancy in the SARG protocol.
There are five sections in this paper. In section 2, we introduce the notations used throughout this paper. In section 3, we calculate the ITER and the QBER for the quantum key distribution protocol introduced in section 2 as a function of the states used by Alice, Bob and Evan analytically. Afterwards, we determine their minima in the presence of intercept–resend eavesdropping attacks for different N s numerically. Geometrical considerations suggest tha tAlice and Bob should use two mutually unbiased bases. Section 4 analyses a concrete protocol based on this idea and shows that mutually unbiased bases indeed guarantee a high ITER and a high QBER in case of eavesdropping. Finally, we summarize our results in section 5.

\subsubsection{Alternative design}

In quantum cryptography there are conventionally three parties, Alice, Bob and Evan. Alice wants to transmit a sequence of secret bits to Bob. To do so, she prepares single photons in certain states and sends them to Bob. Bob measures the state of each incoming photon. Afterwards, Alice and Bob exchange information via classical communication. At the same time, Evan tries to catch the secret bits without revealing his presence. For example, he measures the state of every transmitted photon and listens in to the classical communication between Alice and Bob. The cryptographic protocol is secure as long as Evan’s attempts to obtain information result in an error rate which can be detected easily.

Let us start by introducing sufficient conditions for such a protocol to work:

1. Bob should measure the incoming photons in a randomly chosen basis. Otherwise, Evan simply uses the same measurement basis and the bit error rate remains zero. This means Bob should randomly switch between at least two sets of basis states. In the following weas sume that this is the case and denote these bases e = {|e i i: i = 1, . . . , N }and f ] {| f i i: i = 1, . . . , N }.

Here N is the dimension of the photon states. The only condition imposed on e and f is that they form a basis.

2. Similarly, Alice should encode the information that she wants to transmit to Bob such that it cannot be deduced easily by Evan. To obtain a non zero error rate in case of eavesdropping,she should either use non-orthogonal states (as in B92 [37]) or randomly switch between at least two sets of basis states (as in BB84 [19]). For simplicity, we assume in the following that Alice prepares each photon randomly in one of the basis states of e and f .

3. In order to establish strong correlations between Alice’s input state and Bob’s measurement outcome, Alice needs to reveal some information via classical communication. This information should be enough for Alice and Bob to obtain a shared secret key bit but not enough for the eavesdropper to deduce it. One possibility is that Alice announces which basis, e or f , she used (as in BB84 [19]). Another possibility is that Alice reveals the index i of the respective basis state (as in [35, 36]). This does not reveal any information about the key as long as the states |e i i and | f i i with the same index i encode different bits. In this paper, we consider this second approach and show that it can guarantee relatively high error rates in the presence of an eavesdropper.

4. We now have a closer look at how Bob should interpret his measurement outcomes after Alice told him the index i of her basis state. He can do this by using a table like table 1. If Alice announces that she prepared the photon in a state with index i, Bob obtains ‘n i j ’when he measures |e j i and he obtains ‘u i j ’ when he measures | f j i. The parameters n i j and u i j in the table assume three different values, ‘0,’ ‘1’, or ‘x’, depending on whether Bob obtains a ‘0’, a ‘1’, or no key bit is transmitted.
Suppose Alice sends a photon prepared in |e 1 i in order to transmit a ‘0’. This implies n 11 = ‘0’ or ‘x’(2)in order to avoid that Bob obtains a wrong key bit. The state | f 1 i has to encode a ‘1’ in this case. Otherwise, Evan knows that a ‘0’ is transmitted, when ‘i = 1’ is announced.
Consequently,u 11 = ‘1’ or ‘x’. Moreover, if Alice announces ‘i = 1’ and Bob measures | f j i with j 6= 1, then he knows for sure that she prepared her photon in |e 1 i. Analogously, if Alice announces ‘i = 1’ and Bob measures |e j i with j 6= 1, then he knows that Alice prepared | f 1 i. Alice and Bob should therefore choose n 1 j = ‘1’ and u 1 j = ‘0’, for all j 6= 1.
(4)There is no need for Bob to ignore a measurement outcome with j 6= 1 since he always knows which key bit the photon encodes in this case.

5. It is indeed possible (cf equations (2)–(4)) that table 1 contains no crosses and that every detected photon transmits one bit of the cryptographic key. However, the minimum error rate in the case of eavesdropping is already known to be relatively low in this case [35, 36]. We therefore assume here that Bob ignores the cases where his measured state has the index i announced by Alice and choose n ii = u ii = ‘x’.

This means a key bit can only be obtained when Bob’s measurement basis is different from the one used by Alice to prepare the photon. One can easily check that Alice and Bob always obtain the same secret key bit under ideal conditions.

6. For symmetry reasons, Alice should have equally many states to encode a ‘0’ as she has to encode a ‘1’. Without restrictions we therefore assume in the following that all the states o fe encode a ‘0’ while all states of f encode a ‘1’. This means, Bob obtains a ‘1’ whenever he measures a state |e j i with j different from Alice’s index i. Analogously, he obtains a‘0’ when he measures a state | f j i with j 6= i.

The final protocol is summarized in table 2 for the case where Alice and Bob communicate using four-dimensional photon states. For arbitrary N , the entire scheme works as follows:

1. Alice generates a random key sequence of classical bits and randomly assigns each bit value a random index i = 1, 2, . . . , N .
2. Alice then uses this sequence and sends single photons prepared accordingly either in |e i i or | f i i to Bob.
3. Bob measures the state of every incoming photon, thereby randomly switching the measurement basis between e and f .
4. Alice publicly announces the random sequence of indices i used to establish the cryptographic key.
5. Bob interprets his measurement outcomes accordingly, using, for example, table 2, if N = 4. He obtains a key bit whenever his index is different from the index announced by Alice.
6. Bob tells Alice which photon measurements have been successful and provide a bit of the secret key.
7. Finally, Alice and Bob determine whether an eavesdropper introduced an error into their communication. Whenever this error rate is sufficiently small, Alice and Bob can assume that no eavesdropping has occurred.
Note that no conditions are imposed on e and f in this section other than them being bases. This gives us a lot of flexibility when maximizing the security of the corresponding cryptographic protocol. In fact, the only difference between the above protocol and the direct communication scheme introduced in [35, 36] is that we avoid the assumption of he i | f i i being zero. Alice and Bob therefore have to discard their measurement outcomes when both their states have the same index. As we shall see below, the payoff for the corresponding loss inefficiency is a relatively high error rate in the presence of an eavesdropper.


\subsubsection{Eavesdropping}

In the quantum key distribution protocol proposed here, the index i of the photon state in transmission is the only publicly announced information. This index does not reveal any information about the corresponding key bit since it equally likely encodes a ‘0’ a sit encodes ‘1’. An eavesdropper can therefore only learn about the cryptographic key by performing quantum measurements on the transmitted photons. In the following, we assume that Evan measures the state of every photon using a basis g which is optimal for his purpose. Afterwards, he forwards his measurement outcome to Bob. This eavesdropping strategy is known as an intercept–resend attack. Although it is not the most general eavesdropping attack,the corresponding error rate is a strong indication for the security of a cryptographic protocol. Our aim is to increase the minimum error rate introduced by an eavesdropper above the 25\% of the original BB84 protocol [19]. As already mentioned above, there are different types of errors that Alice and Bob can consider.

In the following, i denotes again the index of the photon state prepared by Alice and j is the index of the basis vector measured by Bob. When Alice and Bob use different bases, j can assume any value between 1 and N , even in the absence of any eavesdropping. However, when Alice and Bob use the same basis, i and j should be the same. To detect Evan’s presence,Alice and Bob could therefore do the following: Alice should randomly select some photons that should not be used to obtain key bits. For these photons, she tells Bob exactly which states she prepared. Comparing this information with his own measurement outcomes, Bob can then easily calculate the ITER.

An index transmission error occurs when a photon prepared in |e i i (| f i i) is measured at Bob’s end as |e j i (| f j i) with i 6= j. Assuming that Alice prepares the 2N basis states of e and f with the same frequency and that Bob measures e and f with the same frequency, we find that the ITER of the proposed protocol equals N for a given set of bases e, f and g. The states |g k i denote Evan’s possible measurement outcomes, which are forwarded to Bob without alteration. In principle, Evan could change the state of the transmitted photon by guessing which state Alice prepared. However, this strategy is not expected to reduce the above error rate significantly. A non zero overlap between the basis states of e and f ensures that there is always a certain probability to guess incorrectly.

The same expression is obtained when calculating this error rate by subtracting the probability of not making an error under the condition that Alice and Bob use the same basis from unity.

While Alice and Bob want the error rate in equation (8) to be as large as possible, Evan wants it to be as small as possible. Both parties, with Alice and Bob on one side and Evan on the other side, try to optimize the choice of the bases e, f and g accordingly. Table 3shows the results of a numerical solution of this double-optimization problem for different dimensions N . To obtain this table, the basis e is kept fixed and a large number of bases f is generated randomly. For each f we then generate another large set of random bases g and determine the minima of the corresponding error rates using equation (8). This is illustrated in figure 1 for the N = 4 case. The ITER in table 3 is the maximum of all the obtained minimum error rates.
For N = 2, we find that the minimum ITER introduced by an eavesdropper equals 25\% when Alice and Bob use an optimal choice of e and f , as in the original BB84 protocol [19]. However, when Alice and Bob increase the dimension N of their photon states, the minimum ITER increases. One can easily see that 50\% constitutes an upper bound for the minimum ITER by considering the case where Evan measures the incoming photons either in the e or in the f basis. Using this eavesdropping strategy, the states of at least half of the transmitted photons remain unaffected.

Alternatively, Alice and Bob can detect a potential eavesdropper by calculating the usual QBER. To do so, both randomly select a certain number of control bits from the obtained key sequence and compare them openly. Note that a key bit is obtained when the index j of the state measured by Bob and the index i of the state prepared by Alice are different. Bob interprets his measurement result correctly only when both states belong to a different bases. A quantum bit error hence occurs when Bob measures |e j i (| f j i), while Alice prepared |e i i (| f i i) with i 6= j. Using equation (6), the QBER for a given set of bases e, f and g can be written asP QBER =P ITER,
2P IC(9)where the index-change (IC) probability P IC ,N. The third column in table 3 shows the minimum QBER introduced by an eavesdropper, when Alice and Bob use optimal bases e and f , for different dimensions N . As in section 3.1,these probabilities have been obtained by comparing probabilities for a large set of randomly generated bases f and g.

Even for N = 2, the minimum QBER can be as high as 50\%. A more detailed analysis of the corresponding protocol shows that Alice and Bob can realize this scenario by choosing e and f almost identical, independent of Evan’s choice of measurement basis g. However, the price they pay for this very high QBER is a steep drop in the efficiency of their quantum key distribution. In the extreme case, where |e 1 i = | f 1 i and |e 2 i = | f 2 i, it becomes impossible to generate secret key bits, since Alice’s and Bob’s state always have the same index i, at least in the absence of any eavesdropping. In the following, we assume therefore that Alice and Bob use the ITER in order to detect the presence of an eavesdropper.

\subsubsection{Optimal choices of e , f and g}

We now address the question of how Alice and Bob can take advantage of the high ITERs shown in table 3 by having a look at possible realizations of the proposed quantum key distribution protocol. First, we consider the N = 2 case, which suggests an optimal strategy for Alice and Bob in higher dimensions. Moreover, we discuss in this section what Evan can do to best cloud his presence.

The problems that Alice, Bob and Evan have to solve in the N = 2 case in order to optimize their strategies are exactly the same as in BB84 [19]. Suppose Alice and Bob choose (cf figure 2)

In order to maximize the minimum of this probability with respect to fi 2 , Alice and Bob should choose fi 1 = 14 pi. In this case, sin 2 (2(fi 1 - fi 2 )) becomes the same as cos 2 (2fi 2 ). This error rate equals 25\% independent of Evan’s choice of fi 2 . For the eavesdropper, every possible strategy is hence an optimal one.

In other words, for N = 2, Alice and Bob’s optimal choice for e and f are two mutually unbiased bases [21]. This means, upon measurement, a photon prepared in any of the basis states of e is found with equal probability in any of the basis states of f and vice versa. As shown in figure 2, e and f should be as far away from each other as possible. A straightforward generalization of this result to higher dimensions suggests that Alice and Bob should always us etwo mutually unbiased bases e and f .

Let us now have a closer look at the case where Alice and Bob communicate with four dimensional photon states. 

Since the second term in the brackets is always positive, one can easily see that P ITER > 37.5 \% .
Indeed, the best strategy for Evan is to choose sin(2a) = 0. This means, Evan should measure either e or f .
Figure 3 shows the error rate introduced by Evan for the above choice of e and f and fora large set of randomly generated gs with N = 4. It confirms that P ITER is always above 37.5\% if Alice and Bob use two mutually unbiased bases. This applies even when no assumption on the form of the states of g is made, as we do for our analytical calculations in equation (18). A comparison of P ITER = 37.5\% with the result in table 3 for N = 4 confirms that this error rate corresponds to (or is at least very close to) an optimal strategy of Alice, Bob and Evan. Both results agree within the error limits of the underlying numerical calculation.

Let us now have a look at the optimal choice of e, f and g for the general case where Alice and Bob use N -dimensional photon states. As suggested at the end of section 4.1, we assume that Alice and Bob use two mutually unbiased bases. More concretely, we assume that their states are given by 2. One can easily check that e and f are orthonormal and mutually unbiased. We then generate a large set of random gs and calculate the corresponding error rates P ITER using equation (8). Table 4 shows the maxima of these rates as a function of N . The given error rates hence correspond to Evan’s optimal intercept–resend eavesdropping strategy.

A comparison of the ITERs in table 4 with the ITERs in table 3 confirms that using two mutually unbiased bases e and f is (at least close to) an optimal strategy for Alice and Bob. For N = 2, we find again that the minimum error rate introduced by Evan equals 25\%. For N = 3this rate equals 33\%, and for higher-dimensional photon states, the values in the third column of table 4 approach their predicted maximum of 50\% (cf section 3). The second column has been obtained from a numerical optimization of Evan’s strategy. Since it is a relatively hard computational problem to find the best eavesdropping measurement basis g numerically, the errors in this column are relatively large, especially for large N .


Let us now have a closer look at the best intercept–resend eavesdropping strategy for Evan. The discussion of the N = 4 case in section 4.2 suggests that Evan should measure either e or f in order to minimize the bit transmission error rate. If e and f are mutually unbiased,then the probability of detecting a photon in |e i i equals 1/N when Alice prepares an f -state. Analogously, the probability of detecting a photon in | f i i equals 1/N when Alice prepares an e-state. Substituting this into equation (8) yields N - 1P ITER.

For completeness we mention that the corresponding QBER (cf equation (11)) equals 33\% independent of N . A comparison with a numerical evaluation of the ITER confirms that measuring either e or f and forwarding the respective measurement outcome to Bob is indeed an optimal (or at least a close to optimal) strategy for Evan, if Alice and Bob test his presence by calculating this error rate.
Let us conclude this subsection by commenting on the efficiency of the described quantum key distribution scheme. As in BB84, Alice and Bob randomly switch between two sets of basis states. Here a key bit can only be obtained when both use a different basis. Moreover, the index of the state measured by Bob should be different from the index of Alice’s state. The probability for this to happen and hence the mean number of bits per transmitted photon equals N - 1P success.

This expression is exactly the same as the ITER in equation (21).

In order to implement the above protocol, Alice needs a single-photon source. As in BB84, the photon can come from a parametric down conversion crystal, a very weak laser pulse, or an on-demand single-photon source. Using path encoding, the states of f can be prepared easily with the help of a Bell multiport beam splitter. Such a beam splitter may consist of a network of beam splitters and phase plates [38, 39], which have to be interferometrically stable. It can also be made by splicing N optical fibers [40]. Spliced fibre constructions are commercially available and can include between three and thirty input and output ports.

The main feature of a symmetric N x N Bell multiport beam splitter is that a photon entering any of its input ports is redirected with equal likelihood to any of its N possible out putports. One way for Alice to prepare the state |e i i in equation (20) is to bypass the beam splitter and to send a single photon directly to output port i. In this case, preparing the state | f i i inequation (20) only requires to send a single photon into input port i [22]. Bob can use the same setup as Alice to decode the key bit. To measure f , he should send the incoming photon first through a Bell multiport beam splitter and then detect it in one of the N output ports. To measure e, he can simply bypass this step. During the transmission, the path encoding should be switched to time-bin encoding, which promises a better protection of the photons against decoherence [20].

\subsubsection{Conclusions}

In this paper, we propose a quantum key distribution protocol where Alice and Bob use higher-dimensional photon states. The scheme does not encode information in a higher dimensional alphabet [23]–[29] and is not a straightforward generalization of the original BB84protocol [19]. Instead, Alice and Bob use two bases e and f with all e-states encoding a ‘0’and all f -states encoding a ‘1’. Under ideal conditions, a key bit is obtained when Alice and Bob use different bases. In section 2, no conditions are imposed on the states of e and f , which gives us the flexibility to maximize the error rate introduced by an eavesdropper in the case of an intercept–resend attack. This is not the only possible eavesdropping strategy but security against this is a strong indication for the general security of a cryptographic protocol.

In section 3, we generate large sets of random basis states and determine the minimum ITER introduced by an eavesdropper numerically. For N = 2, this error rate equals the 25\% QBER of the BB84 protocol [19]. However, the minimum ITER rapidly approaches 50\% in higher dimensions (cf table 3). A detailed analysis of the N = 2 and 4 cases suggests that Alice and Bob should use two mutually unbiased bases e and f . The best an eavesdropper can do to hide his presence is to measure the transmitted photons either in e or f . This hypothesis is consistent with the numerical results in section 4 (cf table 4). Finally, we point out that the proposed quantum key distribution protocol can be implemented for example with the help of a symmetric Bell multiport beam splitter [22] and switching from path to time-bin encoding during the transmission. The mean number of key bits per transmitted photon turns out to be exactly the same as the minimum error rate introduced by Evan (cf equation (22)).
In section 3, we point out that it is in principle possible to obtain a minimum QBER close to 50\%, even for N = 2. This requires Alice and Bob to use two bases e and f , which are almost identical. Unfortunately, this strategy corresponds to a very low efficiency of the proposed cryptographic protocol. For e = f , the key transmission rate drops to zero. Analytic expressions for the QBER and the efficiency of the bit transmission in the presence of an eavesdropper for a given set of bases can be found in equations (10) and (11).

We thank H Zbinden for helpful comments. MMK thanks J Xu for encouraging discussions and acknowledges funding from the NED University of Engineering and Technology, Karachi,Pakistan. AB thanks the project students S So and P Woodward for their initial participation in this project and acknowledges a James Ellis University Research Fellowship from the Royal Society and the GCHQ. This project was supported in part by the UK Engineering and Physical Sciences Research Council through the QIP IRC and the EU Research and Training Network EMALI.

\subsection{\trnas}
ss
\subsection{\review}
ss
\subsection{\dic}
ss

\section{Decoy state quantum key distribution}

\subsection{Article}

\subsubsection*{Abstract}

There has been much interest in quantum key distribution. Experimentally, quantum key distribution over 150 km of commercial Telecom fibers has been successfully performed. The crucial is sue in quantum key distribution is its security. Unfortunately, all recent experiments are, in principle,insecure due to real-life imperfections. Here, we propose a method that can for the first time make most of those experiments secure by using essentially the same hardware. Our method is to use decoy states to detect eavesdropping attacks. As a consequence, we have the best of both worlds—enjoying unconditional security guaranteed by the fundamental laws of physics and yet dramatically surpassing even some of the best experimental performances reported in the literature.

\subsubsection{Introduction}

Quantum key distribution (QKD) allows two users, Alice and Bob, to communicate in absolute security in the presence of an eavesdropper, Eve. Unlike conventional cryptography, the security of QKD is based on the fundamental laws of physics, rather than unproven computational assumptions. The security of QKD has been rigorously proven in a number of recent papers [1]. See also[2]. There has been tremendous interest in experimental QKD [3, 4], with the current world record distance of 150km of Telecom fibers[4].

Unfortunately, all those exciting recent experiments are, in principle, insecure due to real-life imperfections.
More concretely, highly attenuated lasers are often used as sources. But, these sources sometimes produce signals that contain more than one photons. Those multi-photon signals open the door to powerful new eavesdropping attacks including photon splitting attack. For example,Eve can, in principle, measure the photon number of each signal emitted by Alice and selectively suppress single photon signals. She splits multi-photon signals, keeping one copy for herself and sending one copy to Bob. Now,since Eve has an identical copy of what Bob possesses,the unconditional security of QKD (in, for example, standard BB84 protocol[5]) is completely compromised.

In summary, in standard BB84 protocol, only signals originated from single photon pulses emitted by Alice are guaranteed to be secure. Consequently, paraphrasing GLLP [6], the secure key generation rate (per signal state emitted by Alice) can be shown to be given by:S  Q  {H 2 (E  ) + [1  H 2 (e 1 )]}(1)where Q  and E  are respectively the gain and quantum bit error rate (QBER) of the signal state [7], s and e 1 are respectively the fraction and QBER of detection events by Bob that have originated from single-photon signals emitted by Alice and H 2 is the binary Shannon entropy.

It is a priori very hard to obtain a good lower bound on sigma and a good upper bound on e 1 . Therefore, prior art methods (as in GLLP [6]) make the most pessimistic assumption that all multi-photon signals emitted by Alice will be received by Bob. For this reason, until now, it has been widely believed that the demand for unconditional security will severely reduce the performance of QKD systems [6, 8, 9, 10, 11].

In this paper, we present a simple method that will provide very good bounds to sigma and e 1 . Consequently, our method for the first time makes most of the long distance QKD experiments reported in the literature unconditionally secure. Our method has the advantage that it can be implemented with essentially the current hardware. So,unlike prior art solutions based on single-photon sources,our method does not require daunting experimental developments. Our method is based on the decoy state idea first proposed by Hwang [12]. While the idea of Hwang was highly innovative, his security analysis was heuristic. The key point of the decoy state idea is that Alice prepares a set of additional states—decoy states, in addition to standard BB84 states. Those decoy states are used for the purpose of detecting eavesdropping attacks only, whereas the standard BB84 states are used for key generation only. The only difference between the decoy state and the standard BB84 states is their intensities(i.e., their photon number distributions).

By measuring the yields and QBER of decoy states, we will show that Alice and Bob can obtain reliable bounds to s and e 1 , thus allowing them to surpass all prior art results substantially [13]. Here, we give for the first time a rigorous analysis of the security of decoy state QKD. Moreover, we show that the decoy state idea can be combined with the prior art GLLP [6] analysis.

Preliminary versions of our result in this paper have appeared in [14, 15], where we presented not only the general theory, but also proposed the idea of using only a few decoy states (for example, three states—the vacuum,a weak decoy state with u decoy << 1 and a signal state with u = O(1). We call this a Vacuum+Weak decoy state protocol). Subsequently, our protocols for decoy state QKD have been analyzed in [16] and more systematically in [17]. See also [18]. Recently, we have provided the first experimental demonstration of decoy state QKD in [19].

We now present the general theory of our new decoy state schemes. We will assume that Alice can prepare phase-randomized coherent states and can turn her powerup and down for each signal. This may be achieved by using standard commercial variable optical attenuators(VOAs) [20]. Let | ue i0 i denote a weak coherent state emitted by Alice. Assuming that the phase, 0, of all signals is totally randomized, the probability distribution for the number of photons of the signal state follows a Poisson distribution with some parameter u. That is to say that, with a probability p n = e -u u n /n!, Alice’s signal will have n photons. In summary, we have assumed that Alice can prepare any Poissonian (with parameter u) mixture of photon number states and, moreover, Alice can vary the parameter, u, for each individual signal.

Let us consider the gain Q u for a coherent state
| e i i. [Here and thereafter, we actually mean the ran
do m mixture of | ue io i over all values of 0 as the phase is assumed to be totally randomized].

Similarly, the QBER can depend on the photon number. Let us define e n as the QBER of an n-photon signal. The QBER E u for a coherent state | ue i0 i is given.

Essence of the decoy state idea Let us imagine that a decoy state and a signal state have the same characteristics (wavelength, timing information, etc). Therefore,Eve cannot distinguish a decoy state from a signal state and the only piece of information available to Eve is the number of photons in a signal. Therefore, the yield, Y n ,and QBER, e n , can depend on only the photon number,n, but not which distribution (decoy or signal) the state is from. We emphasize that the essence of the decoy state idea can be summarized by the following two equations:Y n (signal) = Y n (decoy) = Y ne n (signal) = e n (decoy) = e n .

While a few decoy states are sufficient, for ease of discussion, we will for the moment consider the case where Alice will pick an infinite number of possible intensities for decoy states. Let us imagine that Alice varies over all non-negative values of u randomly and independently for each signal, Alice and Bob can experimentally measure the yield Q u and the QBER E u . Since the relations between the variables Q u ’s and Y n ’s and between E u ’s and e n ’s are linear, given the set of variables Q u ’s and E u ’s measured from their experiments, Alice and Bob can deduce mathematically with high confidence the variables Y n ’s and e n ’s. This means that Alice and Bob can constrain simultaneously the yields, Y n and QBER e n simultaneously for all n. Suppose Alice and Bob know their channel property well. Then, they know what range of values of Y n ’s and e n ’s is acceptable. Any attack by Eve that will change the value of any one of the Y n ’s and e n ’s substantially will, in principle, be caught with high probability by our decoy state method. Therefore, in order to avoid being detected, the eavesdropper, Eve, has very limited options in her eavesdropping attack. In summary,the ability for Alice and Bob to verify experimentally the values of Y n and e n ’s in the decoy state method greatly strengthens their power in detecting eavesdropping, thus leading to a dramatic improvement in the performance of their QKD system.

The decoy state method allows Alice and Bob to detect deviations from the normal behavior due to eavesdropping attacks. Therefore, in what follows, we will consider normal behavior (i.e., the case of no eavesdropping). Details of QKD set-up model can be seen in [17].

Let us discuss the yields, Y n ’s, in a realistic set-up.
(a) The case n = 0.
In the absence of eavesdropping, Y 0 is simply given by the background detection event rate p dark of the system.
(b) The case n > 1. For n > 1, yield Y n comes from two sources, i) the detection of signal photons n n , and ii) the background event p dark . The combination gives,assuming the independence of background and signal detection event where in the second line we neglect the cross term because the background rate (typically 10 -5 ) and transmission efficiency (typically 10 -3 ) are both very small.

Suppose the overall transmission probability of each photon is n. In a normal channel, it is common to assume independence between the behaviors of the n photons. Therefore, the transmission efficiency for n-photon signals n n is given by:n n = 1 - (1 - n) n ,(7)[For a small n and ignore the dark count, Y n = nn.]QBER Let us discuss the QBERs, e n ’s, in a realistic experiment.
(a) If the signal is a vacuum, Bob’s detection is due to background including dark counts and stray light due to timing pulses. Assuming that the two detectors have equal background event rates, then the output is totally random and the error rate is 50\%. That is, the QBER for the vacuum e 0 = 1/2.
(b) If the signal has n > 1 photons, it also has some error rate, say e n . More concretely, e n comes from two parts, erroneous detections and background contribution, where e detector is independent of n.
The values of Y n and e n can be experimentally verified by Alice and Bob using our decoy state method. Any attempt by Eve to change them significantly will almost always be caught.
Combining decoy state idea with GLLP Suppose key generation is done on signal state | ue i0 i. In principle,Alice and Bob can isolate the single-photon signals and apply privacy amplification to them only. Therefore, generalizing the work in GLLP, Pwe find Eq. (1) where the gain of the signal state, Q u = k=0 Y k e -u u k /k! , [This comes directly from Eq. (2).] and the fraction of Bob’s detection events that have originated from single-photon signals emitted by Alice is the gain for the single photon state.

The derivation of Eq. (1) assumes that error correction protocols can achieve the fundamental (Shannon) limit.
However, practical error correction protocols are generally inefficient. As noted in [22], a simple way to take this inefficiency into account is to introduce a function,f (e) > 1, of the QBER, e. By doing so, we find that the key generation rate for practical protocols is given by:S > q{-Q u f (E u )H 2 (E u ) + Q 1 [1 - H 2 (e 1 )]},(11)where q depends on the implementation (1/2 for the BB84 protocol, because half the time Alice and Bob bases are not compatible, and if we use the efficient BB84 protocol [23], we can have q = 1. For simplicity, we will take q = 1 in this paper.), and f (e) is the error correction efficiency [22].

Let us now compare our result in Eq. (11) with the prior art GLLP result. In the prior art GLLP [6] method,secure key generation rate is shown to be at least where s, the fraction of “untagged” photons, (which is a pessimistic estimation of the fraction of detection events by Bob that have originated from single-photon signals emitted by Alice), is given by 1 - s = p multi /Q u ,(13)where p multi is the probability of Alice’s emitting a multi photon signal. Eq. (13) represents the worst situation where all the multi-photons emitted by Alice will be received by Bob.

Comparing our result (given in Eq. (11)) with the prior art GLLP result (given in Eq. (12)), we see that the main difference is that in our result, a much better lower bound on s and a much better upper bound on e 1can be obtained.

Implication of our result We obtain substantially higher key generation rate than in [6]. In more detail,note that, from Eq. (6), Y n for n > 2 is of similar order to Y 1 . Therefore, from Eq. (11) it is now advantageous for Alice to pick the average photon number in her signal state to be u = O(1). Therefore, the key generation rate in our new method is O(n) where n is the overall transmission probability of the channel. In comparison, in prior art methods for secure QKD, u is chosen to be of order O(n), thus giving a net key generation rate of O(n 2 ).
In summary, we have achieved a substantial increase in net key generation rate from O(n 2 ) to O(n). Moreover, as will be discussed below, our decoy state method allows secure QKD at much longer distances than previously thought possible.

More concretely, we [15] have applied our results to various experiments in the literature. The results are shown in Fig. 1 using the GYS [3] experiment as an example. We found that the optimal averaged number u in GYS that maximizes the key generation rate in our decoy state method in Eq. (11) is, indeed, of O(1) (roughly 0.5). Therefore, the key generation rate is of order O(n). We remark that the calculated optimal value of photon number of 0.5 is, in fact, higher than what experimentalists have been using. Experimentalists often liberally pick 0.1 as a convenient number for average photon number without any security justification. In other words, operating their equipment with the parameters proposed in the present paper will allow experimentalists to not only match, but also surpass their current experimental performance (by having at least five-fold the current experimental key generation rate). This demonstrates clearly the power of decoy state QKD. Moreover, Fig. 1 shows that with our decoy state idea, secure QKD can be done at distances over 140 km with only current technology. In summary, our result shows that we can have the best of both worlds: Enjoy both unconditional security and record-breaking experimental performance. The general principle of decoy state QKD developed here can have widespread applications in other set-ups (e.g. open-air QKD or QKD with other photon sources) and to multiparty quantum cryptographic protocols such as [24]. As demonstrated clearly in [17], one can achieve almost all the benefits of our decoy state method with only one or two decoy states. See also [16]. Recently, we have experimentally demonstrated decoy state QKD in [19].

We have benefitted greatly from enlightening discussions with many colleagues including particularly G. Brassard. Financial support from funding agencies such as CFI, CIPI, CRC program, NSERC, OIT, and PREAare gratefully acknowledged. H.-K. L also thanks travel support from the INI, Cambridge, UK and from the IQIat Caltech through NSF grant EIA-008603

\subsection{\trnas}
\subsubsection*{Аннотация}

Наблюдается большой интерес к квантовому распределению ключей. Экспериментально было успешно выполнено квантовое распределение ключей по 150 км коммерческого телекоммуникационного волокна. Решающим вопросом в квантовом распределении ключей является его безопасность. К сожалению, все последние эксперименты в принципе небезопасны из-за несовершенства реальной среды. Здесь мы предлагаем метод, который впервые может сделать большинство этих экспериментов безопасными, используя практически то же самое оборудование. Наш метод заключается в использовании ложных состояний для обнаружения атак подслушивания. В результате мы получаем лучшее из двух миров - безусловную безопасность, гарантированную фундаментальными законами физики, и при этом значительно превосходящую даже некоторые из лучших экспериментальных результатов, о которых сообщалось в литературе.

\subsubsection{Введение}

Квантовое распределение ключей (QKD) позволяет двум пользователям, Алисе и Бобу, общаться в абсолютной безопасности в присутствии подслушивающего устройства, Евы. В отличие от обычной криптографии, безопасность QKD основана на фундаментальных законах физики, а не на недоказанных вычислительных предположениях. Безопасность QKD была строго доказана в ряде недавних работ [1]. См. также[2]. В последнее время наблюдается огромный интерес к экспериментальному QKD [3, 4], с текущим мировым рекордом расстояния в 150 км по волокнам Telecom[4].

К сожалению, все эти захватывающие недавние эксперименты в принципе небезопасны из-за несовершенства реальной жизни. Более конкретно, в качестве источников часто используются сильно затухающие лазеры. Но эти источники иногда производят сигналы, содержащие более одного фотона. Эти многофотонные сигналы открывают дверь для новых мощных атак подслушивания, включая атаку с расщеплением фотонов. Например, Ева может, в принципе, измерить количество фотонов в каждом сигнале, испускаемом Алисой, и выборочно подавить однофотонные сигналы. Она разделяет многофотонные сигналы, сохраняя одну копию для себя и отправляя одну копию Бобу. Теперь, поскольку у Евы есть идентичная копия того, чем обладает Боб, безусловная безопасность QKD (например, в стандартном протоколе BB84[5]) полностью нарушена.

В целом, в стандартном протоколе BB84 гарантируется безопасность только тех сигналов, которые исходят от однофотонных импульсов, испускаемых Алисой. Следовательно, перефразируя GLLP [6], безопасная скорость генерации ключей (на состояние сигнала, испускаемого Алисой) может быть показана следующим образом:S Q {H 2 (E ) + [1 H 2 (e 1 )]}(1)где Q и E - соответственно коэффициент усиления и квантовая скорость битовой ошибки (QBER) состояния сигнала [7], s и e 1 - соответственно доля и QBER событий обнаружения Бобом, которые произошли от однофотонных сигналов, испускаемых Алисой, а H 2 - двоичная энтропия Шеннона.

Априори очень трудно получить хорошую нижнюю границу для сигмы и хорошую верхнюю границу для e 1 . Поэтому методы предшествующего уровня техники (как в GLLP [6]) делают самое пессимистичное предположение, что все многофотонные сигналы, испускаемые Алисой, будут получены Бобом. По этой причине до сих пор широко распространено мнение, что требование безусловной безопасности сильно снизит производительность систем QKD [6, 8, 9, 10, 11].

В этой статье мы представляем простой метод, который обеспечит очень хорошие границы сигмы и e 1 . Следовательно, наш метод впервые делает большинство экспериментов по QKD на больших расстояниях, о которых сообщалось в литературе, безусловно безопасными. Преимущество нашего метода заключается в том, что он может быть реализован с использованием практически всего современного оборудования. Таким образом, в отличие от предшествующих решений, основанных на однофотонных источниках, наш метод не требует огромных экспериментальных разработок. Наш метод основан на идее состояния приманки, впервые предложенной Хвангом [12]. Хотя идея Хванга была весьма инновационной, его анализ безопасности был эвристическим. Ключевой момент идеи ложных состояний заключается в том, что Алиса готовит набор дополнительных состояний - ложных состояний, в дополнение к стандартным состояниям BB84. Эти состояния-обманки используются только для обнаружения атак подслушивания, в то время как стандартные состояния BB84 используются только для генерации ключей. Единственное различие между состояниями-обманками и стандартными состояниями BB84 заключается в их интенсивности (т.е. в распределении числа фотонов).

Измеряя доходность и QBER состояний приманки, мы покажем, что Алиса и Боб могут получить надежные границы s и e 1 , что позволит им существенно превзойти все результаты предшествующей техники [13]. Здесь мы впервые даем строгий анализ безопасности QKD с ложными состояниями. Более того, мы показываем, что идея состояния приманки может быть объединена с предыдущим анализом GLLP [6].

Предварительные версии нашего результата в этой статье появились в [14, 15], где мы представили не только общую теорию, но и предложили идею использования только нескольких состояний приманки (например, три состояния - вакуум, слабое состояние приманки с u приманки << 1 и сигнальное состояние с u = O(1). Мы называем такой протокол протоколом с состоянием вакуум+слабая приманка). Впоследствии наши протоколы для QKD с состоянием приманки были проанализированы в [16] и более систематически в [17]. См. также [18]. Недавно мы представили первую экспериментальную демонстрацию QKD с ложным состоянием в [19].

Теперь мы представим общую теорию наших новых схем обманных состояний. Мы будем считать, что Алиса может готовить фазово-рандомизированные когерентные состояния и может увеличивать и уменьшать свою мощность для каждого сигнала. Это может быть достигнуто с помощью стандартных коммерческих переменных оптических аттенюаторов (VOA) [20]. Пусть | ue i0 i обозначает слабое когерентное состояние, излучаемое Алисой. Если предположить, что фаза 0 всех сигналов полностью случайна, то распределение вероятности для числа фотонов состояния сигнала соответствует распределению Пуассона с некоторым параметром u. То есть, с вероятностью p n = e -u u n /n!, сигнал Алисы будет содержать n фотонов. Таким образом, мы предположили, что Алиса может подготовить любую пуассоновскую (с параметром u) смесь состояний с числом фотонов и, более того, Алиса может варьировать параметр u для каждого отдельного сигнала.

Рассмотрим коэффициент усиления Q u для когерентного состояния | e i i. [Здесь и далее мы фактически имеем в виду, что сделать m смесь | ue io i по всем значениям 0, так как предполагается, что фаза полностью рандомизирована].

Аналогично, QBER может зависеть от числа фотонов. Определим e n как QBER n-фотонного сигнала. QBER E u для когерентного состояния | ue i0 i дано.

Представим, что состояние приманки и состояние сигнала имеют одинаковые характеристики (длина волны, информация о времени и т.д.). Поэтому Ева не может отличить состояние приманки от состояния сигнала, и единственная информация, доступная Еве, - это количество фотонов в сигнале. Поэтому выход, Y n , и QBER, e n , могут зависеть только от числа фотонов, n, но не от того, из какого распределения (приманка или сигнал) состояние. Подчеркнем, что суть идеи состояния приманки может быть обобщена следующими двумя уравнениями:Y n (сигнал) = Y n (приманка) = Y ne n (сигнал) = e n (приманка) = e n .

Хотя достаточно нескольких ложных состояний, для простоты обсуждения мы пока рассмотрим случай, когда Алиса выберет бесконечное число возможных интенсивностей для ложных состояний. Представим, что Алиса варьирует все неотрицательные значения u случайным образом и независимо для каждого сигнала, Алиса и Боб могут экспериментально измерить выход Q u и QBER E u . Поскольку отношения между переменными Q u и Y n и между E u и e n линейны, учитывая набор переменных Q u и E u, измеренных в ходе экспериментов, Алиса и Боб могут математически с высокой степенью достоверности вывести переменные Y n и e n . Это означает, что Алиса и Боб могут одновременно ограничить урожайность, Y n и QBER e n одновременно для всех n. Предположим, что Алиса и Боб хорошо знают свойство своего канала. Тогда они знают, какой диапазон значений Y n 's и e n 's является приемлемым. Любая атака Евы, которая существенно изменит значение любого из Y n и e n, в принципе, будет с высокой вероятностью поймана нашим методом ложных состояний. Поэтому, чтобы избежать обнаружения, подслушивающая Ева имеет очень ограниченные возможности в своей подслушивающей атаке. В итоге, возможность Алисы и Боба экспериментально проверить значения Y n и e n в методе обманного состояния значительно усиливает их возможности в обнаружении подслушивания, что приводит к значительному улучшению производительности их системы QKD.

Метод ложного состояния позволяет Алисе и Бобу обнаружить отклонения от нормального поведения из-за атак подслушивания. Поэтому в дальнейшем мы будем рассматривать нормальное поведение (т.е. случай отсутствия подслушивания). Подробности модели настройки QKD можно найти в [17].

Давайте обсудим доходность Y n в реалистичной ситуации.
(a) Случай n = 0.
В отсутствие подслушивания Y 0 просто определяется фоновой частотой событий обнаружения p системы.
(b) Случай n > 1. При n > 1 доход Y n поступает из двух источников, i) обнаружение сигнальных фотонов n n , и ii) фоновое событие p. Комбинация дает , предполагая независимость фона и события обнаружения сигнала, где во второй строке мы пренебрегаем перекрестным членом, потому что уровень фона (обычно 10 -5 ) и эффективность передачи (обычно 10 -3 ) очень малы.

Предположим, что общая вероятность передачи каждого фотона равна n. В нормальном канале принято считать, что поведение n фотонов независимо друг от друга. Поэтому эффективность передачи для n-фотонных сигналов n n дается следующим образом:n n = 1 - (1 - n) n ,(7)[Для малого n и игнорирования темнового счета, Y n = nn.]QBER Давайте обсудим QBERs, e n 's, в реалистичном эксперименте.
(a) Если сигналом является вакуум, то обнаружение Боба обусловлено фоном, включающим темные отсчеты и рассеянный свет из-за синхроимпульсов. Если предположить, что оба детектора имеют равную частоту фоновых событий, то выходной сигнал полностью случаен и коэффициент ошибок составляет 50\%. То есть, QBER для вакуума e 0 = 1/2.
(b) Если сигнал имеет n > 1 фотонов, он также имеет некоторый коэффициент ошибок, скажем e n . Более конкретно, e n складывается из двух частей, ошибочных обнаружений и фонового вклада, где e детектора не зависит от n.
Значения Y n и e n могут быть экспериментально проверены Алисой и Бобом с помощью нашего метода ложных состояний. Любая попытка Евы существенно изменить их почти всегда будет поймана.
Объединение идеи состояния приманки с GLLP Предположим, что генерация ключей осуществляется на сигнальном состоянии | ue i0 i. В принципе, Алиса и Боб могут изолировать однофотонные сигналы и применять усиление конфиденциальности только к ним. Поэтому, обобщая работу в GLLP, мы находим уравнение (1), где усиление состояния сигнала, Q u = k=0 Y k e -u u k /k! [Это следует непосредственно из уравнения (2)], а доля событий обнаружения Боба, которые произошли от однофотонных сигналов, испущенных Алисой, является усилением для однофотонного состояния.

Вывод уравнения (1) предполагает, что протоколы коррекции ошибок могут достичь фундаментального (шенноновского) предела.
Однако практические протоколы коррекции ошибок, как правило, неэффективны. Как отмечается в [22], простой способ учесть эту неэффективность - ввести функцию, f (e) > 1, от QBER, e. Сделав это, мы обнаружим, что скорость генерации ключей для практических протоколов дается следующим образом:S > q{-Q u f (E u )H 2 (E u ) + Q 1 [1 - H 2 (e 1 )]},(11)где q зависит от реализации (1/2 для протокола BB84, поскольку в половине случаев базы Алисы и Боба несовместимы, и если мы используем эффективный протокол BB84 [23], мы можем иметь q = 1. Для простоты в данной работе мы будем принимать q = 1.), а f (e) - эффективность коррекции ошибок [22].

Давайте теперь сравним наш результат в уравнении (11) с результатом, полученным в предшествующем методе GLLP. В известном методе GLLP [6] безопасная скорость генерации ключей, как показано, составляет, по меньшей мере, где s - доля "неотмеченных" фотонов (которая является пессимистической оценкой доли событий обнаружения Боба, которые произошли от однофотонных сигналов, испущенных Алисой), дается 1 - s = p multi /Q u ,(13)где p multi - вероятность того, что Алиса испустила многофотонный сигнал. Уравнение (13) представляет наихудшую ситуацию, когда все многофотонные сигналы, испущенные Алисой, будут получены Бобом.

Сравнивая наш результат (приведенный в уравнении (11)) с результатом GLLP (приведенным в уравнении (12)), мы видим, что основное различие заключается в том, что в нашем результате можно получить гораздо лучшее нижнее ограничение на s и гораздо лучшее верхнее ограничение на e 1.

Следствие нашего результата Мы получаем существенно более высокую скорость генерации ключей, чем в [6]. Более подробно, отметим, что из уравнения (6) следует, что Y n для n > 2 имеет тот же порядок, что и Y 1. Поэтому, исходя из уравнения (11), Алисе выгодно выбрать среднее число фотонов в сигнальном состоянии равным u = O(1). Таким образом, скорость генерации ключей в нашем новом методе составляет O(n), где n - общая вероятность передачи данных по каналу. Для сравнения, в известных методах безопасного QKD u выбирается порядка O(n), что дает чистую скорость генерации ключей O(n 2 ).
В итоге мы добились существенного увеличения чистой скорости генерации ключей с O(n 2 ) до O(n). Более того, как будет показано ниже, наш метод обманного состояния позволяет обеспечить безопасное QKD на гораздо больших расстояниях, чем считалось ранее возможным.

Более конкретно, мы [15] применили наши результаты к различным экспериментам в литературе. Результаты показаны на рис. 1 на примере эксперимента GYS [3]. Мы обнаружили, что оптимальное усредненное число u в GYS, которое максимизирует скорость генерации ключей в нашем методе состояния приманки в уравнении (11), действительно имеет порядок O(1) (приблизительно 0,5). Следовательно, скорость генерации ключей имеет порядок O(n). Заметим, что рассчитанное оптимальное значение числа фотонов 0,5 на самом деле выше, чем то, которое используют экспериментаторы. Экспериментаторы часто либерально выбирают 0,1 в качестве удобного числа для среднего числа фотонов без какого-либо обоснования безопасности. Другими словами, эксплуатация оборудования с параметрами, предложенными в настоящей статье, позволит экспериментаторам не только соответствовать, но и превзойти текущую экспериментальную производительность (по крайней мере, в пять раз превысив текущую экспериментальную скорость генерации ключей). Это наглядно демонстрирует возможности QKD с обманным состоянием. Более того, на рис. 1 показано, что с нашей идеей обманного состояния безопасное QKD может быть выполнено на расстоянии более 140 км с использованием только текущей технологии. В целом, наш результат показывает, что мы можем получить лучшее из двух миров: наслаждаться как безусловной безопасностью, так и рекордной экспериментальной производительностью. Разработанный здесь общий принцип QKD с обманным состоянием может иметь широкое применение в других установках (например, QKD под открытым небом или QKD с другими источниками фотонов) и в многосторонних квантовых криптографических протоколах, таких как [24]. Как было ясно продемонстрировано в [17], можно достичь почти всех преимуществ нашего метода ложных состояний только с одним или двумя ложными состояниями. См. также [16]. Недавно мы экспериментально продемонстрировали QKD с ложными состояниями в [19].

Мы получили большую пользу от просветительских дискуссий со многими коллегами, включая, в частности, Г. Брассарда. Финансовая поддержка от таких финансирующих организаций, как CFI, CIPI, CRC program, NSERC, OIT и PREA, выражает нам признательность. H.-K. L также благодарит за поддержку поездок от INI, Кембридж, Великобритания, и от IQI в Калтехе через грант NSF EIA-008603.

\subsection{\review}
Some scientists mistakenly believe that this [6] paper presented a new protocol, but in fact the authors showed a new quantum trap method. The authors of the paper used the method together with the BB84 protocol, but they also claim that it can be used with any quantum key distribution protocol.

The essence of the method is that along with photons containing confidential information, trap photons are sent through the communication channel. They do not contain any meaningful information. When an intruder reads a phototrap, he increases the amount of noise in the communication channel, thereby giving himself away, and does not receive any confidential information.

As a result, the authors were able to increase the distance of secure transmission of information over a quantum communication channel from 30 km to 150 km.


\subsection{\dic}
\begin{multicols}{2}
	\begin{itemize}
		
		\item algorithms - алгоритм
		\item analysis - анализ
		
		\item appropriate - подходящий
		\item approximately - примерно
		
		
		\item basis - основа
		\item beam - луч
		
		\item binary - двоичных
		\item bit - бит
		
		\item capacity - вместимость
		
		\item channel - канал
		
		\item coherent - связный
		\item combination - комбинации
		
		\item communication - связь
		\item compare - сравнить
		
		\item computation - вычисления
		\item computers - компьютеров
		
		\item condition - условие
		\item conjugate - спряжение
		\item considered - рассмотрено
		\item contain - содержат
		
		\item correlation - корреляция
		
		\item cryptography - криптография
		
		\item decode - декодировать
		\item decoy - ловушка
		\item density - плотность
		
		\item dependence - зависимость
		\item detect - обнаружить
		
		\item deterministic - детерминированный
		
		\item developing - разработка
		
		\item difference - разница
		
		\item differentiate - дифференцировать
		
		\item distribution - распределение
		
		\item eavesdropper - подслушиватель
		
		\item encoded - закодировано
		\item entaglement - запутанность
		\item equivalently - эквивалентно
		
		\item imperfections - недостаток
		\item implementation - реализация
		
		\item instances - экземпляров
		
		\item intensity - интенсивность
		\item intercept - перехват
		
		\item interferometer - интерферометр
		
		\item limitation - ограничение
		
		\item lossless - без потерь
		\item lossy - потери
		
		\item malicious - вредоносных
		\item managed - управляемых
		
		\item mathematical - математических
		\item matrix - матрица
		
		\item measurement - измерения
		
		\item method - метод
		\item mimic - имитировать
		
		\item modification - модификация
		
		\item neglected - пренебрегают
		
		\item normalized - нормализовано
		
		\item operator - оператор
		\item optical - оптический
		\item optimal - оптимальный
		
		\item optional - опционально
		
		\item orthogonal - ортогональный
		\item orthogonality - ортогональность
		\item orthonormal - ортонормированный
		
		\item phase - фаза
		\item photon - фотон
		
		\item polarization - поляризация
		
		
		\item probabilistic - вероятностный
		\item probability - вероятность
		\item problem - проблема
		
		\item projecting - проектирование
		
		\item property - свойство
		\item proportion - пропорция
		\item proposal - предложение
		
		\item protocol - протокол
		
		\item prove - доказательство
		
		\item provide - обеспечить
		\item provides - обеспечивает
		
		\item public - общедоступных
		\item pulse - импульс
		
		\item quantum - квантовый
		\item random - случайных
		\item randomize - рандомизировать
		
		\item signature - подпись
		
		\item strategy - стратегия
		
		\item string - строка
		
		\item symmetric - симметричный
		
		\item theoretical - теоретический
		\item theory - теория
		
		\item transmission - передача
		
		\item vacuum - вакуум
		\item value - значение
		
		\item variable - переменная
		
		
		
	\end{itemize}
\end{multicols}


\section{Quantum Key Distribution Protocols: A Review}

\subsection{Article}

\subsubsection*{Abstract}

Quantum key distribution (QKD) provides a way for distribution of secure key in at least two parties which they initially share. And there are many protocols for providing a secure key i.e. BB84 protocol,SARG04 protocol, E91 protocol and many more. In this paper all the concerned protocols that share a secret key is explained and comparative study of all protocols shown.
Keywords : Quantum key distribution, BB84 protocol, BB92 protocol, SARG04 protocol, E91 protocol, COW protocol, DPS protocol, KMB09 Protocol, S09 protocol, S13 protocol.

\subsubsection{Introduction}

Quantum key distribution (QKD) [1] [2] provides a way for two parties to expand a secure key that they initially share. The best known QKD is the BB84 protocol published by Bennett and Brassard in 1984 [1]. The security of BB84 was not proved until many years after its introduction. Among the proofs [3] [4] [5] [6],the one by Shor and Preskill [6] is relevant to this paper. Their simple proof essentially converts an entanglement distillation protocol (EDP) based QKD proposed by Lo and Chau [5] to the BB84 Protocol. The EDP-based QKD has already been shown to be secure by [5] and the conversion successively leads to thesecurity of BB84 protocol.

Security proofs of QKD protocols were further extended to explicitly accommodate the imperfection in practical devices [7] [8]. One important imperfection is that the laser sources used in practice and coherent sources that occasionally emit more than one photon in each signal. Thus they are not single –photon sources that the other security proofs [3] [4] [6] of BB84 assumed. In particular, BB84 may become insecure when coherent sources with strong intensity are used. For instance Eve can launch a photon-number-splitting (PNS)attack puts severe limits on the distance and the key generation rate of unconditionally secure QKD. A novel solution to the problem of imperfect devices in BB84 protocol was proposed by Hwang [9]. Which uses extra test states called the decoy states to learn the properties of the channel and/or eavesdropping on the key generating signal states. An unconditional security proof of decoy-state QKD [10] [11] is presented. Another method to combat PNS attack was by Scarani et.al. [12], who introduced a new protocol called SARG04, which is very similar to the BB84 protocol. The quantum state transmission phase and the measurement phase of SARG04 are the same as that of BB84, as both use the same four quantum state and the same experimental measurement. The only difference between the two protocols is the classical post-processing phase, the protocol becomes secure even when Alice emits two photon, a situation under which BB84 is insecure .This protocol was proved by [13] who also proved the security of SARG04 with a single-photon source. They also proposed a modified SARG04 protocol that uses same six states as the original six state protocols [14] [15]. The security of SAG04 with a single-photon source was also proved by Branciard et.al [16]. They considered SARG04 protocol implemented with single-sources and with realistic sources. For the single photon source case, they provided upper and lower bounds of the bit error rate with one-way classical communications. For the realistic source case they considered only incoherent attack by Eve and showed that SARG04 can achieve higher secret key rate and greater source distance than BB84. Another protocol that is similar to SARG04 is the B92 Protocol [17] which uses two non orthogonal quantum states. The security of B92 with a single-photon source was proved by Tamaki et al [18] [19]. On the other hand Koashi [20] proposed an implementation of B92 with strong phase-reference coherent light that was proved secure.

The Focus of this paper is to survey the most prominent quantum key distribution protocols and their security. In this paper we briefly describe the necessary principles of quantum mechanics from which the protocols are divided in to two categories those based on the Heisenberg Uncertainty Principles and others are based on quantum entanglement Rest of the paper is organised as: In section II description of quantum cryptography and there mechanism is explained. Section III depicts all the Quantum key distribution protocols used Heisenberg's uncertainty principles. IV depicts all the Quantum key distribution protocols used quantum entanglement principles and section V other protocols that both prepare and measure and entanglement based is shown and in Section VI observation table of all the protocols with their applications is depicted. Finally Conclusion is shown in Section VII.

\subsubsection{QUANTUM CRYPTOGRAPHY}
Quantum cryptography is a relatively recent arrival in the information security world. It harnesses the laws of quantum Mechanics to create new cryptographic primitives. There is however, one quantum cryptographic primitive which is achievable with today's technology i.e. Quantum key distribution. By using the quantum properties of light, current lasers, fibre-optics and free space transmission technology can be used for QKD, so that many observers claim security can be based on the law of quantum physics only.

Quantum key distribution is a key establishment protocol which creates symmetric key material by using quantum properties of light to transfer information from Client A to Client B in a manner which, through the incontrovertible results of quantum mechanics, will highlight any eavesdropping by an adversary.
2.1 The Heisenberg Uncertainty Principle. According the Heisenberg Uncertainty principle, it is not possible to measure the quantum state of any system without disturbing that system. Thus the polarization of photon or light particle can only be known at the point when it is measured. This principle play a critical role in thwarting the attempts of eavesdroppers in a cryptosystem based on quantum cryptography.
For any two observable properties linked together like mass and momentum Where A = A and B = B And where= AB – BA

According to the principle two interrelated properties cannot be measured individually without affecting the others. The principle is that since you cannot partition the photon in to two halves measuring the state of photon will affect it value. So if someone tries to detect the state of photons being send to the receiver the error can be detected [21]

2.2 Quantum Entanglement. The other important principle on which QKD can be based is the principle of quantum entanglement. It is possible for two particles to become entangled such that when a particular property is measured in one particle, the opposite state will be observed on the entangled particle instantaneously. This is true regardless of the distance between the entangled particles. It is impossible, however to predict prior to measurement what state will be observed thus it is not possible to communicate via entangled particles without discussing the observation over classical channel. The process of communicating using entangled states, aided by a classical information channel is known as quantum teleportation and is the basis of Eckert's protocol [22].

\subsubsection{Qkd Protocols Using Heisenberg’s Uncertainity Prinicples}
Quantum cryptography exploits the quantum mechanical property that a qubit cannot be copied or amplified without disturbing its original state i.e. No-Cloning Theorem [23] [24]. Key distribution using quantum cryptography would be almost impossible to steal because Quantum key distribution (QKD) [25] [26][27] systems continually and randomly generate new private keys that both parties shares automatically. A compromised key in a QKD system is able to decrypt only a small amount of encoded information because of continuously changes in private key. A secret key can be build from a stream of a single photon where each photon is encoded with a bit value of 0 or 1, typically by a photon superposition state such as polarization. These photons are emitted by a conventional laser as pulses of dim light so that most pulses do not emit a photon. This approach ensures that few pulses contain more than one photon travel through the fiber-optic line. In the end only a small fraction of the received pulses actually contains a photon [28]. The photons that are reached to the receiver are used. The key is generally encoded in either the polarization or the relative phase ofthe photon.
3.1 BB84 protocol. Quantum cryptography is based upon conventional cryptographic methods and extends these through the use of quantum effects. Quantum key Distribution (QKD) is used in quantum cryptography for generating a secret key shared between two parties using a quantum channel and an authenticated classical channel as show in figure 6. The private key obtained then used to encrypt message that are sent over an insecure channel (such as a conventional internet connection).

The BB84 protocol described using Photon polarization state to transmit the information. It was originally developed by Charles Bennett and Gilles Brassard in 1984 [1].

Below are the steps of the BB84 protocol for exchange the secret key in the BB84 protocol [29], client A and client B must do as follow:STAGE 1 PROTOCOL Communication over quantum channel
Client A prepare photon randomly with either rectilinear (+) or diagonal polarization (x) therefore Client A transmit photons in the four polarization states (0, 45, 90,135 degree).
Client A records the polarization of each photon and sends it to Client B.
Client B receives a photon and randomly records its polarization according to the rectilinear or diagonal basis. The Client B records the measurement type (basis used) and the resulting polarization measured.
Client B doesn't know which of the measurement are deterministic, i.e. measured in the same basis as the one used by client A. Half the time Client B will be lucky and chose the same quantum alphabet as the third person. In this case, the bit resulting from his measurement will agree with the bit sent by Client A. However the other half time he will be unlucky and choose the alphabet not used by client A. In this case,the bit resulting from his measurement will agree with the bit sent by client A only 50\% of the time. Afterall these measurement, client B now has in hand a binary sequence 
Client A and Client B now proceed to communicate over the public two-way channel using the following stage 2 protocol.

STAGE 2 PROTOCOL: Communication over a public channel
Phase 1. Raw Key extraction

Over the public channel, client B communicates to client A which quantum alphabet he used for each of his measurements.
In response client A communicate to client B over the public channel which of his measurement were made correct alphabet.

Client A and Client B then delete all bits for which they used incompatible quantum alphabet to produce their resulting raw keys. If the third person has not eavesdropped, then their resulting keys will be the same. If the third person has eavesdropped their resulting key will not be in total agreement.
Phase 2. Error estimation. Over the public channel, Client A and client B compare small portion of their raw keys to estimate the error-rate R, and then delete the disclosed bits from their raw keys to produce their tentative final keys. If through their public disclosures Client A and Client B find no errors (i.e., R=0), then they know that the third person was not eavesdropping and that their tentative keys must be the same final key. If they discover at least one error during their public disclosures (i.e., R>0), then they know that the third person has been eavesdropping. In this case,they discard their tentative final keys and start all over again.
3.2 BB92 protocol. Soon after BB84 protocol was published, Charles Bennett realized that it was not necessary to use two orthogonal basis for encoding and decoding. It turns out that a single non-orthogonal basis can be used instead,without affecting the security of the protocol against eavesdropping. This idea is used in the BB92 protocol [30],which is otherwise identical to BB84 protocol.
The key difference in BB92 is that only two states are necessary rather than the possible 4 polarization states in BB84 protocol.

As shown in figure 3, 0 can be encoded as 0 degrees in the rectilinear basis and 1 can be encoded by 45 degrees in the diagonal basis. Like the BB84 protocol, Client A transmit to Client B a string of photons encoded with randomly chosen bits but this time the bits Client A chooses dictates which bases Client B must use. Client B still randomly chooses a basis by which to measure but if he chooses the wrong basis, he will not measure anything; a condition in quantum mechanics which is known as an erasure. Client B can simply tell Client A after each bit Client B sends whether or not he measured it correctly [31].
3.3 SARG04 protocol. The SARG04 protocol is built when researcher noticed that by using the four states of BB84 with different information encoding they could develop a new protocol which would more robust when attenuated laser pulses are used instead of single- photon sources. SARG04 protocol was proposed in 2004 by Scarani et.al[32].

The SARG04 protocol shares the exact same first phase as BB84. In the second Phase when Client A and Client B determine for which bits their bases matched, Client A does not directly announce her bases rather than Client A announces a pair of non-orthogonal states one of which she used to encode her bit. If Client Bused the correct basis, he will measure the correct state. If he chose incorrectly he will not measure either Client A states and will not be able to determine the bit. If there are no errors, then the length of the key remaining after the sifting stage is 1/4 of the raw key. The SARG04 protocol provides almost identical security to BB84 in perfect single-photon implementations: If the quantum channel is of a given visibility (i.e. with losses) then the QBER of SARG04 is twice that of BB84protocol, and is more sensitive to losses.
However SARG04 protocol provides more security than BB84 in the presence of PNS attack, in both the secret key rate and distance the signal can be carried (limiting distance).
3.4 Six-State protocol. (SSP)The 6-state or 3 bases cryptographic is nothing but the well-known BB84 4-state scheme with an additional basis [33]. Six-State Protocol (SSP) is proposed by Pasquinucci and Gisin in 1999 [34].
When represented on the Poincare sphere the BB84 protocol makes use of four spin-1/2 states corresponding to-x and -y direction. In brief summary Client A sends of the four states to Client B, who measures the qubits he receives in either the X or Y basis. A priori this gives a probability 1/2 that Client A and Client B use the same basis. On an average Client A and Client B have to discard half of the qubits even before they can start extracting their cryptographic key.

In the 6 state protocols the two extra states correspond to -z, i.e. the 6 states are -x, -y, and -z on the Poincare sphere. In this case Client A sends a state chosen freely among the 6 and Client B measures either in the x, y or z-basis. Here the prior probability that Client A and Client B use the same basis is reduced to 1/3,which means that they have to discard 2/3 of the transmitted qubits before they can extract a cryptographic key.
However, this scheme does hold an advantage compared to the BB84 protocol – higher symmetry. As it will be seen this fact together with the use of symmetric eavesdropping strategies dramatically reduced the number of free variables in the problem under investigation.
\subsubsection{Qkd Protocols Using Quantum Entanglement}
A new approach to quantum key distribution where the key is distributed using quantum teleportation
4.1 E91 protocol. The Ekert scheme uses entangled pairs of photons [2]. These can be created by created by Client A, by Client B, or by some source separate from both of them, including eavesdropper Eve. The photons are distributed so that Client A and Client B each up with one photon from each pair.
The Scheme relies on two properties of entanglement. First the entangled states are perfectly correlated in the sense that if Client A and Client B both measure whether their particles have vertical or horizontal polarizations,they will always get the same answer with 100\% probability. The same is true if they both measure any other pair of complementary (orthogonal) polarization However the particular results are completely random, it is impossible for Client A to predict if and Client B will get vertical polarization or horizontal polarization.
Second any attempt at eavesdropping by Eve will destroy these correlations in a way that Client A and Client B can detect.
A typical physical set-up is shown in figure 5, using active polarization rotators (PR), polarizing beam-splitters(PBS) and avalanche photodiodes (APD)

The measurement in the figure 5 is divided into two groups; the first is when different orientations of thea nalyser were used and the second when the same analyser orientation was employed. Any photon which was not registered is discarded. Alice and Bob then reveal the result of the first group only, and check that they correspond to the value expected from Bell's inequality. If this is so then Alice and Bob can be sure that the results they obtained in the second group are anti-correlated and can be used to produce a secret key string. Eve cannot obtain any information from the photons when they are transit as there is simple no information there. Information is only present once the authorized user performs their analyser measurements and key sifting. Eve's only hope is to inject her own data for Alice and Bob, but as she doesn't know their analyser orientations,she will always be detected (the Bell's inequality value will be too low).

4.2 COW protocol. Coherent One-Way protocol (COW protocol) is a new protocol for Quantum cryptography elaborated by Nicolas Gisin et al in 2004 [37].

A new protocol for QKD tailored to work with weak coherent pulses at high bit rates [36]. The advantage of this system are that the setup is experimentally simple and it is tolerant to reduced interference visibility and to photon numbers splitting attacks, thus resulting in a high efficiency in terms of distilled secret bits per qubit. The figure 6 presents the COW protocol. The information is encoded in time. Alice sends Coherent pulses that are either empty or have a mean photon number u < 1. Each logical bit of information is encoded by sequences of two pulses, u-0 for a logical “0” or 0-u for a logical “1”.

For security reason, Alice can also send decoy sequences u-u. To obtain the key, Bob measure the time-of-arrival of the photon on his data-line, detector D B . To ensure the security Bob randomly measures the coherence between successive non-empty pulses, bit sequence “1 -0” or decoy sequence, with the interferometer and detectors D M1 and D M2 . If wavelength of the laser and the phase in the interferometer are well aligned, we have all detection on D M1 and no detection on D M2. A loss of coherence and therefore a reduction of the visibility reveal the presence of an eavesdropper, in which case the key is simply discarded, hence no information will be lost.

4.3 DPS protocol. Differential –phase-shift QKD (DPS-QKD) is a new quantum key distribution scheme that was proposed by K.Inoue et al. [38]. Figure 7 shows the setup of the DPS-QKD scheme.

Alice randomly phase-modulates a pulse train of weak coherent states by {0,$\pi$} for each pulse and sends it to Bob with an average photon number of less than one per pulse. Bob measure the Phase difference between two sequential pulses using a 1-bit delay. Mach-Zehnder interferometer and photon detectors, and records the photon arrival time and which detector clicked. After transmission of the optical pulse train, Bob tells Alice the time instances at which a photon was counted. From this time information and her modulation data. Alice knows which detector clicked at Bob's site. Under an agreement that a click by detector 1 denotes “0” and click by detector 2 denotes “1”, for example Alice and Bob obtain an identical bit string.
The DPS-QKD scheme has certain advantageous features including a simple configuration, efficient time domain use, and robustness against photon number splitting attack [38] [39].

\subsubsection{Qther Protocols}

There are many other protocols in existence, both prepare-and-measures and entanglement based. They are as follows:
5.1. KMB09 protocol. KMB09 protocol is an alternative quantum key distribution protocol [40]. Where Alice and Bob use two mutually unbiased bases with one of them encoding a „0' and the other one encoding a „1'. The security of the scheme is due to a minimum index transmission error rate (ITER) and quantum bit error rate (QBER)introduced by an eavesdropper.
The ITER increase significantly for higher dimensional photon states. This allows for more Noise in the transmission line, thereby increasing the possible distance between Alice and Bob Without the need for intermediate nodes

5.2 S09 protocol. S09 protocol is quantum protocol based on public private key cryptography for secure transmission of data over a public channel [41]. The security of the protocol derives from the fact that Alice and Bob each use secret keys in multiple exchange of the qubit. Unlike the BB84 protocol [1] and its many variants. Bob Know the key to transmit, the qubits are transmitted in only one direction and classical information exchanged thereafter, the communication in the proposed protocol remains quantum in each stage. In the BB84 protocol,each transmitted qubit is in one of four different states in this protocol transmitted qubit can be in any arbitrary states
5.3 S13 protocol. S13 protocol is a new quantum protocol [42] that is identical to the BB84 protocol for all the quantum manipulation, but differs from it by using Private Reconciliation from a Random Seed and Asymmetric Cryptography. Thus allowing the generation of larger secure key

\subsubsection{Conclusions}
QKD Protocols are based on principles from quantum physics and information theory. Quantum key distribution is clearly an unconditionally secure means of establishing secret keys. Combined with unconditionally secure authentication, and an unconditionally secure cryptosystem.
The current commercial systems are aimed mainly at governments and corporations with high security requirements. The major difference of quantum key distribution is the ability to detect any interception of the key, whereas with courier the key security cannot be proven or tested. QKD system has the advantage of being automatic, with greater reliability and lower operating costs than a secure human courier network.

\subsection{\trnas}
ss
\subsection{\review}
ss
\subsection{\dic}
ss

\section{Quantum key distribution protocol with private-public key}
\subsection{Article}

\subsubsection*{Abstract}
A quantum cryptographic protocol based in public key cryptography combinations and private key cryptography is presented. Unlike the BB84 protocol [1] and its many variants[2, 3], two quantum channels are used. The present research does not make reconciliation mechanisms of information to derive the key. A three related system of key distribution are described.

\subsubsection{Introduction}
In cryptography, the objective is to transmit information between two parties (Alice, Bob) that restrict access to an eavesdropper (Eve). In classical cryptography,the information is encrypted by a key which is kept in a secret or public way. The key is distributed among Alice and Bob to decrypt the message. Distribution the key remains a difficult issue. Quantum mechanics provides solutions with protocols that are largely determined by the following phases: preparation (Alice), measurement (Bob), verification and key derive (Alice and Bob).
The last phase made public some details of the phases of preparation or measuring, this is called reconciliation mechanisms of information.
This paper presents a quantum protocol based on public private key cryptography for secure transmission of data over a public channel. The security of the protocol derives from the fact that Alice and Bob each use secret keys in the multiple exchange of the qubit. Unlike the BB84 protocol [1] and its many variants[2, 3], Bob knows the key to transmit, the qubits are transmitted in only one direction and classical information exchanged thereafter, the communication in the proposed protocol remains quantum in each stage. In the BB84 protocol,each transmitted qubit is in one of four different states,in the proposed protocol, the transmitted qubit can be in any arbitrary states.

\subsubsection{Protocol}

Alice took a bit i transforming it in to an element of a secret base B k genering the qubit |$\psi$ i,k i, that sends to Bob through a quantum channel .
1. Bob applies U j that is only known by him, to the qubit |$\psi$ i,k i, returns the resulting qubit to Alice .
2. Alice measures the qubit in the base B k obtaining the bit.

Using this, Bob denotes operation U 0 and U 1 to the identity operator I and XZ operator, respectively, where the bit j l is responsible for operation U j. The bit j, which was chosen by Bob and transmitted over a public channel, has reached Alice. Eve, the eavesdropper, cannot obtain any information by intercepting the transmitted qubits, although she could disrupt the exchange by replacing the transmitted qubits by her own. This can be detected by:
1. appending parity bits, and/or
2. appending previously chosen bit sequences, which could be the destination and sending addresses or their hashed values, or some other mutually agreed sequence.

Since the B k and U j transformations can be changed as frequently as one pleases, Eve cannot obtain any statistical clues to their nature by intercepting the qubits.

\subsubsection{Generalization}
Only Bob is involved in the preparation phase of Key-Message this allows extended to three parties (Alice,Bob, Celine) unlike the standard protocols [7]. I Preparation Phase (Alice,Celine)II Preparation Phase Key-Message (Bob)III Measurement and derivation phase (Alice, Celine)
1. Alice took a bit i transforming it in to an element of a secret base B k , Celine took a bit i transforming it in to an element of a secret base B t , both sendstheir qubit to Bob through a quantum channel.
2. Bob applies U j secret operation on the qubits |$\psi$ i,k i and |$\psi$ s,t i returns qubits resulting to their respective parties
3.Alice and Celine measures the qubits in the base B k and B t obtaining a value sent by Bob.

Generalizing the protocol for n parties, where Bob is central, a quantum key-message distribution network will be obtained.

\subsubsection{Conclusion}
The quantum protocol presented with its variants provides a safe sending of information of direct communication between two or more parties. The generalizations for n parties can create a network of massive sending information for n - 1 parties being one of them the key-message distribution center. This protocol is used to distribute applications key-messages safe over long distances because it allows the sending of massive qubits. Since the proposed protocol does not use classical communication,it is immune to the man-in-the-middle attack on the classical communication channel which BB84 type quantum cryptography protocols suffers from [8]. On the other hand, implementation of this protocol may be harder because the qubits get exchanged multiple times.

\subsection{\trnas}


\subsubsection*{Аннотация}
Представлен квантовый криптографический протокол, основанный на комбинации криптографии с открытым ключом и криптографии с закрытым ключом. В отличие от протокола BB84 [1] и его многочисленных вариантов[2, 3], используются два квантовых канала. В данном исследовании не используются механизмы сверки информации для получения ключа. Описаны три связанные системы распределения ключей.

\subsubsection{Введение}
Цель криптографии - передача информации между двумя сторонами (Алиса, Боб), которая ограничивает доступ подслушивающего устройства (Ева). В классической криптографии информация шифруется ключом, который хранится в секрете или в открытом виде. Ключ распределяется между Алисой и Бобом для расшифровки сообщения. Распределение ключа остается сложной проблемой. Квантовая механика предлагает решения с помощью протоколов, которые в основном определяются следующими фазами: подготовка (Алиса), измерение (Боб), проверка и получение ключа (Алиса и Боб).
На последнем этапе были обнародованы некоторые детали фаз подготовки или измерения, это называется механизмами согласования информации.
В данной работе представлен квантовый протокол, основанный на криптографии с открытым закрытым ключом, для безопасной передачи данных по общедоступному каналу. Безопасность протокола вытекает из того факта, что Алиса и Боб используют секретные ключи при многократном обмене кубитами. В отличие от протокола BB84 [1] и его многочисленных вариантов[2, 3], в котором Боб знает ключ для передачи, кубиты передаются только в одном направлении и после этого происходит обмен классической информацией, коммуникация в предлагаемом протоколе остается квантовой на каждом этапе. В протоколе BB84 каждый передаваемый кубита находится в одном из четырех различных состояний, в предлагаемом протоколе передаваемый кубита может находиться в любом произвольном состоянии.

\subsubsection{Протокол}

Алиса берет бит i, преобразуя его в элемент секретной базы B k, генерируя кубит |$\psi$ i,k i, который отправляет Бобу по квантовому каналу.
1. Боб применяет U j, известное только ему, к кубиту |$\psi$ i,k i, возвращает полученный кубит Алисе.
2. Алиса измеряет кубит в базе B k, получая бит.

Используя это, Боб обозначает операции U 0 и U 1 операторами тождества I и XZ соответственно, где бит j l отвечает за операцию U j. Бит j, выбранный Бобом и переданный по общедоступному каналу, достиг Алисы. Ева, подслушивающее лицо, не может получить никакой информации, перехватив переданные кубиты, хотя она может нарушить обмен, заменив переданные кубиты своими собственными. Это можно обнаружить следующим образом:
1. добавление битов четности
2. добавление ранее выбранных битовых последовательностей, которые могут быть адресами назначения и отправки или их хэшированными значениями, или какой-либо другой взаимно согласованной последовательностью.

Поскольку преобразования B k и U j могут меняться сколь угодно часто, Ева не может получить никаких статистических подсказок об их природе, перехватывая кубиты.

\subsubsection{Обобщение}
В фазе подготовки ключа-сообщения участвует только Боб, что позволяет расширить ее до трех сторон (Алиса, Боб, Селин) в отличие от стандартных протоколов [7]. I Фаза подготовки (Алиса, Селин)II Фаза подготовки ключа-сообщения (Боб)III Фаза измерения и получения (Алиса, Селин)
1. Алиса взяла бит i, преобразовав его в элемент секретной базы B k , Селин взяла бит i, преобразовав его в элемент секретной базы B t , обе отправили свои кубиты Бобу по квантовому каналу.
2. Боб применяет секретную операцию U j к кубитам |$\psi$ i,k i и |$\psi$ s,t i и возвращает полученные кубиты соответствующим сторонам.
3.Алиса и Селин измеряют кубиты в базе B k и B t, получая значение, посланное Бобом.

Обобщая протокол на n сторон, где Боб является центральной, будет получена квантовая сеть распределения ключей-сообщений.

\subsubsection{Вывод}
Представленный квантовый протокол с его вариантами обеспечивает безопасную пересылку информации прямой связи между двумя или более сторонами. Обобщения для n сторон могут создать сеть массовой пересылки информации для n - 1 сторон, одна из которых является центром распределения ключевых сообщений. Этот протокол используется для безопасного распространения прикладных ключевых сообщений на большие расстояния, поскольку позволяет пересылать массивные кубиты. Поскольку предложенный протокол не использует классическую связь, он не подвержен атаке "человек посередине" на классический канал связи, от которой страдают протоколы квантовой криптографии типа BB84 [8]. С другой стороны, реализация этого протокола может быть сложнее, поскольку кубиты обмениваются несколько раз.

\subsection{\review}

The topic of the article is quantum key distribution protocol with private-public key. At the beginning the author describes the new protocol of quantum key distribution. It is based on an additional private and public communication channel. Further the author makes a few critical remarks on other QKD protocol. He shows that his protocol has fewer errors than its predecessors.

In conclusion the author reveals that your protocol is better then BB84. However, it also has a number of drawbacks and possible ways of development.
 

\subsection{\dic}
\begin{multicols}{2}
	\begin{itemize}
		
		\item algorithms - алгоритм
		\item analysis - анализ
		
		\item appropriate - подходящий
		\item approximately - примерно
		
		
		\item basis - основа
		\item beam - луч
		
		\item binary - двоичных
		\item bit - бит
		
		\item capacity - вместимость
		
		\item channel - канал
		
		\item coherent - связный
		\item combination - комбинации
		
		\item communication - связь
		\item compare - сравнить
		
		\item computation - вычисления
		\item computers - компьютеров
		
		\item condition - условие
		\item conjugate - спряжение
		\item considered - рассмотрено
		\item contain - содержат
		
		\item correlation - корреляция
		
		\item cryptography - криптография
		
		\item decode - декодировать
		\item decoy - ловушка
		\item density - плотность
		
		\item dependence - зависимость
		\item detect - обнаружить
		
		\item deterministic - детерминированный
		
		\item developing - разработка
		
		\item difference - разница
		
		\item differentiate - дифференцировать
		
		\item distribution - распределение
		
		\item eavesdropper - подслушиватель
		
		\item encoded - закодировано
		\item entaglement - запутанность
		\item equivalently - эквивалентно
		
		\item imperfections - недостаток
		\item implementation - реализация
		
		\item instances - экземпляров
		
		\item intensity - интенсивность
		\item intercept - перехват
		
		\item interferometer - интерферометр
		
		\item limitation - ограничение
		
		\item lossless - без потерь
		\item lossy - потери
		
		\item malicious - вредоносных
		\item managed - управляемых
		
		\item mathematical - математических
		\item matrix - матрица
		
		\item measurement - измерения
		
		\item method - метод
		\item mimic - имитировать
		
		\item modification - модификация
		
		\item neglected - пренебрегают
		
		\item normalized - нормализовано
		
		\item operator - оператор
		\item optical - оптический
		\item optimal - оптимальный
		
		\item optional - опционально
		
		\item orthogonal - ортогональный
		\item orthogonality - ортогональность
		\item orthonormal - ортонормированный
		
		\item phase - фаза
		\item photon - фотон
		
		\item polarization - поляризация
		
		
		\item probabilistic - вероятностный
		\item probability - вероятность
		\item problem - проблема
		
		\item projecting - проектирование
		
		\item property - свойство
		\item proportion - пропорция
		\item proposal - предложение
		
		\item protocol - протокол
		
		\item prove - доказательство
		
		\item provide - обеспечить
		\item provides - обеспечивает
		
		\item public - общедоступных
		\item pulse - импульс
		
		\item quantum - квантовый
		\item random - случайных
		\item randomize - рандомизировать
		
		\item signature - подпись
		
		\item strategy - стратегия
		
		\item string - строка
		
		\item symmetric - симметричный
		
		\item theoretical - теоретический
		\item theory - теория
		
		\item transmission - передача
		
		\item vacuum - вакуум
		\item value - значение
		
		\item variable - переменная
		
		
		
	\end{itemize}
\end{multicols}

\section{Quantum cryptography protocols robust against photon number splitting attacks for weak laser pulse implementations}
\subsection{Article}

\subsubsection*{Abstract}

We introduce a new class of quantum quantum key distribution protocols, tailored to be robust against photon number splitting (PNS) attacks. We study one of these protocols, which differs from the BB84 only in the classical sifting procedure. This protocol is provably better than BB84 against PNS attacks at zero error.


\subsubsection*{Main}
Quantum cryptography, or more precisely quantum key distribution (QKD) is the only physically secure method for the distribution of a secret key between two distant partners, Alice and Bob [1]. Its security comes from the well-known fact that the measurement of an unknown quantum state modifies the state itself: thus an eavesdopper on the quantum channel, Eve, cannot get information on the key without introducing errors inthe correlations between Alice and Bob. In equivalent terms, QKD is secure because of the no-cloning theorem of quantum mechanics: Eve cannot duplicate the signal and forward a perfect copy to Bob.

In the last years, several long-distance implementations of QKD have been developed, that use photons as information carriers and optical fibers as quantum channels [1]. Most often, although not always [2], Alice sends to Bob a weak laser pulse in which she has encoded the bit. Each pulse is a priori in a coherent state | ue i$\theta$ i of weak intensity, typically u ~ 0.1 photons. However, since no reference phase is available outside Alice’s office, Bob and Eve have no information $\theta$. Consequently, they see the mixed state R d$\theta$ on p = 2pi| ue i$\theta$ ih ue i$\theta$ |. P This state can be re-written as a mixture of Fock states, n p n |nihn|, with the number n of photons distributed according to the Poissonian statistics of mean u, p n = p n (u) = e -u u n /n!. Because two realizations of the same density matrix are indistinguishable, QKD with weak pulses can be re-interpreted as follows: Alice encodes her bit in one photon with frequency p 1 , in two photons with frequency p 2 , and so on, and does nothing with frequency p 0 . Thus, in weak pulses QKD,a rather important fraction of the non-empty pulses actually contain more than one photon. For these pulses,Eve is then no longer limited by the no-cloning theorem:she can simply keep some of the photons while letting the others go to Bob. Such an attack is called photon-number splitting (PNS) attack. Although PNS attacks are far beyond today’s technology [3], if one includes them in the security analysis, the consequences are dramatic [4,5].

In this Letter, we present new QKD protocols that are secure against PNS attack up to significantly longer distances, and that can thus lead to a secure implementation of QKD with weak pulses. These protocols are better tailored than the ones studied before to exploit the correlations that can be established using p. The basic idea is that Alice should encode each bit into a pair of non-orthogonal states belonging to two or more suitable sets.
The structure of the paper is as follows. First, we review the PNS attack on the first and best-known QKD protocol, the BB84 protocol [6], in order to understand why this attack is really devastating when the bit is encoded into pairs of orthogonal states. Then we present the benefits of using non-orthogonal states, mostly by focusing on a specific new protocol which is a simple modification of the BB84.
PNS attacks on the BB84 protocol. Alice encodes each bit in a qubit, either as an eigen state of O x (| + xi coding
0 or | - xi coding 1) or as an eigen state of o z (| + zi coding 0 or | - zi coding 1). The qubit is sent to Bob, who measures either o x or O z . Then comes a classical procedure known as ”sifting” or ”basis-reconciliation”: Alice communicates to Bob through a public classical channel the basis, x or z, in which she prepared each qubit.
When Bob has used the same basis for his measurement,he knows that (in the absence of perturbations, and in particular in the absence of Eve) he has got the correct result. When Bob has used the wrong basis, the partners simply discard that item.
Consider now the implementation of the BB84 protocol with weak pulses. Bob’s raw detection rate is the probability that he detects a photon per pulse sent by Alice. In the absence of Eve, this is given b where N is the quantum efficiency of the detector (typically 10\% at telecom wavelengths), and N B is attenuation due to the losses in the fiber of length l:N B = 10 -B/10 , B = $\alpha$ l [dB] . Below, when we give a distance, we assume the typical value $\alpha$ = 0.25 dB/km. The approximate equality in (1)is valid if N det N B p n n << 1 for all n, which is always the case in weak pulses QKD.
If we endow Eve with unlimited technological power within the laws of physics, the following PNS attack (storage attack) is in principle possible [4,5]: (I) Eve counts the number of photons, using a photon-number quantum non-demolition (QND) measurement; (II) she blocks the single photon pulses, and for the multi-photon pulses she stores one photon in a quantum memory;she forwards the remaining photons to Bob using a perfectly transparent quantum channel, N B = 1 [7]; (III) she waits until Alice and Bob publicly reveal the used base sand correspondingly measures the photons stored in her quantum memory: she has to discriminate between two orthogonal states, and this can be done deterministically.
This way, Eve has obtained full information about Alice’s bits, thence no processing can distill secret keys for the legitimate users; moreover, Eve hasn't introduced any error on Bob’s side.
The unique constraint on PNS attack is that Eve’s presence should not be noticed; in particular, Eve must ensure that the rate of photons received by Bob (1) is not modified [8]. Thus, the PNS attack can be performed on all pulses only when the losses that Bob expects because of the fiber are equal to those introduced by Eve’s storing and blocking photons, that is, when the attenuation in the fiber is larger than a critical value B c BB84 defined by W For u = 0.1, we find B c BB84 = 13 dB, that is l c BB84 ~ 50 km. For shorter distances, Eve can optimize her attack, but won’t be able to obtain full information; Alice and Bob can therefore use a privacy amplification scheme to retrieve a shorter secret key from their data. In conclusion, for B >= B c BB84 , the weak-pulses implementation of the BB84 protocol becomes in principle insecure, even for zero quantum-bit error rate (QBER).
Encoding in non-orthogonal states. The extreme weakness of the BB84 protocol against PNS attacks is due to the fact that whenever Eve can keep one photon, she gets all the information, because after the sifting phase she has to discriminate between two eigen states of a known Hermitian operator. Intuition suggests then that the robustness against PNS attacks can be increased by using protocols that encode the classical bit into pairs of non orthogonal states, that cannot be discriminated deterministically. We prove that this intuition is correct.

To fix the ideas, consider the following protocol using four states: Alice encodes each bit in the state of a qubit,belonging  either to the set A = |0 a i, |1 a i or to the set B = |0 b i, |1 b i , with |h0 a |1 a i| = |h0 b |1 b i| = $\sigma$ 6 = 0(Fig. 1, left). In the absence of an eavesdropper, Bob can be perfectly correlated with Alice: in fact, although the two states are not orthogonal, one can construct a generalized measurement that unambiguously discriminates between the two. The price to pay is that sometimes one gets an inconclusive result [9]. Such a measurement can be realized by a selective filtering, that is filter whose effect is not the same on all states, followed by a von Neumann measurement on the photons that pass the filter [10]. In the example of Fig. 1, the filter that discriminates between the elements of A is given by | + xih1 a | + | - xih0 a  | , where |F  i is the F A =  1+$\sigma$ state orthogonal to |Fi. When the photons are prepared in a state of the pair A, a fraction 1 - $\sigma$ of them pass this filter, and in this case the von-Neumann measurement of O x achieves the discrimination. It is then clear how the cryptography protocol generalizes BB84: Bob randomly applies on each qubit one of the two filters F A or F B , and measures O x on the outcome. Later, Alice discloses for each bit the set A or B: Alice and Bob discard all the items in which Bob has chosen the wrong filter and all the inconclusive results.

Of course, since not all the qubits will pass the filter even when it was correctly chosen, there is a small nuisance on Bob’s side because the net key rate is decreased. This is compensated by increasing u by a factor 1/(1-$\sigma$).
However, the nuisance is by far bigger on Eve’s side, even when the increased mean number of photons u is taken into account. We shall give a detailed analysis of the PNS attacks below for a specific protocol, but a simple estimate shows the origin of the improved robustness. Eve can obtain full information only when (i) she can block all the pulses containing one and two photons, and (ii) on the pulses containing three or more photons, she performs a suitable unambiguous discrimination measurement (see below) and obtains a conclusive outcome, which happens only with probability p ok < 1. Consequently, the critical u attenuation is defined by R raw (B c ) = N det p 3 ( 1-$\sigma$) p ok ,and is determined by p 3 instead of p 2 as in the BB84, see(3). For typical values, B c - B c BB84 ~ 10 dB, which means an improvement of some 40km in the distance [11].

A specific protocol. Here is an astonishingly simple protocol using four non-orthogonal states. Alice sends randomly one of the four states | + xi or | + zi; Bob measures either O x or O z . Thus, at the ”quantum” level,the protocol is identical to BB84, and can be immediately implemented with the existing devices. However,we modify the classical sifting procedure: instead of revealing the basis, Alice announces publicly one of the four pairs of non-orthogonal states A $\omega$,$\omega$  = |$\omega$xi, |$\omega$  zi ,with $\omega$, $\omega$ {+, -}, and with the convention that | + xi code for 0 and | + zi code for 1. Within each set, the overlap of the two states is $\sigma$ =  1 2 . Because of the peculiar choice of states, the usual procedure of choosing randomly between O x or O z turns out to implement the most effective unambiguous discrimination. For definiteness, suppose that for a given qubit Alice has sent | + xi,and that she has announced the set A +,+ . If Bob has measured O x , which happens with probability 2 1 , he has certainly got the result +1; but since this result is possible for both states in the set A +,+ , he has to discard it.
If Bob has measured O z and got +1, again he cannot discriminate. But if he has measured O z and got -1, then he knows that Alice has sent | + xi and adds a 0 to his key.
By symmetry, we see that after this sifting procedure Bob is left with 14 of the raw list of bits, compared to the 12 of the original BB84 protocol. Thus, for a fair comparison with BB84 using u = 0.1, we shall take here u = 0.2, so that the net key rates without eavesdropper at a given distance are the same for both protocols. In spite of the fact that a larger u is used (that is, multi-photon pulses are more frequent), this new protocol is provably better than BB84 against PNS attacks at QBER= 0. This is our main claim, and is demonstrated in the following.

PNS attacks at QBER=0. First, let us prove something that we mentioned above, namely: for protocols using four states like the one under study, Eve can obtain full information from three-photon pulses by using strategies based on unambiguous state-discrimination.
Such strategies have also been considered for BB84, because (although worse than the storage attack for an all-powerful Eve) they don’t require a quantum memory[12], and in their simplest implementation the photon number QND measurement is not required either [13].
The most powerful of these attacks, against which any protocol using four states becomes completely insecure for the three-photon pulses, goes as follows [14]. A pulse containing three photons is necessarily in one of the four x3x3x3 states |f 1 i = | + zi , |F 2 i = | + xi , |F 3 i = | - zi ,x3|F 4 i = | - xi ; that is, in the symmetric subspace of 3 qubits. The dimension of this subspace is 4, and it can be shown that all the |F k i x3 are linearly independent[15]. Therefore, there exist a measurement M that distinguishes unambiguously among them, with some probability of success. In the present case, there exist even four orthogonal states of three qubits, |$\Phi$ k i, k = 1,  , 4,such that |h$\Phi$ i |F j i| =  1 2 B ij [16]. The measurement M is then any von-Neumann measurement discriminating the|$\Phi$ k i; it will give a conclusive outcome with probability p ok = 12 , which is optimal [15,17].
It is then clear that Eve can obtain full information if she can block all the one- and two-photons pulses and half of the three-photon pulses, by applying the following PNS attack: (I) she measures the number of photons;(II) she discards all pulses containing less than 3 photons;(III) on the pulses containing at least 3 photons, she performs M, and if the result is conclusive (which happens with probability p ok > 2 ) she sends a new photon prepared in the good state to Bob. We refer to this attack as to intercept-resend with unambiguous discrimination(IRUD) attack. Neither the quantum memory is needed,nor is the lossless channel, since the new state can be prepared by a friend of Eve located close to Bob.

The critical attenuation B c at which the IRUD attack becomes always possible is defined by N B c u = p ok p 3 (u);for u = 0.2, this gives B c = 25.6 dB ~ 2B c BB84 . Thus, the ultimate limit of robustness (in the case of zero errors) is shifted from - 50km up to - 100km by using our simpl emodification of the BB84 protocol. To further increase the limit of 100km, one can move to protocols using six or more non-orthogonal states [17].
Figure 2 plots Eve’s information for the best PNS attack at QBER= 0, as a function of the attenuation. Note that the new protocol is better than BB84 at any distance. For almost all B < B c , the best PNS attack is not the IRUD but a storage attack, in which Eve keeps one or two photons in a quantum memory and waits for the announcements of the sifting phase. Recall that in BB84,this kind of attack provides Eve with full information. In our protocol Alice announces sets of two non-orthogonal states, so storage attacks give Eve only a limited amount of information.
 If Eve keeps n photons and the overlap can obtain is $\sigma$ (here, 1/ 2), the largest information she p is I(n, $\sigma$) = 1 - H(P, 1 - P ) with P = 2 (1 + 1 - $\sigma$ 2n )[9]. In particular, Eve obtains I(1,  1 2 ) ~ 0.4 bits/pulse for the attenuation B 1 at which she can always keep one photon (B 1 = 11 dB for u = 0.2).

In conclusion: in the limiting case of QBER= 0, our protocol is always more secure than BB84 against PNS attacks, and can be made provably secure against such attacks in regions where BB84 is already provably insecure. Recall that the comparison is made by fixing the net key rates without eavesdropper at a given distance.
Attacks at QBER>0 on the new protocol. In real experiments, dark counts in the detectors and misalignement of optical elements always introduce some errors.
It is then important to show that the specific protocol we presented does not break down if a small amount of error on Bob’s side is allowed. Several attacks at nonzero QBER are described in detail in Ref. [17]. Here, we sketch the analysis of two individual attacks.
First, let us suppose that Eve uses the phase-covariant cloning machine that is the optimal individual attack against BB84 [18]. In the case of the present protocol, Eve can extract less information from her clones,again because Alice does not disclose a basis but a set of non-orthogonal states. As a consequence, the condition I Bob = I Eve is fulfilled up to QBER=15\% [17], a value which is slightly higher than the 14,67\% obtained for BB84. So our new protocol, designed to avoid PNS attacks in a weak-pulses implementation, seems to be robust also against individual eavesdropping in a single photon implementation. Incidentally note that, in the case of a single-photon implementation, our protocol is at least as secure as the B92 protocol in the sens of ”unconditional security” proofs [19]. This is because our protocol can be seen as a modified B92, where Alice chooses randomly between four sets of non-orthogonal states [20].
The second kind of individual attacks that we like to discuss, and that we call PNS+cloning attacks, are specific to imperfect sources. Focus on the range B =10 - 20 dB (see Fig. 2), where one-photon pulses can be blocked and the occurrence of three or more photons is still comparatively rare. Because for the BB84 Eve has already full information in this range, such attacks have never been considered before. Eve could take the two photons, apply an asymmetric 2 -> 3 cloning machine and send one of the clones to Bob; she keeps two clones and some information in the machine. By a suitable choice of the cloning machine, the QBER at which I Bob = I Eve is lowered down to - 9\% [17]. In Ref. [21], a successful qubit distribution over 67km with u = 0.2 and QBER= 5\% has been reported. Under the considered PNS attacks, such distribution is provably insecure using the sifting procedure of BB84, while it can yield a secret key if our sifting procedure is used.

In summary, we have shown that by encoding a classical bit in sets of non-orthogonal qubit states, quantum cryptography can be made significantly more robust against photon-number splitting attacks. We have presented a specific protocol, which is identical to the BB84protocol for all the manipulations at the quantum level and differs only in the classical sifting procedure. Under the studied attacks, our protocol is secure in a region where BB84 is provably insecure. Preliminary studies of more complex attacks suggest that it is at least as robust as BB84 in any situation, and could then replace it. Moreover, our encoding can easily be combined with more complex procedures on the quantum level, e.g. [22].

We thank Norbert Lütkenhaus and Daniel Collins for insightful comments. We acknowledge financial supports by the Swiss OFES and NSF within the European IST project EQUIP and the NCCR "Quantum Photonics".


\subsection{\trnas}
\subsubsection*{Аннотация}

Мы представляем новый класс протоколов квантового распределения ключей, разработанных с учетом устойчивости к атакам с расщеплением фотонного числа (PNS). Мы исследуем один из этих протоколов, который отличается от BB84 только классической процедурой просеивания. Этот протокол доказательно лучше BB84 против атак PNS при нулевой ошибке.


\subsubsection*{Основная часть}
Квантовая криптография, или, точнее, квантовое распределение ключей (КРК), является единственным физически безопасным методом распределения секретного ключа между двумя удаленными друг от друга партнерами, Алисой и Бобом [1]. Его безопасность обусловлена хорошо известным фактом, что измерение неизвестного квантового состояния изменяет само состояние: таким образом, подслушивающий квантовый канал, Ева, не может получить информацию о ключе без внесения ошибок в корреляции между Алисой и Бобом. В эквивалентных терминах, QKD является безопасным благодаря теореме квантовой механики о невозможности клонирования: Ева не может продублировать сигнал и передать Бобу его идеальную копию.

В последние годы было разработано несколько реализаций QKD на больших расстояниях, которые используют фотоны в качестве носителей информации и оптические волокна в качестве квантовых каналов [1]. Чаще всего, хотя и не всегда [2], Алиса посылает Бобу слабый лазерный импульс, в котором она закодировала бит. Каждый импульс априори находится в когерентном состоянии | ue i$\theta$ i слабой интенсивности, обычно u ~ 0,1 фотона. Однако, поскольку за пределами офиса Алисы нет эталонной фазы, Боб и Ева не имеют информации $\theta$. Следовательно, они видят смешанное состояние R d$\theta$ на p = 2pi| ue i$\theta$ ih ue i$\theta$ |. P Это состояние может быть переписано как смесь состояний Фока, n p n |nihn|, с числом n фотонов, распределенных согласно пуассоновской статистике среднего u, p n = p n (u) = e -u u n /n! Поскольку две реализации одной и той же матрицы плотности неразличимы, QKD со слабыми импульсами можно интерпретировать следующим образом: Алиса кодирует свой бит в один фотон с частотой p 1 , в два фотона с частотой p 2 , и так далее, и ничего не делает с частотой p 0 . Таким образом, в слабых импульсах QKD, довольно значительная часть непустых импульсов на самом деле содержит более одного фотона. Для этих импульсов Ева больше не ограничена теоремой об отсутствии клонирования: она может просто оставить себе некоторые фотоны, а остальные отдать Бобу. Такая атака называется атакой с разделением числа фотонов (PNS). Хотя атаки PNS находятся далеко за пределами сегодняшней технологии [3], если включить их в анализ безопасности, последствия будут плохими [4,5].

В этом письме мы представляем новые протоколы QKD, которые защищены от атак PNS на значительно больших расстояниях, и которые, таким образом, могут привести к безопасной реализации QKD со слабыми импульсами. Эти протоколы лучше, чем ранее изученные, приспособлены для использования корреляций, которые могут быть установлены с помощью p. Основная идея заключается в том, что Алиса должна закодировать каждый бит в пару неортогональных состояний, принадлежащих двум или более подходящим наборам.
Структура статьи выглядит следующим образом. Сначала мы рассмотрим атаку PNS на первый и самый известный протокол QKD, протокол BB84 [6], чтобы понять, почему эта атака действительно разрушительна, когда бит кодируется в пары ортогональных состояний. Затем мы представим преимущества использования неортогональных состояний, сосредоточившись в основном на конкретном новом протоколе, который является простой модификацией BB84.
PNS-атаки на протокол BB84. Алиса кодирует каждый бит в кубите либо как собственное состояние O x (| + xi кодирование
0 или | - xi кодирование 1) или как собственное состояние o z (| + zi кодирование 0 или | - zi кодирование 1). Кубит отправляется Бобу, который измеряет либо o x, либо O z . Затем следует классическая процедура, известная как "просеивание" или "согласование базиса": Алиса сообщает Бобу по общедоступному классическому каналу базис, x или z, в котором она подготовила каждый кубит.
Если Боб использовал одну и ту же основу для своего измерения, он знает, что (в отсутствие возмущений и, в частности, в отсутствие Евы) он получил правильный результат. Если Боб использовал неправильную основу, партнеры просто отбрасывают этот элемент.
Рассмотрим теперь реализацию протокола BB84 со слабыми импульсами. Необработанный коэффициент обнаружения Боба - это вероятность того, что он обнаружит фотон в каждом импульсе, посланном Алисой. В отсутствие Евы, эта вероятность равна b, где N - квантовая эффективность детектора (обычно 10\% на телекоммуникационных длинах волн), а N B - затухание из-за потерь в волокне длиной l:N B = 10 -B/10 , B = $\alpha$ l [дБ]. Ниже, когда мы указываем расстояние, мы принимаем типичное значение $\alpha$ = 0,25 дБ/км. Приблизительное равенство в (1)справедливо, если N det N B p n n << 1 для всех n, что всегда имеет место в слабых импульсах QKD.
Если наделить Еву неограниченной технологической мощью в рамках законов физики, то в принципе возможна следующая атака ПНС (атака хранения) [4,5]: (I) Ева подсчитывает количество фотонов, используя измерение квантового неуничтожения фотонов (QND); (II) она блокирует однофотонные импульсы, а для многофотонных импульсов сохраняет один фотон в квантовой памяти; остальные фотоны она пересылает Бобу, используя абсолютно прозрачный квантовый канал, N B = 1 [7]; (III) она ждет, пока Алиса и Боб публично раскроют использованную базу, соответственно измеряет фотоны, хранящиеся в ее квантовой памяти: она должна различать два ортогональных состояния, и это может быть сделано детерминированно.
Таким образом, Ева получила полную информацию о битах Алисы, поэтому никакая обработка не сможет извлечь секретные ключи для законных пользователей; кроме того, Ева не внесла никакой ошибки на стороне Боба.
Уникальное ограничение на PNS-атаку заключается в том, что присутствие Евы не должно быть замечено; в частности, Ева должна гарантировать, что скорость фотонов, полученных Бобом (1), не будет изменена [8]. Таким образом, атака PNS может быть выполнена на всех импульсах только тогда, когда потери, которые Боб ожидает из-за волокна, равны потерям, вносимым Евой при хранении и блокировании фотонов, то есть когда затухание в волокне больше критического значения B c BB84, определяемого W Для u = 0,1 находим B c BB84 = 13 дБ, то есть l c BB84 ~ 50 км. Для меньших расстояний Ева может оптимизировать свою атаку, но не сможет получить полную информацию; поэтому Алиса и Боб могут использовать схему усиления конфиденциальности для получения более короткого секретного ключа из своих данных. В заключение следует отметить, что для B >= B c BB84 реализация протокола BB84 с использованием слабых импульсов становится в принципе небезопасной, даже при нулевой квантовой частоте битовых ошибок (QBER).
Кодирование в неортогональных состояниях. Крайняя слабость протокола BB84 против атак ПНС объясняется тем, что всякий раз, когда Ева может сохранить один фотон, она получает всю информацию, поскольку после фазы просеивания ей приходится различать два собственных состояния известного гермитианского оператора. Интуиция подсказывает, что устойчивость к атакам PNS можно повысить, используя протоколы, которые кодируют классический бит в пары неортогональных состояний, которые нельзя детерминированно различать. Мы доказываем, что эта интуиция верна.

Чтобы закрепить эти идеи, рассмотрим следующий протокол, использующий четыре состояния: Алиса кодирует каждый бит в состоянии кубита, принадлежащего либо множеству A = |0 a i, |1 a i, либо множеству B = |0 b i, |1 b i , при этом |h0 a |1 a i| = |h0 b |1 b i| = $\sigma$ 6 = 0 (рис. 1, слева). В отсутствие подслушивающего устройства Боб может быть идеально коррелирован с Алисой: на самом деле, хотя эти два состояния не ортогональны, можно построить обобщенное измерение, которое однозначно различает эти два состояния. Платой за это является то, что иногда получается неубедительный результат [9]. Такое измерение может быть реализовано с помощью селективной фильтрации, то есть фильтра, эффект которого не одинаков для всех состояний, с последующим фон Неймановским измерением фотонов, прошедших фильтр [10]. В примере рис. 1 фильтр, различающий элементы A, задается | + xih1 a | + | - xih0 a | , где |F i - состояние F A = 1+$\sigma$, ортогональное к |Fi. Когда фотоны подготовлены в состоянии пары A, часть из них 1 - $\sigma$ проходит этот фильтр, и в этом случае фон-Неймановское измерение O x достигает дискриминации. Тогда становится ясно, как криптографический протокол обобщает BB84: Боб случайным образом применяет к каждому биту один из двух фильтров F A или F B , и измеряет O x по результату. Позже Алиса раскрывает для каждого бита набор A или B: Алиса и Боб отбрасывают все элементы, в которых Боб выбрал неправильный фильтр, и все неубедительные результаты.

Конечно, поскольку не все кубиты пройдут фильтр, даже если он был выбран правильно, на стороне Боба возникает небольшое неудобство, поскольку чистая ключевая скорость уменьшается. Это компенсируется увеличением u в 1/(1-$\sigma$) раз.
Тем не менее, неприятность на стороне Евы гораздо больше, даже если принять во внимание увеличенное среднее число фотонов u. Ниже мы дадим подробный анализ атак PNS для конкретного протокола, но простая оценка показывает происхождение улучшенной стойкости. Ева может получить полную информацию только тогда, когда (i) она может блокировать все импульсы, содержащие один и два фотона, и (ii) на импульсах, содержащих три или более фотонов, она выполняет подходящее однозначное измерение дискриминации (см. ниже) и получает окончательный результат, что происходит только с вероятностью p ok < 1. Следовательно, критическое затухание u определяется R raw (B c ) = N det p 3 ( 1-$\sigma$) p ok , и определяется p 3, а не p 2, как в BB84, см.(3). Для типичных значений, B c - B c BB84 ~ 10 дБ, что означает улучшение расстояния примерно на 40 км [11].

Конкретный протокол. Вот удивительно простой протокол, использующий четыре неортогональных состояния. Алиса посылает случайным образом одно из четырех состояний | + xi или | + zi; Боб измеряет либо O x, либо O z . Таким образом, на "квантовом" уровне протокол идентичен BB84 и может быть немедленно реализован с помощью существующих устройств. Однако мы модифицируем классическую процедуру отсеивания: вместо того, чтобы раскрывать базис, Алиса публично объявляет одну из четырех пар неортогональных состояний A $\omega$,$\omega$ = |$\omega$xi, |$\omega$ zi , с $\omega$, $\omega$ {+, -}, и с условием, что | + xi означает 0, а | + zi - 1. В пределах каждого набора перекрытие двух состояний равно $\sigma$ = 1 2 . Из-за особого выбора состояний обычная процедура случайного выбора между O x и O z оказывается наиболее эффективной для однозначной дискриминации. Для определенности предположим, что для данного кубита Алиса послала | + xi ,и что она объявила набор A +,+ . Если Боб измерил O x , что происходит с вероятностью 2 1 , он, конечно, получил результат +1; но поскольку этот результат возможен для обоих состояний в наборе A +,+ , он должен отбросить его.
Если Боб измерил O z и получил +1, то он снова не может отличить. Но если он измерил O z и получил -1, то он знает, что Алиса послала | + xi и добавляет 0 к своему ключу.
По симметрии, мы видим, что после этой процедуры отсеивания у Боба остается 14 необработанных битов, по сравнению с 12 в оригинальном протоколе BB84. Таким образом, для справедливого сравнения с BB84, использующим u = 0,1, мы возьмем здесь u = 0,2, так что скорости передачи ключей без подслушивающего устройства на заданном расстоянии одинаковы для обоих протоколов. Несмотря на то, что используется большее u (то есть многофотонные импульсы происходят чаще), этот новый протокол доказательно лучше BB84 против PNS-атак при QBER = 0. Это наше главное утверждение, которое демонстрируется ниже.

Атаки PNS при QBER=0. Во-первых, давайте докажем то, что мы упоминали выше, а именно: для протоколов, использующих четыре состояния, подобных исследуемому, Ева может получить полную информацию из трехфотонных импульсов, используя стратегии, основанные на однозначной дискриминации состояний.
Такие стратегии также рассматривались для BB84, поскольку (хотя они и хуже атаки хранения для всемогущей Евы) они не требуют квантовой памяти[12], и в их простейшей реализации измерение числа фотонов QND также не требуется[13].
Самая мощная из этих атак, против которой любой протокол, использующий четыре состояния, становится совершенно небезопасным для трехфотонных импульсов, происходит следующим образом [14]. Импульс, содержащий три фотона, обязательно находится в одном из четырех состояний x3x3x3 |f 1 i = | + zi , |F 2 i = | + xi , |F 3 i = | - zi ,x3|F 4 i = | - xi ; то есть в симметричном подпространстве из 3 кубитов. Размерность этого подпространства равна 4, и можно показать, что все |F k i x3 линейно независимы[15]. Поэтому существует измерение M, которое однозначно различает их с некоторой вероятностью успеха. В нашем случае существует даже четыре ортогональных состояния трех кубитов, |$\Phi$ k i, k = 1, , 4, таких, что |h$\Phi$ i |F j i| = 1 2 B ij [16]. Тогда измерение M - это любое фон-Неймановское измерение, дискриминирующее|$\Phi$ k i; оно даст окончательный результат с вероятностью p ok = 12, что является оптимальным [15,17].
Тогда становится ясно, что Ева может получить полную информацию, если она сможет заблокировать все одно- и двухфотонные импульсы и половину трехфотонных импульсов, применив следующую атаку PNS: (I) она измеряет количество фотонов; (II) она отбрасывает все импульсы, содержащие менее 3 фотонов; (III) на импульсах, содержащих по крайней мере 3 фотона, она выполняет M, и если результат окончательный (что происходит с вероятностью p ok > 2 ), она посылает Бобу новый фотон, подготовленный в хорошем состоянии. Мы называем эту атаку атакой перехвата-отправки с однозначной дискриминацией (IRUD). Ни квантовая память не нужна, ни канал без потерь, так как новое состояние может быть подготовлено другом Евы, находящимся рядом с Бобом.

Критическое затухание B c, при котором атака IRUD становится всегда возможной, определяется N B c u = p ok p 3 (u); для u = 0,2 это дает B c = 25,6 дБ ~ 2B c BB84 . Таким образом, предел стойкости (в случае нулевых ошибок) сдвигается с - 50 км до - 100 км при использовании нашей простой модификации протокола BB84. Для дальнейшего увеличения предела в 100 км можно перейти к протоколам, использующим шесть и более неортогональных состояний [17].
На рисунке 2 показана информация Евы для лучшей атаки PNS при QBER = 0, как функция затухания. Обратите внимание, что новый протокол лучше BB84 на любом расстоянии. Почти для всех B < B c , лучшей атакой PNS является не IRUD, а атака хранения, при которой Ева хранит один или два фотона в квантовой памяти и ждет объявления фазы просеивания. Напомним, что в BB84 такая атака дает Еве полную информацию. В нашем протоколе Алиса объявляет наборы из двух неортогональных состояний, поэтому атаки хранения дают Еве лишь ограниченный объем информации.
Если Ева сохраняет n фотонов и перекрытие может получить $\sigma$ (здесь 1/ 2), то наибольшая информация, которую она p получает, это I(n, $\sigma$) = 1 - H(P, 1 - P ) при P = 2 (1 + 1 - $\sigma$ 2n )[9]. В частности, Ева получает I(1, 1 2 ) ~ 0,4 бит/импульс для затухания B 1, при котором она всегда может сохранить один фотон (B 1 = 11 дБ для u = 0,2).

В заключение: в предельном случае QBER = 0 наш протокол всегда более безопасен, чем BB84, против атак PNS, и может быть доказательно безопасен против таких атак в областях, где BB84 уже доказательно небезопасен. Напомним, что сравнение проводится путем фиксации чистой скорости передачи ключей без подслушивающего устройства на заданном расстоянии.
Атаки при QBER>0 на новый протокол. В реальных экспериментах отсчеты в детекторах и несовершенность оптических элементов всегда вносят некоторые ошибки.
Затем важно показать, что представленный нами конкретный протокол не разрушается, если допускается небольшое количество ошибок на стороне Боба. Несколько атак при ненулевом QBER подробно описаны в Ref. [17]. Здесь мы приводим анализ двух отдельных атак.
Во-первых, предположим, что Ева использует фазово-ковариантную машину клонирования, которая является оптимальной индивидуальной атакой против BB84 [18]. В случае настоящего протокола Ева может извлечь меньше информации из своих клонов, опять же потому, что Алиса раскрывает не базис, а набор неортогональных состояний. Как следствие, условие I Bob = I Eve выполняется до QBER=15\% [17], что немного больше, чем 14,67\%, полученных для BB84. Таким образом, наш новый протокол, разработанный для предотвращения атак PNS в слабоимпульсной реализации, похоже, устойчив и к индивидуальному подслушиванию в однофотонной реализации. Кстати, отметим, что в случае однофотонной реализации наш протокол, по крайней мере, так же безопасен, как и протокол B92 в смысле доказательств "безусловной безопасности" [19]. Это потому, что наш протокол можно рассматривать как модифицированный B92, в котором Алиса выбирает случайным образом между четырьмя наборами неортогональных состояний [20].
Второй вид индивидуальных атак, которые мы хотим обсудить и которые мы называем PNS+cloning attacks, специфичны для несовершенных источников. Сосредоточьтесь на диапазоне B =10 - 20 дБ (см. рис. 2), где однофотонные импульсы могут быть заблокированы, а появление трех и более фотонов все еще сравнительно редко. Поскольку для BB84 Ева уже имеет полную информацию в этом диапазоне, такие атаки никогда ранее не рассматривались. Ева может взять два фотона, применить асимметричную машину клонирования 2 -> 3 и послать один из клонов Бобу; она сохраняет два клона и некоторую информацию в машине. При подходящем выборе машины клонирования, QBER, при котором I Боба = I Евы, снижается до - 9\% [17]. В работе [21] [21] сообщается об успешном распространении кубитов на расстояние 67 км с u = 0.2 и QBER = 5\%. При рассмотренных атаках PNS такое распределение доказательно небезопасно при использовании процедуры просеивания BB84, в то время как оно может дать секретный ключ при использовании нашей процедуры просеивания.

Подводя итог, мы показали, что, кодируя классический бит в наборы неортогональных состояний кубитов, можно значительно повысить устойчивость квантовой криптографии к атакам с расщеплением числа фотонов. Мы представили конкретный протокол, который идентичен протоколу BB84 по всем манипуляциям на квантовом уровне и отличается только классической процедурой просеивания. При изученных атаках наш протокол безопасен в области, где BB84 доказательно небезопасен. Предварительные исследования более сложных атак показывают, что он, по крайней мере, так же надежен, как BB84 в любой ситуации, и может заменить его. Более того, наше кодирование может быть легко объединено с более сложными процедурами на квантовом уровне, например, [22].

Мы благодарим Норберта Люткенхауса и Дэниела Коллинза за содержательные комментарии. Мы признательны за финансовую поддержку со стороны Швейцарского OFES и NSF в рамках европейского проекта IST EQUIP и NCCR "Квантовая фотоника".
\subsection{\review}
ss
\subsection{\dic}
ss

\section{Quantum key distribution based on random grouping bell state measurement}
\subsection{Article}

\subsubsection*{Abstract}

We propose a four-qubit quantum key distribution protocol via two Bell states which constitute a transmitted unit from the sender to the receiver in each communication.
An encryption here is designed by randomly grouping four qubits of a unit into two new couples, which is a way to increase the possibility of detecting the eavesdropper. Ultimately,the receiver randomly measures this grouped unit with two Bell state measurements. From the comparison of grouping information of these four qubits, we find that the two sides in a valid communication can discover the illegal party in the channel. In the proposed protocol, the receiver measures the unit when he receives it instantly, which is an efficient way to overcome the ultrashort storage time of quantum state.
Index Terms— Quantum communication, quantum key distribution, Bell state measurement.

\subsubsection{Introduction}
Quantum key distribution (QKD) is a promising technology to protect the security of classical information in quantum epoch [1]. It enables two parties to share a secret key with unconditional security, which can then be used to encrypt and decrypt messages. Many works have been conducted to promote the development of QKD protocols. In 1984, it was Bennett and Brassard who first introduced the QKD protocol by using two mutually unbiased bases of photon’s polarization degree of freedom [2]. Later Ekert proposed another QKD protocol which was called E91 based on Einstein-Podolsky Rosen (EPR) pairs [3]. After that, various QKD protocol sare theoretically proposed and experimentally realized, such as QKD protocols designed with single-photon [4], multiple states [5] and Bell state [8]. Among these works, photons are extensively used to carry information because they are easy to manipulate and they transmit at light speed.
As the quantum channel, Bell state was firstly proposed by [6] and verified to be the maximally entangled state of a two-qubit quantum system. Compared with other multi-qubit states, such as W state, GHZ state and cluster state, Bell state is easier to prepare via nonlinear process [7]. In reference [8],two parties share the secret key by comparing the form of initial Bell state and the outcome of entanglement swapping.
Then, [9] improved the total efficiency of the communication to 100\% compared with the former 50\% in [8]. Reference [10]presented the first authenticated semi-quantum key distribution protocol without using authenticated classical channels based on Bell states. In this letter, we propose a protocol to prevent the eavesdropper with lower qubit error rate and shorter detecting key bits based on a four-qubit state which consists of two couples of Bell states.
In our protocol, a group of four-qubit state are prepared each time and sent from the sender to the receiver. The receiver performs quantum state measurement immediately after he receives the qubits. Compared with the two-way protocols where the quantum state needs to be preserved until the tranmission is finished in [8] and [9], our protocol can overcome the ultrashort coherence time of quantum states.
Every four qubits form a unit to transfer information from the sender to the receiver in each communication. The four qubits are sent in random order by the sender and received by the receiver in a randomly grouping measurement. Our calculation shows that the qubit error rate is 4.17\%, which is lower than 46.875\% in [11]. Furthermore, only 11 bits are needed to detect the eavesdropper in our QKD protocol which is smaller than 72 bits in BB84 protocol [2] with the same security.
\subsubsection{QKD protocol based on four-qubit states}

Quantum Channel With Bell State References [8] and [9] proposed two QKD protocols, both used Bell states distributed from the sender to the receiver. In their protocols, two pairs of Bell states are shared between two legal parties of communication. The sender and the receiver both keep two qubits entangled with each other. After the simultaneous Bell state measurement (BSM) of the two parties, there exists an entanglement swapping among these four qubits.
To be more specific, let’s denote the four qubits of two Bell states as P 1 , P 2 , P 3 and P 4 . Entanglement exists between P 1and P 2 , P 3 and P 4 . After BSM of the two sides, P 1 and P 3 ,P 2 and P 4 become entangled, respectively. However, the two sides still need to keep their qubits during communication, and it is challenging to store qubits in the state of the art.
Consider the ultrashort storage time of qubits, a novel QKD protocol is proposed with a four-qubit state consisting of two couples of Bell states expressed as equation (1). A group of four-qubit state is prepared each time to send to Bob for measurement immediately. What is different from [8] and [9]is, the sender sends all the qubits to the receiver and the reciever performs quantum state measurement immediately when he receives the qubits. This is a one way process. The two parties don’t need to store the qubits thus the qubits are measured before they decoherence.

From (1) we can see that the state becomes superposition of four states which means we can obtain four different outcomes by combinations of BSMs. Note that equation (1)represents only one special case, other two forms are shownfor comparison in Table II with different groupings of qubits.
In brief, there exists three forms of random grouping of these four qubits, of which only equation (1) is defined as the right one. This is the primary technique to encrypt the information during communication. The proposed protocol is illustrated in Fig. 1, where Alice and Bob are the legitimate sender and receiver, respectively.
B. Preliminaries In this section, we show the details of the protocol in Fig. 1.
a) Step 1 state preparation: Alice prepares one group of the four-qubit state in equation (1). Each group consists of four qubits P y , y $\in$ {1, 2, 3, 4}. With such dense coding, the key information G of each group of the two Bell states is shown in Table I. Alice needs to record the information of qubits and the corresponding information G.
b) Step 2 qubit distribution: According to equation (1),Alice knows that the group order of these four qubits is{(P 1 , P 3 ), (P 2 , P 4 )}. Then, Alice rearranges these four qubits randomly and sends them to Bob via quantum channel.
c) Step 3 grouping measurement: Bob receives the qubits and divides them into two parts randomly. Then he performs BSMs on the two parts then sends the grouping information and the measurement results to Alice via classical channel.
d) Step 4 results comparison: Alice compares the results from Bob with her reserved information of P y . If Alice gets a coincident comparison, she announces ‘1’, and then the communication can proceed to step 5 or returns to step 1.
Otherwise, she announces ‘0’ and the communication returns to step 1 or ends. The different groupings of the four qubits by Bob are shown in Table II.
e) Step 5 raw key acquirement: Alice and Bob obtain a long binary sequence as the raw key R after multiple communications. If we define R A and R B as the raw keys belonging to Alice and Bob, respectively. Alice randomly selects parts of R A in different positions as her agreement key C A and announces the corresponding positions. Then, Bob chooses the agreed key C B in the same location in R B .
f) Step 6 privacy amplification: Bob chooses a set of C B bits as the parity bits D B and announces D B along with its corresponding positions. Alice selects her D A in the same way and compares it with D B . If the bit error rate is smaller than the threshold, the communication is secure and they can proceed to step 7; if not, they need to return to step 1 or terminate this communication.
g) Step 7 final key acquirement: The final keys R A and R Bare used to encrypt the secret message through the communication. Theoretically, R A= R B, R A is the raw key R A without C A under ideal condition, the same as R BC. Analyses of the Qubit Error Rate With Different Groupings We can see that there are three different kinds of groupings from Table II. One is right, another two are wrong. Therefore,the probability of correctly grouping four qubits by Bob is 13 .
If there is no eavesdropper in the communication channel,the qubit error rate $\epsilon$ 0 , which is the threshold during one communication, can be calculated as following.
(1) Bob divides the four qubits into the wrong grouping{(P 1 , P 2 ), (P 3 , P 4 )}.
(2)|C 1234 =In this case, he will only get |$\Phi$ -  12 |$\Phi$ +  34 . Bob will get the right binary random sequence G = 01 with the probability of 1.

By BSM, he can obtain four correct measurement results {|$\Phi$ + 1, |$\Phi$ - 1}, {|$\Phi$ - 1, |$\Phi$ + 1}, {|$\Psi$ + 1, |$\Psi$ - 1} and {|$\Psi$ - 1, |$\Psi$ + 1}. As a result, Bob will get the right binary random sequence G with the probability of 44 = 1 even he makes a wrong grouping.
In summary, the results of the BSM can only be partially right in the first case or totally right in the second case.
Therefore, Bob can acquire the right binary random sequence with the probability of 12 x 14 + 12 x 44 = 58 when grouping wrongly. Hence, the qubit error rate $\epsilon$ 0 is, 11 5= 0.0417. (3)$\epsilon$ 0 = 1 - ( + ) = 3 824
Note that if Bob chooses the right grouping with probability 1, he can obtain the right binary random sequence with the probability of 1. Ultimately, the threshold in our protocol can be set to be 4.17\%.

\subsubsection{Security Analyses}
In this section, we analyze the security of our protocol under two major attacks: the intercept-resend attack and the Trojan Horse attack. Meanwhile, we assume the existence ofan eavesdropper Eve in the communication channel.
A. The Intercept-Resend Attack The eavesdropper Eve can interact and resend a new four qubit state to Bob so that he can acquire the information of the state. In this case, Eve plays the same role as Bob does in the communication. He can group the four-qubit state and measure it with Bell bases.
After step 2 of our protocol, the four qubits sent by Alice will transmit through the communication channel. Let’s assume that Eve intercepts these four qubits and processes them in the same way as Bob does. Then, Eve resends his decoy four qubits to Bob. Let us discuss the different cases of the communication between Eve and Bob.
(1) Eve chooses the right grouping and acquires the right measurement results. Bob simultaneously chooses the right grouping. Now, there are three parties, Alice, Bob and Eve in the communication. There is a probability of p 1 that Eve can successfully filch the information, p 1 = ( x 1)(4)x ( x 1).

Bob(2) Similarly, Eve can divide the qubits and obtain the measurement results correctly. Then, he sends the right decoy four-qubit state to Bob. After receiving them,Bob chooses a wrong grouping but yields a right measurement result. However, this group of state will be abandoned since in the comparison stage Alice and Bob can detect the wrong grouping.
Eve chooses a wrong grouping but he gets a right measurement result while Bob chooses a right grouping.
Bob(4) In the same way, Eve transmits the decoy state with a wrong grouping but Bob gets the right results. Meanwhile, Bob divides this decoy four-qubit state into wrong groups but obtain the right measurement result. In this case, the qubits will be abandoned because of the wrong grouping of Bob.
In conclusion, there is a probability that Bob gets a wrong measurement result, i.e., the qubit error rate with existence of eavesdropper in our protocol can be calculated as (8).
Suppose Alice and Bob need to compare n bits of binary random sequence to detect Eve with the probability ofp d = 0.999999999 with
92 n). We can find a minimum value of n = 11 from equation (9).
On the other hand, in BB84 protocol, Alice and Bob need to compare n = 72 bits to detect Eve with the same probability. From the aforementioned analyses, we can calculate the mutual information between Alice, Bob and Eve. The mutual information between Alice and Bob is p d = 1 - (1 - $\epsilon$ e ) = 1 - (I(A; B) = 1 - [--(1 - $\epsilon$ 0 ) log(1 - $\epsilon$ 0 ) - $\epsilon$ 0 log $\epsilon$ 0 ]
The mutual information between Alice and Eve is I(A; E) = 1 - [-(1 - $\epsilon$ e ) log(1 - $\epsilon$ e ) - $\epsilon$ e log $\epsilon$ e ] = 0.3664.
Therefore, the communication is secure since I(A; B) >I(A; E). Furthermore, I(A; B) in our protocol is greater than I (A; B) = 0.1887 bit in BB84 protocol.
The theoretical secret key rate is R = I(A; B) - I(A; E) = 0.7501 - 0.3664 = 0.3837 > 0.(11)So, the binary random bits can be distilled after Alice and Bob perform key reconciliation and privacy amplification the final key R A and R B. The transmission distance of the secret key is given by the qubit error rate $\epsilon$ [12], which is:$\epsilon$ =u10 -$\alpha$/10$\eta$, u10 -$\alpha$/10 $\eta$ + 2P e(12)where u = 4 is the averaged qubit flux leaving from Alice, 10 -$\alpha$·/10 represents the fiber attenuation through distance, $\eta$ is the detection efficiency. Here, we set the channel loss to be $\alpha$ = 0.2 and $\alpha$ = 0.5. Referring to [12], P e = 8.5 x 10 -7is the probability of an error count per clock cycle. We set the values of $\eta$ to be 0.2, 0.5 and 0.99. Figure 2 shows the relation between (km) and the qubit error rate $\epsilon$. We can see that decreases with the increase of the channel loss. Meanwhile,the higher the detection efficiency of the communication $\eta$ is,the larger the qubit error rate $\epsilon$ is. From Fig. 2, we can infer that there exists a further distance when $\epsilon$ e = 0.0417 which is the qubit error rate in our protocol.

B. Trojan Horse Attack. A QKD system may be probed by Eve by sending a bright light into the quantum channel and analyzing the back reflection in a Trojan-horse attack. Eve occupies part of the quantum channel to probe the laser sent by Alice with an auxiliary source. Note that his detection scheme relies on the feature of the auxiliary source. Eve needs to remove part of the legitimate signal, compensating the introduced loss by an improved quantum channel. Hence, Eve needs to prepare a better channel which has less attenuation than the legitimate channel. Then, he can measure the intercepted state with a quantum memory.
With the Trojan Horse attack, the measurement that maximizes Eve’s information gain is known [13],(13)I E (|$\nu$| 2 ) = 1 - H(p), where p = 12 (1 + 1 - |$\nu$, 0|0, $\nu$| 2 ) = 1+ 2 2|$\nu$| and H(·) is the binary entropy. |$\nu$| 2 is the mean photon number of Eve. Hence, (13) can be rewritten as.

From [13], if Alice’s monitoring detector sets a limit to Eve’s backscattered signal of 0.1 photon, then 0.095 < I E (|$\nu$| 2 ) < 0.135. The lower bound is I E (|$\nu$| 2 ) = 1 - exp(-|$\nu$| 2 ).

In our protocol, the maximum mean photon number of Eve can be 0.4, so, 0.329 M I E (|$\nu$| 2 ) < 0.448. We can see that I E(max) (|$\nu$| 2 ) < I(A; B). Hence, the Trojan Horse attack can be prevented in our protocol.

\subsubsection{Conclusion}
We have analyzed a four-qubit QKD protocol theoretically. Our protocol can provide the encryption without inserting decoy qubits in the qubit sequence to detect the eavesdropper. Reference [14] designed a QKD protocol based on decoy-state,which is more applicable in practical, with longer transmission distance in the state of the art. This is follow-up work since we analyse our protocol in practical implementation. With the increase of the randomness of grouping of four qubits in each unit by Bob, it will be more difficult for Eve to decode the classical information. Our results show that the security can be guaranteed and the storage of the qubits during the communication is not required. Furthermore, our protocol can be extended to distributing multiple qubits.

\subsection{\trnas}
\subsubsection*{Аннотация}

Мы предлагаем протокол квантового распределения ключей на четыре кубита через два состояния Белла, которые составляют передаваемую единицу от отправителя к получателю в каждой коммуникации.
Шифрование здесь разрабатывается путем случайной группировки четырех кубитов единицы в две новые пары, что является способом увеличения возможности обнаружения подслушивающего устройства. В конечном итоге, приемник случайным образом измеряет эту сгруппированную единицу с помощью двух измерений состояния Белла. Из сравнения информации о группировке этих четырех кубитов следует, что обе стороны в действительной коммуникации могут обнаружить нелегальную сторону в канале. В предложенном протоколе приемник измеряет кубит, когда получает его мгновенно, что является эффективным способом преодоления сверхкороткого времени хранения квантового состояния.
Индексные термины - квантовая связь, квантовое распределение ключей, измерение состояния Белла.

\subsubsection{Введение}
Квантовое распределение ключей (QKD) является перспективной технологией для защиты безопасности классической информации в квантовую эпоху [1]. Она позволяет двум сторонам обмениваться секретным ключом с безусловной безопасностью, который затем может быть использован для шифрования и дешифрования сообщений. Было проведено много работ, направленных на развитие протоколов QKD. В 1984 году Беннетт и Брассард впервые представили протокол QKD, используя два взаимно несмещенных основания степени свободы поляризации фотона [2]. Позже Экерт предложил другой протокол QKD, который был назван E91 на основе пар Эйнштейн-Подольский-Розен (ЭПР) [3]. После этого были теоретически предложены и экспериментально реализованы различные протоколы QKD, такие как протоколы QKD, разработанные с использованием одного фотона [4], нескольких состояний [5] и состояния Белла [8]. Среди этих работ фотоны широко используются для переноса информации, поскольку ими легко манипулировать и они передают информацию со скоростью света.
В качестве квантового канала состояние Белла было впервые предложено в [6] и подтверждено как максимально запутанное состояние двухквантовой квантовой системы. По сравнению с другими многоквантовыми состояниями, такими как W-состояние, GHZ-состояние и кластерное состояние, состояние Белла легче подготовить с помощью нелинейного процесса [7]. В работе [8] две стороны делятся секретным ключом, сравнивая форму начального состояния Белла и результат обмена запутанностью.
Затем в [9] общая эффективность коммуникации была повышена до 100\% по сравнению с прежними 50\% в [8]. В [10]представлен первый аутентифицированный полуквантовый протокол распределения ключей без использования аутентифицированных классических каналов на основе состояний Белла. В этом письме мы предлагаем протокол для защиты от подслушивающего устройства с более низкой частотой ошибок на кубитах и более коротким обнаружением ключевых битов на основе четырехквантового состояния, которое состоит из двух пар состояний Белла.
В нашем протоколе каждый раз подготавливается группа из четырех кубитов и отправляется от отправителя к получателю. Приемник выполняет измерение квантового состояния сразу после получения кубитов. По сравнению с двусторонними протоколами, в которых квантовое состояние должно быть сохранено до завершения передачи в [8] и [9], наш протокол может преодолеть сверхкороткое время когерентности квантовых состояний.
Каждые четыре кубита образуют единое целое для передачи информации от отправителя к получателю в каждом сообщении. Четыре кубита отправляются в случайном порядке отправителем и принимаются получателем в случайном порядке группировки. Наши расчеты показывают, что коэффициент ошибок в кубитах составляет 4,17\%, что ниже, чем 46,875\% в [11]. Более того, для обнаружения подслушивающего устройства в нашем QKD протоколе требуется всего 11 бит, что меньше, чем 72 бита в протоколе BB84 [2] при той же безопасности.
\subsubsection{Протокол QKD, основанный на четырехкубитном состоянии}

Квантовый канал с состояниями Белла [8] и [9] предложили два протокола QKD, оба использовали состояния Белла, распределенные от отправителя к получателю. В их протоколах две пары состояний Белла разделяются между двумя законными сторонами коммуникации. Отправитель и получатель хранят два кубита, запутанные друг с другом. После одновременного измерения состояния Белла (BSM) двух сторон, существует обмен запутанностью между этими четырьмя кубитами.
Более конкретно, обозначим четыре кубита двух состояний Белла как P 1 , P 2 , P 3 и P 4 . Между P 1 и P 2 , P 3 и P 4 существует запутанность. После BSM двух сторон, P 1 и P 3 , P 2 и P 4 становятся запутанными, соответственно. Однако обеим сторонам все еще необходимо хранить свои кубиты во время коммуникации, а хранение кубитов на современном уровне техники является сложной задачей.
Учитывая сверхкороткое время хранения кубитов, предлагается новый протокол QKD с четырехкбитными состояниями, состоящими из двух пар состояний Белла, выраженных уравнением (1). Каждый раз готовится группа из четырех кубитов, чтобы немедленно отправить Бобу для измерения. Отличие от [8] и [9]заключается в том, что отправитель посылает все кубиты получателю, а получатель выполняет измерение квантового состояния сразу после получения кубитов. Это односторонний процесс. Обеим сторонам не нужно хранить кубиты, таким образом, кубиты измеряются до того, как они декогерентны.

Из (1) видно, что состояние становится суперпозицией четырех состояний, что означает, что мы можем получить четыре различных исхода путем комбинаций BSMs. Заметим, что уравнение (1) представляет только один частный случай, две другие формы показаны для сравнения в таблице II с различными группировками кубитов.
Вкратце, существует три формы случайной группировки этих четырех кубитов, из которых только уравнение (1) определяется как правильное. Это основная техника для шифрования информации во время коммуникации. Предлагаемый протокол показан на рис. 1, где Алиса и Боб являются законными отправителем и получателем, соответственно.
B. Предварительные сведения В этом разделе мы покажем детали протокола на рис. 1.
a) Шаг 1 подготовки состояния: Алиса готовит одну группу из четырех кубитов в уравнении (1). Каждая группа состоит из четырех кубитов P y , y $\in$ {1, 2, 3, 4}. При таком плотном кодировании ключевая информация G каждой группы двух состояний Белла показана в таблице I. Алисе необходимо записать информацию о кубитах и соответствующую информацию G.
b) Распределение кубитов на этапе 2: Согласно уравнению (1), Алиса знает, что групповой порядок этих четырех кубитов таков{(P 1 , P 3 ), (P 2 , P 4 )}. Затем Алиса переставляет эти четыре кубита случайным образом и отправляет их Бобу по квантовому каналу.
c) Шаг 3 измерения группировки: Боб получает кубиты и делит их на две части случайным образом. Затем он выполняет BSM на этих двух частях и отправляет информацию о группировке и результаты измерений Алисе по классическому каналу.
d) Сравнение результатов на этапе 4: Алиса сравнивает результаты, полученные от Боба, со своей зарезервированной информацией о P y . Если Алиса получает совпадающее сравнение, она объявляет '1', после чего коммуникация может перейти к шагу 5 или вернуться к шагу 1.
В противном случае она объявляет '0', и коммуникация возвращается к шагу 1 или завершается. Различные группировки четырех кубитов Бобом показаны в таблице II.
e) Шаг 5 - получение необработанного ключа: Алиса и Боб получают длинную двоичную последовательность в качестве необработанного ключа R после многократного обмена данными. Если мы определим R A и R B как необработанные ключи, принадлежащие Алисе и Бобу, соответственно. Алиса случайным образом выбирает части R A в различных позициях в качестве своего ключа согласия C A и объявляет соответствующие позиции. Затем Боб выбирает согласованный ключ C B в том же месте в R B .
f) Шаг 6 усиление конфиденциальности: Боб выбирает набор битов C B в качестве битов четности D B и объявляет D B вместе с соответствующими позициями. Алиса выбирает свой D A таким же образом и сравнивает его с D B . Если коэффициент битовых ошибок меньше порогового значения, связь безопасна и они могут перейти к шагу 7; если нет, им нужно вернуться к шагу 1 или прервать эту связь.
g) Шаг 7. Получение окончательного ключа: Окончательные ключи R A и R B используются для шифрования секретного сообщения посредством связи. Теоретически, R A= R B, R A - это необработанный ключ R A без C A при идеальных условиях, такой же, как и R BC. Анализ коэффициента ошибок Кубита при различных группировках Из таблицы II видно, что существует три различных вида группировок. Одна из них правильная, две другие - неправильные. Таким образом, вероятность того, что Боб правильно сгруппирует четыре кубита, равна 13.
Если в канале связи нет подслушивающего устройства, то коэффициент ошибок на кубитах $\epsilon$ 0, который является пороговым для одной связи, может быть рассчитан следующим образом.
(1) Боб делит четыре кубита на неправильные группы{(P 1 , P 2 ), (P 3 , P 4 )}.
(2)|C 1234 =В этом случае он получит только |$\Phi$ - 12 |$\Phi$ + 34 . Боб получит правильную двоичную случайную последовательность G = 01 с вероятностью 1.

С помощью BSM он может получить четыре правильных результата измерений {|$\Phi$ + 1, |$\Phi$ - 1}, {|$\Phi$ - 1, |$\Phi$ + 1}, {|$\Psi$ + 1, |$\Psi$ - 1} и {|$\Psi$ - 1, |$\Psi$ + 1}. В результате Боб получит правильную двоичную случайную последовательность G с вероятностью 44 = 1, даже если он сделает неправильную группировку.
Таким образом, результаты BSM могут быть только частично верными в первом случае или полностью верными во втором.
Поэтому Боб может получить правильную двоичную случайную последовательность с вероятностью 12 x 14 + 12 x 44 = 58 при неправильной группировке. Следовательно, коэффициент ошибок кубита $\epsilon$ 0 составляет, 11 5 = 0,0417. (3)$\epsilon$ 0 = 1 - ( + ) = 3 824
Обратите внимание, что если Боб выбирает правильную группировку с вероятностью 1, он может получить правильную двоичную случайную последовательность с вероятностью 1. В конечном итоге, порог в нашем протоколе может быть установлен равным 4,17\%.

\subsubsection{Анализ безопасности}
В этом разделе мы анализируем безопасность нашего протокола при двух основных атаках: атаке перехвата-отправки и атаке "Троянский конь". При этом мы предполагаем существование подслушивающего устройства Ева в канале связи.
A. Атака перехвата-передачи Подслушивающая Ева может взаимодействовать и повторно отправить Бобу новое состояние из четырех кубитов, чтобы он мог получить информацию об этом состоянии. В этом случае Ева играет ту же роль, что и Боб в коммуникации. Он может сгруппировать состояние четырех кубитов и измерить его с помощью базиса Белла.
После шага 2 нашего протокола четыре кубита, отправленные Алисой, пройдут по каналу связи. Предположим, что Ева перехватывает эти четыре кубита и обрабатывает их так же, как и Боб. Затем Ева повторно посылает Бобу свои ложные четыре кубита. Давайте обсудим различные случаи коммуникации между Евой и Бобом.
(1) Ева выбирает правильную группировку и получает правильные результаты измерений. Боб одновременно выбирает правильную группировку. Теперь в коммуникации участвуют три стороны, Алиса, Боб и Ева. Существует вероятность p 1 того, что Ева сможет успешно перехватить информацию, p 1 = ( x 1)(4)x ( x 1).

Боб(2) Аналогично, Ева может разделить кубиты и получить правильные результаты измерений. Затем он посылает Бобу правильное состояние четырех кубитов. Получив их, Боб выбирает неправильную группировку, но получает правильный результат измерения. Однако эта группа состояний будет отброшена, так как на этапе сравнения Алиса и Боб могут обнаружить неправильную группировку.
Ева выбирает неправильную группировку, но получает правильный результат измерения, в то время как Боб выбирает правильную группировку.
Боб(4) Таким же образом, Ева передает ложное состояние с неправильной группировкой, но Боб получает правильные результаты. Между тем, Боб делит это ложное состояние из четырех кубитов на неправильные группы, но получает правильный результат измерения. В этом случае кубиты будут отброшены из-за неправильной группировки Боба.
В заключение, существует вероятность того, что Боб получит неверный результат измерения, т.е. коэффициент ошибок кубитов при наличии подслушивающего устройства в нашем протоколе может быть рассчитан как (8).
Предположим, что Алисе и Бобу нужно сравнить n битов двоичной случайной последовательности, чтобы обнаружить Еву с вероятностью p d = 0,999999999 с 92 n). Из уравнения (9) можно найти минимальное значение n = 11.
С другой стороны, в протоколе BB84 Алисе и Бобу нужно сравнить n = 72 бита, чтобы обнаружить Еву с одинаковой вероятностью. Исходя из вышеупомянутого анализа, мы можем рассчитать взаимную информацию между Алисой, Бобом и Евой. Взаимная информация между Алисой и Бобом равна p d = 1 - (1 - $\epsilon$ e ) = 1 - (I(A; B) = 1 - [--(1 - $\epsilon$ 0 ) log(1 - $\epsilon$ 0 ) - $\epsilon$ 0 log $\epsilon$ 0 ].
Взаимная информация между Алисой и Евой равна I(A; E) = 1 - [-(1 - $\epsilon$ e ) log(1 - $\epsilon$ e ) - $\epsilon$ e log $\epsilon$ e ] = 0,3664.
Следовательно, коммуникация безопасна, поскольку I(A; B) >I(A; E). Более того, I(A; B) в нашем протоколе больше, чем I (A; B) = 0,1887 бит в протоколе BB84.
Теоретическая скорость передачи секретного ключа равна R = I(A; B) - I(A; E) = 0.7501 - 0.3664 = 0.3837 > 0.(11)Таким образом, двоичные случайные биты могут быть дистиллированы после того, как Алиса и Боб выполнят согласование ключей и усиление секретности окончательного ключа R A и R B. Расстояние передачи секретного ключа задается коэффициентом ошибок $\epsilon$ [12], который имеет вид:$\epsilon$ =u10 -$\alpha$/10$\eta$, u10 -$\alpha$/10$\eta$ + 2P e(12)где u = 4 - усредненный поток кубитов, уходящих от Алисы, 10 -$\alpha$-/10 - затухание волокна через расстояние, $\eta$ - эффективность обнаружения. Здесь мы задаем потери в канале $\alpha$ = 0.2 и $\alpha$ = 0.5. Согласно [12], P e = 8.5 x 10 -7 - это вероятность ошибки за тактовый цикл. Мы установили значения $\eta$ равными 0.2, 0.5 и 0.99. На рисунке 2 показана зависимость между (km) и коэффициентом ошибок на кубитах $\epsilon$. Видно, что с увеличением потери канала она уменьшается. При этом, чем выше эффективность обнаружения связи $\eta$, тем больше коэффициент ошибок на кубитах $\epsilon$. Из рис. 2 мы можем сделать вывод, что существует дальнейшее расстояние, когда $\epsilon$ e = 0.0417, что является коэффициентом ошибок кубитов в нашем протоколе.

B. Атака "Троянский конь". Система QKD может быть прозондирована Евой путем посылки яркого света в квантовый канал и анализа обратного отражения в атаке "троянский конь". Ева занимает часть квантового канала, чтобы прозондировать лазер, посланный Алисой, с помощью вспомогательного источника. Обратите внимание, что его схема обнаружения полагается на особенность вспомогательного источника. Еве нужно удалить часть легитимного сигнала, компенсируя вносимые потери улучшенным квантовым каналом. Следовательно, Еве нужно подготовить лучший канал, который имеет меньшее затухание, чем легитимный канал. Затем он может измерить перехваченное состояние с помощью квантовой памяти.
При атаке "Троянский конь" известно измерение, которое максимизирует информационный выигрыш Евы [13], (13)I E (|$\nu$| 2 ) = 1 - H(p), где p = 12 (1 + 1 - |$\nu$, 0|0, $\nu$| 2 ) = 1 + 2 2|$\nu$| и H(-) - бинарная энтропия. |$\nu$| 2 - среднее число фотонов Евы. Следовательно, (13) может быть переписано как.

Из [13] следует, что если следящий детектор Алисы устанавливает предел для обратно рассеянного сигнала Евы в 0,1 фотона, то 0,095 < I E (|$\nu$| 2 ) < 0,135. Нижняя граница равна I E (|$\nu$| 2 ) = 1 - exp(-|$\nu$| 2 ).

В нашем протоколе максимальное среднее число фотонов Евы может быть 0,4, поэтому 0,329 M I E (|$\nu$| 2 ) < 0,448. Мы видим, что I E(max) (|$\nu$| 2 ) < I(A; B). Следовательно, атака "Троянский конь" может быть предотвращена в нашем протоколе.

\subsubsection{Вывод}
Мы теоретически проанализировали четырехкбитную QKD-программу. Наш протокол может обеспечить шифрование без вставки ложных кубитов в последовательность кубитов для обнаружения подслушивающего устройства. В работе [14] разработан протокол QKD на основе ложного состояния, который более применим в практических условиях при большом расстоянии передачи. Данная работа является продолжением, поскольку мы анализируем наш протокол в практической реализации. С увеличением случайности группировки четырех кубитов в каждом блоке Бобом, Еве будет сложнее декодировать классическую информацию. Наши результаты показывают, что безопасность может быть гарантирована, а хранение кубитов во время коммуникации не требуется. Более того, наш протокол может быть расширен для распределения нескольких кубитов.


\subsection{\review}
ss
\subsection{\dic}
ss


\section{Quantum-key-distribution (qkd) networks enabled by software-defined networks (sdn}
ss

\section{Quantum key distribution: a networking perspective}
\subsection{Article}

\subsubsection*{Abstract}
The convergence of quantum cryptography with applications used in everyday life is a topic drawing attention from the industrial and academic worlds. The development of quantum electronics has led to the practical achievement of quantum devices that are already available on the market and waiting for their first application on a broader scale. A major aspect of quantum cryptography is the methodology of Quantum Key Distribution (QKD), which is used to generate and distribute symmetric cryptographic keys between two geographically separate users using the principles of quantum physics. In previous years, several successful QKD networks have been created to test the implementation and interoperability of different practical solutions. This article surveys previously applied methods, showing techniques for deploying QKD networks and current challenges of QKD networking. Unlike studies focusing on optical channels and optical equipment, this survey focuses on the network aspect by considering network organization, routing and signaling protocols, simulation techniques, and a software-defined QKD networking approach.


\subsubsection{Introduction}
Establishing secure cryptographic keys through untrusted networks is one of the most fundamental cryptographic tasks [1]. While the use of public key infrastructure based on computationally complex mathematical problems and assumptions about the computational power of eavesdroppers prevail, these belong to the group of theoretically breakable computational security solutions.
They are therefore under threat as computational power continues to increase and as quantum computing algorithms emerge that can break some widely used computationally complex mathematical problems in polynomial time [2, 3]. Quantum Key Distribution, known as QKD [4], is based on the principles of quantum information theory and allows to establish information-secure cryptographic keys that do not depend on these constraints, at least on a protocol level. A suitable message authentication scheme, such as Wegman-Carter [5], should be combined with QKD to this end [6, 7].
QKD networks differ significantly from traditional telecommunication networks due to the specificity of QKD links and network organization. Restrictions such as limited key generation rate and reachable distance (Section 2), present lack of quantum repeaters (Section 3.2), specific routing due to the use of public and quantum channels in quantum links (Section 6), and network organization that for now has to employ a hop-by-hop key transport approach (Section 5.2.2) are the motivations for this survey. Although several studies on QKD link and QKD quantum channel scan be found [8–10], this survey focuses on QKD networking, network organization, routing and signaling protocols, and software-defined QKD networking techniques. After reading this survey,interested readers will have an insight into quantum networks from an engineering perspective and be familiar with the modes of functioning, realization, existing solutions, and methods of simulating quantum cryptographic networks. This survey provides a high-level view of QKD networks and is of use and interest to researchers, practitioners of QKD network design, and PhD students in the field of applied quantum cryptography.
The survey is organized as follows: Section 2 introduces the features of QKD links. Section 3 summarizes the limitations and basic characteristics of QKD networks and explains how they are practically implemented. QKD network types are described in Section 4. Section 5 covers previously deployed QKD networks. QKD network routing techniques are discussed in Section 6. Section 7 provides an overview of QKD software-defined networking. Section 8 concludes this survey.

The survey includes the supplementary material with additional QKD networks listed in Section 1 and simulation techniques discussed in Section 2. Overview of signaling network protocols is given in Section 3, while QKD header and QKD packet encapsulation are discussed in supplementary material, Section 4. Section 5 provides an overview of work in QKD standardization process.
The graphical outline of the survey structure is shown in Figure 1.
\subsubsection{QKD LINKS}
A QKD link, or simply, "link," denotes a logical connection between two remote QKD nodes connected by a quantum channel used for transmitting photons and a public channel used for postprocessing the exchanged information, respectively. The disadvantage of this type of link is reflected in a limited quantum channel key generation rate, available to the parties connected by a direct optical fiber or free line-of-sight in a point-to-point (P2P) manner over a certain distance.
However, it is also a necessary condition for secure key generation.
Although fiber is a good and commonly used medium for transmitting qubits, the installation of a dedicated optical channel for QKD purposes is not practical in all circumstances. 1 A free space link is sometimes convenient, although it has its drawbacks, since it needs suitable atmospheric conditions, a visible light path, and an acceptable signal-to-noise ratio (SNR) that strictly limits usage time. Nevertheless, the results obtained from experiments in Los Alamos [17] and Munich in which a link between the ground and an aircraft flying at 290 km/h was established [18] demonstrated promise with satellite connections [17–23]. After performing a sequence of free space QKD experiments on the ground, China successfully launched the quantum satellite “Micius,” which demonstrated a satellite-to-ground QKD over a distance of 645 to 1200 kilometers [24].
The maximum distance, over which key can be generated, decreases with increasing losses and optical detector noise. For a given detector and settings, the detector’s dark-count 2 rate is constant, but key rate decreases with distance due to increase of cumulative losses. In current commercial optical fiber systems, the distance of a QKD link is roughly limited to 100 km, while the key rate is limited to a few tens or hundreds of [26, 27]. Due to the limited key rate, key storage is installed at both endpoints of the corresponding link. This storage is gradually filled with new key material, and the available key material is subsequently used to encrypt/decrypt data flows [28].
The amount of data to be encrypted and the encryption algorithm type determine the rate of key storage discharge, or, simply, the key consumption rate. The key rate of the link is otherwise referred to as the key charging rate [28–31]. The QKD link can be designated “currently unavailable” when no available key material in key storage is found, as no cryptographic operations can be performed [32]. It is also worth noting that an apparently optimal strategy for QKD devices is to continuously generate keys with maximum intensity until the storage is full (which depends on how it is implemented) [28, 33].
A key can be used to encrypt communication over a public channel using a One Time Pad(OTP) cipher and ITS authentication scheme such as Wegman-Carter [34, 35]. Since an OTP cipher requires the same amount of key that corresponds to the length of the message being encrypted and additional keys for ITS authentication, this approach consumes more key material than the message being transmitted. If not enough key material is available, OTP cannot be used, and the use of alternative cryptographic techniques such as Advanced Encryption Standard (AES), which does not require such a large amount of key consumption, is the most common choice [36].

\subsubsection{QKD NETWORKS}
QKD networks are used to extend the range of QKD systems and consist of static nodes that represent secure access points considered to have unlimited processing power and power supply. Because of the point-to-point behavior of the links connecting nodes, previously deployed testbeds [29, 37, 38] have shown that secure keys in QKD networks can be transmitted from node to node in a hop-by-hop manner (Section 5.2.2) or through a key repeater concept (Section 5.1.5).
Common to both networks is the assumption that all nodes in a network should be trusted[32, 39]. This assumption can be avoided if multipath communication Quantum Network Coding techniques are used [40]. In this survey, previously deployed QKD networks are briefly discussed,focusing on methods of communication, routing protocols, and network organization.
To facilitate organization, a QKD network has often been described using several layers[41, 42]:
• A quantum layer where a secure symmetrical key is established.
• A key management layer used to verify and manage the previously established key.
• A communication layer where the established key is used to secure data traffic.
As mentioned above, QKD is a key agreement primitive and as such is located in the lowest(basic) layer of the QKD network architecture. Taking into account different rates of key material consumption by different applications, a situation in which not enough key material is available to meet the needs of higher layers is not desirable. The quantum layer therefore needs to continuously establish key material. To provide a guaranteed level of service, the QKD network should have a detailed view in its resources and capacities. Previously deployed QKD networks did not have defined strategies for a quality assurance service. For example, the SECOQC QKD network,discussed in Section 5.2, was committed to the basic Best Effort service type, which only defines the average key rate and traffic burst, while the Guaranteed Key Rate service type had been suggested for improved versions of QKD networks [33].
Considering the comprehensive and detailed documentation available on quantum optical communications [26, 43–47], the emphasis of this publication is on the two upper layers. These layers can have different and independent network organization, as communication between nodes is achieved through existing standard connections, such as the Internet, where an arbitrary number of intermediate devices can be included (Figure 2). The key management layer is in charge of managing the key storage resources, routing protocols, quality of service (QoS), and so on. The topmost communication layer uses previously established key material to encrypt data traffic by using an existing security protocol suite, such as Internet Protocol Security (IPSec) [14, 48].However, the described hierarchy distributes the responsibility for security across all three layers.
\subsubsection*{QKD Network Attributes}
QKD represents a new generation of security solutions that do not rely on the computational assumptions of problems presumed difficult. However, QKD networks must be integrated into the existing environment and need to meet certain criteria and conditions. Some of the most common requirements from QKD networks are listed below.
3.1.1 Key Rate. One of the vital parameters describing a QKD network is the average key rate of a QKD link. Since encryption and decryption operations cannot be performed without sufficient key material, the competition between the rate at which key material is stored in the key storage and the rate at which it is consumed for encryption and decryption operations has a major influence on network performance.
Comparing previously deployed QKD networks and testbeds chronologically, a rapid improvement in the development of quantum equipment is evident. QKD systems implemented in 2002 in the DARPA QKD network could achieve a key rate of approx. 400 bps over 10 km [29]. In 2007,in SECOQC, the maximum key rate was 3.1 kbps over 33 km [37]. The best performed solutions presented in Tokyo in 2009 achieved a key rate of 304 kbps over 45 km [38]. In 2017, China built the 2,000-km Beijing-Shanghai backbone QKD network with devices typically achieving key rates of 250 kbps over 43 km.
In the past 20 years, a steadily increasing secret key rate has been obtained with improved optical components and better electronics mainly in the detectors. For the latest jump to achieve record-high rates of around 10 Mbps [49], digital signal processing in FPGA was optimized. The throughput of measured qubits to enhance key rates has also been enhanced, especially for shorter links, by removing limitations without FPGA. A second race is open to achieving longer single span transmission distances [50–53] based on protocol enhancements as well as technological improvements leading to detectors with ever-decreasing dark count rates. 3 It could be argued that the development to improve single links on rates at short distances and maximum span will make QKD networks needless. The opposite is true, however, as the opportunity to open a mass market with these improvements at the link level seems low and unlikely to cover broad deployment scenarios specifically using the technical improvements from recent years. What the latest improvements genuinely enable is increased diversity in the links that can be potentially deployed in QKD networks.
It is therefore reasonable to expect that in the future, an optimal solution will significantly exceed the present key rate and distance values, although the race between generation and consumption of key material will remain.
3.1.2 Link Length. The fundamental constraint of a QKD link is the length over which secure key material can be generated (due to scattering and absorption of polarized photons and other factors [27, 44, 45, 54]), which limits the ability of quantum channels (direct optical links or free line-of-sight) to a certain distance. It is interesting to compare the lengths of links in previously built QKD networks. 4 The maximum length in the DARPA QKD network was a 29 km connection through the optical switch between Harvard and Boston Universities [29]. In SECOQC, the maximum length of the link was 82 km between the BREIT and St. Pölten nodes [37], while in Tokyo, the maximum connection between the nodes was a record 90 km between the Koganei-1and Koganei-2 nodes [58]. In the Beijing-Shanghai Backbone QKD network, the maximum link length is 89.3 km between Hefei and Wuwei.
In current systems with optical fibers, the distance over which QKD links can be effectively applied is limited to roughly 100 km [26, 27].
3.1.3 Protection of Key Material. The main reason for interest in QKD is the privacy of the established key material. This means that the nodes of a QKD network must be secured with a strong probability that the established key material is unique and inaccessible to third parties. The security of key material is evaluated not only when it is established but also when it is managed,stored, and eventually used. It is therefore important to secure each level of the QKD network architecture.
3.1.4 Key Usage. Because of scarce resources (generation key rate), communication in a network is reduced to a minimum, since each additional packet means spending an additional amount of previously established key material. Since communication is usually performed on a hop-by-hop basis that requires the trustworthiness of all nodes in the path, selecting the shortest routing path is necessary to minimize the number of nodes that can potentially be abducted or attacked by an eavesdropper. Also, involving longer paths requires a higher consumption of key material. During network congestion or problems in communication, used key material is deliberately discarded and new key material for retransmission is applied to reduce the risk of leaks [28]. Therefore,minimizing the number of hops is preferable.
3.1.5 Robustness. Because of the cost and manner of implementation, QKD networks will slowly integrate into traditional and everyday telecommunication environments. It is important then to ensure robustness, which is reflected in the gradual and seamless addition of new nodes and establishment of new links. A QKD network needs to provide adequate replacement paths to avoid defective nodes or nodes under severe attack. Regardless of the security techniques, remembering that attackers can easily find ways of terminating optical links and breaking QKD connections is important. A QKD network must have an adequate response to such situations.

Because traffic can be connected and directed between different network domains, network repeaters have a fundamental role in modern networking. Although theoretical and pioneering results in the field of quantum repeaters are available [59–62], in practice they remain unachievable with current technology [10, 27]. The idea behind a quantum repeater is to employ quantum entanglement of photons to communicate over different quantum links. Quantum entanglement is a key aspect in applying quantum communications and quantum information. In short, quantum entanglement implies that multiple particles are connected together in such a manner that the measurement of one particle’s quantum state determines the possible quantum states of the other particles. Even when particles are separated by large distances, they still make up a joint quantum system. Entanglement fidelity is a property used to describe how well the entanglement between two subsystems is preserved in a quantum process.
In theory, however, the application of entangled states and entanglement swapping is hindered by two main roadblocks. The first is that the greater the distance between two entangled systems,the lower the fidelity. In fact, the achievable fidelity of a quantum state decreases exponentially with the distance because of lossy quantum channels [27, 63]. 5 In this context, the concept of entanglement purification [64, 65] can be used to increase the fidelity of a single entangled state by using a number of noisy entangled states (as described in Reference [60]). However, this increases the number of required resources for transmitting each qubit over a quantum repeater (i.e., the number of entangled states). The second roadblock in achieving a quantum repeater following the scheme, for example, in References [60, 61], is that quantum memory is required a technology that is not practically available as of today. The use of quantum repeaters is essentially based on the idea of creating “chains” of entangled photons using a technique called entanglement swapping. Concepts either with quantum memories [66] or without [67] have been developed. Different building blocks for matching the transfer of the wavelengths of these flying qubits to quantum memory have been practically demonstrated [68, 69]. Internal loss and fidelity need to be improved to implement chains with one or more intermediate nodes working at higher rates. The first work to integrate future quantum repeaters in the overall infrastructure was recently published [70]. Each node in a QDK network therefore acts as a repeater and forwards packets or enanglement states of other nodes to enable quantum information sharing between QKD hosts.

\subsubsection{QKD NETWORK TYPES}
Although many hybrid realizations have been proposed, QKD networks can be grouped into two distinct categories: switched QKD networks and trusted repeater QKD networks.

Switched QKD networks consist of nodes connected to a dedicated, fully optical network. This network contains a switching mechanism used to establish a direct optical point-to-point QKD connection between any two nodes in the QKD network. The limitations on distance in point to-point QKD links restrict these networks to a metropolitan or regional scale [10]. Since every optical switch adds at least several dB of loss to the photonic path, optical switches can significantly reduce a network’s range.
The main drawback of switched QKD networks is the requirement of dedicated optical infrastructure for quantum channels, which is often not economically feasible. By contrast, the major advantage of this class of networks is the reliance on an optical switch that allows establishing a connection between two nodes without the active participation of other network nodes (Figure 3(a)).

Another drawback of switched QKD networks is the consistency of the applied QKD technique.
Combining different QKD techniques such as free-space QKD and QKD over fiber is not possible,since no suitable devices that could perform this transformation in the path are available. The first switched all-pass QKD network was described in Reference [71]. Four nodes were connected through an optical switch, and each of the QKD terminals was designed as a transceiver so they could establish a QKD link to one of the other three simultaneously.

In trusted repeater QKD networks, the security of each node along the transmission path is essential for securely transmitting information (hence the name). Point-to-point communication between two nodes provides identical keys to the nodes and thus enables secure communication (Sections 5.1.5 and 5.2.2). Taking into account the lack of a quantum repeater, nodes are also responsible for routing and forwarding mechanisms (Figure 3(b)). Organizing a network in this manner is its greatest drawback, because the security of transfer depends on the security of all the nodes in the path. However, trusted repeater networks are not limited by distance or node numbers and can be made up of different QKD devices implementing different QKD technologies.

Since the quantum channels can be “given” to the eavesdropper without compromising the security of QKD, a rational adversary would rather target the weaker link, being the node. The usual assumption is letting the nodes be “invulnerable,” which is the trusted repeater hypothesis. However,given that the optical device controls at some point will most likely have a conventional computer control logic, the security of the device is no better than the security of a classic computer running QKD algorithms and its physical protection.
The admittedly strong assumption of fully trusted repeaters can be relaxed in at least three ways:(i) use measurement device independent (MDI) QKD, (ii) use quantum repeaters, and (iii) rely on multiple paths.

This first approach has been described by Reference [72] and adds the assumption of perfect state preparation achievable by communication parties, as well as adding a potentially unstrusted location to the quantum channel. Measurements using Bell states and formal arguments for “unconditional security” have been supported with experimental demonstrations [73, 74]. Of course,the absence of the trusted repeater assumption in these proofs makes security much stronger than those assuming trusted repeater QKD. Note, however, that MDI QKD essentially prolongs the quantum channel but the two sender stations must still be situated in Trusted Repeater Nodes. This also holds true for the other alternatives outlined next (except possibly in the case when end-to-end quantum links could be established without intermediate Trusted Repeater Nodes).

The concept of quantum repeaters was discussed above (see Section 3.2). While practical demonstrations have been presented [61, 66, 67], the spatial distances the technology can overcome (as of today) strongly depend on the amount of fidelity induced by the entanglement swapping and the degree to which it can be handled (Section 3.2).
The third and most practical method today resorts to classic technology and employs multiple paths and threshold cryptographic techniques to mitigate the risk of eavesdropping. Roughly speaking, multipath transmission quantum networks trade trust in the repeaters for the assumption of the repeater being vulnerable to eavesdropping, the attacker being forced to intercept many of the intermediate devices to discover the message. Indeed, it can be shown that in absence of trusted repeaters, multiple paths are a theoretical necessity. At the same time, path redundancy also mitigates the issue of all QKD implementations being vulnerable to denial-of-service attacks(the adversary may passively eavesdrop not to get information, but to make the local quantum key stores run dry to enforce the endpoints to switch to conventional transmission techniques [75]).
Advanced routing mechanisms can be put to use to bypass lines with detected eavesdroppers.
Indeed, otherwise, attackers could try to break the security, employing passive eavesdropping to redirect traffic over vulnerable repeaters and thus get a hold of the secret key [76]. It can be shown[77, 78] that “end-to-end security” without trusted repeaters in quantum networks (without quantum repeaters) can be restored only under weak assumptions of the attack resilience of nodes [79].
Furthermore, using the same techniques, simultaneous multi-level security against other attacks can be achieved along the same lines to an arbitrarily selected level of service quality [80]. The topology of a quantum network generally has a strong impact on achievable security, and despite theoretical and practical progress in the construction of quantum networks, even without trusted repeaters [33, 36, 54], the problem remains computationally (in fact, NP-) hard in its most general form [81]. Methods for foiling covert channels and malicious classical post-processing units have been discussed in Reference [82].
\subsubsection*{QKD Overlay Networks}
While the previously described QKD network types relate to the organization of quantum channels, the QKD overlay network type refers to public channel realization. The primary goal of the overlay network is achieving the higher hierarchy network with the aim of providing a better QoS and utilizing the resources of lower-level networks. In doing so, the overlay network aims to be independent of the defined paths from Internet Service Providers (ISP). Finding alternative routes that can provide a service with a higher degree of quality and quick rerouting in the case of interrupt detection or using multipath communications are key features of the overlay network approach. The use of multipath connections is an often suggested solution for improving network work loads through protecting against network failures, network load balancing, large bandwidth implementation, low-delay time selection, and more [83–86]. Studies have shown that at least four link-disjoint paths between large ISPs are present in 90\% of point-of-presence pairs [87, 88].

It is known that routing between network domains using external routing protocols such as Border Gateway Protocol (BGP) results in slow response and recovery from network outages. Due to the time required to obtain information about interruptions or congestion on network links and the BGP minimum route advertisement interval timer settings, which is usually within minutes,the time needed to obtain a consistent view of the network after a link outage can reach tens of minutes, which is a long period for network applications. BGP also propagates only one route, and detecting the alternative route network nodes need in different situations is difficult [89].
The overlay network can help overcome these challenges by establishing the network with a peer-to-peer approach. The overlay network connects nodes in different domains and allows the use of alternative paths by encapsulating traffic to the traffic in the lower network. When an intermediate node in the path received the packet, the node will unpack the packet, analyze the IP address of the recipient, re-encapsulate packet again, and forward it further to network nodes that may be in other domains. Simply, it is a hop-by-hop approach popularly applied in QKD networking (Figure 4). Considering the encapsulation principle, overlay nodes independently perform link state measurements and can respond more quickly to link congestion by redirecting traffic to other less-congested links. Overlay networks can offer new functionality that is difficult to perform in lower-layer networks. The overlay QKD approach is attractive, since it can be used to bypass “untrusted” nodes and perform quick rerouting when trust in nodes is no longer valid or multipath communication is required [28, 33].
\subsubsection{PREVIOUSLY DEPLOYED QKD NETWORKS}
This section briefly discusses some previously deployed QKD networks. Since much of the literature deals with quantum optical infrastructure, the focus is placed on the logical structure of networks and topology, key storage and management solutions, key usage, and the solution’s performance.
\subsubsection{DARPA QKD Network}
The world’s first QKD network was the DARPA QKD Network, presented in December 2002 by BBN Technologies and Harvard and Boston Universities [14]. Initially, the network consisted of a weak-coherent BB84 transmitter pair (Anna and Alice), a pair of compatible receivers (Boris and Bob) and one 2x2 optical switch that could connect any sender to any receiver (Figure 5). Later, the network was extended with two free space QKD links, and the third planned free space link from QinetiQ (UK) was not explained in any official project documentation. The DARPA QKD network combined two previously explained types in a hybrid solution. The DARPA QKD network laid the foundation for the further development of trusted repeater QKD networks, but it also demonstrated practically the disadvantages of a switched QKD network type.
Two nodes (Alice and Bob) and a switch were located at BBN, while Anna and Boris were located at Harvard and Boston Universities (BU), respectively. BBN designed its own 2x2 optical switch and used it to connect Anna, Alice, Bob, and Boris. This switch was optically passive and therefore did not disturb the quantum state of photons. The switch was constructed by modifying a standard telecommunications facility switch. It operates by moving reflective elements that change the internal light path to produce either a BAR or CROSS connection. It is controlled through a direct line from Alice’s optical process computer (OPC) by applying a TTL-level pulse to either the BAR or CROSS pin for 20 ms to switch the activated position. According to Reference [29], switching time was 8 ms and optical loss was less than 1 dB.

A previously performed set of experiments presented results that measured degradation in the phase-modulated QDK incurred by optical switches [90]. A demonstration of QKD transmission and the results of insertion loss, which was the principal effect on QKD throughput in three different types of optical switches, yielded the following: 2x1 optical-mechanical switch (4.7 dB loss),
2x2 LiNbO3 switch (5.4 dB loss) and four-port MEMS switch (5.3 to 5.9 dB of loss).

5.1.1 BBN Protocol Suite. Considering that the DARPA QKD network was the first QKD network, no predefined protocols could be used for QKD communication over the public channel.
BBN therefore developed its own QKD protocol stack in C programming language. All messages were packed in IP datagrams to convey the control messages through the Internet [91].
Table 1 presents only the list of technologies used, while interested readers may refer to References [29, 92]. It can be seen that several techniques were used for different post-processing stages. The aim was to minimize the number of messages exchanged to speed up the key generation rate and reduce congestion caused by the sudden transfer of a large number of packets over a public channel. Figure 6(a) shows the basic format of BBN’s QKD Protocol datagram. Each datagram contains a packet header with details of the permitted, reliable, in-order transmission of the message, crash detection, and so on.
The datagram is filled with one or more messages of variable length, carrying the details needed to describe commands or a response to an action. It is important to emphasize that these datagrams are protected by an IPSec security mechanism in standard mode.
The aim was to create a secure tunnel between Quantum Protocol Daemons so all traffic over the public channel was encrypted, authenticated, and integrity-checked. An example of messages exchanged by BBN’s QKD Protocol is shown in Figure 6(b).
5.1.2 Key Management and Usage. In the final technical DARPA report [29], details for the originally planned and used technologies are given. It is interesting to note that the authors assumed the Diffie-Hellman (DF) key agreement primitive would be broken by 2015. Since the average key rate of a QKD device was 1 kbps, the goal was to introduce QKD as a new key agreement solution and integrate it with existing IPSec and IKE (Internet Key Exchange) key management protocols. Later, when the key rate of QKD devices increased, IKE could be abandoned and the use of OTP forced, which would lead to a highly secure network architecture.
The authors proposed the use of a QKD network between sensitive areas only, in which the QKD endpoints would be used to further distribute obtained information into private networks(enclaves). The QKD endpoints would have had the same function as border routers in standard IP networks. Connecting the end-user directly to the QKD network was never planned. The main idea was to create a storage “reservoir” at both ends of the corresponding QKD link that would be gradually filled with key material established through QKD. This keying material would later be used with an IPSec protocol suite and employed to encrypt virtual private network (VPN) tunnels.
When the traffic was received by the corresponding endpoint at the receiving end of the VPN tunnel, it would be forwarded to a final user located in the private network (enclave).
To simplify the process but also use the software of different platforms and manufacturers, the QKD endpoint was separated into two distinct computers. The first computer, called Optical Process Control (OPC), used the LabView software to control associated QKD equipment, while the other computer was used for IPSec communication, routing, network protocols, and QKD protocols (sifting, privacy amplification, etc.). These two computers used local 100 Mbps Ethernet connections. For synchronous data exchange, a specialized set of BBN-supplied UDP 6 protocols was used.

Another important issue was synchronization between these computers and synchronization between QKD and the VPN protocol suite. More precisely, solving the management of key material produced by optical devices was necessary. The procedure was as follows:

(1) The OPC computer delivers a fixed-size raw Qframe block of symbols transmitted through optical devices. It contains an indication of the bases that were used to encode the information in photons. These Qframe blocks are further processed by a QKD daemon with the QKD post-processing suite (sifting, QBER, privacy amplification, or similar), producing a Qblock as output.

(2) A Qblock is a fixed-size block of shared bits, each Qblock having its own 16-bit Qblock identifier (ID). Qblocks are stored in a storage reservoir of key material at both ends. These blocks are stored continuously, regardless of consumption.

(3) The IKE daemon uses Qblock IDs to establish a final key, which is then used by IPSec,since both ends have the same key material stored in their respective storage reservoirs.


5.1.3 IPSec Protocol Suite. Internet Protocol Security (IPSec) is a protocol suite for the purpose of ensuring the integrity, authenticity, and confidentiality of connections over the public internet. IPSec operates at the Internet Protocol (IP) layer and as a perimeter between protected and unprotected network interfaces by requiring a protection level. By default, IPSec uses the Internet Key Exchange (IKE) method for automatic keying. The basic concept of IKE protocol is simple and takes place in two phases. The first phase establishes an authenticated, bi-directional, secure link(the Internet Security Association) and the Key Management Protocol (ISAKMP) SA by exchanging random nonce and half-keys for the Diffie-Hellman key exchange. Authentication of a Phase 1channel is performed by exchanging messages encrypted with a session key. Random secret bits from Phase 1 that are used to establish ISAKMP are conventionally termed SKEYID. These bits are considered the most sensitive point in IKE points, since they are used as a partial input for creating Phase 2 SA keys and for protecting traffic through a given Phase 2 SA. Replacing these bits from time to time in order not to compromise the system’s security is therefore important.
The second phase uses SKEYID bits to negotiate the IPSec SAs between two gateways that carry message traffic for a certain VPN traffic flow. Each IKE security association has a maximum lifetime that limits the use of key material for the previously established association. These limitations can be defined in time (seconds) or in encrypted data (kilobytes) and are stored in SPD entry for a given SA. After the lifetime expires, a new SA must be negotiated with fresh key material. It is important to note that there is no standard for using OTP with IPSec. Various solutions have therefore been proposed, such as References [14, 48, 97].

The DARPA QKD Network employed IKE because at the time of its development (January 2002),IKE was the most widely deployed internet key agreement protocol. Two extensions exist that depend on a later-used type of cipher:

• The Quantum Perfect Forward Secrecy (QPFS) extension, which is based on the use of QKD techniques, for agreeing on secret keys employed as seeds for conventional symmetric ciphers such as AES or 3DES. Since the security of these symmetrical ciphers may be compromised in the following years, continual and automatic reseeding with fresh QKD bits is advisable. In the DARPA QKD network, AES keys were refreshed about once per minute [15] by omitting the Phase 1 negotiation and using QKD bits as a direct input to the IKE Phase 2. This solution increased the security of IPSec associations, since the keys were derived fromQKD instead of the Diffie-Hellman (DH) key exchange.

• An extension based on the use of QKD techniques for agreeing on secret key bits used with a one-time pad (OTP) cipher. This solution lowers the data rate to the QKD key rate, since the key material in the storage reservoir is charged only by QKD.
To implement the listed extensions, the DARPA QKD network team extended the IKE Phase 2 by adding an option to the QPFS extension that works in the same structural manner as a regular IKE Phase 2 PFS but uses QKD bits rather than bits obtained from the DH key exchange. The solution was implemented in Net BSD with the “raccoon” IKE daemon. The modifications included policy mechanisms to specify when and which extension should be used, with the possibility to specify values (re-key rates, cryptographic algorithms, keys, etc.) for each VPN gateway.

5.1.4 Routing in the DARPA QKD Network. A routing mechanism is required in situations when two nodes do not have a direct point-to-point QKD link between them and therefore need to agree on a path through a trusted repeater network.

Each node has a database of the full link state of the network. For each network node, it keeps the node’s ID and a list of neighboring nodes. DARPA modified well-known Open Shortest Path First(OSPF) routing protocol [98] to use the specific QKD networks metric [92]. The idea is that eachnode exchanges a certain number of bits with its neighboring node, thereby measuring the rate of exchange and the total number of bits exchanged (measuring the quality of the connection) [99].
Link quality is calculated using link metric m and stored in the database of the corresponding node: q > t, 100 + 1000q-t ,(1)m =,q < t,where q denotes the number of Qblocks expected to be available on the link in one Link State Announcement (LSA) update interval, m is the link metric, and t is the threshold (default value 5)for a minimum number of Qblocks to be maintained on an active link.
Later, when a route between distant nodes is requested, the route with the smallest total metricis selected. For the purposes of finding this path, Dijkstra’s algorithm is used. To refresh the records in link-state databases, periodic messages ROUT1LSA are exchanged [29]. These messages carry the node ID of the sending node, the node ID of the neighboring node, and the corresponding 32-bit link metric. ROUT1LSA messages are exchanged at every LSA update interval, which is a configurable parameter set to one minute by default. Each node has an individual LSA timer that does not depend on other nodes in the network.

However, it is evident that described modification of OSPFv2 protocol does not take into account the parameters of the public link. The metric m defined in Equation (1) only examines the amount of available key material, without considering other parameters such as link load or delay [99].
Routing protocols are discussed further in Section 6.

5.1.5 BBN’s Key Repeater Protocols for Trusted Networks. As noted above, a QKD network is used to overcome the limitations of the length of the QKD link. The DARPA QKD network laid the foundation for the Key Repeater Protocol and represents the first implementation of a Trusted Repeater QKD network. This implementation will be explained briefly here. More details can be found in Reference [29].
When two distant nodes in the QKD network (i.e., node A and node D) want to establish secure communication and no direct point-to-point link exists between them, they need to agree on a path through the network. This path is calculated with a routing protocol, and the nodes use a Key Repeater strategy to establish key material. The source node is always the node with the higher node ID. The source node (node A in Figure 7(a)) sends reservation requests to each node in the path (intermediate nodes) and to the destination (node D in Figure 7(a)). Each node in the path then negotiates with its predecessor for a Qblock (Figure 7(a)) and informs the source when the negotiation process has been successfully completed. If reservations are successful, the source requests a key from the destination and all the intermediate nodes. The intermediate nodes send the XOR of two Qblocks established with neighboring nodes, while the destination node sends the XOR of the previous hop Qblock and a new random Qblock n (Figure 7(b)). This Qblock n is the final key shared by source node A and destination node D.
From the above, it is obvious that BBN’s Key Repeater method of establishing key material takes time and requires the absolute trust of each node involved in communication. Authentication techniques therefore have a special significance in the entire process. As already discussed, the most effective way to circumvent compromised nodes is to use multiple independent paths.


5.1.6 Summary. The DARPA network was the first network to demonstrate QKD networking.
The performance achieved by this network (maximum distance of 29 km via the optical switch between Harvard and Boston Universities [91] and maximum key rate of 400 bps) is considered the basis for further QKD deployment. The system involved trusted repeater and switched QKD networking, demonstrating the advantages and disadvantages of both methods. A brief summary of the DARPA QKD network is given in Table 2.

However, the DARPA network was shut down in 2006, and no other field deployments by US government agencies have been reported since. In 2017, the Quantum National Initiative was announced, fueled by China’s successful launch of the “Micius” satellite [100, 101]. In 2018, the startup company Quantum Xchange announced plans for the first commercial quantum communication network “Phio” in the USA [102, 103]. Using its own exclusive trusted nodes, Quantum Xchange provides secure key transmission over long distances. This QKD network operates in Washington,D.C., and New York City, including the link connecting financial markets on Wall Street with data centers in New Jersey [104]. To achieve double network capacity, collaboration with Toshiba was announced [105].

\subsubsection*{SECOQC QKD Network}
In 2004, the European Commission’s (EC) integrated FP6 Project SECOQC (Secure Communication based on Quantum Cryptography) brought together 41 research and industrial partners from 11 European Union countries, Russia, and Switzerland. The main aim of the SECOQC project was to firmly define the practical application of QKD technologies and systematically treat the issue of QKD networks, including their security, design and architecture, communications protocols,implementation, demonstration, and test operation of QKD network protocols.
The SECOQC approach was to define QKD networks as infrastructure based on point-to-point QKD capabilities that aimed for ITS key agreement and secure communication [106]. Taking into account that the first results of the DARPA QKD network were available [92], SECOQC decided to further improve the trusted repeater QKD network type. The “Quantum Backbone” QBB network of metropolitan distance (6–85 km) consisting of seven fiber-bound key distribution links plus one short distance free-space link was deployed for testing purposes in Vienna [57]. Five nodes-SIE,BRT, GUD, FRM, and ERD-were located at the Siemens premises, and the STP node was hosted by a repeater station near St. Pölten on the communications line from Vienna to Munich, Germany.

5.2.1 QBB Links and Nodes. As shown in Figure 8, SECOQC integrated eight links belonging to six different systems:

• Attenuated Laser Pulse — a modified, commercially available “Cerberis” solution implemented by Swiss company idQuantique.

• One-way Weak Coherent Pulse system with decoy states — implemented by Toshiba UK.

• Coherent-One-Way — implemented by the N. Gisin’s team at GAP, University of Geneva.

• Entangled photons — provided by the Austrian Research Centers (ARC) and Royal Institute of Technology of Kista KTH, Sweden.

• Continuous Variables — implemented by the CNRS-Thales-ULB consortium from France/Belgium.

• Free Space link — developed by Ludwig Maximilian University of Munich, Germany.
All of these systems had to comply with the following requirements:

• A key rate greater than 1 kbps over 25 km of fibers (6 dB loss with a fiber attenuation of approximately 0.25 dB/km over standard telecommunications fiber).

• Autonomously deliver the key for more than six months without human interaction.

• A latency time of one minute for a new start-up. 

• All equipment used must fit into a standard 19" telecommunications rack.

• Each QKD-device must communicate with its peer over a standard interface provided bythe node module controlling the share management commands. 

SECOQC network included several different QKD solutions:

• The free space QKD system employed the BB84 protocol with decoy states, which resulted in a secure key rate of up to 17 kbps over 80 m between the ERD and FRM nodes.

• The idQuantique QKD system implemented the BB84 and SARG04 protocols using a commercial Cerberis system, which resulted in an almost equal value prescribed by the SECOQC criteria (1 kbps).

• Toshiba Research Europe Ltd (TREL) implemented a weak coherent pulse (WCP) decoy state plus vacuum state BB84 protocol and obtained a 5.7 kbps key rate over a fiber length of
25 km.

• A coherent One-Way (COW) System designed by GAP (Group of Applied Physics at the University of Geneva) achieved a novel distributed phase reference COW protocol, which can be seen as a BB84 modification with phase relations between pulses [28].

• The entanglement-based QKD (ENT) developed by an Austrian-Swedish consortium implemented BBM92 for entangled states between the ERD and SIE nodes over a 16-km fiber and provided a reliable key rate of over 2 kbps.

• The Continuous-Variable (CV) system was developed in cooperation between Charles Fabryde l’Institut d’Optique, THALES Research \& Technology France and University Libre de Bruxelles. Their system achieved a distribution rate of 8 kbps over a 6.2-km standard optical fiber (attenuation of the fiber was approximately 2.8 dB, while the length of an equivalent fiber with a loss of 0.2 dB/km would be 14 km).

SECOQC nodes followed the DARPA approach of storing key material in storage reservoirs.
Considering that the QKD links between nodes must be achieved in a point-to-point manner, anode needs to possess a dedicated QKD device for each connection to other nodes. Key material from QKD devices is first deployed in Pickup Stores. This temporary storage keeps the key material until it is confirmed that the same material is found in both QKD nodes forming the corresponding QKD link. After successfully confirming the existence of the same key material at both ends,the key material is then forwarded to a Common Store. There is only a single Common Store for the Q3P link (which can contain one or more QKD links between two nodes), and key material in this storage is uniquely identified by the key material block. When use of the key material is requested, keys are forwarded to In or Out key buffers and used for encryption or decryption purposes by a Crypto Engine. Organizing key storage in a described manner, QKD nodes can tolerate fluctuations in key consumption by buffering the generated key material. More details about KeyStore organization can be found in References [28, 106].

5.2.2 Hop-by-hop Message Transmission. The SECOQC network has laid bare the basics of the hop-by-hop approach to QKD network communication. This mode is known as the “Store \& Forward” technique and implies the use of a separate key for each link in the path. As shown in Figure 9, each node decrypts the message, verifies the authentication tag, and re-encrypts the message using a key that matches the connection to the next node. The procedure is repeated in each node on the path until the message reaches its destination [106].


5.2.3 Routing in the SECOQC QKD Network. SECOQC suggested using an IPv4 address structure and geographical division of the QKD network in multiple routing areas for the following reasons:

• A QKD network is a private network that has the freedom to use any available area of IPv4address space.

• Expecting a rapid and extensive spread of the QKD network is not reasonable. IPv4 address space should therefore be enough to address current and future nodes.

• The distance limitations of QKD links do not allow the QKD network to be divided betwee nthe backbone and an arbitrary number of autonomous systems. Treating all nodes in the network equally is therefore necessary.

• The current lack of quantum repeaters means QKD nodes must be seen as an access point for end-user applications, not just as forwarding nodes in the network.

To meet the requirements for addressing and routing, SECOQC proposed forming a node with the following components:

• Q3P modules responsible for link-level communication with other nodes.

• A routing module to collect and maintain routing tables.

• A forwarding module to create paths and make forwarding decisions.

• Other modules for node management, random session key generation, security monitoring,and so on.

The function of a routing module is to maintain a local table of routing information and informother nodes about updates in the network so they can also update their routing tables. Well-known routing protocols can be modified and used in the QKD network, but it is necessary to take into account the lack of a quantum repeaters, which means that each network node must be ready to receive traffic from its neighboring node and forward it along the best path to the requested destination (more details in Section 6). This is the forwarding module’s task. The module receive san incoming packet, checks the TTL value and authentication checksum, and depending on the results, decides either to forward or discard the packet.

In SECOQC, a modified version of the OSPFv2 protocol was used. It is interesting that OSPFv2does not support QoS routing, which is necessary to guarantee the required service type. OSPFv2was implemented with the aim of accelerating the development process [33]. In the standard OSPF,the forwarding decision is made based on the destination address and the shortest path information in the routing table. However, considering the low key rate of QKD links, other parameters must be taken into account when calculating the best path.


To compute the shortest path tree data structure, the Dijkstra algorithm is used. Each node calculates a unique shortest path tree and uses a modified version of the OSPFv2 to compute the routing table. The main difference is that the modified OSPFv2 calculates multiple paths to each destination instead of a single shortest path. Multiple paths are needed to fulfill the requirements of load balancing and spare paths. Each QKD node therefore computes as many routing tables as the number of its interfaces. OSPFv2 delivers periodic LSA messages to other nodes in the network with the aim of spreading information about the current state of the network.

Furthermore, each node computes an Extended Routing Table in which all costs in increasing magnitude to every other node are listed. This table is used to merge all routing tables in a single place. The table structure is similar to the standard routing table, the difference being that it has as many entries for each destination as the number of outgoing node links [33]. Now, the node can find multiple paths to the destination node, but it also needs to know an approximate load of selected links in the path. If the load of a link is greater than the calculated threshold, the next best link is looked up, and so on. The third Load State Database table is computed to store details about the approximate load of each outgoing link. It is used to verify whether the link has sufficient resources for transmitting the message [28]. The approximate load of the outgoing link i at discrete time t is denoted by L i (t ) and calculated using a low-pass filter with Equation (2):L i (t ) = 1 -(2)· L i (t - 1) + · l i (t ), where l i (t ) is the instant load of the outgoing link i and w is the filter constant. The instant load l i (t ) is calculated as a ratio of the number of transmitted bits in the previous unit time. More details about routing and forwarding modules can be found in References [33] and [28].


5.2.4 Summary. It should be emphasized that application development was not the SECOQC project’s task. SECOQC, however, conducted several experiments to test the solutions created. During the SECOQC QKD conference from October 8–10, 2008, a demonstration of telephone communications and video conferencing was given. A VPN tunnel was established between the nodes and AES encryption was used. The AES key was refreshed every 20 seconds, 9 and at certain moments, AES encryption was replaced by OTP [106]. The main objective was to test routing mechanisms, measure key material consumption and generation, and highlight basic mechanisms of the SECOQC network functionality. It is worth noting that SECOQC investigated the establishment of a QKD connection to the end-user [28].

The SECOQC network has laid the groundwork for a hop-by-hop networking approach that greatly simplifies views on implementing routing decisions. The hop-by-hop approach allows each node to decide which further path to direct the message, which offers more flexibility in implementing routing protocols. However, the BBN key-repeater approach described in Section 5.1.5,requires having a global, up-to-date view of the network before establishing and reserving resources on the path, which can be demanding due to the dynamic consumption of generatin gkey rates. The SECOQC network also demonstrated interoperability between different equipment manufacturers and showed that the QKD network could achieve ranges of almost 100 km (the maximum link length was 82 km between the BREIT and St. Pölten nodes) [37]. A brief summary of the SECOQC network is given in Table 3.

Interest in quantum cryptography in the EU has been accompanied by projects funded under the Quantum Technologies Flagship, Quant Era, COST, and EuraMet programs [107–112]. In 2019, the EU Horizont2020 project OPENQKD with a consortium of 38 partners from industry and academia was announced [113]. OPENQKD aims to lay the foundations for future European quantum infrastructure and the convergence of quantum technology with practical telecommunications systems in Europe within three years.
\subsubsection*{Tokyo UQCC QKD Network}
Two years after SECOQC, nine organizations from Japan and the European Union participated in the Tokyo UQCC QKD testbed network (“Japan Giga Bit Network 2 plus” - JGN2plus). The network consisted of parts of the National Institute of Information and Communications Technology of Japan (NICT) open testbed network called “Japan Giga Bit Network 2 plus” (JGN2plus) [38].
The Tokyo QKD network contained four access points, Hakusan, Hongo, Koganei, and Otemachi,and six nodes connected by commercial optical fibers installed at these access points (Figure 10).
Since half of chosen fibers were aerial, large losses occurred on the links. The link between the Kogenei and Otemachi nodes had a loss rate of about 0.3 dB/km, while on other links this rate reached even 0.5 dB/km.

Similarly to SECOQC, the project’s participants implemented certain network links, allocated as follows:

• A 24-km link between the Otemachi and Hakusan nodes was provided by the Mitsubishi Electric Corporation and NTT Company. They implemented the BB84 protocol with a maximum key rate of 2 kbps and QBER of 4.5%.

• A 45-km link between the Koganei and Otemachi nodes was provided by NEC, implementing the decoy state BB84 protocol with a NICT superconducting single photon detector(SSPD). The maximum key rate was 81.7 kbps with an average QBER of 2.7%.

• NTT used DPS-QKD on the longest link in the network, which was 90 km between the Koganei-1 and Koganei-2 nodes. They also used an SSPD detector and achieved a maximum key rate of 15 kbps with an average QBER of 2.3% [114].

• Three organizations from Austria, including the AIT, the Institute of Quantum Opticsand Quantum Information (IQOQI), and the University of Vienna, formed a single team called “All Vienna.” They presented their SECOQC QKD device. This device was based on entanglement of the QKD BBM92 protocol, which was placed between the Koganei-2 and Koganei-3 nodes with a maximum key rate of 0.25 kbps.

• Toshiba Research Europe Ltd. demonstrated their decoy-state BB84 system with selfdifferencing avalanche photodiodes (SPAPDs) between the Koganei-2 and Otemachi-2nodes on a 45-km link. The maximum key rate was a record 304 kbps with an averageQBER of 3.8\%. This was by far the highest sustained QKD bit rate produced to date.

• Finally, a 13-km link between the Otemachi and Hongo nodes was provided by idQuantique from Switzerland, making use of the SARG04 protocol from their commercial Cerberis solution. The maximum key rate was 1.5 kbps.

The Tokyo UQCC QKD network followed a similar three-layer network architecture based on the trusted repeater approach as it was implemented in the SECOQC project. The main difference was the use of a Key Management Server (KMS) for centralized management. The Tokyo QKD network attempted to test a government-chartered network scenario, which often has a central dispatcher or central data server. The KMS was installed in Koganei-1, Koganei-2, Otemachi-1, and Otemachi-2. All nodes implemented Key Management Agents, whose main task was to save the key material and store link statistical data, such as QBER and key generation rate. Later, these statistical data were forwarded to the KMS, which coordinated with all the links in the network [38].

5.3.1 Summary. In October 2010, a live demonstration of secure TV conferencing, eavesdropping detection, and QKD link rerouting on the Tokyo UQCC QKD network was performed. Layer
2-VPN encryption with OTP between the Otemachi-2 and Koganei-1 nodes was established. Two routes were used to demonstrate the routing algorithm when the links were attacked by the eavesdropper. The KMS detected the attacks because of an increase in the QBER and rerouted the communication through a spare link.

As noted in Table 4, the Tokyo QKD network showed that QKD technology can reach speeds of several hundred bits per second. The network also confirmed communication capabilities in QKDlink distance, achieving a record link of 90 km between the Koganei-1 and Koganei-2 nodes [58]. However, what sets this network apart is the introduction of a hierarchical view into the organization of QKD networks. The key management servers implement a management layer and have complete insight into the state of the QKD network in their domain. Organization in this manner has brought the QKD network closer to the SDN perspective discussed in Section 7.

\subsubsection*{QKD Networks in China}
China has been constructing QKD networks on a national scale. These efforts started by constructing testbed metropolitan QKD networks in Hefei, where a three-node network [115] and five-node network [71] were constructed in 2009 and 2010, respectively. Other efforts to construct fiber-based QKD networks have been reported in References [116–119], and a satellite-based QKD network is also being formed [24]. This section provides an overview of these developments by looking at some recently constructed fiber-based networks.

5.4.1 Beijing-Shanghai Backbone QKD Network. In September 2017, the 2,000 km Beijing-Shanghai backbone QKD network commenced operation [120]. To date, it is the longest QKD network in the world. The project is led by the University of Science and Technology of China(USTC). Other participants include China Cable Television Network Co., Shandong Academy of Information \& Communication Technology, Industrial and Commercial Bank of China (ICBC),Xinhua Financial Information Exchange, and others. The network was completed in September 2016 and was tested for one year before commencing operation.

The backbone network consists of 32 physical nodes linearly connected by QKD links(Figure 11). Among these nodes, Beijing, Jinan, Fuli, Hefei, Nanjing, and Shanghai are the access points, while the rest are trusted repeater nodes. The backbone network has 135 links in total.
Two to eight multiple QKD links lie between adjacent nodes. To conserve fiber resources, the network uses quantum wavelength division multiplexing technology, which combines four quantum channels into a single fiber. The network rents dark fibers deployed by China Cable Television Network Co. The distance between adjacent nodes along the backbone line varies from 34 km to
89 km, with fiber loss varying from 7.26 dB to 22.27 dB.
The backbone network deploys the QKD devices provided by QuantumCTek Co. The device implements a polarization-coding-based decoy state BB84 protocol. Some of the devices integrate the up-conversion single photon detection technique and thereby achieve a 25\% single photon detection rate.

The backbone network is designed to function as a high bandwidth channel that feeds quantum keys between metropolitan and QKD networks located in different cities. Up to now, the backbone network has been connected to the metropolitan QKD networks already established in Beijing,Shanghai, Jian, and Hefei. A wide area QKD network has been thus formed and provides end-users,including banks, government agencies, and large enterprises, with versatile security services [121].

In November 2018, an extension of the Beijing-Shanghai backbone network was completed by establishing a backbone QKD link between Wuhan and Hefei. The purpose was to connect the Wuhan metropolitan QKD network to the backbone network. The Wuhan-Hefei backbone line is operated by CAS Quantum Network Co. In the long term, the backbone network will be further extended to cover a wider area of China.

5.4.2 Jinan Government Private QKD Network. The Jinan government private QKD network commenced construction in April 2017 and was completed in August 2017. The network covers an 8,000 km 2 area of the city and consists of 32 nodes, including a centralized control station node, eight trusted repeater nodes, and 23 end-user nodes. QuantumCTek Co., Ltd provides the QKD systems and the network design solutions, while China Union Shandong Branch provides the fiber resources [122, 123]. The network has 33 QKD links in total (Figure 12). The length of the links varies from 1.7 km to 64.7 km, with the fiber loss varying from 1.48 dB to 25.2 dB. The quantum signals are transmitted through dark fibers provided by China Union Shandong Branch.
The network deploys QKD systems implementing the polarization-coding- based decoy state BB84protocol. All systems are provided by QuantumCTek.

The network adopts a salable, self-built service channel, which provides secure data transmission services and a minimum bandwidth of 512 Mbps. The Jinan private network integrates VoIP telephone and video conferencing services supplied with quantum keys. The security service is accessed through repeaters that implement an IPSec VPN protocol supplied with quantum keys.
The network supports the OTP and several other symmetric encryption algorithms. The typical key refresh rate of symmetric encryption algorithms is once per second [124].

5.4.3 Wuhan Metropolitan Area QKD Network. The Wuhan metropolitan area QKD network was constructed in 2017 from January to December. The network consists of one command center,one centralized control station, 10 trusted repeater nodes, and 60 end-user nodes (Figure 13).

The network has 74 QKD links in total. The centralized control station and two central trusted repeater nodes are interconnected to form a network ring. Their connections adopt a dual-link structure (two QKD links are established between every other node). The longest QKD link was 16.5 km, and the optical channel loss incurred by the fiber, WDM device, and optical switch was a maximum of 14.6 dB over 6.7 km. This high loss largely results from the complex metropolitan fiber environment. The key rate of the QKD links ranges from 2.8 kbps to 141 kbps. The QKDsystems are provided by QuantumCTek [125].

One of the network’s features is classic quantum WDM technology [126], which integrates the classic service signal, quantum signal, and classic QKD post-processing signal into a single fiber. The security service is accessed via an encryption repeater that implements an IPSec VPN protocol supplied with quantum keys. The typical key refresh rate of the symmetric encryption algorithmsis once every five seconds.

5.4.4 Hefei, Chaohu, and Wuhu Wide Area QKD Network. A QKD wide area network connecting the cities of Hefei, Chaohu. and Wuhu (HCW) in China was reported in 2014 in Reference [119].
The entire HCW QKD network, which has a complete technical description available, operated for more than 5,000 hours from 21 December 2011 to 19 July 2012 and was installed in the Anhui provincial telecommunications network of China Mobile Ltd., with over 150 km of coverage area.

Thirteen QKD devices over nine nodes were employed in this network [119]. The HCW wide area network consists of two metropolitan networks: the Hefei QKD network, which has five nodes [116, 117], and the Wuhu QKD network [118], which has three nodes. These two networks were connected with an intercity QKD link, which combined Hefei and Wuhu metropolitan area QKD networks through a trusted repeater node at the Chaohu Branch of China Mobile Ltd. [119].

The network deployed QKD systems that implement the phase-coding-based decoy state BB84protocol. The maximum key rate in the HCW QKD network was 16.15 kbps between the West Campus and North Campus nodes over a 3.1 km link connected to an optical switch located in the Campus Library [119].

5.4.5 Networking Strategies. The Beijing-Shanghai backbone QKD network and several other metropolitan area QKD networks employ a number of networking strategies to improve the performance and robustness of the network:

• Some metropolitan area networks adopt a ring topology in the network’s design to improve its disaster tolerance.

• Backbone connections adopt a multiple parallel link networking strategy to improve bandwidth and the network’s stability.

• Some core nodes such as the centralized control stations deploy backup devices that reduce the probability of system service interruption caused by a single point of failure.

5.4.6 Network Key Routing. Key routing in the Beijing-Shanghai QKD backbone network, Jinan government-private QKD network, and Wuhan metropolitan area QKD network uses a client server architecture to maximize channel utilization and provide on-demand quantum keys to end-users.
The centralized control station of each network implements a key routing server that is in charge of managing the routing table for each network node. Based on information collected from the network (the running status of the QKD links, the remaining key storage capacity, and other information), the routing table of each node is periodically updated. The updated routing tables are exchanged with other network nodes to provide information about suitable paths until the next update [127].

The key routing server supports multiple queuing strategies adapted to different network topology structures. In the case of an emergency fault, such as the failure of the key management machine or unavailability of the key, the node device actively reports the event to the routing server.
The server then recalculates routing tables for affected nodes. In the Beijing-Shanghai backbone network, key routing is managed by dividing the network into multiple sub-networks. Each subnetwork adopts the above client-server structure with the key routing server located at the access nodes [127, 128].

5.4.7 Summary. With a 2,000-km link connecting Shanghai and Beijing and metropolitan networks in Hefei and Jinan, China is currently leading the QKD race in terms of practical developments [129]. In their described methodologies, unique approaches in implementing existing and available technology can be seen in these QKD networks. However, in addition to using discrete QKD protocols that guarantee high performance but require expensive single-photon detectors,experiments that rely on continuous-variable and measurement-device-independent (MDI) QKD have also been reported. For CV-QKD, results of 5.77 kbps over 50 km have been achieved [73, 130]. The experiments with MDI-QKD have resulted the higher rates (up to the channel attenuation): 98.2 kbps over 49.1 km and up to 1 Mbps over a dozen kilometers’ distance [131]. Moreover, QKD systems based on MDI can work efficiently not only in symmetric channels with similar losses,but even with channels with asymmetric losses [132]. The optimization methods [133] can extend the secure transmission distance in such MDI-QKD implementations by more than 20–50 km in standard telecom fiber. A brief summary of the networks described above with publicly available references is listed in Table 5.

Not discussed here for reasons of space, it is, however, important to mention that China is leading in the field of space-oriented quantum technology. In 2017, the 640-kg “Micius” satellite was launched [134]. In a 273-second satellite pass and using a 1-m telescope on the ground, sifted key rates of about 12 kbps at 645 km to 1 kbps at 1,200 km were expected [135]. After post-processing, 1.1 kbps for the secure key was obtained.
6
\subsubsection{QUALITY OF SERVICE IN QKD NETWORKS}

Similarities between QKD and Mobile Ad Hoc Network Technologies. The specific QKD issues and constraints described above pose significant challenges in QKD network design. However, analysis of the characteristics of QKD networks has shown similarities in Reference [136] to Mobile Ad Hoc Networks (MANET) and Vehicular Ad Hoc Networks(VANET) [137–139].

The main characteristics of QKD technology from a simple point of view can be listed:

• QKD links such as those described above are always implemented in point-to-point behavior and can be roughly characterized by two features: limited distance and key rate(exponentially) decreasing with distance. Links may become unavailable when there is note nough key material or when the public channel is congested. This is similar to Wi-Fi links,which are limited in range and whose communication speeds depend on the user’s distance from the transmitter’s antenna.

• One of the main features of current QKD networks is the lack of a quantum repeater (Section 3.2), and communication is therefore usually performed on a hop-by-hop basis.

In MANET networks, communication takes place on a hop-by-hop basis, and mobile nodes are typically powered with energy-aware solutions such as batteries. The nodes connect themselves in a self-organizing, decentralized manner with no authority in charge of controlling and network management. The main drawback of MANET networks is the unpredictable mobility of nodes,which can often lead to unstable routing paths [114]. The amount of battery power and the mobility of MANET nodes can be easily linked to the amount of key material in QKD key storage. The limited range of wireless links is much like the limitations in the length of a QKD link. The lack of dedicated network infrastructure (such as routers) is another similarity between these two technologies. The poor mobility of QKD nodes, however, makes it similar to VANET technology,in which communication takes place along a predefined path.

Although at first glance MANET and QKD networks have nothing in common, a simple analysis of the features of these networks reveals their similarity. What clearly distinguishes these two networks, though, is their purpose. MANET networks are designed for fast and straightforward communication in situations where pre-existing installed infrastructure is not available (e.g.,search-and-rescue operations during natural disasters or in war zones). By contrast, the primary goal of QKD is to provide ITS communication. This may have a significant impact when choosing network solutions, since a solution required in one situation may not be suitable in another. For example, consider routing solutions based on network flooding. QKD networks rely on the assumption that all nodes are trusted when communication is performed in a hop-by-hop or key repeating manner [32, 39], and by following this assumption strictly, an eavesdropper is restricted to attacking QKD links only. Because of the nature of QKD, the eavesdropper is not able to gain any information about the key being transported through the link, but service may be denied to disable the communication. Although results have been obtained by combining multiple paths to establish secure key material [27, 82], it is thought that the amount of routing information being sent to the nodes should be reduced to a minimum. To prevent a denial-of-service attack, no node(except source and destination) in a network should know the routing request. Therefore, the number of broadcast packets should be minimized. Furthermore, considering the primary objective of QKD, which is to provide ITS communication, routing packets must be either authenticated and encrypted or at least authenticated [140]. This means that the number of routing packets in the network needs to be minimized (routing overhead) concerning the material to be preserved for the protection of data, which is the primary goal of secure communication. From this, it follows that protocols based on flooding are not preferred in QKD networks.
\subsubsection*{Routing Protocols}
In previously deployed QKD networks, emphasis was placed on quantum channels. The public channel, though, was largely neglected and assumed that it was somehow achievable without any difficulties. The prioritization of network traffic and signaling protocols were ignored, and the solutions in existing conventional networks have consequently been modified for the needs of QKD networks. The first such solution, which is based on modifying the well-known Open Shortest Path First (OSPF) routing protocol [98], was implemented in the DARPA BBN QKD network builtin 2004 in the US [92]. Instead of using the routing hop-count metric, a modified OSPFv2 protocol was used to determine link quality according to the amount of key material in key storage. As discussed in Section 5.1.4, the modified version of the OSPFv2 does not take into account the status of the public channel [99].

A similar approach was offered in Reference [141], where the author proposed using unencrypted and non-authenticated communication to disseminate OSPFv2 routing packets. Obviously,this kind of network is easy prey for an eavesdropper with unlimited resources at their disposal,especially in terms of passive eavesdropping [76]. Since the described solution is based on the use of key material in key storage as a routing metric, it cannot provide efficient routing because of alack of information about the state of the public channel.

In the SECOQC network, another modified version of the OSPFv2 protocol was introduced [28,
33]. It was based on a local load-balancing policy calculated as the ratio of the number of transmitted bits over a period of time. As discussed in Section 5.2.3, a solution such as this does not consider the available amount of key material, which means that the algorithm may choose a path with insufficient key material for data transmission.

When the Chinese HCW QKD network was developed, a Quantum Key Reservation Approach(QKDRA) based on the IntServ model was applied [142, 143]. OSPFv2 is used to find the path from the source to the destination node. After the path is determined, the source node issues a key reservation request to all nodes in the path. After receiving a request, the intermediate node responds with a key reservation result message. Finally, the destination node determines the possibility of establishing the connection. Since OSPFv2 focuses on finding the shortest path,hence the name, solutions presented in References [142, 143] find the shortest path between the source and destination and reserve a sufficient amount of key material on a selected path. Note that this path may not be optimal. More specifically, the path is the shortest, but it may not bead equate in terms of QoS. It is known that minimum hop-count (shortest) routing typically finds routes with a significantly lower throughput than the best available [144], since it does not consider other link parameters. OSPFv2 in its original form does not consider QoS constraints; therefore,it cannot guarantee that traffic on the selected path will be adequately served. Reservation of resources on the quantum channel, in this case, does not provide a gain, since the path for the public channel may be inappropriate. However, even an extended version of OSPFv2 that includes QoS constraints [145] may not be optimal for QKD networks. Implementation of OSPFv2 in this way can find the path that has the best characteristics of the public channel but does not consider the parameters of the quantum channel.
Yang proposed using the Dijkstra algorithm to identify multiple paths but without considering the status of the public channel [146]. The idea is to use thresholds to exclude links that have a lower key material amount and periodically flood routing details, such as the amount of available key material.

The impact of public channel states on the key rate can be found in Reference [147]. This study shows that a public channel should not be excluded in route calculations, since the performance of the public channel affects the quantum channel and vice versa. Therefore, novel metrics are introduced to uniquely describe the state of the public and quantum channels as well as the overall QKDlink [147, 148]. With the aim of minimizing key consumption, network flooding should be avoided and a single-layer network organization and Greedy Perimeter Stateless Routing Protocol for QKD networks (GPSRQ) was introduced [136]. The GPSRQ routing protocol uses distributed geography and reactive routing to achieve high-level scalability. It is equipped with a caching mechanism and detection of returning loops, enabling forwarding while minimizing key material consumption. However, GPSRQ applications are limited to planar topologies only, because geographic routing in networks with non-planar topologies are not able to quickly determine the shortest path, leading to unnecessary forwarding and increased consumption of scarce key material.

Routing in QKD networks depends primarily on the architecture of the network organization(hierarchical or distributed architecture, overlay or single stack network, hop-by-hop or key repeater networking). Unlike conventional networks, routing solutions in QKD networks need to take into account both channels of the QKD link. Based on the requirement to minimize key consumption, it is necessary to reduce the amount of routing packets that have to be encrypted and authenticated or at least authenticated to avoid active and passive eavesdropping QKD network attacks [28, 140].

Considering the efforts to extend the QKD network to metropolitan areas, which involves a significant number of network nodes, a hierarchical organization was considered in previously implemented networks [42, 125, 149–151]. This approach, which is based on a key management layer, converges to a software-based network paradigm and is discussed in more detail in Section 7.

\subsubsection{QKD SOFTWARE DEFINED NETWORKING}
A more evolutionary strategy for adopting QKD in transport networks is taking advantage of the latest developments in networking technologies, more specifically, in network management.
Software-defined networking (SDN) [152, 153] allows the control (management) and data (forwarding) planes to be separated. Its popularity has increased in both academic and industrial spheres since its creation in 2008. SDN allows the integration of new technologies and services at a faster pace while enabling centralized management and optimization based on network programmability and configure ability principles. Although the approach for SDN has changed over they ears from OpenFlow-based device programmability towards open and standard interfaces, this transformation has helped network operators increasingly adopt SDN in their systems to reduce the time-to-market and vendor lock-in.

An SDN network is conceptually organized into three layers. The control and management layer knows the status of the entire network and can optimize its behavior through a centralized entity known as the SDN controller. The controller identifies the capabilities of the devices installed in the infrastructure layer through a set of standard mechanisms (southbound interface). It also knows the requirements of the different applications running in a network through standard interfaces (northbound interface). Its role is to optimize resources and provide the means for devices and services to fulfill their tasks. A QKD system installed in infrastructure can export its requirements to the controller so it can create a specific path with the required optical characteristics(i.e., maximum tolerated noise, attenuation, etc.) to connect the emitter to the receiver (either on a single or multi-hop path) and satisfy an application’s requirements. This allows an unprecedented means of creating a fully integrated classic/quantum network and genuinely zero-configuration QKD devices that can be directly plugged into a standard telecommunications network.

Before this technology, demonstrations required either an ad-hoc, separate network just for the quantum channel (i.e., typically a network of dark fibers) or specific network modifications for each link [154, 155]. These are very expensive and entirely orthogonal deployments for common telecommunications activities, in which devices are expected to work out of the box and share the fiber with many other conventional communications channels. For QKD to become mainstream, it is critically important that QKD systems follow the trends and architectures used in the transport network segment.

Other projects and demonstrations have shown initial steps towards automating QKD networks.
In References [156, 157], the authors implemented a mechanism for automating the switching of a quantum channel between a transmitter and two simulated receivers using optical cross-connect switches that were OpenFlow-enabled. In this sense and despite the enabler, in this case, being a software-defined optical network (SDON) controller, the research focused more on applying secure virtual machine migrations in a distributed data center scenario.
The most advanced contribution towards Software-Defined QKD Networks was presented in References [158–160] (Figure 14(b)). Three production sites in the Telefonica in Spain network were connected. The proposed architecture and demonstrations aimed to demonstrate the technological maturity of QKD systems for integration into production networks. The CV-QKD systems used for the trial were implemented so they could be managed and optimized with software processes and were robust enough to coexist with traditional communications channels. The software integrated the first version of an SDN interface defined by the Industry Specification Group (ISG) for QKD at the European Standards Telecommunications Institute (ETSI). With this interface, the QKD systems and the key delivery processes are centrally managed by an SDN controller, allowing quantum channels (via optical switches) to be dynamically instantiated, multi-hop associations to be created, and demands for keys from external applications to be identified. This setup was also designed so any control and data channel could integrate QKD-derived keys to secure communications related to either the QKD network or traditional telecommunications services running in the production network.

QKD can also be seen as an additional security layer for transport networks. The integration of QKD in SDN is a mutually beneficial relationship, since QKD-derived keys can be used to secure the different layers of a transport network. Apart from the demonstration conducted in Reference [157] in which the encryption algorithm (AES) was used to provide security, the authors in Reference [161] showed how existing security protocols used in the control plane could integrate quantum cryptography in a seamless evolutionary manner without affecting current schemes. Being composable in both cryptosystems (QKD and traditional or even post-quantum schemes), the security of the proposed system brings the best of both: certification from traditional schemes is still applicable to the hybrid system, while the security of the resulting solution is the highest possible, because breaking the final key means both cryptosystems must be compromised. This solution is deployed in control channels orchestrating an SDN controller and a network function virtualization (NFV) architecture through SSH and TLS protocols.

The SDN-based experiment of monitoring and mitigation of physical layer attacks was reported [162]. Real-time monitoring of QBER and the secret key rate was used to recalculate routes for the quantum channel establishment.
Other cases have focused more on data plane security and service establishment. Marksteiner presented an integration of QKD-derived keys in IPSec channels, focusing its research on the security and scalability of the solution depending on the service throughput [163]. In addition to this research, the approach reported in Reference [164] focuses on service automation for encrypted channels in an end-to-end network. Automation was suggested for data center scenarios (implementing extensions in OpenFlow) and for transport segments (using MPLS and NETCONF for configuration). This was integrated into virtual network functions implementing the extensions and the security channel using IPSec, as in Reference [163]. Mavromat is demonstrated the usage of QKD for energy-efficient SDN management of Internet of Things devices [165].
We also point out experiments with the use of SDN to control the WDM organization of QKDlinks [156, 166–168] as well as the use of machine learning (ML) models for the prediction ofCh-QKD quality in QKD-DWDM networks with increasing efficiency of SDN-enabled optical networks [169].

In a broad QKD network where multiple QKD tenants share the same underlying infrastructure,addressing the secure-key assignment is essential for efficient network managment. Cao proposed the SDN-based secret-key rate sharing approach using heuristic algorithm using simulations [170].
The multi-tenant organization can be served using Key as a Service (KaaS) approach where key pools (KP) defined at the control layer of SDN hierarchy mapped to virtual key pools using RESTful API at the application layer.
These results show how SDN must be seen as a technological enabler for QKD’s integration into transport networks. At the same time, QKD also benefits the network, since it implements an additional ITS layer for critical infrastructures. This integrative approach allows QKD systems to be smoothly integrated into the network and for QKD to be commercialized at different service levels (self-healed network infrastructures, end-to-end services at different OSI layers, etc.).

\subsubsection{CONCLUSION}
Quantum cryptography is an attractive cryptographic technology that has received the attention of various organizations in academic and industrial communities. In recent years, notable progress in the development of optical equipment has been reflected through a number of successful demonstrations of QKD technology. These demonstrations show great achievements in quantum cryptography and highlight the practical difficulties that still need to be resolved.
We provide a summary of the major key points related to QKD networks in Table 6. Trusted repeaters are necessary to extend the secure transmission distance of quantum channels. Solutions for integrating QKD networks into existing optical communications networks are currently the hot topic in optical research. Real quantum cryptography networks employed by end-users for real-life information transfer applications will be the next milestone. In terms of the industry,standards for security evaluation, production, and application of QKD are already being defined[189, 190].

Currently, a person finding himself in a QKD laboratory and asking for the maximum achievable key rate will receive a response with a question about the distance she/he is looking to cover.
As mentioned in Section 3, one of the main drawbacks of QKD links is length limitations. However, the networks discussed in this document demonstrate the significant development in optical equipment in recent times. In 2002, QKD systems achieved a key rate of 1 kbps [29], which was used in the DARPA QKD network. In 2007 in SECOQC, this key rate increased tenfold [37], while in 2011 in the Tokyo QKD network, a key rate of 300 kbps was achieved [38]. This key rate was sufficient to establish a video conference secured with an OTP cipher provided by QKD. It is also interesting to compare the length of links in these networks. The maximum length in the DARPAQKD network was a 29-km connection via the optical switch between Harvard and Boston Universities [91]. In SECOQC, the maximum length of the QKD link was 82 km between the BREIT and St. Pölten nodes [37]. In Tokyo, the maximum distance was a record 90 km between the Koganei-1and Koganei-2 nodes [171]. In Hefei-Chaohu-Wuhu (HCW) in China, the maximum distance was
85.1 km via the HCW intercity link between Hefei and Chaohu [119, 191].
It is reasonable therefore to expect a higher key rate and longer distances in the coming years.
Since optical quantum repeaters are predicted to become available for practical use in the future [57], QKD networks are currently implemented solely through the Trusted Repeater Approach(TRA). TRA is essential for overcoming the distance limitations between QKD links and in providing routing in QKD networks. TRA, however, has several restrictions that will have to be resolved if a QKD network is to be applied in everyday life and integrated with conventional IP networks.
One means for widespread application of QKD technology is integration with telecommunications networks using an approach such as SDN-QKD.

\subsection{\trnas}

\subsubsection*{Аннотация}
Сближение квантовой криптографии с приложениями, используемыми в повседневной жизни, является темой, привлекающей внимание промышленного и академического мира. Развитие квантовой электроники привело к практическому достижению квантовых устройств, которые уже доступны на рынке и ждут своего первого применения в более широком масштабе. Одним из основных аспектов квантовой криптографии является методология квантового распределения ключей (QKD), которая используется для генерации и распределения симметричных криптографических ключей между двумя географически разделенными пользователями с использованием принципов квантовой физики. В предыдущие годы было создано несколько успешных сетей QKD для проверки реализации и совместимости различных практических решений. В данной статье проводится обзор ранее применявшихся методов, показываются приемы развертывания QKD-сетей и текущие проблемы создания QKD-сетей. В отличие от исследований, посвященных оптическим каналам и оптическому оборудованию, данный обзор фокусируется на сетевом аспекте, рассматривая организацию сети, протоколы маршрутизации и сигнализации, методы моделирования и программно-определяемый подход к созданию сети QKD.


\subsubsection{Введение}
Установление безопасных криптографических ключей через недоверенные сети является одной из самых фундаментальных криптографических задач [1]. Хотя использование инфраструктуры открытых ключей, основанной на вычислительно сложных математических задачах и предположениях о вычислительной мощности подслушивающих лиц, преобладает, они относятся к группе теоретически разрушаемых решений вычислительной безопасности.
Поэтому они находятся под угрозой, поскольку вычислительная мощность продолжает расти и появляются алгоритмы квантовых вычислений, которые могут решить некоторые широко используемые вычислительно сложные математические задачи за полиномиальное время [2, 3]. Квантовое распределение ключей, известное как QKD [4], основано на принципах квантовой теории информации и позволяет устанавливать информационно-безопасные криптографические ключи, которые не зависят от этих ограничений, по крайней мере, на уровне протокола. Для этого схема аутентификации сообщений, такая как схема Вегмана-Картера [5], должна быть объединена с QKD [6, 7].
QKD-сети значительно отличаются от традиционных телекоммуникационных сетей из-за специфики QKD-соединений и организации сети. Такие ограничения, как ограниченная скорость генерации ключей и достижимое расстояние (раздел 2), отсутствие квантовых ретрансляторов (раздел 3.2), специфическая маршрутизация из-за использования публичных и квантовых каналов в квантовых каналах (раздел 6), а также организация сети, в которой на данный момент используется подход передачи ключей по принципу hop-by-hop (раздел 5.2.2), являются мотивами для данного обзора. Хотя можно найти несколько работ по сканированию QKD-соединений и квантовых каналов QKD [8-10], данный обзор посвящен QKD-сетям, организации сетей, протоколам маршрутизации и сигнализации, а также программно-определяемым методам QKD-сетей. После прочтения этого обзора заинтересованные читатели получат представление о квантовых сетях с инженерной точки зрения и будут знакомы с режимами функционирования, реализацией, существующими решениями и методами моделирования квантовых криптографических сетей. Данный обзор дает высокоуровневое представление о QKD-сетях и будет полезен и интересен исследователям, практикам, занимающимся проектированием QKD-сетей, и аспирантам в области прикладной квантовой криптографии.
Обзор организован следующим образом: В разделе 2 представлены особенности QKD-соединений. В разделе 3 кратко описаны ограничения и основные характеристики QKD-сетей, а также объясняется, как они практически реализуются. Типы QKD-сетей описаны в разделе 4. В разделе 5 рассматриваются ранее развернутые QKD-сети. В разделе 6 обсуждаются методы маршрутизации QKD-сетей. В разделе 7 представлен обзор программно-определяемых сетей QKD. Раздел 8 завершает данный обзор.

Обзор включает дополнительный материал с перечислением дополнительных сетей QKD в Разделе 1 и методами моделирования, обсуждаемыми в Разделе 2. Обзор протоколов сигнальных сетей приведен в разделе 3, а инкапсуляция QKD-заголовков и QKD-пакетов обсуждается в дополнительном материале, разделе 4. В разделе 5 представлен обзор работы в процессе стандартизации QKD.
%Графическая схема структуры исследования показана на рисунке 1.
\subsubsection{QKD связи}
QKD-связь, или просто "связь", обозначает логическое соединение между двумя удаленными QKD-узлами, соединенными квантовым каналом, используемым для передачи фотонов, и публичным каналом, используемым для постобработки обмениваемой информации, соответственно. Недостаток этого типа связи выражается в ограниченной скорости генерации ключей квантового канала, доступной сторонам, соединенным прямым оптическим волокном или свободной прямой видимостью в режиме "точка-точка" (P2P) на определенном расстоянии.
Однако это также является необходимым условием для безопасной генерации ключей.
Хотя оптоволокно является хорошей и широко используемой средой для передачи кубитов, установка выделенного оптического канала для целей QKD не является практичной во всех обстоятельствах. 1 Иногда удобно использовать канал связи в свободном пространстве, хотя он имеет свои недостатки, поскольку требует подходящих атмосферных условий, пути видимого света и приемлемого отношения сигнал/шум (SNR), что строго ограничивает время использования. Тем не менее, результаты, полученные в ходе экспериментов в Лос-Аламосе [17] и Мюнхене, в которых была установлена связь между землей и самолетом, летящим со скоростью 290 км/ч [18], продемонстрировали перспективность спутниковых соединений [17-23]. После проведения ряда экспериментов по ККД в свободном космосе и на земле, Китай успешно запустил квантовый спутник "Micius", который продемонстрировал ККД со спутника на землю на расстоянии от 645 до 1200 километров [24].
Максимальное расстояние, на котором может быть сгенерирован ключ, уменьшается с увеличением потерь и шума оптического детектора. Для данного детектора и настроек, скорость темного счета детектора постоянна, но скорость ключа уменьшается с расстоянием из-за увеличения кумулятивных потерь. В современных коммерческих оптоволоконных системах расстояние QKD-канала примерно ограничено 100 км, а скорость передачи ключей - несколькими десятками или сотнями [26, 27]. Из-за ограниченной скорости передачи ключей хранилище ключей устанавливается в обеих конечных точках соответствующей линии связи. Это хранилище постепенно заполняется новым ключевым материалом, а имеющийся ключевой материал впоследствии используется для шифрования/дешифрования потоков данных [28].
Объем данных, подлежащих шифрованию, и тип алгоритма шифрования определяют скорость разрядки хранилища ключей, или, проще говоря, скорость потребления ключей. Скорость использования ключей звеном иначе называется скоростью зарядки ключей [28-31]. QKD-соединение может быть обозначено как "в настоящее время недоступное", когда в хранилище ключей нет доступного ключевого материала, так как никакие криптографические операции не могут быть выполнены [32]. Стоит также отметить, что, по-видимому, оптимальной стратегией для QKD-устройств является непрерывная генерация ключей с максимальной интенсивностью до полного заполнения хранилища (что зависит от способа реализации) [28, 33].
Ключ может быть использован для шифрования связи по публичному каналу с помощью шифра One Time Pad (OTP) и схемы аутентификации ITS, такой как Wegman-Carter [34, 35]. Поскольку для шифра OTP требуется такое же количество ключа, которое соответствует длине шифруемого сообщения, и дополнительные ключи для аутентификации ITS, этот подход потребляет больше ключевого материала, чем передаваемое сообщение. Если ключевого материала недостаточно, OTP не может быть использован, и наиболее распространенным выбором является использование альтернативных криптографических методов, таких как Advanced Encryption Standard (AES), который не требует такого большого расхода ключей [36].

\subsubsection{QKD сети}
Сети QKD используются для расширения диапазона действия систем QKD и состоят из статических узлов, которые представляют собой безопасные точки доступа, считающиеся имеющими неограниченную вычислительную мощность и источник питания. Из-за того, что каналы, соединяющие узлы, работают по принципу "точка-точка", ранее развернутые тестовые площадки [29, 37, 38] показали, что защищенные ключи в сетях QKD могут передаваться от узла к узлу по принципу "хоп-хоп" (раздел 5.2.2) или через концепцию ретранслятора ключей (раздел 5.1.5).
Общим для обеих сетей является предположение, что все узлы в сети должны быть доверенными[32, 39]. Этого предположения можно избежать, если использовать методы квантового сетевого кодирования для многопутевой связи [40]. В этом обзоре кратко рассмотрены ранее развернутые сети QKD, сфокусированные на методах связи, протоколах маршрутизации и организации сети.
Чтобы облегчить организацию, сеть QKD часто описывается с использованием нескольких уровней[41, 42]:
- Квантовый уровень, на котором устанавливается защищенный симметричный ключ.
- Уровень управления ключами, используемый для проверки и управления ранее установленным ключом.
- Уровень связи, на котором установленный ключ используется для защиты трафика данных.
Как упоминалось выше, QKD является примитивом согласования ключей и как таковой находится на самом нижнем (базовом) уровне архитектуры сети QKD. Принимая во внимание различные скорости потребления ключевого материала различными приложениями, ситуация, в которой недостаточно ключевого материала для удовлетворения потребностей более высоких уровней, нежелательна. Поэтому квантовый уровень должен постоянно создавать ключевой материал. Чтобы обеспечить гарантированный уровень обслуживания, сеть QKD должна иметь детальное представление о своих ресурсах и возможностях. Ранее развернутые сети QKD не имели определенных стратегий для обеспечения качества услуг. Например, сеть SECOQC QKD, рассмотренная в разделе 5.2, придерживалась базового типа сервиса Best Effort, который определяет только среднюю ключевую скорость и разрыв трафика, в то время как тип сервиса Guaranteed Key Rate был предложен для улучшенных версий сетей QKD [33].
Учитывая наличие исчерпывающей и подробной документации по квантовой оптической связи [26, 43-47], акцент в данной публикации сделан на двух верхних уровнях. Эти уровни могут иметь различную и независимую сетевую организацию, поскольку связь между узлами осуществляется через существующие стандартные соединения, такие как Интернет, куда может быть включено произвольное количество промежуточных устройств (Рисунок 2). Уровень управления ключами отвечает за управление ресурсами хранения ключей, протоколами маршрутизации, качеством обслуживания (QoS) и так далее. Самый верхний коммуникационный уровень использует ранее созданный ключевой материал для шифрования трафика данных с помощью существующего набора протоколов безопасности, таких как Internet Protocol Security (IPSec) [14, 48].Однако описанная иерархия распределяет ответственность за безопасность между всеми тремя уровнями.
\subsubsection*{Атрибуты сети QKD}
QKD представляет собой новое поколение решений в области безопасности, которые не полагаются на вычислительные предположения проблем, считающихся сложными. Однако сети QKD должны быть интегрированы в существующую среду и должны соответствовать определенным критериям и условиям. Некоторые из наиболее распространенных требований к сетям QKD перечислены ниже.
3.1.1 Скорость передачи ключей. Одним из жизненно важных параметров, описывающих сеть QKD, является средняя скорость передачи ключей по каналу QKD. Поскольку операции шифрования и дешифрования не могут быть выполнены без достаточного количества ключевого материала, конкуренция между скоростью хранения ключевого материала в хранилище ключей и скоростью его потребления для операций шифрования и дешифрования оказывает большое влияние на производительность сети.
Сравнивая ранее развернутые сети QKD и испытательные стенды в хронологическом порядке, можно заметить быстрое улучшение в развитии квантового оборудования. Системы QKD, реализованные в 2002 году в сети DARPA QKD, могли достичь скорости передачи ключей около 400 бит/с на расстоянии 10 км [29]. В 2007 году в SECOQC максимальная ключевая скорость составила 3,1 кбит/с на расстоянии 33 км [37]. Лучшие решения, представленные в Токио в 2009 году, достигли ключевой скорости 304 кбит/с на расстоянии 45 км [38]. В 2017 году в Китае была построена магистральная QKD-сеть Пекин-Шанхай протяженностью 2 000 км, устройства которой обычно достигают скорости передачи ключей 250 кбит/с на расстоянии 43 км.
За последние 20 лет благодаря усовершенствованию оптических компонентов и улучшению электроники, в основном в детекторах, была получена постоянно растущая скорость передачи секретных ключей. Для достижения рекордно высоких скоростей около 10 Мбит/с [49] была оптимизирована цифровая обработка сигнала в ПЛИС. Пропускная способность измеряемых кубитов для повышения скорости передачи ключей также была увеличена, особенно для более коротких линий связи, за счет устранения ограничений без использования ПЛИС. Вторая гонка открыта для достижения более длинных расстояний однопролетной передачи [50-53] на основе усовершенствования протоколов, а также технологических улучшений, ведущих к детекторам с постоянно уменьшающейся скоростью темновых отсчетов. 3 Можно утверждать, что развитие, направленное на улучшение скорости передачи данных по одиночным каналам на короткие расстояния и максимальный пролет, сделает сети QKD ненужными. Однако верно и обратное, поскольку возможность открыть массовый рынок с помощью этих улучшений на уровне каналов связи представляется низкой и вряд ли позволит охватить широкие сценарии развертывания, специально используя технические усовершенствования последних лет. Что действительно позволяют последние усовершенствования, так это увеличить разнообразие каналов, которые потенциально могут быть развернуты в сетях QKD.
Поэтому разумно ожидать, что в будущем оптимальное решение будет значительно превышать нынешние значения ключевой скорости и расстояния, хотя гонка между генерацией и потреблением ключевого материала сохранится.
3.1.2 Длина канала связи. Фундаментальным ограничением QKD канала является длина, на которой может быть сгенерирован безопасный ключевой материал (из-за рассеяния и поглощения поляризованных фотонов и других факторов [27, 44, 45, 54]), что ограничивает возможности квантовых каналов (прямые оптические линии или свободная линия прямой видимости) определенным расстоянием. Интересно сравнить длины связей в ранее построенных сетях QKD. 4 Максимальная длина в QKD-сети DARPA составляла 29 км через оптический коммутатор между Гарвардским и Бостонским университетами [29]. В SECOQC максимальная длина соединения составила 82 км между узлами BREIT и St. Pölten [37], а в Токио максимальное соединение между узлами составило рекордные 90 км между узлами Koganei-1 и Koganei-2 [58]. В магистральной QKD-сети Пекин-Шанхай максимальная длина соединения составляет 89,3 км между узлами Хэфэй и Вувэй.
В современных системах с оптическими волокнами расстояние, на котором можно эффективно применять QKD-связь, ограничено примерно 100 км [26, 27].
3.1.3 Защита ключевого материала. Основной причиной интереса к QKD является конфиденциальность установленного ключевого материала. Это означает, что узлы сети QKD должны быть защищены с большой вероятностью того, что установленный ключевой материал уникален и недоступен для третьих лиц. Безопасность ключевого материала оценивается не только при его создании, но и при управлении, хранении и, в конечном итоге, использовании. Поэтому важно обеспечить безопасность каждого уровня архитектуры сети QKD.
3.1.4 Использование ключей. Из-за дефицита ресурсов (скорость генерации ключей), коммуникация в сети сводится к минимуму, поскольку каждый дополнительный пакет означает расходование дополнительного количества ранее созданного ключевого материала. Поскольку связь обычно осуществляется по принципу hop-by-hop, что требует достоверности всех узлов на пути, выбор кратчайшего пути маршрутизации необходим для минимизации количества узлов, которые потенциально могут быть похищены или атакованы подслушивающим устройством. Кроме того, использование более длинных путей требует большего расхода ключевого материала. Во время перегруженности сети или проблем со связью использованный ключевой материал намеренно отбрасывается, а для повторной передачи применяется новый ключевой материал, чтобы снизить риск утечки [28]. Поэтому минимизация количества переходов является предпочтительной.
3.1.5 Надежность. Из-за стоимости и способа реализации, сети QKD будут медленно интегрироваться в традиционные и повседневные телекоммуникационные среды. Поэтому важно обеспечить устойчивость, которая выражается в постепенном и беспрепятственном добавлении новых узлов и установлении новых связей. QKD-сеть должна обеспечивать адекватные пути замены, чтобы избежать дефектных узлов или узлов, подвергающихся серьезным атакам. Независимо от методов защиты, важно помнить, что злоумышленники могут легко найти способы прерывания оптических связей и разрыва QKD-соединений. Сеть QKD должна иметь адекватную реакцию на такие ситуации.

Поскольку трафик может быть соединен и направлен между различными сетевыми доменами, сетевые повторители играют фундаментальную роль в современных сетях. Хотя теоретические и пионерские результаты в области квантовых повторителей уже имеются [59-62], на практике они остаются недостижимыми с нынешней технологией [10, 27]. Идея квантового повторителя заключается в использовании квантовой запутанности фотонов для связи по различным квантовым каналам. Квантовая запутанность является ключевым аспектом в применении квантовых коммуникаций и квантовой информации. Вкратце, квантовая запутанность подразумевает, что несколько частиц связаны друг с другом таким образом, что измерение квантового состояния одной частицы определяет возможные квантовые состояния других частиц. Даже если частицы разделены большим расстоянием, они все равно составляют единую квантовую систему. Верность запутанности - это свойство, используемое для описания того, насколько хорошо сохраняется запутанность между двумя подсистемами в квантовом процессе.
В теории, однако, применение запутанных состояний и обмена запутанными состояниями затруднено двумя основными препятствиями. Первое заключается в том, что чем больше расстояние между двумя запутанными системами, тем ниже точность. Фактически, достижимая точность квантового состояния уменьшается экспоненциально с расстоянием из-за квантовых каналов с потерями [27, 63]. 5 В этом контексте концепция очистки запутанности [64, 65] может быть использована для повышения верности одного запутанного состояния путем использования ряда зашумленных запутанных состояний (как описано в ссылке [60]). Однако это увеличивает количество необходимых ресурсов для передачи каждого кубита через квантовый повторитель (т.е. количество запутанных состояний). Вторым препятствием на пути создания квантового повторителя по схеме, приведенной, например, в ссылках [60, 61], является то, что для квантовой памяти требуется технология, которая на сегодняшний день практически недоступна. Использование квантовых повторителей основано на идее создания "цепочек" запутанных фотонов с помощью техники, называемой обменом запутанностью. Были разработаны концепции как с квантовой памятью [66], так и без нее [67]. Различные строительные блоки для согласования передачи длин волн этих летающих кубитов с квантовой памятью были практически продемонстрированы [68, 69]. Внутренние потери и верность должны быть улучшены для реализации цепочек с одним или несколькими промежуточными узлами, работающими на более высоких скоростях. Недавно была опубликована первая работа по интеграции будущих квантовых повторителей в общую инфраструктуру [70]. Таким образом, каждый узел в сети QDK действует как ретранслятор и пересылает пакеты или состояния переключения других узлов, чтобы обеспечить обмен квантовой информацией между узлами QKD.

\subsubsection{QKD виды сетей}
Хотя было предложено множество гибридных реализаций, сети QKD можно разделить на две отдельные категории: сети QKD с коммутацией и сети QKD с доверенным ретранслятором.

Коммутируемые сети QKD состоят из узлов, подключенных к выделенной, полностью оптической сети. Эта сеть содержит механизм коммутации, используемый для установления прямого оптического QKD-соединения "точка-точка" между любыми двумя узлами сети QKD. Ограничения на расстояние в QKD-соединениях "точка-точка" ограничивают эти сети масштабами мегаполиса или региона [10]. Поскольку каждый оптический коммутатор добавляет по крайней мере несколько дБ потерь в фотонный канал, оптические коммутаторы могут значительно уменьшить радиус действия сети.
Основным недостатком коммутируемых QKD-сетей является требование выделенной оптической инфраструктуры для квантовых каналов, что часто экономически нецелесообразно. Напротив, основным преимуществом этого класса сетей является зависимость от оптического коммутатора, который позволяет установить соединение между двумя узлами без активного участия других узлов сети (Рисунок 3(a)).

Другим недостатком коммутируемых сетей QKD является согласованность применяемой техники QKD.
Объединение различных методов QKD, таких как QKD в свободном пространстве и QKD по волокну, невозможно, так как нет подходящих устройств, которые могли бы выполнить это преобразование в канале. Первая коммутируемая всепроходная сеть QKD была описана в статье [71]. Четыре узла были соединены через оптический коммутатор, и каждый из QKD-терминалов был разработан как приемопередатчик, чтобы они могли установить QKD-связь с одним из трех других одновременно.

В QKD-сетях с доверенным ретранслятором безопасность каждого узла на пути передачи необходима для безопасной передачи информации (отсюда и название). Связь "точка-точка" между двумя узлами обеспечивает идентичные ключи для узлов и, таким образом, обеспечивает безопасную связь (разделы 5.1.5 и 5.2.2). Учитывая отсутствие квантового ретранслятора, узлы также отвечают за механизмы маршрутизации и пересылки (Рисунок 3(b)). Организация сети таким образом является ее самым большим недостатком, поскольку безопасность передачи зависит от безопасности всех узлов на пути. Однако сети доверенных ретрансляторов не ограничены расстоянием или количеством узлов и могут состоять из различных QKD-устройств, реализующих различные технологии QKD.

Поскольку квантовые каналы могут быть "отданы" подслушивающему лицу без ущерба для безопасности QKD, рациональный противник скорее нацелится на более слабое звено - узел. Обычно предполагается, что узлы "неуязвимы", что является гипотезой доверенного ретранслятора. Однако, учитывая, что оптическое устройство управления в какой-то момент, скорее всего, будет иметь обычную компьютерную логику управления, безопасность устройства не лучше, чем безопасность классического компьютера, выполняющего алгоритмы QKD, и его физическая защита.
Признанное сильное предположение о полностью доверенных повторителях может быть ослаблено по крайней мере тремя способами: (i) использовать QKD, независимые от измерительных устройств (MDI), (ii) использовать квантовые повторители, и (iii) полагаться на множественные пути.

Этот первый подход был описан в [72] и добавляет предположение о совершенной подготовке состояния, достижимой сторонами связи, а также добавляет потенциально ненадежное место в квантовый канал. Измерения с использованием состояний Белла и формальные аргументы в пользу "безусловной безопасности" были подкреплены экспериментальными демонстрациями [73, 74]. Конечно, отсутствие предположения о доверенном ретрансляторе в этих доказательствах делает безопасность намного сильнее, чем в тех, которые предполагают доверенный ретранслятор QKD. Заметим, однако, что MDI QKD по существу удлиняет квантовый канал, но две станции отправителя все равно должны быть расположены в узлах доверенного ретранслятора. Это справедливо и для других альтернатив, описанных далее (за исключением, возможно, случая, когда сквозные квантовые каналы могут быть установлены без промежуточных узлов доверенного ретранслятора).

Концепция квантовых повторителей обсуждалась выше (см. раздел 3.2). Хотя были представлены практические демонстрации [61, 66, 67], пространственные расстояния, которые может преодолеть технология (на сегодняшний день), сильно зависят от количества верности, вызванной обменом запутанности, и степени, в которой с ней можно справиться (раздел 3.2).
Третий и наиболее практичный метод сегодня прибегает к классической технологии и использует множественные пути и пороговые криптографические методы для снижения риска подслушивания. Грубо говоря, квантовые сети с многопутевой передачей обменивают доверие к ретрансляторам на предположение о том, что ретранслятор уязвим для подслушивания, и злоумышленник вынужден перехватить множество промежуточных устройств, чтобы обнаружить сообщение. Действительно, можно показать, что в отсутствие доверенных ретрансляторов множественные пути являются теоретической необходимостью. В то же время, избыточность путей также смягчает проблему уязвимости всех реализаций QKD к атакам типа "отказ в обслуживании" (противник может пассивно подслушивать не для получения информации, а для того, чтобы локальные хранилища квантовых ключей иссякли и заставили конечные точки перейти на обычные методы передачи [75]).
Усовершенствованные механизмы маршрутизации могут быть использованы для обхода линий с обнаруженными подслушивающими устройствами.
Действительно, в противном случае злоумышленники могут попытаться нарушить безопасность, используя пассивное подслушивание для перенаправления трафика через уязвимые повторители и, таким образом, завладеть секретным ключом [76]. Можно показать[77, 78], что "сквозная безопасность" без доверенных повторителей в квантовых сетях (без квантовых повторителей) может быть восстановлена только при слабых предположениях об устойчивости узлов к атакам [79].
Более того, используя те же методы, можно добиться одновременной многоуровневой защиты от других атак по той же схеме до произвольно выбранного уровня качества обслуживания [80]. Топология квантовой сети обычно оказывает сильное влияние на достижимую безопасность, и, несмотря на теоретический и практический прогресс в построении квантовых сетей, даже без доверенных повторителей [33, 36, 54], проблема остается вычислительно (фактически, NP-) трудной в своей наиболее общей форме [81]. Методы борьбы со скрытыми каналами и вредоносными классическими блоками постобработки обсуждались в статье [82].
\subsubsection*{QKD оверлейные сети}
Если описанные ранее типы QKD сетей относятся к организации квантовых каналов, то тип оверлейной сети QKD относится к реализации публичных каналов. Основной целью оверлейной сети является достижение сети более высокой иерархии с целью обеспечения лучшего QoS и использования ресурсов сетей более низкого уровня. При этом оверлейная сеть стремится быть независимой от определенных путей от поставщиков интернет-услуг (ISP). Поиск альтернативных маршрутов, которые могут предоставить услугу с более высокой степенью качества, и быстрая перемаршрутизация в случае обнаружения прерывания или использования многопутевых соединений являются ключевыми особенностями подхода оверлейной сети. Использование многопутевых соединений является часто предлагаемым решением для повышения рабочей нагрузки сети за счет защиты от сбоев сети, балансировки нагрузки сети, реализации большой пропускной способности, выбора времени с малой задержкой и многого другого [83-86]. Исследования показали, что в 90\% пар точек присутствия между крупными интернет-провайдерами существует по крайней мере четыре пути с раздельными соединениями [87, 88].

Известно, что маршрутизация между сетевыми доменами с использованием внешних протоколов маршрутизации, таких как протокол Border Gateway Protocol (BGP), приводит к медленному реагированию и восстановлению после перебоев в сети. Из-за времени, необходимого для получения информации о прерываниях или перегрузках на сетевых линиях, и настройки таймера минимального интервала объявления маршрута BGP, который обычно составляет несколько минут, время, необходимое для получения согласованного представления о сети после прерывания связи, может достигать десятков минут, что является длительным периодом для сетевых приложений. BGP также распространяет только один маршрут, и определить альтернативный маршрут, необходимый узлам сети в различных ситуациях, довольно сложно [89].
Оверлейная сеть может помочь преодолеть эти проблемы путем создания сети с одноранговым подходом. Оверлейная сеть соединяет узлы в разных доменах и позволяет использовать альтернативные пути, инкапсулируя трафик в трафик в нижестоящей сети. Когда промежуточный узел на пути получает пакет, он распаковывает его, анализирует IP-адрес получателя, снова инкапсулирует пакет и передает его дальше узлам сети, которые могут находиться в других доменах. Проще говоря, это подход hop-by-hop, популярно применяемый в сетях QKD (рис. 4). Учитывая принцип инкапсуляции, узлы оверлейной сети независимо выполняют измерения состояния канала и могут быстрее реагировать на перегрузку канала, перенаправляя трафик на другие, менее загруженные каналы. Оверлейные сети могут предложить новые функциональные возможности, которые трудно реализовать в сетях нижнего уровня. Оверлейный QKD-подход привлекателен, поскольку его можно использовать для обхода "недоверенных" узлов и быстрой перемаршрутизации, когда доверие к узлам утрачено или требуется многопутевая связь [28, 33].
\subsubsection{Созданные сети QKD}
В этом разделе кратко обсуждаются некоторые ранее развернутые сети QKD. Поскольку большая часть литературы посвящена квантовой оптической инфраструктуре, основное внимание уделяется логической структуре сетей и топологии, решениям по хранению и управлению ключами, использованию ключей и производительности решения.
\subsubsection{Сеть QKD DARPA}
Первой в мире QKD-сетью стала сеть DARPA QKD Network, представленная в декабре 2002 года компанией BBN Technologies и Гарвардским и Бостонским университетами [14]. Первоначально сеть состояла из слабокогерентной пары передатчиков BB84 (Анна и Алиса), пары совместимых приемников (Борис и Боб) и одного оптического коммутатора 2x2, который мог соединить любого отправителя с любым приемником (рис. 5). Позже сеть была расширена за счет двух бесплатных космических QKD-каналов, а третий запланированный бесплатный космический канал от QinetiQ (Великобритания) не был описан ни в одной официальной документации проекта. QKD-сеть DARPA объединила два ранее описанных типа в гибридное решение. Сеть DARPA QKD заложила основу для дальнейшего развития доверенных ретрансляционных QKD-сетей, но также практически продемонстрировала недостатки коммутируемого типа QKD-сетей.
Два узла (Алиса и Боб) и коммутатор находились в BBN, а Анна и Борис - в Гарвардском и Бостонском университетах (BU), соответственно. BBN разработал свой собственный оптический коммутатор 2x2 и использовал его для соединения Анны, Алисы, Боба и Бориса. Этот переключатель был оптически пассивным и поэтому не нарушал квантовое состояние фотонов. Коммутатор был сконструирован путем модификации стандартного коммутатора телекоммуникационного оборудования. Он работает за счет перемещения отражающих элементов, которые изменяют внутренний световой путь для создания соединения BAR или CROSS. Он управляется по прямой линии от оптического технологического компьютера (OPC) Alice путем подачи импульса уровня TTL на контакт BAR или CROSS в течение 20 мс для переключения активированного положения. Согласно ссылке [29], время переключения составило 8 мс, а оптические потери - менее 1 дБ.

В ранее проведенном наборе экспериментов были представлены результаты, в которых измерялась деградация фазомодулированного QDK, вызванная оптическими коммутаторами [90]. Демонстрация передачи QKD и результаты вносимых потерь, которые были основным эффектом на пропускную способность QKD в трех различных типах оптических коммутаторов, дали следующее: Оптико-механический коммутатор 2x1 (потери 4,7 дБ),
Коммутатор 2x2 LiNbO3 (потери 5,4 дБ) и четырехпортовый MEMS-коммутатор (потери от 5,3 до 5,9 дБ).

5.1.1 Набор протоколов BBN. Учитывая, что сеть DARPA QKD была первой сетью QKD, для связи QKD по общедоступному каналу не могли использоваться предопределенные протоколы.
Поэтому BBN разработал свой собственный стек протоколов QKD на языке программирования C. Все сообщения были упакованы в IP-датаграммы для передачи управляющих сообщений через Интернет [91].
В таблице 1 представлен только список использованных технологий, в то время как заинтересованные читатели могут обратиться к ссылкам [29, 92]. Видно, что для различных этапов постобработки использовалось несколько технологий. Целью было минимизировать количество обмениваемых сообщений, чтобы ускорить скорость генерации ключей и уменьшить перегрузку, вызванную внезапной передачей большого количества пакетов по общедоступному каналу. На рисунке 6(a) показан основной формат датаграммы протокола BBN QKD. Каждая дейтаграмма содержит заголовок пакета с подробной информацией о разрешенной, надежной, упорядоченной передаче сообщения, обнаружении сбоев и так далее.
Дейтаграмма заполняется одним или несколькими сообщениями переменной длины, несущими детали, необходимые для описания команд или ответа на действие. Важно подчеркнуть, что эти дейтаграммы защищены механизмом безопасности IPSec в стандартном режиме.
Целью было создание безопасного туннеля между демонами квантового протокола, чтобы весь трафик по публичному каналу был зашифрован, аутентифицирован и проверен на целостность. Пример сообщений, которыми обменивается QKD-протокол BBN, показан на рисунке 6(b).

5.1.2 Управление ключами и их использование. В окончательном техническом отчете DARPA [29] приведены детали первоначально запланированных и использованных технологий. Интересно отметить, что авторы предполагали, что примитив согласования ключей Диффи-Хеллмана (DF) будет разрушен к 2015 году. Поскольку средняя ключевая скорость устройства QKD составляла 1 кбит/с, целью было представить QKD как новое решение для согласования ключей и интегрировать его с существующими протоколами управления ключами IPSec и IKE (Internet Key Exchange). Позже, когда ключевая скорость QKD-устройств увеличится, от IKE можно будет отказаться и принудительно использовать OTP, что приведет к созданию высокозащищенной сетевой архитектуры.
Авторы предложили использовать QKD-сеть только между чувствительными зонами, в которых конечные точки QKD использовались бы для дальнейшего распределения полученной информации по частным сетям (анклавам). Конечные точки QKD выполняли бы ту же функцию, что и пограничные маршрутизаторы в стандартных IP-сетях. Подключение конечного пользователя непосредственно к сети QKD никогда не планировалось. Основная идея заключалась в создании "резервуара" на обоих концах соответствующей QKD-связи, который постепенно заполнялся бы ключевым материалом, созданным с помощью QKD. Этот ключевой материал впоследствии будет использоваться с набором протоколов IPSec и применяться для шифрования туннелей виртуальных частных сетей (VPN).
Когда трафик принимался соответствующей конечной точкой на принимающем конце VPN-туннеля, он пересылался конечному пользователю, находящемуся в частной сети (анклаве).
Чтобы упростить процесс, но при этом использовать программное обеспечение различных платформ и производителей, конечная точка QKD была разделена на два разных компьютера. Первый компьютер, называемый Optical Process Control (OPC), использовал программное обеспечение LabView для управления соответствующим оборудованием QKD, а другой компьютер использовался для связи IPSec, маршрутизации, сетевых протоколов и протоколов QKD (просеивание, усиление конфиденциальности и т.д.). Эти два компьютера использовали локальные соединения Ethernet 100 Мбит/с. Для синхронного обмена данными использовался специализированный набор протоколов UDP 6, поставляемых BBN.

Другим важным вопросом была синхронизация между этими компьютерами и синхронизация между QKD и набором протоколов VPN. Точнее, необходимо было решить вопрос управления ключевым материалом, производимым оптическими устройствами. Процедура была следующей:

(1) Компьютер OPC поставляет блок необработанных символов фиксированного размера Qframe, переданных через оптические устройства. Он содержит указание на основания, которые были использованы для кодирования информации в фотонах. Эти блоки Qframe далее обрабатываются демоном QKD с помощью набора средств постобработки QKD (просеивание, QBER, усиление конфиденциальности или подобное), в результате чего на выходе получается Qblock.

(2) Q-блок - это блок общих битов фиксированного размера, каждый Q-блок имеет свой 16-битный идентификатор Q-блока (ID). Qблоки хранятся в резервуаре ключевого материала на обоих концах. Эти блоки хранятся непрерывно, независимо от потребления.

(3) Демон IKE использует идентификаторы Qblock для создания окончательного ключа, который затем используется IPSec, поскольку оба конца имеют одинаковый ключевой материал, хранящийся в их соответствующих резервуарах.


5.1.3 Набор протоколов IPSec. Защита протокола Интернета (IPSec) - это набор протоколов для обеспечения целостности, подлинности и конфиденциальности соединений через публичный Интернет. IPSec работает на уровне Интернет-протокола (IP) и является периметром между защищенными и незащищенными сетевыми интерфейсами, требуя уровень защиты. По умолчанию IPSec использует метод Internet Key Exchange (IKE) для автоматической передачи ключей. Основная концепция протокола IKE проста и происходит в две фазы. Первая фаза устанавливает аутентифицированный, двунаправленный, безопасный канал (Ассоциация безопасности Интернета) и протокол управления ключами (ISAKMP) SA путем обмена случайными полуключами для обмена ключами Диффи-Хеллмана. Аутентификация канала Фазы 1 осуществляется путем обмена сообщениями, зашифрованными сеансовым ключом. Случайные секретные биты Фазы 1, которые используются для установления ISAKMP, условно называются SKEYID. Эти биты считаются наиболее чувствительным моментом в точках IKE, поскольку они используются как частичный вход для создания ключей SA фазы 2 и для защиты трафика через заданную SA фазы 2. Поэтому замена этих битов время от времени, чтобы не нарушить безопасность системы, очень важна.
Вторая фаза использует биты SKEYID для согласования IPSec SAs между двумя шлюзами, которые передают трафик сообщений для определенного потока VPN-трафика. Каждая ассоциация безопасности IKE имеет максимальный срок службы, который ограничивает использование ключевого материала для ранее созданной ассоциации. Эти ограничения могут быть определены во времени (секунды) или в зашифрованных данных (килобайты) и хранятся в записи SPD для данной SA. После истечения срока действия необходимо согласовать новую SA со свежим ключевым материалом. Важно отметить, что не существует стандарта для использования OTP в IPSec. Поэтому были предложены различные решения, например, ссылки [14, 48, 97].

В сети DARPA QKD Network использовался IKE, поскольку на момент его разработки (январь 2002 года) IKE был наиболее широко распространенным протоколом согласования ключей в Интернете. Существуют два расширения, которые зависят от более позднего типа шифра:

- Расширение Quantum Perfect Forward Secrecy (QPFS), основанное на использовании техники QKD, для согласования секретных ключей, используемых в качестве семян для обычных симметричных шифров, таких как AES или 3DES. Поскольку безопасность этих симметричных шифров может быть скомпрометирована в последующие годы, целесообразно постоянное и автоматическое засеивание свежими битами QKD. В сети DARPA QKD ключи AES обновлялись примерно раз в минуту [15] путем отказа от фазы 1 переговоров и использования битов QKD в качестве прямого входа в фазу 2 IKE. Это решение повысило безопасность ассоциаций IPSec, поскольку ключи были получены из QKD вместо обмена ключами Диффи-Хеллмана (DH).

- Расширение, основанное на использовании техники QKD для согласования битов секретного ключа, используемого с шифром с одноразовым блокнотом (OTP). Это решение снижает скорость передачи данных до скорости передачи ключа QKD, поскольку ключевой материал в резервуаре для хранения заряжается только QKD.
Для реализации перечисленных расширений сетевая команда DARPA QKD расширила IKE Phase 2, добавив опцию в расширение QPFS, которая работает так же структурно, как и обычная PFS IKE Phase 2, но использует биты QKD, а не биты, полученные при обмене ключами DH. Решение было реализовано в Net BSD с помощью демона IKE "raccoon". Модификации включали механизмы политики для указания, когда и какое расширение должно использоваться, с возможностью задания значений (скорости повторного ключа, криптографических алгоритмов, ключей и т.д.) для каждого VPN-шлюза.

5.1.4 Маршрутизация в QKD-сети DARPA. Механизм маршрутизации необходим в ситуациях, когда два узла не имеют прямой QKD-связи "точка-точка" между собой и поэтому должны согласовать путь через доверенную сеть ретрансляторов.

Каждый узел имеет базу данных о полном состоянии каналов сети. Для каждого узла сети он хранит идентификатор узла и список соседних узлов. DARPA модифицировала известный протокол маршрутизации Open Shortest Path First (OSPF) [98] для использования специфической метрики сетей QKD [92]. Идея заключается в том, что каждый узел обменивается определенным количеством битов с соседним узлом, таким образом измеряя скорость обмена и общее количество обмениваемых битов (измеряя качество соединения) [99].
Качество соединения рассчитывается с использованием метрики соединения m и сохраняется в базе данных соответствующего узла: q > t, 100 + 1000q-t ,(1)m =,q < t, где q обозначает количество Qблоков, ожидаемых к наличию на соединении за один интервал обновления Link State Announcement (LSA), m - метрика соединения, а t - порог (значение по умолчанию 5) для минимального количества Qблоков, которое должно поддерживаться на активном соединении.
В дальнейшем, когда запрашивается маршрут между удаленными узлами, выбирается маршрут с наименьшей общей метрикой. Для поиска такого маршрута используется алгоритм Дейкстры. Для обновления записей в базах данных состояния соединений происходит периодический обмен сообщениями ROUT1LSA [29]. Эти сообщения содержат идентификатор узла отправляющего узла, идентификатор соседнего узла и соответствующую 32-битную метрику соединения. Сообщения ROUT1LSA обмениваются через каждый интервал обновления LSA, который является настраиваемым параметром, по умолчанию установленным на одну минуту. Каждый узел имеет индивидуальный таймер LSA, который не зависит от других узлов в сети.

Однако очевидно, что описанная модификация протокола OSPFv2 не учитывает параметры канала общего пользования. Метрика m, определенная в уравнении (1), рассматривает только количество доступного ключевого материала, не учитывая другие параметры, такие как загрузка канала или задержка [99].
Протоколы маршрутизации рассматриваются далее в разделе 6.

5.1.5 Протоколы повторителя ключей BBN для доверенных сетей. Как отмечалось выше, сеть QKD используется для преодоления ограничений, связанных с длиной канала QKD. Сеть DARPA QKD заложила основу для протокола повторителя ключей и представляет собой первую реализацию доверенной сети повторителя QKD. Здесь будет дано краткое описание этой реализации. Более подробную информацию можно найти в ссылке [29].
Когда два удаленных узла в сети QKD (т.е. узел A и узел D) хотят установить безопасную связь и между ними нет прямого соединения "точка-точка", им необходимо договориться о пути через сеть. Этот путь рассчитывается с помощью протокола маршрутизации, а узлы используют стратегию ключевого повторителя для создания ключевого материала. Узлом-источником всегда является узел с более высоким идентификатором узла. Узел-источник (узел A на рисунке 7(a)) посылает запросы на резервирование каждому узлу на пути (промежуточные узлы) и узлу назначения (узел D на рисунке 7(a)). Затем каждый узел на пути ведет переговоры со своим предшественником за Q-блок (рисунок 7(a)) и сообщает источнику об успешном завершении процесса переговоров. Если резервирование прошло успешно, источник запрашивает ключ у получателя и всех промежуточных узлов. Промежуточные узлы посылают XOR двух Q-блоков, установленных с соседними узлами, а узел назначения посылает XOR Q-блока предыдущего hop'а и новый случайный Q-блок n (рис. 7(b)). Этот Q-блок n является окончательным ключом, совместно используемым узлом-источником A и узлом назначения D.
Из вышесказанного очевидно, что метод ключевого повторителя BBN для установления ключевого материала требует времени и абсолютного доверия каждого узла, участвующего в коммуникации. Поэтому методы аутентификации имеют особое значение во всем процессе. Как уже говорилось, наиболее эффективным способом обхода скомпрометированных узлов является использование нескольких независимых путей.


5.1.6 Резюме. Сеть DARPA была первой сетью, продемонстрировавшей сетевое QKD.
Производительность, достигнутая этой сетью (максимальное расстояние 29 км через оптический коммутатор между Гарвардским и Бостонским университетами [91] и максимальная ключевая скорость 400 бит/с), считается основой для дальнейшего развертывания QKD. В системе использовались доверенный ретранслятор и коммутируемая QKD-сеть, демонстрируя преимущества и недостатки обоих методов. Краткая информация о QKD-сети DARPA приведена в таблице 2.

Однако сеть DARPA была закрыта в 2006 году, и с тех пор не сообщалось о других развертываниях в полевых условиях государственными учреждениями США. В 2017 году была объявлена Квантовая национальная инициатива, подкрепленная успешным запуском Китаем спутника "Micius" [100, 101]. В 2018 году стартап-компания Quantum Xchange объявила о планах создания первой коммерческой сети квантовой связи "Phio" в США [102, 103]. Используя собственные эксклюзивные доверенные узлы, Quantum Xchange обеспечивает безопасную передачу ключей на большие расстояния. Эта QKD-сеть работает в Вашингтоне, округ Колумбия, и Нью-Йорке, включая канал, соединяющий финансовые рынки на Уолл-стрит с центрами обработки данных в Нью-Джерси [104]. Для достижения удвоенной пропускной способности сети было объявлено о сотрудничестве с компанией Toshiba [105].

\subsubsection*{SECOQC QKD сеть}
В 2004 году интегрированный проект Европейской Комиссии (ЕК) FP6 SECOQC (Secure Communication based on Quantum Cryptography) объединил 41 научного и промышленного партнера из 11 стран Европейского Союза, России и Швейцарии. Основной целью проекта SECOQC было твердое определение практического применения технологий QKD и систематическое рассмотрение вопроса сетей QKD, включая их безопасность, дизайн и архитектуру, протоколы связи, реализацию, демонстрацию и тестовую эксплуатацию протоколов сети QKD.
Подход SECOQC заключался в определении QKD сетей как инфраструктуры, основанной на возможностях QKD "точка-точка", которые направлены на согласование ключей ITS и безопасную связь [106]. Учитывая, что первые результаты работы QKD-сети DARPA уже были доступны [92], SECOQC решил продолжить совершенствование QKD-сети типа доверенного ретранслятора. Для тестирования в Вене была развернута сеть QBB "Quantum Backbone" с городским расстоянием (6-85 км), состоящая из семи волоконно-оптических линий распределения ключей и одной короткой линии связи в свободном пространстве [57]. Пять узлов - SIE, BRT, GUD, FRM и ERD - были расположены на территории Siemens, а узел STP был размещен на ретрансляционной станции около Санкт-Пёльтена на линии связи Вена - Мюнхен, Германия.

5.2.1 Связи и узлы QBB. Как показано на рисунке 8, SECOQC объединил восемь связей, принадлежащих шести различным системам:

- Аттенюированный лазерный импульс - модифицированное, коммерчески доступное решение "Cerberis", реализованное швейцарской компанией idQuantique.

- Система одностороннего слабого когерентного импульса с ложными состояниями - реализована компанией Toshiba UK.

- Coherent-One-Way - реализована командой Н. Гизина в GAP, Женевский университет.

- Запутанные фотоны - предоставлены Австрийскими исследовательскими центрами (ARC) и Королевским технологическим институтом Киста KTH, Швеция.

- Непрерывные переменные - реализуется консорциумом CNRS-Thales-ULB из Франции/Бельгии.

- Free Space link - разработана Мюнхенским университетом Людвига Максимилиана, Германия.
Все эти системы должны были соответствовать следующим требованиям:

- Ключевая скорость более 1 кбит/с на 25 км волокна (потери 6 дБ при затухании волокна приблизительно 0,25 дБ/км в стандартном телекоммуникационном волокне).

- Автономная доставка ключа в течение более чем шести месяцев без участия человека.

- Время задержки составляет одну минуту для нового запуска. 

- Все используемое оборудование должно помещаться в стандартную 19-дюймовую телекоммуникационную стойку.

- Каждое QKD-устройство должно взаимодействовать со своим аналогом через стандартный интерфейс, предоставляемый модулем узла, управляющим командами управления общим доступом. 

Сеть SECOQC включала несколько различных решений QKD:

- Система QKD в свободном пространстве использовала протокол BB84 с ложными состояниями, что обеспечило безопасную скорость передачи ключей до 17 кбит/с на расстоянии 80 м между узлами ERD и FRM.

- Система idQuantique QKD реализовала протоколы BB84 и SARG04 с помощью коммерческой системы Cerberis, что позволило получить почти равное значение, предписанное критериями SECOQC (1 кбит/с).

- Компания Toshiba Research Europe Ltd (TREL) реализовала протокол BB84 "слабый когерентный импульс (WCP) состояние обмана плюс вакуумное состояние" и получила скорость передачи ключей 5,7 кбит/с по волокну длиной 25 км.

- В когерентной односторонней системе (COW), разработанной GAP (Группа прикладной физики Женевского университета), был реализован новый распределенный фазовый опорный протокол COW, который можно рассматривать как модификацию BB84 с фазовыми соотношениями между импульсами [28].

- QKD на основе запутанности (ENT), разработанная австрийско-шведским консорциумом, реализует BBM92 для запутанных состояний между узлами ERD и SIE по 16-км оптоволокну и обеспечивает надежную скорость передачи ключей более 2 кбит/с.

- Система Continuous-Variable (CV) была разработана в сотрудничестве между Институтом оптики Чарльза Фабрида, исследовательским центром THALES Research \& Technology France и Университетом Libre de Bruxelles. Их система достигла скорости распространения 8 кбит/с по стандартному оптическому волокну длиной 6,2 км (затухание волокна составляло приблизительно 2,8 дБ, в то время как длина эквивалентного волокна с потерями 0,2 дБ/км составила бы 14 км).

Узлы SECOQC следовали подходу DARPA, предусматривающему хранение ключевых материалов в резервуарах. Учитывая, что QKD-связи между узлами должны осуществляться по принципу "точка-точка", анод должен иметь выделенное QKD-устройство для каждого соединения с другими узлами. Ключевой материал из QKD-устройств сначала помещается в Pickup Stores. В этом временном хранилище ключевой материал хранится до тех пор, пока не будет подтверждено, что один и тот же материал находится в обоих QKD-узлах, образующих соответствующую QKD-связь. После успешного подтверждения существования одинакового ключевого материала на обоих концах, ключевой материал пересылается в общее хранилище. Для Q3P-связи (которая может содержать одну или несколько QKD-связей между двумя узлами) существует только одно общее хранилище, и ключевой материал в этом хранилище однозначно идентифицируется блоком ключевого материала. Когда запрашивается использование ключевого материала, ключи пересылаются в буферы ключей In или Out и используются для шифрования или дешифрования криптопроцессором. Организуя хранение ключей описанным образом, узлы QKD могут переносить колебания в потреблении ключей путем буферизации сгенерированного ключевого материала. Более подробную информацию об организации хранилища ключей можно найти в ссылках [28, 106].

5.2.2 Передача сообщений по принципу "хоп-хоп". Сеть SECOQC заложила основы подхода "хоп-хоп" к передаче сообщений в сети QKD. Этот способ известен как техника "Store \& Forward" и подразумевает использование отдельного ключа для каждого звена пути. Как показано на рисунке 9, каждый узел расшифровывает сообщение, проверяет аутентификационную метку и повторно шифрует сообщение, используя ключ, соответствующий соединению со следующим узлом. Эта процедура повторяется в каждом узле на пути, пока сообщение не достигнет пункта назначения [106].


5.2.3 Маршрутизация в сети SECOQC QKD. SECOQC предложил использовать структуру адресов IPv4 и географическое разделение сети QKD на несколько областей маршрутизации по следующим причинам:

- Сеть QKD - это частная сеть, которая может свободно использовать любую доступную область адресного пространства IPv4.

- Ожидать быстрого и широкого распространения сети QKD нецелесообразно. Поэтому адресного пространства IPv4 должно быть достаточно для адресации текущих и будущих узлов.

- Ограничения по расстоянию между линиями QKD не позволяют разделить сеть QKD между магистралью и произвольным числом автономных систем. Поэтому все узлы в сети должны рассматриваться одинаково.

- Отсутствие в настоящее время квантовых повторителей означает, что узлы QKD должны рассматриваться как точки доступа для приложений конечных пользователей, а не только как пересылочные узлы в сети.

Для удовлетворения требований к адресации и маршрутизации SECOQC предложил сформировать узел из следующих компонентов:

- Модули Q3P, отвечающие за связь на уровне канала с другими узлами.

- Модуль маршрутизации для сбора и ведения таблиц маршрутизации.

- Модуль пересылки для создания путей и принятия решений о пересылке.

- Другие модули для управления узлами, генерации случайных сеансовых ключей, мониторинга безопасности и так далее.

Функция модуля маршрутизации заключается в ведении локальной таблицы маршрутной информации и информировании других узлов об обновлениях в сети, чтобы они также могли обновить свои таблицы маршрутизации. Известные протоколы маршрутизации могут быть модифицированы и использованы в сети QKD, но при этом необходимо учитывать отсутствие квантового повторителя, что означает, что каждый узел сети должен быть готов принять трафик от соседнего узла и переслать его по наилучшему пути к запрашиваемому месту назначения (подробнее в разделе 6). В этом и заключается задача модуля пересылки. Модуль получает входящий пакет, проверяет значение TTL и контрольную сумму аутентификации, и в зависимости от результатов принимает решение о пересылке или отбрасывании пакета.

В SECOQC использовалась модифицированная версия протокола OSPFv2. Интересно, что OSPFv2 не поддерживает маршрутизацию QoS, которая необходима для обеспечения требуемого типа обслуживания. OSPFv2 был внедрен с целью ускорения процесса разработки [33]. В стандартном OSPF решение о пересылке принимается на основе адреса назначения и информации о кратчайшем пути в таблице маршрутизации. Однако, учитывая низкую ключевую скорость каналов QKD, при расчете наилучшего пути необходимо учитывать другие параметры.


Для вычисления структуры данных дерева кратчайших путей используется алгоритм Дейкстры. Каждый узел вычисляет уникальное дерево кратчайших путей и использует модифицированную версию OSPFv2 для составления таблицы маршрутизации. Основное отличие заключается в том, что модифицированный OSPFv2 рассчитывает несколько путей к каждому пункту назначения вместо одного кратчайшего пути. Множество путей необходимо для выполнения требований балансировки нагрузки и запасных путей. Поэтому каждый узел QKD вычисляет столько таблиц маршрутизации, сколько у него интерфейсов. OSPFv2 периодически отправляет сообщения LSA другим узлам сети с целью распространения информации о текущем состоянии сети.

Кроме того, каждый узел вычисляет расширенную таблицу маршрутизации, в которой перечислены все затраты по возрастающей величине до каждого другого узла. Эта таблица используется для объединения всех таблиц маршрутизации в одном месте. Структура таблицы похожа на стандартную таблицу маршрутизации, с той лишь разницей, что она имеет столько записей для каждого пункта назначения, сколько исходящих связей узла [33]. Теперь узел может найти несколько путей к узлу назначения, но ему также необходимо знать приблизительную нагрузку выбранных звеньев на пути. Если нагрузка на звено больше рассчитанного порога, то ищется следующее лучшее звено, и так далее. Третья таблица Load State Database вычисляется для хранения информации о приблизительной нагрузке каждого исходящего звена. Она используется для проверки того, достаточно ли ресурсов у канала для передачи сообщения [28]. Приблизительная нагрузка исходящего канала i в дискретное время t обозначается L i (t ) и рассчитывается с помощью фильтра низких частот по уравнению (2):L i (t ) = 1 -(2)- L i (t - 1) + - l i (t ), где l i (t ) - мгновенная нагрузка исходящего канала i, а w - постоянная фильтра. Мгновенная нагрузка l i (t ) рассчитывается как отношение количества переданных битов за предыдущую единицу времени. Более подробную информацию о модулях маршрутизации и пересылки можно найти в ссылках [33] и [28].


5.2.4 Резюме. Следует подчеркнуть, что разработка приложений не входила в задачи проекта SECOQC. SECOQC, однако, провел несколько экспериментов для проверки созданных решений. Во время конференции SECOQC QKD 8-10 октября 2008 года была проведена демонстрация телефонной связи и видеоконференций. Между узлами был создан VPN-туннель и использовалось шифрование AES. Ключ AES обновлялся каждые 20 секунд, а в определенные моменты шифрование AES заменялось OTP [106]. Основной целью было протестировать механизмы маршрутизации, измерить потребление и генерацию ключевого материала, а также продемонстрировать основные механизмы функционирования сети SECOQC. Стоит отметить, что в SECOQC исследовалось установление QKD-соединения с конечным пользователем [28].

Сеть SECOQC заложила основу для сетевого подхода hop-by-hop, который значительно упрощает представление о реализации решений по маршрутизации. Подход hop-by-hop позволяет каждому узлу решать, по какому дальнейшему пути направить сообщение, что обеспечивает большую гибкость при реализации протоколов маршрутизации. Однако, подход BBN key-repeater, описанный в разделе 5.1.5, требует глобального, актуального представления сети перед созданием и резервированием ресурсов на пути, что может быть требовательным из-за динамического потребления скорости генерации ключей. Сеть SECOQC также продемонстрировала совместимость между различными производителями оборудования и показала, что сеть QKD может достигать дальности почти 100 км (максимальная длина канала связи составила 82 км между узлами BREIT и St. Pölten) [37]. Краткая информация о сети SECOQC приведена в Таблице 3.

Интерес к квантовой криптографии в ЕС сопровождался проектами, финансируемыми в рамках программ Quantum Technologies Flagship, Quant Era, COST и EuraMet [107-112]. В 2019 году был объявлен проект ЕС Horizont2020 OPENQKD с консорциумом из 38 партнеров из промышленности и научных кругов [113]. Цель OPENQKD - заложить основы будущей европейской квантовой инфраструктуры и конвергенции квантовых технологий с практическими телекоммуникационными системами в Европе в течение трех лет.
\subsubsection*{Токийская сеть UQCC QKD}
Через два года после SECOQC девять организаций из Японии и Европейского Союза приняли участие в сети испытательного стенда QKD в Токио UQCC ("Japan Giga Bit Network 2 plus" - JGN2plus). Сеть состояла из частей открытой тестовой сети Национального института информационных и коммуникационных технологий Японии (NICT) под названием "Japan Giga Bit Network 2 plus" (JGN2plus) [38].
Токийская сеть QKD содержала четыре точки доступа, Хакусан, Хонго, Коганей и Отемачи, и шесть узлов, соединенных коммерческими оптическими волокнами, установленными в этих точках доступа (Рисунок 10).
Поскольку половина выбранных волокон были воздушными, на линиях связи возникали большие потери. На линии связи между узлами Когеней и Отемачи уровень потерь составил около 0,3 дБ/км, в то время как на других линиях этот показатель достигал даже 0,5 дБ/км.

Аналогично SECOQC, участники проекта реализовали определенные сетевые связи, распределенные следующим образом:

- 24-километровый канал связи между узлами Otemachi и Hakusan был предоставлен Mitsubishi Electric Corporation и NTT Company. Они реализовали протокол BB84 с максимальной скоростью передачи ключей 2 кбит/с и QBER 4,5\%.

- 45-километровый канал связи между узлами Коганей и Отемачи был предоставлен компанией NEC, реализующей протокол BB84 с ложным состоянием и сверхпроводящим однофотонным детектором (SSPD) компании NICT. Максимальная скорость передачи ключей составила 81,7 кбит/с при среднем QBER 2,7\%.

- NTT использовала DPS-QKD на самом длинном канале в сети, который составлял 90 км между узлами Koganei-1 и Koganei-2. Они также использовали детектор SSPD и достигли максимальной скорости передачи ключей 15 кбит/с при среднем QBER 2,3\% [114].

- Три организации из Австрии, включая AIT, Институт квантовой оптики и квантовой информации (IQOQI) и Венский университет, сформировали единую команду под названием "Вся Вена". Они представили свое устройство SECOQC QKD. Это устройство было основано на запутывании протокола QKD BBM92, который был размещен между узлами Koganei-2 и Koganei-3 с максимальной скоростью передачи ключей 0,25 кбит/с.

- Toshiba Research Europe Ltd. продемонстрировала свою систему BB84 с самодифференцирующимися лавинными фотодиодами (SPAPD) между узлами Koganei-2 и Otemachi-2 на 45-км линии связи. Максимальная ключевая скорость составила рекордные 304 кбит/с при среднем QBER 3,8\%. Это была самая высокая устойчивая скорость передачи ключей QKD на сегодняшний день.

- Наконец, 13-километровый канал связи между узлами Отемачи и Хонго был предоставлен компанией idQuantique из Швейцарии, использующей протокол SARG04 из своего коммерческого решения Cerberis. Максимальная скорость передачи ключей составляла 1,5 кбит/с.

Токийская сеть UQCC QKD следовала аналогичной трехслойной сетевой архитектуре, основанной на подходе доверенного ретранслятора, как это было реализовано в проекте SECOQC. Основным отличием было использование сервера управления ключами (KMS) для централизованного управления. Токийская сеть QKD пыталась протестировать сценарий правительственной сети, которая часто имеет центрального диспетчера или центральный сервер данных. KMS была установлена в узлах Koganei-1, Koganei-2, Otemachi-1 и Otemachi-2. На всех узлах были установлены агенты управления ключами, основной задачей которых было сохранение ключевого материала и хранение статистических данных по каналу связи, таких как QBER и скорость генерации ключей. Позже эти статистические данные передавались в KMS, которая координировала работу всех звеньев сети [38].

5.3.1 Резюме. В октябре 2010 года была проведена демонстрация безопасной телеконференции, обнаружения подслушивания и перенаправления каналов QKD в токийской сети UQCC QKD. 
Было установлено 2-VPN шифрование с OTP между узлами Otemachi-2 и Koganei-1. Для демонстрации алгоритма маршрутизации были использованы два маршрута, когда каналы были атакованы подслушивающим устройством. KMS обнаружила атаки из-за увеличения QBER и перенаправила связь через запасной канал.

Как отмечено в таблице 4, токийская сеть QKD показала, что технология QKD может достигать скорости в несколько сотен бит в секунду. Сеть также подтвердила возможности связи на расстоянии QKD-линии, достигнув рекордной связи в 90 км между узлами Koganei-1 и Koganei-2 [58]. Однако отличительной особенностью этой сети является внедрение иерархического подхода к организации QKD-сетей. Серверы управления ключами реализуют уровень управления и имеют полное представление о состоянии сети QKD в своем домене. Организация таким образом приблизила сеть QKD к перспективе SDN, обсуждаемой в разделе 7.

\subsubsection*{QKD-сети в Китае}
Китай строит сети QKD в национальном масштабе. Эти усилия начались со строительства тестовых городских сетей QKD в Хэфэй, где в 2009 и 2010 годах были построены трехузловая сеть [115] и пятиузловая сеть [71] соответственно. О других попытках построить оптоволоконные сети QKD сообщалось в ссылках [116-119], также формируется спутниковая сеть QKD [24]. В данном разделе представлен обзор этих разработок на примере некоторых недавно построенных оптоволоконных сетей.

5.4.1 Магистральная сеть QKD Пекин-Шанхай. В сентябре 2017 года начала работу магистральная QKD-сеть Пекин-Шанхай протяженностью 2 000 км [120]. На сегодняшний день это самая длинная сеть QKD в мире. Проект возглавляет Университет науки и технологий Китая (USTC). Среди других участников - China Cable Television Network Co, Шаньдунская академия информационных и коммуникационных технологий, Промышленный и коммерческий банк Китая (ICBC), Биржа финансовой информации Синьхуа и другие. Строительство сети было завершено в сентябре 2016 года, и перед началом эксплуатации она тестировалась в течение одного года.

Магистральная сеть состоит из 32 физических узлов, линейно соединенных QKD-связями (Рисунок 11). Среди этих узлов Пекин, Цзинань, Фули, Хэфэй, Нанкин и Шанхай являются точками доступа, а остальные - доверенными узлами-ретрансляторами. Магистральная сеть имеет в общей сложности 135 связей.
Между соседними узлами проложено от двух до восьми многократных QKD-каналов. Для экономии ресурсов волокна в сети используется технология квантового мультиплексирования с разделением по длине волны, которая объединяет четыре квантовых канала в одном волокне. Сеть арендует темное волокно, развернутое компанией China Cable Television Network Co. Расстояние между соседними узлами вдоль магистральной линии варьируется от 34 км до
89 км, при этом потери в волокне варьируются от 7,26 дБ до 22,27 дБ.
В магистральной сети используются устройства QKD, предоставленные компанией QuantumCTek Co. В устройствах реализован протокол BB84, основанный на кодировании поляризации и обманном состоянии. Некоторые из устройств интегрируют технику обнаружения одиночных фотонов с преобразованием вверх и таким образом достигают скорости обнаружения одиночных фотонов 25\%.

Магистральная сеть предназначена для функционирования в качестве канала с высокой пропускной способностью, по которому квантовые ключи передаются между столичными и QKD-сетями, расположенными в разных городах. До настоящего времени магистральная сеть была подключена к городским QKD-сетям, уже созданным в Пекине, Шанхае, Цзяне и Хэфэе. Таким образом, была сформирована широкомасштабная сеть QKD, которая предоставляет конечным пользователям, включая банки, правительственные учреждения и крупные предприятия, универсальные услуги безопасности [121].

В ноябре 2018 года было завершено расширение магистральной сети Пекин-Шанхай путем создания магистрального канала QKD между Уханем и Хэфэем. Цель заключалась в подключении городской сети QKD в Ухане к магистральной сети. Магистральная линия Ухань-Хэфэй управляется компанией CAS Quantum Network Co. В долгосрочной перспективе магистральная сеть будет расширена, чтобы охватить более обширную территорию Китая.

5.4.2 Цзинаньская государственная частная сеть QKD. Цзинаньская государственная частная сеть QKD начала строиться в апреле 2017 года и была завершена в августе 2017 года. Сеть охватывает территорию города площадью 8 000 км 2 и состоит из 32 узлов, включая узел централизованной станции управления, восемь узлов доверенных ретрансляторов и 23 узла конечных пользователей. Компания QuantumCTek Co., Ltd предоставляет системы QKD и решения по проектированию сети, а филиал China Union Shandong предоставляет оптоволоконные ресурсы [122, 123]. Всего в сети 33 линии QKD (Рисунок 12). Длина линий варьируется от 1,7 км до 64,7 км, а потери в волокне - от 1,48 дБ до 25,2 дБ. Квантовые сигналы передаются по темному волокну, предоставленному филиалом China Union Shandong.
В сети развернуты системы QKD, реализующие основанный на поляризационном кодировании протокол BB84 с ложным состоянием. Все системы предоставлены компанией QuantumCTek.

В сети используется продаваемый, самостоятельно построенный канал обслуживания, который обеспечивает безопасную передачу данных и минимальную пропускную способность 512 Мбит/с. Частная сеть Цзинань объединяет телефонную связь VoIP и услуги видеоконференций, снабженные квантовыми ключами. Доступ к услугам безопасности осуществляется через ретрансляторы, реализующие протокол IPSec VPN, снабженный квантовыми ключами.
Сеть поддерживает OTP и несколько других алгоритмов симметричного шифрования. Типичная частота обновления ключей в алгоритмах симметричного шифрования составляет один раз в секунду [124].

5.4.3 Сеть QKD столичного района Ухань. Сеть QKD в столичном округе Ухань была построена в 2017 году с января по декабрь. Сеть состоит из одного командного центра, одной централизованной станции управления, 10 доверенных узлов ретрансляторов и 60 узлов конечных пользователей (рис. 13).

Всего в сети 74 QKD-связи. Централизованная станция управления и два центральных доверенных узла ретранслятора соединены между собой, образуя сетевое кольцо. Их соединения имеют двухзвенную структуру (между каждым другим узлом устанавливается два QKD-канала). Длина самого длинного QKD-канала составила 16,5 км, а потери оптического канала, понесенные волокном, WDM-устройством и оптическим коммутатором, составили максимум 14,6 дБ на протяжении 6,7 км. Такие высокие потери в основном являются результатом сложной городской волоконно-оптической среды. Ключевая скорость каналов QKD варьируется от 2,8 кбит/с до 141 кбит/с. Системы QKD предоставлены компанией QuantumCTek [125].

Одной из особенностей сети является классическая квантовая технология WDM [126], которая объединяет классический служебный сигнал, квантовый сигнал и классический сигнал постобработки QKD в одном волокне. Доступ к службе безопасности осуществляется через ретранслятор шифрования, который реализует протокол IPSec VPN, снабженный квантовыми ключами. Типичная частота обновления ключей для симметричных алгоритмов шифрования составляет один раз в пять секунд.

5.4.4 Широкозональная QKD-сеть Хэфэй, Чаоху и Уху. О широкозональной сети QKD, соединяющей города Хэфэй, Чаоху и Вуху (HCW) в Китае, было сообщено в 2014 году в статье [119].
Вся сеть HCW QKD, на которую имеется полное техническое описание, работала более 5 000 часов с 21 декабря 2011 года по 19 июля 2012 года и была установлена в телекоммуникационной сети провинции Аньхой компании China Mobile Ltd. с зоной покрытия более 150 км.

В этой сети использовалось тринадцать QKD-устройств на девяти узлах [119]. Широкозональная сеть HCW состоит из двух городских сетей: сети QKD Хэфэй, которая имеет пять узлов [116, 117], и сети QKD Уху [118], которая имеет три узла. Эти две сети были соединены междугородним каналом QKD, который объединил столичные сети QKD Хэфэй и Уху через доверенный узел ретранслятора в филиале Чаоху компании China Mobile Ltd. [119]. [119].

В сети были развернуты системы QKD, реализующие основанный на фазовом кодировании протокол BB84 с ложным состоянием. Максимальная ключевая скорость в сети HCW QKD составляла 16,15 кбит/с между узлами Западного кампуса и Северного кампуса по линии связи длиной 3,1 км, подключенной к оптическому коммутатору, расположенному в библиотеке кампуса [119].

5.4.5 Сетевые стратегии. Магистральная сеть QKD Пекин-Шанхай и несколько других сетей QKD в городских районах используют ряд сетевых стратегий для повышения производительности и надежности сети:

- Некоторые городские сети используют кольцевую топологию при проектировании сети для повышения ее устойчивости к авариям.

- Магистральные соединения используют стратегию построения сети с несколькими параллельными каналами для повышения пропускной способности и стабильности сети.

- На некоторых основных узлах, таких как централизованные станции управления, установлены резервные устройства, которые снижают вероятность прерывания обслуживания системы, вызванного единственной точкой отказа.

5.4.6 Маршрутизация ключей в сети. Маршрутизация ключей в магистральной сети QKD Пекин-Шанхай, государственной частной сети QKD Цзинань и городской сети QKD Ухань использует архитектуру клиент-сервер для максимального использования канала и предоставления квантовых ключей конечным пользователям по требованию.
Централизованная станция управления каждой сети реализует сервер маршрутизации ключей, который отвечает за управление таблицей маршрутизации для каждого узла сети. На основе информации, собранной в сети (состояние работы QKD-соединений, оставшийся объем хранилища ключей и другая информация), таблица маршрутизации каждого узла периодически обновляется. Обновленные таблицы маршрутизации обмениваются с другими узлами сети, чтобы предоставить информацию о подходящих путях до следующего обновления [127].

Сервер маршрутизации ключей поддерживает несколько стратегий очередей, адаптированных к различным топологическим структурам сети. В случае аварийного сбоя, такого как отказ машины управления ключами или недоступность ключа, устройство узла активно сообщает об этом событии серверу маршрутизации.
Затем сервер пересчитывает таблицы маршрутизации для затронутых узлов. В магистральной сети Пекин-Шанхай маршрутизация ключей управляется путем разделения сети на несколько подсетей. Каждая подсеть принимает вышеупомянутую структуру клиент-сервер с сервером маршрутизации ключей, расположенным на узлах доступа [127, 128].

5.4.7 Резюме. С 2000-километровой линией связи, соединяющей Шанхай и Пекин, и городскими сетями в Хэфэй и Цзинань, Китай в настоящее время лидирует в гонке QKD с точки зрения практических разработок [129]. В описанных ими методологиях можно увидеть уникальные подходы к реализации существующих и доступных технологий в этих сетях QKD. Однако, в дополнение к использованию дискретных протоколов QKD, которые гарантируют высокую производительность, но требуют дорогостоящих однофотонных детекторов, также были представлены эксперименты, основанные на непрерывно-переменных и независимых от устройства измерения (MDI) QKD. Для CV-QKD были достигнуты результаты 5,77 кбит/с на расстоянии 50 км [73, 130]. Эксперименты с MDI-QKD привели к более высоким скоростям (до затухания канала): 98,2 кбит/с на расстоянии 49,1 км и до 1 Мбит/с на расстоянии в десятки километров [131]. Более того, системы QKD на основе MDI могут эффективно работать не только в симметричных каналах с одинаковыми потерями, но даже в каналах с асимметричными потерями [132]. Методы оптимизации [133] могут увеличить безопасное расстояние передачи в таких реализациях MDI-QKD более чем на 20-50 км в стандартном телекоммуникационном волокне. Краткая сводка описанных выше сетей с общедоступными ссылками приведена в таблице 5.

Не рассматривая здесь этот вопрос по соображениям компактности, важно, однако, отметить, что Китай лидирует в области квантовых технологий, ориентированных на космос. В 2017 году был запущен 640-килограммовый спутник "Micius" [134]. В ходе 273-секундного пролета спутника и с использованием 1-метрового телескопа на земле ожидалась скорость просеянного ключа от примерно 12 кбит/с на расстоянии 645 км до 1 кбит/с на расстоянии 1200 км [135]. После постобработки было получено 1,1 кбит/с для защищенного ключа.

\subsubsection{Качество обслуживания в сетях QKD}

Сходство между QKD и технологиями мобильных сетей Ad Hoc. Специфические проблемы и ограничения QKD, описанные выше, создают значительные трудности при проектировании сетей QKD. Однако анализ характеристик QKD-сетей показал сходство [136] с мобильными ад-хок сетями (MANET) и транспортными ад-хок сетями (VANET) [137-139].

Основные характеристики технологии QKD с простой точки зрения можно перечислить:

- QKD каналы, такие как описанные выше, всегда реализуются в режиме "точка-точка" и могут быть приблизительно охарактеризованы двумя особенностями: ограниченным расстоянием и скоростью передачи ключа (экспоненциально), уменьшающейся с расстоянием. Связи могут стать недоступными, когда ключевого материала недостаточно или когда канал общего пользования перегружен. Это похоже на каналы Wi-Fi, которые ограничены по дальности и скорость связи которых зависит от расстояния пользователя от антенны передатчика.

- Одной из основных особенностей существующих сетей QKD является отсутствие квантового ретранслятора (раздел 3.2), и поэтому связь обычно осуществляется по принципу "хоп-хоп".

В сетях MANET связь осуществляется по принципу "хоп-хоп", а мобильные узлы обычно питаются от энергосберегающих решений, таких как батареи. Узлы соединяются между собой самоорганизующимся, децентрализованным образом, без какого-либо органа, отвечающего за контроль и управление сетью. Основным недостатком сетей MANET является непредсказуемая мобильность узлов, которая часто может привести к нестабильным путям маршрутизации [114]. Количество заряда батареи и мобильность узлов MANET можно легко связать с количеством ключевого материала в хранилище ключей QKD. Ограниченный радиус действия беспроводных каналов связи во многом схож с ограничениями на длину QKD-связи. Отсутствие специализированной сетевой инфраструктуры (например, маршрутизаторов) - еще одно сходство между этими двумя технологиями. Слабая мобильность узлов QKD, однако, делает ее похожей на технологию VANET, в которой связь происходит по заранее определенному пути.

Хотя на первый взгляд сети MANET и сети QKD не имеют ничего общего, простой анализ особенностей этих сетей показывает их сходство. Однако, что четко отличает эти две сети, так это их назначение. Сети MANET предназначены для быстрой и простой связи в ситуациях, когда нет возможности использовать уже существующую инфраструктуру (например, поисково-спасательные операции во время стихийных бедствий или в зонах военных действий). В отличие от этого, основной целью QKD является обеспечение связи ИТС. Это может оказать существенное влияние при выборе сетевых решений, поскольку решение, необходимое в одной ситуации, может оказаться непригодным в другой. Например, рассмотрим решения маршрутизации, основанные на сетевом наводнении. Сети QKD опираются на предположение, что все узлы являются доверенными, когда связь осуществляется по принципу hop-by-hop или с повторением ключа [32, 39], и, строго следуя этому предположению, подслушивающее устройство ограничивается атакой только на каналы QKD. Из-за природы QKD подслушивающее устройство не может получить никакой информации о ключе, передаваемом по каналу связи, но ему может быть отказано в обслуживании, чтобы отключить связь. Хотя были получены результаты по объединению нескольких путей для создания безопасного ключевого материала [27, 82], считается, что объем маршрутной информации, передаваемой узлам, должен быть сведен к минимуму. Для предотвращения атаки типа "отказ в обслуживании" ни один узел (кроме источника и получателя) в сети не должен знать запрос на маршрутизацию. Поэтому количество широковещательных пакетов должно быть сведено к минимуму. Более того, учитывая основную цель QKD, которая заключается в обеспечении связи ITS, пакеты маршрутизации должны быть либо аутентифицированы и зашифрованы, либо, по крайней мере, аутентифицированы [140]. Это означает, что количество маршрутных пакетов в сети должно быть минимизировано (накладные расходы на маршрутизацию) относительно материала, который должен быть сохранен для защиты данных, что является основной целью безопасной связи. Из этого следует, что протоколы, основанные на переполнении, не являются предпочтительными в сетях QKD.
\subsubsection*{Протоколы маршрутизации}
В ранее развернутых сетях QKD особое внимание уделялось квантовым каналам. Общественный канал, однако, в основном игнорировался и предполагалось, что он каким-то образом достижим без каких-либо трудностей. Приоритетность сетевого трафика и сигнальные протоколы игнорировались, и решения в существующих обычных сетях были, соответственно, модифицированы для нужд QKD-сетей. Первое такое решение, основанное на модификации известного протокола маршрутизации Open Shortest Path First (OSPF) [98], было реализовано в сети DARPA BBN QKD, построенной в 2004 году в США [92]. Вместо использования метрики маршрутизации hop-count, был использован модифицированный протокол OSPFv2 для определения качества соединения в зависимости от количества ключевого материала в хранилище ключей. Как обсуждалось в разделе 5.1.4, модифицированная версия OSPFv2 не учитывает статус канала общего пользования [99].

Аналогичный подход был предложен в статье [141], где автор предложил использовать незашифрованную и неаутентифицированную связь для распространения пакетов маршрутизации OSPFv2. Очевидно, что такая сеть является легкой добычей для подслушивающего устройства, имеющего в своем распоряжении неограниченные ресурсы, особенно в плане пассивного подслушивания [76]. Поскольку описанное решение основано на использовании ключевого материала в хранилище ключей в качестве метрики маршрутизации, оно не может обеспечить эффективную маршрутизацию из-за отсутствия информации о состоянии открытого канала.

В сети SECOQC была внедрена еще одна модифицированная версия протокола OSPFv2 [28,
33]. Оно было основано на локальной политике балансировки нагрузки, рассчитываемой как отношение количества переданных битов за определенный период времени. Как обсуждалось в разделе 5.2.3, подобное решение не учитывает доступное количество ключевого материала, что означает, что алгоритм может выбрать путь с недостаточным количеством ключевого материала для передачи данных.

При создании китайской сети HCW QKD был применен подход к резервированию квантовых ключей (QKDRA), основанный на модели IntServ [142, 143]. Для поиска пути от источника к узлу назначения используется протокол OSPFv2. После определения пути узел источника отправляет запрос на резервирование ключа всем узлам на пути. После получения запроса промежуточный узел отвечает сообщением о результате резервирования ключа. Наконец, узел назначения определяет возможность установления соединения. Поскольку OSPFv2 фокусируется на поиске кратчайшего пути, отсюда и название, решения, представленные в ссылках [142, 143], находят кратчайший путь между источником и пунктом назначения и резервируют достаточное количество ключевого материала на выбранном пути. Обратите внимание, что этот путь может быть не оптимальным. Точнее, путь является кратчайшим, но он может быть неоптимальным с точки зрения QoS. Известно, что маршрутизация с минимальным количеством hop (кратчайшая) обычно находит маршруты со значительно меньшей пропускной способностью, чем наилучший из доступных [144], поскольку она не учитывает другие параметры соединения. OSPFv2 в своем первоначальном виде не учитывает ограничения QoS; поэтому он не может гарантировать, что трафик на выбранном пути будет адекватно обслуживаться. Резервирование ресурсов на квантовом канале, в этом случае, не дает выигрыша, так как путь для публичного канала может оказаться неподходящим. Однако даже расширенная версия OSPFv2, включающая ограничения QoS [145], может оказаться неоптимальной для сетей QKD. Реализация OSPFv2 таким образом может найти путь, который имеет наилучшие характеристики публичного канала, но не учитывает параметры квантового канала.
Янг предложил использовать алгоритм Дейкстры для определения нескольких путей, но без учета состояния канала общего пользования [146]. Идея состоит в том, чтобы использовать пороговые значения для исключения связей, которые имеют меньшее количество ключевого материала, и периодически передавать подробности маршрутизации, такие как количество доступного ключевого материала.

Влияние состояний открытого канала на ключевую скорость можно найти в ссылке [147]. Это исследование показывает, что канал общего пользования не должен исключаться из расчетов маршрута, поскольку производительность канала общего пользования влияет на квантовый канал и наоборот. Поэтому вводятся новые метрики для уникального описания состояния публичного и квантового каналов, а также всей QKD-связи [147, 148]. С целью минимизации потребления ключей следует избегать флудинга сети, поэтому была представлена одноуровневая организация сети и протокол маршрутизации Greedy Perimeter Stateless Routing Protocol for QKD networks (GPSRQ) [136]. Протокол маршрутизации GPSRQ использует распределенную географию и реактивную маршрутизацию для достижения высокого уровня масштабируемости. Он оснащен механизмом кэширования и обнаружения возвращающихся петель, что позволяет осуществлять пересылку при минимизации потребления ключевого материала. Однако применение GPSRQ ограничено только планарной топологией, поскольку географическая маршрутизация в сетях с непланарной топологией не может быстро определить кратчайший путь, что приводит к ненужной пересылке и повышенному расходу дефицитного ключевого материала.

Маршрутизация в сетях QKD зависит в первую очередь от архитектуры организации сети (иерархическая или распределенная архитектура, оверлейная или одностековая сеть, сеть hop-by-hop или сеть ключевых повторителей). В отличие от обычных сетей, решения по маршрутизации в QKD-сетях должны учитывать оба канала QKD-связи. Исходя из требования минимизации потребления ключей, необходимо уменьшить количество маршрутных пакетов, которые должны быть зашифрованы и аутентифицированы или, по крайней мере, аутентифицированы, чтобы избежать активных и пассивных подслушивающих атак QKD-сети [28, 140].

Учитывая усилия по распространению сети QKD на мегаполисы, что предполагает значительное количество сетевых узлов, в ранее реализованных сетях рассматривалась иерархическая организация [42, 125, 149-151]. Этот подход, основанный на уровне управления ключами, сходится к парадигме сети на основе программного обеспечения и более подробно рассматривается в разделе 7.

\subsubsection{QKD разработка сетевых технологий}
Более эволюционная стратегия применения QKD в транспортных сетях заключается в использовании преимуществ последних разработок в области сетевых технологий, а точнее, в области управления сетями.
Программно-определяемые сети (SDN) [152, 153] позволяют разделить плоскости контроля (управления) и данных (пересылки). С момента своего появления в 2008 году эта технология приобрела большую популярность как в академической, так и в промышленной сферах. SDN позволяет быстрее интегрировать новые технологии и услуги, обеспечивая при этом централизованное управление и оптимизацию на основе принципов программируемости и конфигурируемости сети. Хотя подход к SDN изменился по сравнению с программируемостью устройств на основе OpenFlow в сторону открытых и стандартных интерфейсов, эта трансформация помогла сетевым операторам все больше внедрять SDN в свои системы, чтобы сократить время выхода на рынок и уменьшить привязку к поставщикам.

Сеть SDN концептуально состоит из трех уровней. Уровень контроля и управления знает состояние всей сети и может оптимизировать ее поведение с помощью централизованной структуры, известной как контроллер SDN. Контроллер определяет возможности устройств, установленных на инфраструктурном уровне, с помощью набора стандартных механизмов (southbound interface). Он также знает требования различных приложений, работающих в сети, через стандартные интерфейсы (северный интерфейс). Его роль заключается в оптимизации ресурсов и предоставлении средств для выполнения задач устройствами и сервисами. Система QKD, установленная в инфраструктуре, может экспортировать свои требования контроллеру, чтобы он мог создать определенный путь с необходимыми оптическими характеристиками (например, максимально допустимый шум, затухание и т.д.) для соединения излучателя с приемником (одно или многоходовый путь) и удовлетворить требования приложения. Это позволяет создать беспрецедентное средство для создания полностью интегрированной классической/квантовой сети и действительно QKD-устройств с нулевой конфигурацией, которые могут быть напрямую подключены к стандартной телекоммуникационной сети.

До появления этой технологии демонстрации требовали либо создания отдельной сети ad-hoc только для квантового канала (т.е., как правило, сети темных волокон), либо специальных модификаций сети для каждого канала [154, 155]. Это очень дорогие и совершенно ортогональные развертывания для обычной телекоммуникационной деятельности, в которой устройства, как ожидается, будут работать "из коробки" и делить волокно со многими другими обычными каналами связи. Для того чтобы QKD стало стандартом, критически важно, чтобы системы QKD следовали тенденциям и архитектурам, используемым в сегменте транспортных сетей.

Другие проекты и демонстрации показали первые шаги к автоматизации сетей QKD.
В ссылках [156, 157] авторы реализовали механизм автоматизации переключения квантового канала между передатчиком и двумя смоделированными приемниками с помощью оптических кросс-коннект-коммутаторов с поддержкой OpenFlow. В этом смысле и несмотря на то, что в данном случае средством обеспечения является контроллер программно-определяемой оптической сети (SDON), исследование больше сосредоточено на применении безопасной миграции виртуальных машин в сценарии распределенного центра обработки данных.
Самый передовой вклад в программно-определяемые сети QKD был представлен в ссылках [158-160] (Рисунок 14(b)). Были подключены три производственных объекта в сети Telefonica в Испании. Предложенная архитектура и демонстрации были направлены на демонстрацию технологической зрелости систем QKD для интеграции в производственные сети. Системы CV-QKD, использованные для испытаний, были реализованы таким образом, чтобы ими можно было управлять и оптимизировать с помощью программных процессов, и были достаточно надежными, чтобы сосуществовать с традиционными каналами связи. В программное обеспечение была интегрирована первая версия интерфейса SDN, определенного Группой отраслевых спецификаций (ISG) для QKD в Европейском институте телекоммуникационных стандартов (ETSI). С помощью этого интерфейса системы QKD и процессы доставки ключей централизованно управляются контроллером SDN, что позволяет динамически создавать квантовые каналы (через оптические коммутаторы), создавать многоходовые ассоциации и определять запросы на ключи от внешних приложений. Эта установка также была разработана таким образом, чтобы любой канал управления и передачи данных мог интегрировать ключи, полученные с помощью QKD, для защиты коммуникаций, связанных либо с сетью QKD, либо с традиционными телекоммуникационными услугами, работающими в производственной сети.

QKD также можно рассматривать как дополнительный уровень безопасности для транспортных сетей. Интеграция QKD в SDN является взаимовыгодной, поскольку ключи, полученные с помощью QKD, могут использоваться для защиты различных уровней транспортной сети. Помимо демонстрации, проведенной в статье [157], в которой для обеспечения безопасности использовался алгоритм шифрования (AES), авторы в статье [161] показали, как существующие протоколы безопасности, используемые в плоскости управления, могут интегрировать квантовую криптографию плавным эволюционным способом, не затрагивая текущие схемы. Будучи совместимой в обеих криптосистемах (QKD и традиционных или даже постквантовых схемах), безопасность предложенной системы обеспечивает лучшее из обоих: сертификация традиционных схем по-прежнему применима к гибридной системе, а безопасность полученного решения является максимально возможной, поскольку взлом конечного ключа означает, что обе криптосистемы должны быть скомпрометированы. Данное решение развернуто в каналах управления, организующих работу контроллера SDN и архитектуры виртуализации сетевых функций (NFV) посредством протоколов SSH и TLS.

В [162] сообщалось об эксперименте по мониторингу и смягчению атак физического уровня на основе SDN. Мониторинг QBER и скорости секретного ключа в реальном времени использовался для перерасчета маршрутов для установления квантового канала.
В других случаях больше внимания уделялось безопасности плоскости данных и созданию сервисов. Marksteiner представил интеграцию QKD-производных ключей в каналы IPSec, сосредоточив свое исследование на безопасности и масштабируемости решения в зависимости от пропускной способности сервиса [163]. В дополнение к этому исследованию, подход, о котором сообщается в справочнике [164], сосредоточен на автоматизации услуг для зашифрованных каналов в сквозной сети. Автоматизация была предложена для сценариев центра обработки данных (реализация расширений в OpenFlow) и для транспортных сегментов (использование MPLS и NETCONF для конфигурации). Это было интегрировано в виртуальные сетевые функции, реализующие расширения и канал безопасности с использованием IPSec, как в ссылке [163]. Компания Mavromat продемонстрировала использование QKD для энергоэффективного SDN-управления устройствами Интернета вещей [165].
Мы также отмечаем эксперименты с использованием SDN для управления WDM-организацией QKDлинков [156, 166-168], а также использование моделей машинного обучения (ML) для прогнозирования качества Ch-QKD в сетях QKD-DWDM с повышением эффективности оптических сетей с поддержкой SDN [169].

В широкой сети QKD, где несколько арендаторов QKD используют одну и ту же базовую инфраструктуру, решение проблемы безопасного распределения ключей имеет важное значение для эффективного управления сетью. Cao предложил подход к распределению скорости секретного ключа на основе SDN с использованием эвристического алгоритма с помощью моделирования [170].
Многопользовательская организация может обслуживаться с использованием подхода "ключ как услуга" (KaaS), при котором пулы ключей (KP), определенные на уровне управления иерархии SDN, отображаются на виртуальные пулы ключей с помощью RESTful API на уровне приложений.
Эти результаты показывают, что SDN следует рассматривать как технологический инструмент для интеграции QKD в транспортные сети. В то же время, QKD также приносит пользу сети, поскольку реализует дополнительный уровень ITS для критических инфраструктур. Такой интеграционный подход позволяет плавно интегрировать системы QKD в сеть и коммерциализировать QKD на различных уровнях обслуживания (самовосстанавливающиеся сетевые инфраструктуры, сквозные услуги на различных уровнях OSI и т.д.).

\subsubsection{Заключение}
Квантовая криптография является привлекательной криптографической технологией, которая привлекла внимание различных организаций в академических и промышленных сообществах. В последние годы заметный прогресс в развитии оптического оборудования отразился в ряде успешных демонстраций технологии QKD. Эти демонстрации показывают большие достижения в квантовой криптографии и подчеркивают практические трудности, которые еще предстоит решить.
Мы приводим краткое изложение основных ключевых моментов, связанных с QKD-сетями. Доверенные повторители необходимы для увеличения безопасного расстояния передачи квантовых каналов. Решения для интеграции сетей QKD в существующие оптические сети связи в настоящее время являются актуальной темой в оптических исследованиях. Реальные сети квантовой криптографии, используемые конечными пользователями для реальных приложений передачи информации, станут следующей вехой. С точки зрения промышленности, стандарты для оценки безопасности, производства и применения QKD уже определяются[189, 190].

В настоящее время человек, оказавшийся в лаборатории QKD и спрашивающий о максимально достижимой ключевой скорости, получит ответ с вопросом о расстоянии, которое он/она хочет преодолеть.
Как упоминалось в разделе 3, одним из основных недостатков QKD-соединений является ограничение по длине. Однако сети, обсуждаемые в этом документе, демонстрируют значительное развитие оптического оборудования в последнее время. В 2002 году системы QKD достигли скорости передачи ключей в 1 кбит/с [29], которая использовалась в сети DARPA QKD. В 2007 году в SECOQC эта ключевая скорость увеличилась в десять раз [37], а в 2011 году в токийской сети QKD была достигнута ключевая скорость 300 кбит/с [38]. Этой скорости передачи ключей было достаточно для организации видеоконференции, защищенной шифром OTP, предоставленным QKD. Интересно также сравнить длину связей в этих сетях. Максимальная длина соединения в сети DARPAQKD составляла 29 км через оптический коммутатор между Гарвардским и Бостонским университетами [91]. В SECOQC максимальная длина QKD-связи составила 82 км между узлами BREIT и St. Pölten [37]. В Токио максимальное расстояние составило рекордные 90 км между узлами Коганей-1 и Коганей-2 [171]. В Хэфэй-Чаоху-Вуху (HCW) в Китае максимальное расстояние составило
85,1 км по междугородней линии HCW между Хэфэй и Чаоху [119, 191].
Поэтому разумно ожидать повышения ключевой ставки и увеличения расстояния в ближайшие годы.
Поскольку прогнозируется, что оптические квантовые повторители станут доступными для практического использования в будущем [57], в настоящее время сети QKD реализуются исключительно с помощью подхода доверенных повторителей (TRA). TRA необходим для преодоления ограничений расстояния между QKD-каналами и для обеспечения маршрутизации в QKD-сетях. Однако TRA имеет ряд ограничений, которые необходимо устранить, если сеть QKD будет применяться в повседневной жизни и интегрирована с обычными IP-сетями.
Одним из способов широкого применения технологии QKD является интеграция в телекоммуникационные сети с использованием такого подхода, как SDN-QKD.
\subsection{\review}
This article talks about quantum networks for transmitting information. The article can be divided into several logical parts.  In the first part the authors consider existing technical solutions, in the second part the prospects of quantum key distribution development, and in the third part they make conclusions about the current state of quantum technologies

This article is a review, the authors do not describe new discoveries, they only describe the current state of science in the field of quantum technologies: quantum key distribution, quantum Internet and quantum data networks.

\subsection{\dic}
ss

\clearpage
\section{Scientific report}
%Я Галов Кирилл Алексеевич. Я учусь в аспирантуре Донского Технического Университета на кафедре "Кибербезопасность информационных систем". Основным направлением моей научной деятельности является квантовая криптография.

%Квантовая криптография - это раздел криптографии, который занимается изучением способов защиты информации при помощи квантовых технологий и свойств квантовых объектов. Это важное направление науки, потому что современная классическая криптография основывается на предположении об ограниченности вычислительной мощности злоумышленника и на том, что не существует эффективного алгоритма для взлома шифров. Квантовая криптография основывается на нерушимых законах физики, что делает ее более надежной, чем классическую.

%Я только начал свой путь молодого ученого и еще не успел опубликовать значимого количества научных работ. Однако, за первые пол года учебы в аспирантуре я успел поучаствовать в конференции, материалы которой попадут в РИНЦ и стать соавтором работы по теме безопасности в сети интернет, которая в скором времени должна быть опубликована в Scopus.

%Мне нравится заниматься наукой, это очень интересно и полезно для общества. Надеюсь через несколько лет опубликовать достаточное количество научных статей и защитить диссертацию по теме квантового распределения ключей.

I am Kirill Alekseyevich Galov. I am a postgraduate student at Don State Technical University, department of Cyber Security of Information Systems. The main direction of my scientific activity is quantum cryptography.

Quantum cryptography - is a section of cryptography, which is engaged in the study information protection methods with the quantum technologies and properties of quantum objects. It is an important branch of science because modern classical cryptography is based on the assumption that the computing power of the attacker is limited and that there is no effective algorithm for breaking ciphers. Quantum cryptography is based on the unbreakable laws of physics, which makes it more secure than classical cryptography.

I have just begun my journey as a young scientist and have not yet had time to publish a significant number of scientific papers. However, in the first half year of my postgraduate studies I had time to participate in a conference, the materials of which will be published in the Russian Science Citation Index, and become co-author of papers on Internet security, which are due for publication in Scopus in the near future.

I like doing science, it is very interesting and useful for society. I hope in a few years to publish a sufficient number of scientific articles and defend my dissertation on quantum key distribution.


\clearpage

\pagenumbering{gobble}
\section*{Отзыв научного руководителя на используемую литературу}
Я, Черкесова Лариса Владимировна, научный руководитель аспиранта первого года обучения Галова Кирилла Алексеевича даю согласие на использование аспирантом следующей литературы:
\begin{enumerate}
	\item Huttner B. et al. Quantum cryptography with coherent states //Physical Review A. – 1995. – Т. 51. – №. 3. – С. 1863.
	\item Bennett C. H., Brassard G. Quantum cryptography: Public key distribution and coin tossing //arXiv preprint arXiv:2003.06557. – 2020.
	\item Gisin N. et al. Towards practical and fast quantum cryptography //arXiv preprint quant-ph/0411022. – 2004.
	\item Ekert A. K. Quantum cryptography based on Bell’s theorem //Physical review letters. – 1991. – Т. 67. – №. 6. – С. 661.
	\item Khan M. M., Murphy M., Beige A. High error-rate quantum key distribution for long-distance communication //New Journal of Physics. – 2009. – Т. 11. – №. 6. – С. 063043.
	\item Lo H. K., Ma X., Chen K. Decoy state quantum key distribution //Physical review letters. – 2005. – Т. 94. – №. 23. – С. 230504.
	\item Singh H., Gupta D. L., Singh A. K. Quantum key distribution protocols: a review //Journal of Computer Engineering. – 2014. – Т. 16. – №. 2. – С. 1-9.
	\item Serna E. H. Quantum key distribution protocol with private-public key //arXiv preprint arXiv:0908.2146. – 2009.
	\item Scarani V. et al. Quantum cryptography protocols robust against photon number splitting attacks for weak laser pulse implementations //Physical review letters. – 2004. – Т. 92. – №. 5. – С. 057901.
	\item Song D., Chen D. Quantum Key Distribution Based on Random Grouping Bell State Measurement //IEEE Communications Letters. – 2020. – Т. 24. – №. 7. – С. 1496-1499.
	\item Wang H., Zhao Y., Nag A. Quantum-key-distribution (qkd) networks enabled by software-defined networks (sdn) //Applied Sciences. – 2019. – Т. 9. – №. 10. – С. 2081.
	\item Mehic M. et al. Quantum key distribution: a networking perspective //ACM Computing Surveys (CSUR). – 2020. – Т. 53. – №. 5. – С. 1-41.
\end{enumerate}
Содержание данной литературы соответствует специальности 09.06.01 «Информатика и вычислительная техника». Информация является актуальной и полезной для самостоятельного изучения.\\
\bigskip

%\begin{centering}
\rightline{проф. каф. «КБИС», д.ф.-м.н. \hspace{0.5cm} \underline{\hspace{3cm}} \hspace{0.5cm} Черкесова Л. В}

%\end{centering}


\end{document}